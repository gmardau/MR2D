\section{Introduction}

\paragraph{Design optimisation and fluid dynamic simulators} In many engineering domains, and especially in aeronautics, \textit{fluid dynamic simulators} are commonly used to optimise various system designs. However, the complexity of such simulators is increasing rapidly and high-fidelity simulators are computationally heavy, leading to simulations that may take intolerable amounts of time to finish. In \textit{engineering design optimisation}, these computer simulations are often driven by an optimisation process. The optimisation method, given a set of parameters concerning the shape of the model, repeatedly perturbs these parameters until a solution close to optimal is reached.
Fluid dynamic simulators work with a partitioning of the space around the design in the form of \textit{meshes}, also known as grids, to discretise the problem. The purpose of this work is to develop methods to adapt previous meshes to new designs, thereby preserving useful information between iterations. This should allow the simulator to perform more efficiently in an optimisation context, and thus decrease the overall computational cost of the optimisation process.

\paragraph{High quality meshes and Delaunay triangulations} It is known that the quality of input meshes have a considerable influence over the quality and precision of the results of fluid dynamic simulators. Therefore, when it comes to be able to generate guaranteed good-quality meshes, using \textit{Delaunay triangulations} is a natural choice. Among the optimal properties that they exhibit, the one that stands out the most is that the Delaunay triangulation maximizes the minimum angle among all possible triangulations of a fixed set of points. Moreover, Delaunay triangulations have been object of extensive study over the past years and good algorithms are widely available for their construction and refinement.

\paragraph{Mesh deformation} Incremental approaches to re-meshing based on \textit{mesh deformation} have been studied in the past. These operate by representing the edges and vertices of the mesh as tension and torsion springs, respectively, which, upon perturbations to the domain boundary, are moved so that the forces applied on them are in equilibrium. However, it is not possible to guarantee that the new mesh produced by these methods is still Delaunay, and thus that it retains its optimal properties, without a subsequent verification and correction steps for all the elements of the mesh that have been modified. Additionally, the number of elements in the mesh is not altered when using mesh deformation methods, which may not be suitable for large perturbations.

\paragraph{Mesh remodelling} In this work it is proposed a new incremental approach to re-meshing, denominated \textit{mesh remodelling}. Unlike mesh deformation, mesh remodelling has the possibility of removing and inserting elements to the mesh, and therefore vary their number as needed. Moreover, the new method has the benefit of maintaining a Delaunay triangulation, thus preserving its optimal properties and allowing the application of further refinement algorithms to guarantee high mesh quality in every iteration, crucial for the simulation process.

\paragraph{Document structure} This document is organised as follows. Section 2 presents the concept of Delaunay triangulations, their properties, and methods for their refinement on a two-dimensional space. The mesh remodelling method is proposed in Section 3. Section 4 describes the context in which the new method was developed; both the models used in this work, a specialised version of mesh remodelling, and a practical application are addressed. Finally, Section 5 is dedicated to the experimentation and the analysis of its results.