\section{Case study}

\paragraph{Airfoils and design optimisation} This work is carried out in the context of engineering design optimisation, being airfoils the design in question and the resulting meshes to be used by a fluid dynamics simulator. An airfoil is a cross-section of an airplane wing from a lateral standpoint. In the section that follows, the parametrisation used to create the airfoil models, the specialised version of the mesh remodelling method, and the practical application in which the method is to be used are described.

\subsection{Model parametrisation}

\paragraph{Class Shape Transformation} The airfoil designs used in this study were created using a parametrisation called Class Shape Transformation. It was chosen due to being specifically designed to model the many components of an aircraft, such as fuselages, nacelles, winglets, airfoils, among others. Additionally, it is possible to achieve a great variety of shapes within a particular class of model by changing a single component of the parametrisation, simplifying the process of design optimisation. Also, the fact that one can predict the changes in shape from the parameters variation makes for a very intuitive and easy to work with parametrisation.
\begin{equation*}
\begin{gathered}
a(t) = \begin{cases}
\eta_u(1-2t) & \text{, } 0 \leq t < \frac{1}{2}\\
\eta_l(2t-1) & \text{, } \frac{1}{2} \leq t < 1
\end{cases}\\[1pc]
\left.
\begin{aligned}
\eta_u(\psi) &= C_u(\psi) \times S_u(\psi) + T_u(\psi)\\[0.75pc]
C_u(\psi) &= \psi ^{e_1} \times (1 - \psi) ^ {e_2}\\
S_u(\psi) &= \sum_{i=0}^{n} A_{ui} \times B_{i,n}(\psi)\\
T_u(\psi) &= \psi \times \Delta\eta_u
\end{aligned}
\quad\quad\right|\quad\quad
\begin{aligned}
\eta_l(\psi) &= C_l(\psi) \times S_l(\psi) + T_l(\psi)\\[0.75pc]
C_l(\psi) &= \psi ^{e_1} \times (1 - \psi) ^ {e_2}\\
S_l(\psi) &= \sum_{i=0}^{n} A_{li} \times B_{i,n}(\psi)\\
T_l(\psi) &= \psi \times \Delta\eta_l
\end{aligned}\\[1pc]
\psi = \dfrac{x}{c}
\quad\quad\quad\quad
\eta = \dfrac{y}{c}
\end{gathered}
\end{equation*}

\paragraph{Formulation} The shape of the model is given by the \textit{airfoil function}\footnote{Not a part of the original scheme.}, $a(t)$, which purpose is to merge the functions of the upper and lower surfaces of the airfoil into one, describing a counter-clockwise path that starts and ends at the trailing edge of the airfoil. In this formulation, $y$ is a function of $x$, which in turn is a function of $t$. The upper surface functions and parameters contain the subscript $u$, while the lower surface functions and parameters contain the subscript $l$. The \textit{Class function}, $C(\psi)$, defines the class of the model, giving it an initial shape. The coefficients $e_1$ and $e_2$ control the shape of the model at its leading and trailing edges, respectively, assuming the values 0.5 and 1 in the ``NACA airfoil'' class --- the one used in this work. The \textit{Shape function}, $S(\psi)$, adjusts the basic shape of the model with the help of $A$, a vector of \textit{shape coefficients} --- object of study in design optimisation. $B_{i,n}(\psi)$ represents the $i^\text{th}$ Bernstein basis polynomial of degree $n$. Finally, the \textit{Trailing edge function}\footnote{Not presented as a separate function in the original scheme.}, $T(\psi)$, is used to translate the trailing edge of the airfoil along the $y$ axis, being $\Delta\eta$ the translation amount. The variable $c$ denotes the length of the airfoil chord, i.e.\ the horizontal distance between the leading and trailing edges of the airfoil.

\subsection{Mesh remodelling specialisation}

\paragraph{Model characteristics and premises} Given the specific characteristics of the CST parametrisation as well as some design choices, it is possible to optimise the more general mesh remodelling method into a faster and still as robust version, although more complex, so that it takes full advantage of the models in use. Some of these characteristics and choices are as follows:
\begin{itemize}
\item The model is defined by two functions, guaranteeing that for a given value of $x$ there is a unique value of $y$;
\item The chord length, $c$, is set to 1, therefore limiting the $x$-domain of all models to the interval $[0,1]$;
\item The trailing edge translation value, $\Delta\eta$, is set to 0 on both surfaces, ensuring the continuous presence of a vertex at the coordinates $(1,0)$;
\item The number of vertices in the initial discretisation is maintained throughout the optimisation process, allowing a one-to-one correlation between vertices of different models. Also, due to $y$ being a function of $x$ and $x$ a function of $t$, these vertices differ only in their $y$-value.
\end{itemize}

\paragraph{Adjustment} The first step of the specialised version of mesh remodelling aims to speed-up the method by adjusting the coordinates of some, if not all of the boundary vertices, thus reducing the number of circular region removals and vertex replacements to be performed in a latter stage. This is only possible due to the singular properties of the models addressed in this work, especially the first and fourth previously listed, which guarantee that no boundary edges of the same surface are going to intersect regardless of the magnitude of the adjustments. This process is applied to all boundary vertices, whether Steiner or non-Steiner. Let $v$ be the vertex being currently checked and $v_n$ its new position; let $U$ be $v$'s neighbours and $U_n$ their new positions. Vertex $v$ can only be adjusted if all the following prove true:
\begin{itemize}
\item $v_n$ is not be beyond the current opposite surface;
\item Considering $v_n$ and $U$: $v$'s surrounding triangles are valid;
\item Considering $v_n$ and $U_n$: $v$'s surrounding triangles are valid, Delaunay, and respect the mesh's quality and gradation constraints.
\end{itemize}
The first verification prevents the two surfaces from intersecting when one surface gets adjusted and the other does not. The second verification guarantees that, whichever combination of vertices is adjusted, the resulting mesh is still valid. The third and last verification covers the properties that any triangle in a fully refined Delaunay triangulation must possess, which is the reason why, in the case that every boundary vertex gets adjusted, not only the next step of mesh remodelling but also the subsequent refinement procedure can be skipped. If $v$ passes all checks, then its coordinates are updated to $v_n$. If that is not the case and $v$ happens to be non-Steiner, then the Steiner vertices between $v$ and its two non-Steiner boundary neighbours are not even considered for adjustment. This is done to prevent the accumulation of vertices in regions that need to be rebuilt, which could lead to over-refinement.

\paragraph{Removal and replacement} The second and final stage of the specialised mesh remodelling version is very similar to the original method, the only difference being the number of circular removal regions employed; one for each pair of old/new vertices, given the one-to-one correlation, instead of one for each new and old vertex. The centre of such removal regions is located at the outermost vertex --- highest $y$-value for vertices belonging to the upper surface; the opposite for the lower surface. Regarding the radius of these regions, although it is recommended that it be determined in the same way as the original, one can use the vertical distance between the pair of vertices instead and still achieve good results. If the vertex was already adjusted, the use of a removal region for that particular pair of vertices can be avoided altogether.
\begin{figure}[!h]
\begin{center}
\begin{tikzpicture}[gnuplot, scale=1.5]
%% generated with GNUPLOT 5.0p5 (Lua 5.3; terminal rev. 99, script rev. 100)
%% Thu 29 Dec 2016 05:38:31 PM WET
%\path (0.000,0.000) rectangle (12.500,8.750);
%
\gpfill{rgb color={0.000,0.000,0.000},opacity=0.15} (10.995,4.375)--(10.994,4.375)--(10.994,4.375)--(10.994,4.375)%
--(10.994,4.376)--(10.994,4.376)--(10.994,4.376)--(10.994,4.376)--(10.994,4.377)%
--(10.994,4.377)--(10.994,4.377)--(10.994,4.377)--(10.994,4.377)--(10.993,4.378)%
--(10.993,4.378)--(10.993,4.378)--(10.993,4.378)--(10.993,4.378)--(10.992,4.379)%
--(10.992,4.379)--(10.992,4.379)--(10.992,4.379)--(10.992,4.379)--(10.991,4.379)%
--(10.991,4.379)--(10.991,4.379)--(10.991,4.379)--(10.990,4.379)--(10.990,4.379)%
--(10.990,4.379)--(10.990,4.380)--(10.989,4.379)--(10.989,4.379)--(10.989,4.379)%
--(10.988,4.379)--(10.988,4.379)--(10.988,4.379)--(10.988,4.379)--(10.987,4.379)%
--(10.987,4.379)--(10.987,4.379)--(10.987,4.379)--(10.987,4.379)--(10.986,4.378)%
--(10.986,4.378)--(10.986,4.378)--(10.986,4.378)--(10.986,4.378)--(10.985,4.377)%
--(10.985,4.377)--(10.985,4.377)--(10.985,4.377)--(10.985,4.377)--(10.985,4.376)%
--(10.985,4.376)--(10.985,4.376)--(10.985,4.376)--(10.985,4.375)--(10.985,4.375)%
--(10.985,4.375)--(10.985,4.375)--(10.985,4.374)--(10.985,4.374)--(10.985,4.374)%
--(10.985,4.373)--(10.985,4.373)--(10.985,4.373)--(10.985,4.373)--(10.985,4.372)%
--(10.985,4.372)--(10.985,4.372)--(10.985,4.372)--(10.985,4.372)--(10.986,4.371)%
--(10.986,4.371)--(10.986,4.371)--(10.986,4.371)--(10.986,4.371)--(10.987,4.370)%
--(10.987,4.370)--(10.987,4.370)--(10.987,4.370)--(10.987,4.370)--(10.988,4.370)%
--(10.988,4.370)--(10.988,4.370)--(10.988,4.370)--(10.989,4.370)--(10.989,4.370)%
--(10.989,4.370)--(10.990,4.370)--(10.990,4.370)--(10.990,4.370)--(10.990,4.370)%
--(10.991,4.370)--(10.991,4.370)--(10.991,4.370)--(10.991,4.370)--(10.992,4.370)%
--(10.992,4.370)--(10.992,4.370)--(10.992,4.370)--(10.992,4.370)--(10.993,4.371)%
--(10.993,4.371)--(10.993,4.371)--(10.993,4.371)--(10.993,4.371)--(10.994,4.372)%
--(10.994,4.372)--(10.994,4.372)--(10.994,4.372)--(10.994,4.372)--(10.994,4.373)%
--(10.994,4.373)--(10.994,4.373)--(10.994,4.373)--(10.994,4.374)--(10.994,4.374)--(10.994,4.374)--cycle;
%
\gpfill{rgb color={0.000,0.000,0.000},opacity=0.15} (10.999,4.375)--(10.998,4.375)--(10.998,4.376)--(10.998,4.377)%
--(10.998,4.377)--(10.998,4.378)--(10.998,4.379)--(10.998,4.379)--(10.997,4.380)%
--(10.997,4.380)--(10.997,4.381)--(10.996,4.382)--(10.996,4.382)--(10.996,4.383)%
--(10.995,4.383)--(10.995,4.384)--(10.994,4.384)--(10.994,4.385)--(10.993,4.385)%
--(10.993,4.385)--(10.992,4.386)--(10.991,4.386)--(10.991,4.386)--(10.990,4.387)%
--(10.990,4.387)--(10.989,4.387)--(10.988,4.387)--(10.988,4.387)--(10.987,4.387)%
--(10.986,4.387)--(10.986,4.388)--(10.985,4.387)--(10.984,4.387)--(10.983,4.387)%
--(10.983,4.387)--(10.982,4.387)--(10.981,4.387)--(10.981,4.387)--(10.980,4.386)%
--(10.980,4.386)--(10.979,4.386)--(10.978,4.385)--(10.978,4.385)--(10.977,4.385)%
--(10.977,4.384)--(10.976,4.384)--(10.976,4.383)--(10.975,4.383)--(10.975,4.382)%
--(10.975,4.382)--(10.974,4.381)--(10.974,4.380)--(10.974,4.380)--(10.973,4.379)%
--(10.973,4.379)--(10.973,4.378)--(10.973,4.377)--(10.973,4.377)--(10.973,4.376)%
--(10.973,4.375)--(10.973,4.375)--(10.973,4.374)--(10.973,4.373)--(10.973,4.372)%
--(10.973,4.372)--(10.973,4.371)--(10.973,4.370)--(10.973,4.370)--(10.974,4.369)%
--(10.974,4.369)--(10.974,4.368)--(10.975,4.367)--(10.975,4.367)--(10.975,4.366)%
--(10.976,4.366)--(10.976,4.365)--(10.977,4.365)--(10.977,4.364)--(10.978,4.364)%
--(10.978,4.364)--(10.979,4.363)--(10.980,4.363)--(10.980,4.363)--(10.981,4.362)%
--(10.981,4.362)--(10.982,4.362)--(10.983,4.362)--(10.983,4.362)--(10.984,4.362)%
--(10.985,4.362)--(10.986,4.362)--(10.986,4.362)--(10.987,4.362)--(10.988,4.362)%
--(10.988,4.362)--(10.989,4.362)--(10.990,4.362)--(10.990,4.362)--(10.991,4.363)%
--(10.991,4.363)--(10.992,4.363)--(10.993,4.364)--(10.993,4.364)--(10.994,4.364)%
--(10.994,4.365)--(10.995,4.365)--(10.995,4.366)--(10.996,4.366)--(10.996,4.367)%
--(10.996,4.367)--(10.997,4.368)--(10.997,4.369)--(10.997,4.369)--(10.998,4.370)%
--(10.998,4.370)--(10.998,4.371)--(10.998,4.372)--(10.998,4.372)--(10.998,4.373)--(10.998,4.374)--cycle;
%
\gpfill{rgb color={0.000,0.000,0.000},opacity=0.15} (10.994,4.378)--(10.993,4.379)--(10.993,4.380)--(10.993,4.381)%
--(10.993,4.382)--(10.993,4.383)--(10.992,4.384)--(10.992,4.385)--(10.992,4.386)%
--(10.991,4.387)--(10.991,4.388)--(10.990,4.389)--(10.989,4.390)--(10.989,4.391)%
--(10.988,4.392)--(10.987,4.392)--(10.987,4.393)--(10.986,4.394)--(10.985,4.394)%
--(10.984,4.395)--(10.983,4.396)--(10.982,4.396)--(10.981,4.397)--(10.980,4.397)%
--(10.979,4.397)--(10.978,4.398)--(10.977,4.398)--(10.976,4.398)--(10.975,4.398)%
--(10.974,4.398)--(10.973,4.399)--(10.971,4.398)--(10.970,4.398)--(10.969,4.398)%
--(10.968,4.398)--(10.967,4.398)--(10.966,4.397)--(10.965,4.397)--(10.964,4.397)%
--(10.963,4.396)--(10.962,4.396)--(10.961,4.395)--(10.960,4.394)--(10.959,4.394)%
--(10.958,4.393)--(10.958,4.392)--(10.957,4.392)--(10.956,4.391)--(10.956,4.390)%
--(10.955,4.389)--(10.954,4.388)--(10.954,4.387)--(10.953,4.386)--(10.953,4.385)%
--(10.953,4.384)--(10.952,4.383)--(10.952,4.382)--(10.952,4.381)--(10.952,4.380)%
--(10.952,4.379)--(10.952,4.378)--(10.952,4.376)--(10.952,4.375)--(10.952,4.374)%
--(10.952,4.373)--(10.952,4.372)--(10.953,4.371)--(10.953,4.370)--(10.953,4.369)%
--(10.954,4.368)--(10.954,4.367)--(10.955,4.366)--(10.956,4.365)--(10.956,4.364)%
--(10.957,4.363)--(10.958,4.363)--(10.958,4.362)--(10.959,4.361)--(10.960,4.361)%
--(10.961,4.360)--(10.962,4.359)--(10.963,4.359)--(10.964,4.358)--(10.965,4.358)%
--(10.966,4.358)--(10.967,4.357)--(10.968,4.357)--(10.969,4.357)--(10.970,4.357)%
--(10.971,4.357)--(10.973,4.357)--(10.974,4.357)--(10.975,4.357)--(10.976,4.357)%
--(10.977,4.357)--(10.978,4.357)--(10.979,4.358)--(10.980,4.358)--(10.981,4.358)%
--(10.982,4.359)--(10.983,4.359)--(10.984,4.360)--(10.985,4.361)--(10.986,4.361)%
--(10.987,4.362)--(10.987,4.363)--(10.988,4.363)--(10.989,4.364)--(10.989,4.365)%
--(10.990,4.366)--(10.991,4.367)--(10.991,4.368)--(10.992,4.369)--(10.992,4.370)%
--(10.992,4.371)--(10.993,4.372)--(10.993,4.373)--(10.993,4.374)--(10.993,4.375)--(10.993,4.376)--cycle;
%
\gpfill{rgb color={0.000,0.000,0.000},opacity=0.15} (10.982,4.382)--(10.981,4.383)--(10.981,4.385)--(10.981,4.386)%
--(10.981,4.388)--(10.980,4.389)--(10.980,4.391)--(10.980,4.392)--(10.979,4.394)%
--(10.978,4.395)--(10.977,4.397)--(10.977,4.398)--(10.976,4.399)--(10.975,4.400)%
--(10.974,4.402)--(10.973,4.403)--(10.972,4.404)--(10.970,4.405)--(10.969,4.406)%
--(10.968,4.407)--(10.967,4.407)--(10.965,4.408)--(10.964,4.409)--(10.962,4.410)%
--(10.961,4.410)--(10.959,4.410)--(10.958,4.411)--(10.956,4.411)--(10.955,4.411)%
--(10.953,4.411)--(10.952,4.412)--(10.950,4.411)--(10.948,4.411)--(10.947,4.411)%
--(10.945,4.411)--(10.944,4.410)--(10.942,4.410)--(10.941,4.410)--(10.939,4.409)%
--(10.938,4.408)--(10.937,4.407)--(10.935,4.407)--(10.934,4.406)--(10.933,4.405)%
--(10.931,4.404)--(10.930,4.403)--(10.929,4.402)--(10.928,4.400)--(10.927,4.399)%
--(10.926,4.398)--(10.926,4.397)--(10.925,4.395)--(10.924,4.394)--(10.923,4.392)%
--(10.923,4.391)--(10.923,4.389)--(10.922,4.388)--(10.922,4.386)--(10.922,4.385)%
--(10.922,4.383)--(10.922,4.382)--(10.922,4.380)--(10.922,4.378)--(10.922,4.377)%
--(10.922,4.375)--(10.923,4.374)--(10.923,4.372)--(10.923,4.371)--(10.924,4.369)%
--(10.925,4.368)--(10.926,4.367)--(10.926,4.365)--(10.927,4.364)--(10.928,4.363)%
--(10.929,4.361)--(10.930,4.360)--(10.931,4.359)--(10.933,4.358)--(10.934,4.357)%
--(10.935,4.356)--(10.937,4.356)--(10.938,4.355)--(10.939,4.354)--(10.941,4.353)%
--(10.942,4.353)--(10.944,4.353)--(10.945,4.352)--(10.947,4.352)--(10.948,4.352)%
--(10.950,4.352)--(10.952,4.352)--(10.953,4.352)--(10.955,4.352)--(10.956,4.352)%
--(10.958,4.352)--(10.959,4.353)--(10.961,4.353)--(10.962,4.353)--(10.964,4.354)%
--(10.965,4.355)--(10.967,4.356)--(10.968,4.356)--(10.969,4.357)--(10.970,4.358)%
--(10.972,4.359)--(10.973,4.360)--(10.974,4.361)--(10.975,4.363)--(10.976,4.364)%
--(10.977,4.365)--(10.977,4.367)--(10.978,4.368)--(10.979,4.369)--(10.980,4.371)%
--(10.980,4.372)--(10.980,4.374)--(10.981,4.375)--(10.981,4.377)--(10.981,4.378)--(10.981,4.380)--cycle;
%
\gpfill{rgb color={0.000,0.000,0.000},opacity=0.15} (10.962,4.388)--(10.961,4.390)--(10.961,4.392)--(10.961,4.394)%
--(10.961,4.396)--(10.960,4.398)--(10.960,4.400)--(10.959,4.401)--(10.958,4.403)%
--(10.957,4.405)--(10.956,4.407)--(10.955,4.409)--(10.954,4.410)--(10.953,4.412)%
--(10.951,4.414)--(10.950,4.415)--(10.949,4.416)--(10.947,4.418)--(10.945,4.419)%
--(10.944,4.420)--(10.942,4.421)--(10.940,4.422)--(10.938,4.423)--(10.936,4.424)%
--(10.935,4.425)--(10.933,4.425)--(10.931,4.426)--(10.929,4.426)--(10.927,4.426)%
--(10.925,4.426)--(10.923,4.427)--(10.920,4.426)--(10.918,4.426)--(10.916,4.426)%
--(10.914,4.426)--(10.912,4.425)--(10.910,4.425)--(10.909,4.424)--(10.907,4.423)%
--(10.905,4.422)--(10.903,4.421)--(10.901,4.420)--(10.900,4.419)--(10.898,4.418)%
--(10.896,4.416)--(10.895,4.415)--(10.894,4.414)--(10.892,4.412)--(10.891,4.410)%
--(10.890,4.409)--(10.889,4.407)--(10.888,4.405)--(10.887,4.403)--(10.886,4.401)%
--(10.885,4.400)--(10.885,4.398)--(10.884,4.396)--(10.884,4.394)--(10.884,4.392)%
--(10.884,4.390)--(10.884,4.388)--(10.884,4.385)--(10.884,4.383)--(10.884,4.381)%
--(10.884,4.379)--(10.885,4.377)--(10.885,4.375)--(10.886,4.374)--(10.887,4.372)%
--(10.888,4.370)--(10.889,4.368)--(10.890,4.366)--(10.891,4.365)--(10.892,4.363)%
--(10.894,4.361)--(10.895,4.360)--(10.896,4.359)--(10.898,4.357)--(10.900,4.356)%
--(10.901,4.355)--(10.903,4.354)--(10.905,4.353)--(10.907,4.352)--(10.909,4.351)%
--(10.910,4.350)--(10.912,4.350)--(10.914,4.349)--(10.916,4.349)--(10.918,4.349)%
--(10.920,4.349)--(10.923,4.349)--(10.925,4.349)--(10.927,4.349)--(10.929,4.349)%
--(10.931,4.349)--(10.933,4.350)--(10.935,4.350)--(10.936,4.351)--(10.938,4.352)%
--(10.940,4.353)--(10.942,4.354)--(10.944,4.355)--(10.945,4.356)--(10.947,4.357)%
--(10.949,4.359)--(10.950,4.360)--(10.951,4.361)--(10.953,4.363)--(10.954,4.365)%
--(10.955,4.366)--(10.956,4.368)--(10.957,4.370)--(10.958,4.372)--(10.959,4.374)%
--(10.960,4.375)--(10.960,4.377)--(10.961,4.379)--(10.961,4.381)--(10.961,4.383)--(10.961,4.385)--cycle;
%
\gpfill{rgb color={0.000,0.000,0.000},opacity=0.15} (10.931,4.395)--(10.930,4.397)--(10.930,4.399)--(10.930,4.402)%
--(10.929,4.404)--(10.929,4.406)--(10.928,4.409)--(10.927,4.411)--(10.927,4.413)%
--(10.925,4.415)--(10.924,4.418)--(10.923,4.420)--(10.922,4.422)--(10.920,4.423)%
--(10.919,4.425)--(10.917,4.427)--(10.915,4.429)--(10.913,4.430)--(10.912,4.432)%
--(10.910,4.433)--(10.908,4.434)--(10.905,4.435)--(10.903,4.437)--(10.901,4.437)%
--(10.899,4.438)--(10.896,4.439)--(10.894,4.439)--(10.892,4.440)--(10.889,4.440)%
--(10.887,4.440)--(10.885,4.441)--(10.882,4.440)--(10.880,4.440)--(10.877,4.440)%
--(10.875,4.439)--(10.873,4.439)--(10.870,4.438)--(10.868,4.437)--(10.866,4.437)%
--(10.864,4.435)--(10.862,4.434)--(10.859,4.433)--(10.857,4.432)--(10.856,4.430)%
--(10.854,4.429)--(10.852,4.427)--(10.850,4.425)--(10.849,4.423)--(10.847,4.422)%
--(10.846,4.420)--(10.845,4.418)--(10.844,4.415)--(10.842,4.413)--(10.842,4.411)%
--(10.841,4.409)--(10.840,4.406)--(10.840,4.404)--(10.839,4.402)--(10.839,4.399)%
--(10.839,4.397)--(10.839,4.395)--(10.839,4.392)--(10.839,4.390)--(10.839,4.387)%
--(10.840,4.385)--(10.840,4.383)--(10.841,4.380)--(10.842,4.378)--(10.842,4.376)%
--(10.844,4.374)--(10.845,4.372)--(10.846,4.369)--(10.847,4.367)--(10.849,4.366)%
--(10.850,4.364)--(10.852,4.362)--(10.854,4.360)--(10.856,4.359)--(10.857,4.357)%
--(10.859,4.356)--(10.862,4.355)--(10.864,4.354)--(10.866,4.352)--(10.868,4.352)%
--(10.870,4.351)--(10.873,4.350)--(10.875,4.350)--(10.877,4.349)--(10.880,4.349)%
--(10.882,4.349)--(10.885,4.349)--(10.887,4.349)--(10.889,4.349)--(10.892,4.349)%
--(10.894,4.350)--(10.896,4.350)--(10.899,4.351)--(10.901,4.352)--(10.903,4.352)%
--(10.905,4.354)--(10.908,4.355)--(10.910,4.356)--(10.912,4.357)--(10.913,4.359)%
--(10.915,4.360)--(10.917,4.362)--(10.919,4.364)--(10.920,4.366)--(10.922,4.367)%
--(10.923,4.369)--(10.924,4.372)--(10.925,4.374)--(10.927,4.376)--(10.927,4.378)%
--(10.928,4.380)--(10.929,4.383)--(10.929,4.385)--(10.930,4.387)--(10.930,4.390)--(10.930,4.392)--cycle;
%
\gpfill{rgb color={0.000,0.000,0.000},opacity=0.15} (10.894,4.404)--(10.893,4.406)--(10.893,4.409)--(10.893,4.412)%
--(10.892,4.415)--(10.892,4.418)--(10.891,4.420)--(10.890,4.423)--(10.889,4.426)%
--(10.888,4.428)--(10.886,4.431)--(10.885,4.433)--(10.883,4.436)--(10.881,4.438)%
--(10.879,4.440)--(10.877,4.442)--(10.875,4.444)--(10.873,4.446)--(10.871,4.448)%
--(10.868,4.450)--(10.866,4.451)--(10.863,4.453)--(10.861,4.454)--(10.858,4.455)%
--(10.855,4.456)--(10.853,4.457)--(10.850,4.457)--(10.847,4.458)--(10.844,4.458)%
--(10.841,4.458)--(10.839,4.459)--(10.836,4.458)--(10.833,4.458)--(10.830,4.458)%
--(10.827,4.457)--(10.824,4.457)--(10.822,4.456)--(10.819,4.455)--(10.816,4.454)%
--(10.814,4.453)--(10.811,4.451)--(10.809,4.450)--(10.806,4.448)--(10.804,4.446)%
--(10.802,4.444)--(10.800,4.442)--(10.798,4.440)--(10.796,4.438)--(10.794,4.436)%
--(10.792,4.433)--(10.791,4.431)--(10.789,4.428)--(10.788,4.426)--(10.787,4.423)%
--(10.786,4.420)--(10.785,4.418)--(10.785,4.415)--(10.784,4.412)--(10.784,4.409)%
--(10.784,4.406)--(10.784,4.404)--(10.784,4.401)--(10.784,4.398)--(10.784,4.395)%
--(10.785,4.392)--(10.785,4.389)--(10.786,4.387)--(10.787,4.384)--(10.788,4.381)%
--(10.789,4.379)--(10.791,4.376)--(10.792,4.374)--(10.794,4.371)--(10.796,4.369)%
--(10.798,4.367)--(10.800,4.365)--(10.802,4.363)--(10.804,4.361)--(10.806,4.359)%
--(10.809,4.357)--(10.811,4.356)--(10.814,4.354)--(10.816,4.353)--(10.819,4.352)%
--(10.822,4.351)--(10.824,4.350)--(10.827,4.350)--(10.830,4.349)--(10.833,4.349)%
--(10.836,4.349)--(10.839,4.349)--(10.841,4.349)--(10.844,4.349)--(10.847,4.349)%
--(10.850,4.350)--(10.853,4.350)--(10.855,4.351)--(10.858,4.352)--(10.861,4.353)%
--(10.863,4.354)--(10.866,4.356)--(10.868,4.357)--(10.871,4.359)--(10.873,4.361)%
--(10.875,4.363)--(10.877,4.365)--(10.879,4.367)--(10.881,4.369)--(10.883,4.371)%
--(10.885,4.374)--(10.886,4.376)--(10.888,4.379)--(10.889,4.381)--(10.890,4.384)%
--(10.891,4.387)--(10.892,4.389)--(10.892,4.392)--(10.893,4.395)--(10.893,4.398)--(10.893,4.401)--cycle;
%
\gpfill{rgb color={0.000,0.000,0.000},opacity=0.15} (10.848,4.414)--(10.847,4.417)--(10.847,4.420)--(10.847,4.423)%
--(10.846,4.427)--(10.845,4.430)--(10.844,4.433)--(10.843,4.436)--(10.842,4.439)%
--(10.841,4.442)--(10.839,4.445)--(10.837,4.448)--(10.835,4.451)--(10.833,4.453)%
--(10.831,4.456)--(10.829,4.458)--(10.827,4.460)--(10.824,4.462)--(10.822,4.464)%
--(10.819,4.466)--(10.816,4.468)--(10.813,4.470)--(10.810,4.471)--(10.807,4.472)%
--(10.804,4.473)--(10.801,4.474)--(10.798,4.475)--(10.794,4.476)--(10.791,4.476)%
--(10.788,4.476)--(10.785,4.477)--(10.781,4.476)--(10.778,4.476)--(10.775,4.476)%
--(10.771,4.475)--(10.768,4.474)--(10.765,4.473)--(10.762,4.472)--(10.759,4.471)%
--(10.756,4.470)--(10.753,4.468)--(10.750,4.466)--(10.747,4.464)--(10.745,4.462)%
--(10.742,4.460)--(10.740,4.458)--(10.738,4.456)--(10.736,4.453)--(10.734,4.451)%
--(10.732,4.448)--(10.730,4.445)--(10.728,4.442)--(10.727,4.439)--(10.726,4.436)%
--(10.725,4.433)--(10.724,4.430)--(10.723,4.427)--(10.722,4.423)--(10.722,4.420)%
--(10.722,4.417)--(10.722,4.414)--(10.722,4.410)--(10.722,4.407)--(10.722,4.404)%
--(10.723,4.400)--(10.724,4.397)--(10.725,4.394)--(10.726,4.391)--(10.727,4.388)%
--(10.728,4.385)--(10.730,4.382)--(10.732,4.379)--(10.734,4.376)--(10.736,4.374)%
--(10.738,4.371)--(10.740,4.369)--(10.742,4.367)--(10.745,4.365)--(10.747,4.363)%
--(10.750,4.361)--(10.753,4.359)--(10.756,4.357)--(10.759,4.356)--(10.762,4.355)%
--(10.765,4.354)--(10.768,4.353)--(10.771,4.352)--(10.775,4.351)--(10.778,4.351)%
--(10.781,4.351)--(10.785,4.351)--(10.788,4.351)--(10.791,4.351)--(10.794,4.351)%
--(10.798,4.352)--(10.801,4.353)--(10.804,4.354)--(10.807,4.355)--(10.810,4.356)%
--(10.813,4.357)--(10.816,4.359)--(10.819,4.361)--(10.822,4.363)--(10.824,4.365)%
--(10.827,4.367)--(10.829,4.369)--(10.831,4.371)--(10.833,4.374)--(10.835,4.376)%
--(10.837,4.379)--(10.839,4.382)--(10.841,4.385)--(10.842,4.388)--(10.843,4.391)%
--(10.844,4.394)--(10.845,4.397)--(10.846,4.400)--(10.847,4.404)--(10.847,4.407)--(10.847,4.410)--cycle;
%
\gpfill{rgb color={0.000,0.000,0.000},opacity=0.15} (10.796,4.426)--(10.795,4.429)--(10.795,4.433)--(10.795,4.437)%
--(10.794,4.440)--(10.793,4.444)--(10.792,4.448)--(10.791,4.451)--(10.789,4.455)%
--(10.788,4.458)--(10.786,4.462)--(10.784,4.465)--(10.782,4.468)--(10.779,4.471)%
--(10.777,4.474)--(10.774,4.476)--(10.772,4.479)--(10.769,4.481)--(10.766,4.484)%
--(10.763,4.486)--(10.760,4.488)--(10.756,4.490)--(10.753,4.491)--(10.749,4.493)%
--(10.746,4.494)--(10.742,4.495)--(10.738,4.496)--(10.735,4.497)--(10.731,4.497)%
--(10.727,4.497)--(10.724,4.498)--(10.720,4.497)--(10.716,4.497)--(10.712,4.497)%
--(10.709,4.496)--(10.705,4.495)--(10.701,4.494)--(10.698,4.493)--(10.694,4.491)%
--(10.691,4.490)--(10.688,4.488)--(10.684,4.486)--(10.681,4.484)--(10.678,4.481)%
--(10.675,4.479)--(10.673,4.476)--(10.670,4.474)--(10.668,4.471)--(10.665,4.468)%
--(10.663,4.465)--(10.661,4.462)--(10.659,4.458)--(10.658,4.455)--(10.656,4.451)%
--(10.655,4.448)--(10.654,4.444)--(10.653,4.440)--(10.652,4.437)--(10.652,4.433)%
--(10.652,4.429)--(10.652,4.426)--(10.652,4.422)--(10.652,4.418)--(10.652,4.414)%
--(10.653,4.411)--(10.654,4.407)--(10.655,4.403)--(10.656,4.400)--(10.658,4.396)%
--(10.659,4.393)--(10.661,4.390)--(10.663,4.386)--(10.665,4.383)--(10.668,4.380)%
--(10.670,4.377)--(10.673,4.375)--(10.675,4.372)--(10.678,4.370)--(10.681,4.367)%
--(10.684,4.365)--(10.688,4.363)--(10.691,4.361)--(10.694,4.360)--(10.698,4.358)%
--(10.701,4.357)--(10.705,4.356)--(10.709,4.355)--(10.712,4.354)--(10.716,4.354)%
--(10.720,4.354)--(10.724,4.354)--(10.727,4.354)--(10.731,4.354)--(10.735,4.354)%
--(10.738,4.355)--(10.742,4.356)--(10.746,4.357)--(10.749,4.358)--(10.753,4.360)%
--(10.756,4.361)--(10.760,4.363)--(10.763,4.365)--(10.766,4.367)--(10.769,4.370)%
--(10.772,4.372)--(10.774,4.375)--(10.777,4.377)--(10.779,4.380)--(10.782,4.383)%
--(10.784,4.386)--(10.786,4.390)--(10.788,4.393)--(10.789,4.396)--(10.791,4.400)%
--(10.792,4.403)--(10.793,4.407)--(10.794,4.411)--(10.795,4.414)--(10.795,4.418)--(10.795,4.422)--cycle;
%
\gpfill{rgb color={0.000,0.000,0.000},opacity=0.15} (10.733,4.439)--(10.732,4.443)--(10.732,4.447)--(10.732,4.451)%
--(10.731,4.455)--(10.730,4.459)--(10.729,4.463)--(10.727,4.467)--(10.726,4.471)%
--(10.724,4.474)--(10.722,4.478)--(10.720,4.482)--(10.717,4.485)--(10.715,4.488)%
--(10.712,4.491)--(10.709,4.494)--(10.706,4.497)--(10.703,4.500)--(10.700,4.502)%
--(10.697,4.505)--(10.693,4.507)--(10.689,4.509)--(10.686,4.511)--(10.682,4.512)%
--(10.678,4.514)--(10.674,4.515)--(10.670,4.516)--(10.666,4.517)--(10.662,4.517)%
--(10.658,4.517)--(10.654,4.518)--(10.649,4.517)--(10.645,4.517)--(10.641,4.517)%
--(10.637,4.516)--(10.633,4.515)--(10.629,4.514)--(10.625,4.512)--(10.621,4.511)%
--(10.618,4.509)--(10.614,4.507)--(10.610,4.505)--(10.607,4.502)--(10.604,4.500)%
--(10.601,4.497)--(10.598,4.494)--(10.595,4.491)--(10.592,4.488)--(10.590,4.485)%
--(10.587,4.482)--(10.585,4.478)--(10.583,4.474)--(10.581,4.471)--(10.580,4.467)%
--(10.578,4.463)--(10.577,4.459)--(10.576,4.455)--(10.575,4.451)--(10.575,4.447)%
--(10.575,4.443)--(10.575,4.439)--(10.575,4.434)--(10.575,4.430)--(10.575,4.426)%
--(10.576,4.422)--(10.577,4.418)--(10.578,4.414)--(10.580,4.410)--(10.581,4.406)%
--(10.583,4.403)--(10.585,4.399)--(10.587,4.395)--(10.590,4.392)--(10.592,4.389)%
--(10.595,4.386)--(10.598,4.383)--(10.601,4.380)--(10.604,4.377)--(10.607,4.375)%
--(10.610,4.372)--(10.614,4.370)--(10.618,4.368)--(10.621,4.366)--(10.625,4.365)%
--(10.629,4.363)--(10.633,4.362)--(10.637,4.361)--(10.641,4.360)--(10.645,4.360)%
--(10.649,4.360)--(10.654,4.360)--(10.658,4.360)--(10.662,4.360)--(10.666,4.360)%
--(10.670,4.361)--(10.674,4.362)--(10.678,4.363)--(10.682,4.365)--(10.686,4.366)%
--(10.689,4.368)--(10.693,4.370)--(10.697,4.372)--(10.700,4.375)--(10.703,4.377)%
--(10.706,4.380)--(10.709,4.383)--(10.712,4.386)--(10.715,4.389)--(10.717,4.392)%
--(10.720,4.395)--(10.722,4.399)--(10.724,4.403)--(10.726,4.406)--(10.727,4.410)%
--(10.729,4.414)--(10.730,4.418)--(10.731,4.422)--(10.732,4.426)--(10.732,4.430)--(10.732,4.434)--cycle;
%
\gpfill{rgb color={0.000,0.000,0.000},opacity=0.15} (10.663,4.454)--(10.662,4.458)--(10.662,4.463)--(10.661,4.467)%
--(10.661,4.472)--(10.660,4.476)--(10.658,4.480)--(10.657,4.485)--(10.655,4.489)%
--(10.653,4.493)--(10.651,4.497)--(10.648,4.501)--(10.646,4.505)--(10.643,4.508)%
--(10.640,4.512)--(10.637,4.515)--(10.634,4.518)--(10.630,4.521)--(10.627,4.524)%
--(10.623,4.526)--(10.619,4.529)--(10.615,4.531)--(10.611,4.533)--(10.607,4.535)%
--(10.602,4.536)--(10.598,4.538)--(10.594,4.539)--(10.589,4.539)--(10.585,4.540)%
--(10.580,4.540)--(10.576,4.541)--(10.571,4.540)--(10.566,4.540)--(10.562,4.539)%
--(10.557,4.539)--(10.553,4.538)--(10.549,4.536)--(10.544,4.535)--(10.540,4.533)%
--(10.536,4.531)--(10.532,4.529)--(10.528,4.526)--(10.524,4.524)--(10.521,4.521)%
--(10.517,4.518)--(10.514,4.515)--(10.511,4.512)--(10.508,4.508)--(10.505,4.505)%
--(10.503,4.501)--(10.500,4.497)--(10.498,4.493)--(10.496,4.489)--(10.494,4.485)%
--(10.493,4.480)--(10.491,4.476)--(10.490,4.472)--(10.490,4.467)--(10.489,4.463)%
--(10.489,4.458)--(10.489,4.454)--(10.489,4.449)--(10.489,4.444)--(10.490,4.440)%
--(10.490,4.435)--(10.491,4.431)--(10.493,4.427)--(10.494,4.422)--(10.496,4.418)%
--(10.498,4.414)--(10.500,4.410)--(10.503,4.406)--(10.505,4.402)--(10.508,4.399)%
--(10.511,4.395)--(10.514,4.392)--(10.517,4.389)--(10.521,4.386)--(10.524,4.383)%
--(10.528,4.381)--(10.532,4.378)--(10.536,4.376)--(10.540,4.374)--(10.544,4.372)%
--(10.549,4.371)--(10.553,4.369)--(10.557,4.368)--(10.562,4.368)--(10.566,4.367)%
--(10.571,4.367)--(10.576,4.367)--(10.580,4.367)--(10.585,4.367)--(10.589,4.368)%
--(10.594,4.368)--(10.598,4.369)--(10.602,4.371)--(10.607,4.372)--(10.611,4.374)%
--(10.615,4.376)--(10.619,4.378)--(10.623,4.381)--(10.627,4.383)--(10.630,4.386)%
--(10.634,4.389)--(10.637,4.392)--(10.640,4.395)--(10.643,4.399)--(10.646,4.402)%
--(10.648,4.406)--(10.651,4.410)--(10.653,4.414)--(10.655,4.418)--(10.657,4.422)%
--(10.658,4.427)--(10.660,4.431)--(10.661,4.435)--(10.661,4.440)--(10.662,4.444)--(10.662,4.449)--cycle;
%
\gpfill{rgb color={0.000,0.000,0.000},opacity=0.15} (10.586,4.469)--(10.585,4.473)--(10.585,4.478)--(10.584,4.483)%
--(10.583,4.488)--(10.582,4.493)--(10.581,4.498)--(10.579,4.503)--(10.577,4.507)%
--(10.575,4.512)--(10.573,4.516)--(10.570,4.520)--(10.567,4.524)--(10.564,4.528)%
--(10.561,4.532)--(10.558,4.536)--(10.554,4.539)--(10.550,4.542)--(10.546,4.545)%
--(10.542,4.548)--(10.538,4.551)--(10.534,4.553)--(10.529,4.555)--(10.525,4.557)%
--(10.520,4.559)--(10.515,4.560)--(10.510,4.561)--(10.505,4.562)--(10.500,4.563)%
--(10.495,4.563)--(10.491,4.564)--(10.486,4.563)--(10.481,4.563)--(10.476,4.562)%
--(10.471,4.561)--(10.466,4.560)--(10.461,4.559)--(10.456,4.557)--(10.452,4.555)%
--(10.447,4.553)--(10.443,4.551)--(10.439,4.548)--(10.435,4.545)--(10.431,4.542)%
--(10.427,4.539)--(10.423,4.536)--(10.420,4.532)--(10.417,4.528)--(10.414,4.524)%
--(10.411,4.520)--(10.408,4.516)--(10.406,4.512)--(10.404,4.507)--(10.402,4.503)%
--(10.400,4.498)--(10.399,4.493)--(10.398,4.488)--(10.397,4.483)--(10.396,4.478)%
--(10.396,4.473)--(10.396,4.469)--(10.396,4.464)--(10.396,4.459)--(10.397,4.454)%
--(10.398,4.449)--(10.399,4.444)--(10.400,4.439)--(10.402,4.434)--(10.404,4.430)%
--(10.406,4.425)--(10.408,4.421)--(10.411,4.417)--(10.414,4.413)--(10.417,4.409)%
--(10.420,4.405)--(10.423,4.401)--(10.427,4.398)--(10.431,4.395)--(10.435,4.392)%
--(10.439,4.389)--(10.443,4.386)--(10.447,4.384)--(10.452,4.382)--(10.456,4.380)%
--(10.461,4.378)--(10.466,4.377)--(10.471,4.376)--(10.476,4.375)--(10.481,4.374)%
--(10.486,4.374)--(10.491,4.374)--(10.495,4.374)--(10.500,4.374)--(10.505,4.375)%
--(10.510,4.376)--(10.515,4.377)--(10.520,4.378)--(10.525,4.380)--(10.529,4.382)%
--(10.534,4.384)--(10.538,4.386)--(10.542,4.389)--(10.546,4.392)--(10.550,4.395)%
--(10.554,4.398)--(10.558,4.401)--(10.561,4.405)--(10.564,4.409)--(10.567,4.413)%
--(10.570,4.417)--(10.573,4.421)--(10.575,4.425)--(10.577,4.430)--(10.579,4.434)%
--(10.581,4.439)--(10.582,4.444)--(10.583,4.449)--(10.584,4.454)--(10.585,4.459)--(10.585,4.464)--cycle;
%
\gpfill{rgb color={0.000,0.000,0.000},opacity=0.15} (10.500,4.486)--(10.499,4.491)--(10.499,4.496)--(10.498,4.501)%
--(10.497,4.507)--(10.496,4.512)--(10.495,4.517)--(10.493,4.522)--(10.491,4.527)%
--(10.488,4.532)--(10.486,4.537)--(10.483,4.541)--(10.480,4.545)--(10.477,4.550)%
--(10.473,4.554)--(10.470,4.558)--(10.466,4.561)--(10.462,4.565)--(10.457,4.568)%
--(10.453,4.571)--(10.449,4.574)--(10.444,4.576)--(10.439,4.579)--(10.434,4.581)%
--(10.429,4.583)--(10.424,4.584)--(10.419,4.585)--(10.413,4.586)--(10.408,4.587)%
--(10.403,4.587)--(10.398,4.588)--(10.392,4.587)--(10.387,4.587)--(10.382,4.586)%
--(10.376,4.585)--(10.371,4.584)--(10.366,4.583)--(10.361,4.581)--(10.356,4.579)%
--(10.351,4.576)--(10.347,4.574)--(10.342,4.571)--(10.338,4.568)--(10.333,4.565)%
--(10.329,4.561)--(10.325,4.558)--(10.322,4.554)--(10.318,4.550)--(10.315,4.545)%
--(10.312,4.541)--(10.309,4.537)--(10.307,4.532)--(10.304,4.527)--(10.302,4.522)%
--(10.300,4.517)--(10.299,4.512)--(10.298,4.507)--(10.297,4.501)--(10.296,4.496)%
--(10.296,4.491)--(10.296,4.486)--(10.296,4.480)--(10.296,4.475)--(10.297,4.470)%
--(10.298,4.464)--(10.299,4.459)--(10.300,4.454)--(10.302,4.449)--(10.304,4.444)%
--(10.307,4.439)--(10.309,4.435)--(10.312,4.430)--(10.315,4.426)--(10.318,4.421)%
--(10.322,4.417)--(10.325,4.413)--(10.329,4.410)--(10.333,4.406)--(10.338,4.403)%
--(10.342,4.400)--(10.347,4.397)--(10.351,4.395)--(10.356,4.392)--(10.361,4.390)%
--(10.366,4.388)--(10.371,4.387)--(10.376,4.386)--(10.382,4.385)--(10.387,4.384)%
--(10.392,4.384)--(10.398,4.384)--(10.403,4.384)--(10.408,4.384)--(10.413,4.385)%
--(10.419,4.386)--(10.424,4.387)--(10.429,4.388)--(10.434,4.390)--(10.439,4.392)%
--(10.444,4.395)--(10.449,4.397)--(10.453,4.400)--(10.457,4.403)--(10.462,4.406)%
--(10.466,4.410)--(10.470,4.413)--(10.473,4.417)--(10.477,4.421)--(10.480,4.426)%
--(10.483,4.430)--(10.486,4.435)--(10.488,4.439)--(10.491,4.444)--(10.493,4.449)%
--(10.495,4.454)--(10.496,4.459)--(10.497,4.464)--(10.498,4.470)--(10.499,4.475)--(10.499,4.480)--cycle;
%
\gpfill{rgb color={0.000,0.000,0.000},opacity=0.15} (10.412,4.504)--(10.411,4.510)--(10.411,4.516)--(10.410,4.521)%
--(10.409,4.527)--(10.408,4.533)--(10.406,4.539)--(10.404,4.545)--(10.402,4.550)%
--(10.399,4.556)--(10.396,4.561)--(10.393,4.566)--(10.390,4.571)--(10.386,4.576)%
--(10.382,4.580)--(10.378,4.585)--(10.373,4.589)--(10.369,4.593)--(10.364,4.597)%
--(10.359,4.600)--(10.354,4.603)--(10.349,4.606)--(10.343,4.609)--(10.338,4.611)%
--(10.332,4.613)--(10.326,4.615)--(10.320,4.616)--(10.314,4.617)--(10.309,4.618)%
--(10.303,4.618)--(10.297,4.619)--(10.290,4.618)--(10.284,4.618)--(10.279,4.617)%
--(10.273,4.616)--(10.267,4.615)--(10.261,4.613)--(10.255,4.611)--(10.250,4.609)%
--(10.244,4.606)--(10.239,4.603)--(10.234,4.600)--(10.229,4.597)--(10.224,4.593)%
--(10.220,4.589)--(10.215,4.585)--(10.211,4.580)--(10.207,4.576)--(10.203,4.571)%
--(10.200,4.566)--(10.197,4.561)--(10.194,4.556)--(10.191,4.550)--(10.189,4.545)%
--(10.187,4.539)--(10.185,4.533)--(10.184,4.527)--(10.183,4.521)--(10.182,4.516)%
--(10.182,4.510)--(10.182,4.504)--(10.182,4.497)--(10.182,4.491)--(10.183,4.486)%
--(10.184,4.480)--(10.185,4.474)--(10.187,4.468)--(10.189,4.462)--(10.191,4.457)%
--(10.194,4.451)--(10.197,4.446)--(10.200,4.441)--(10.203,4.436)--(10.207,4.431)%
--(10.211,4.427)--(10.215,4.422)--(10.220,4.418)--(10.224,4.414)--(10.229,4.410)%
--(10.234,4.407)--(10.239,4.404)--(10.244,4.401)--(10.250,4.398)--(10.255,4.396)%
--(10.261,4.394)--(10.267,4.392)--(10.273,4.391)--(10.279,4.390)--(10.284,4.389)%
--(10.290,4.389)--(10.297,4.389)--(10.303,4.389)--(10.309,4.389)--(10.314,4.390)%
--(10.320,4.391)--(10.326,4.392)--(10.332,4.394)--(10.338,4.396)--(10.343,4.398)%
--(10.349,4.401)--(10.354,4.404)--(10.359,4.407)--(10.364,4.410)--(10.369,4.414)%
--(10.373,4.418)--(10.378,4.422)--(10.382,4.427)--(10.386,4.431)--(10.390,4.436)%
--(10.393,4.441)--(10.396,4.446)--(10.399,4.451)--(10.402,4.457)--(10.404,4.462)%
--(10.406,4.468)--(10.408,4.474)--(10.409,4.480)--(10.410,4.486)--(10.411,4.491)--(10.411,4.497)--cycle;
%
\gpfill{rgb color={0.000,0.000,0.000},opacity=0.15} (10.322,4.522)--(10.321,4.528)--(10.321,4.535)--(10.320,4.542)%
--(10.319,4.549)--(10.317,4.556)--(10.315,4.562)--(10.313,4.569)--(10.310,4.575)%
--(10.307,4.581)--(10.304,4.588)--(10.300,4.593)--(10.296,4.599)--(10.292,4.605)%
--(10.288,4.610)--(10.283,4.615)--(10.278,4.620)--(10.273,4.624)--(10.267,4.628)%
--(10.261,4.632)--(10.256,4.636)--(10.249,4.639)--(10.243,4.642)--(10.237,4.645)%
--(10.230,4.647)--(10.224,4.649)--(10.217,4.651)--(10.210,4.652)--(10.203,4.653)%
--(10.196,4.653)--(10.190,4.654)--(10.183,4.653)--(10.176,4.653)--(10.169,4.652)%
--(10.162,4.651)--(10.155,4.649)--(10.149,4.647)--(10.142,4.645)--(10.136,4.642)%
--(10.130,4.639)--(10.124,4.636)--(10.118,4.632)--(10.112,4.628)--(10.106,4.624)%
--(10.101,4.620)--(10.096,4.615)--(10.091,4.610)--(10.087,4.605)--(10.083,4.599)%
--(10.079,4.593)--(10.075,4.588)--(10.072,4.581)--(10.069,4.575)--(10.066,4.569)%
--(10.064,4.562)--(10.062,4.556)--(10.060,4.549)--(10.059,4.542)--(10.058,4.535)%
--(10.058,4.528)--(10.058,4.522)--(10.058,4.515)--(10.058,4.508)--(10.059,4.501)%
--(10.060,4.494)--(10.062,4.487)--(10.064,4.481)--(10.066,4.474)--(10.069,4.468)%
--(10.072,4.462)--(10.075,4.456)--(10.079,4.450)--(10.083,4.444)--(10.087,4.438)%
--(10.091,4.433)--(10.096,4.428)--(10.101,4.423)--(10.106,4.419)--(10.112,4.415)%
--(10.118,4.411)--(10.124,4.407)--(10.130,4.404)--(10.136,4.401)--(10.142,4.398)%
--(10.149,4.396)--(10.155,4.394)--(10.162,4.392)--(10.169,4.391)--(10.176,4.390)%
--(10.183,4.390)--(10.190,4.390)--(10.196,4.390)--(10.203,4.390)--(10.210,4.391)%
--(10.217,4.392)--(10.224,4.394)--(10.230,4.396)--(10.237,4.398)--(10.243,4.401)%
--(10.249,4.404)--(10.256,4.407)--(10.261,4.411)--(10.267,4.415)--(10.273,4.419)%
--(10.278,4.423)--(10.283,4.428)--(10.288,4.433)--(10.292,4.438)--(10.296,4.444)%
--(10.300,4.450)--(10.304,4.456)--(10.307,4.462)--(10.310,4.468)--(10.313,4.474)%
--(10.315,4.481)--(10.317,4.487)--(10.319,4.494)--(10.320,4.501)--(10.321,4.508)--(10.321,4.515)--cycle;
%
\gpfill{rgb color={0.000,0.000,0.000},opacity=0.15} (10.225,4.541)--(10.224,4.548)--(10.224,4.556)--(10.223,4.564)%
--(10.221,4.571)--(10.219,4.579)--(10.217,4.587)--(10.215,4.594)--(10.212,4.601)%
--(10.208,4.608)--(10.205,4.615)--(10.200,4.622)--(10.196,4.628)--(10.191,4.634)%
--(10.186,4.640)--(10.181,4.646)--(10.175,4.651)--(10.169,4.656)--(10.163,4.661)%
--(10.157,4.665)--(10.150,4.670)--(10.143,4.673)--(10.136,4.677)--(10.129,4.680)%
--(10.122,4.682)--(10.114,4.684)--(10.106,4.686)--(10.099,4.688)--(10.091,4.689)%
--(10.083,4.689)--(10.076,4.690)--(10.068,4.689)--(10.060,4.689)--(10.052,4.688)%
--(10.045,4.686)--(10.037,4.684)--(10.029,4.682)--(10.022,4.680)--(10.015,4.677)%
--(10.008,4.673)--(10.001,4.670)--(9.994,4.665)--(9.988,4.661)--(9.982,4.656)%
--(9.976,4.651)--(9.970,4.646)--(9.965,4.640)--(9.960,4.634)--(9.955,4.628)%
--(9.951,4.622)--(9.946,4.615)--(9.943,4.608)--(9.939,4.601)--(9.936,4.594)%
--(9.934,4.587)--(9.932,4.579)--(9.930,4.571)--(9.928,4.564)--(9.927,4.556)%
--(9.927,4.548)--(9.927,4.541)--(9.927,4.533)--(9.927,4.525)--(9.928,4.517)%
--(9.930,4.510)--(9.932,4.502)--(9.934,4.494)--(9.936,4.487)--(9.939,4.480)%
--(9.943,4.473)--(9.946,4.466)--(9.951,4.459)--(9.955,4.453)--(9.960,4.447)%
--(9.965,4.441)--(9.970,4.435)--(9.976,4.430)--(9.982,4.425)--(9.988,4.420)%
--(9.994,4.416)--(10.001,4.411)--(10.008,4.408)--(10.015,4.404)--(10.022,4.401)%
--(10.029,4.399)--(10.037,4.397)--(10.045,4.395)--(10.052,4.393)--(10.060,4.392)%
--(10.068,4.392)--(10.076,4.392)--(10.083,4.392)--(10.091,4.392)--(10.099,4.393)%
--(10.106,4.395)--(10.114,4.397)--(10.122,4.399)--(10.129,4.401)--(10.136,4.404)%
--(10.143,4.408)--(10.150,4.411)--(10.157,4.416)--(10.163,4.420)--(10.169,4.425)%
--(10.175,4.430)--(10.181,4.435)--(10.186,4.441)--(10.191,4.447)--(10.196,4.453)%
--(10.200,4.459)--(10.205,4.466)--(10.208,4.473)--(10.212,4.480)--(10.215,4.487)%
--(10.217,4.494)--(10.219,4.502)--(10.221,4.510)--(10.223,4.517)--(10.224,4.525)--(10.224,4.533)--cycle;
%
\gpfill{rgb color={0.000,0.000,0.000},opacity=0.15} (10.120,4.561)--(10.119,4.569)--(10.119,4.578)--(10.117,4.586)%
--(10.116,4.595)--(10.114,4.603)--(10.111,4.612)--(10.108,4.620)--(10.105,4.628)%
--(10.101,4.636)--(10.097,4.644)--(10.093,4.651)--(10.088,4.658)--(10.083,4.665)%
--(10.077,4.672)--(10.071,4.678)--(10.065,4.684)--(10.058,4.690)--(10.051,4.695)%
--(10.044,4.700)--(10.037,4.704)--(10.029,4.708)--(10.021,4.712)--(10.013,4.715)%
--(10.005,4.718)--(9.996,4.721)--(9.988,4.723)--(9.979,4.724)--(9.971,4.726)%
--(9.962,4.726)--(9.954,4.727)--(9.945,4.726)--(9.936,4.726)--(9.928,4.724)%
--(9.919,4.723)--(9.911,4.721)--(9.902,4.718)--(9.894,4.715)--(9.886,4.712)%
--(9.878,4.708)--(9.871,4.704)--(9.863,4.700)--(9.856,4.695)--(9.849,4.690)%
--(9.842,4.684)--(9.836,4.678)--(9.830,4.672)--(9.824,4.665)--(9.819,4.658)%
--(9.814,4.651)--(9.810,4.644)--(9.806,4.636)--(9.802,4.628)--(9.799,4.620)%
--(9.796,4.612)--(9.793,4.603)--(9.791,4.595)--(9.790,4.586)--(9.788,4.578)%
--(9.788,4.569)--(9.788,4.561)--(9.788,4.552)--(9.788,4.543)--(9.790,4.535)%
--(9.791,4.526)--(9.793,4.518)--(9.796,4.509)--(9.799,4.501)--(9.802,4.493)%
--(9.806,4.485)--(9.810,4.478)--(9.814,4.470)--(9.819,4.463)--(9.824,4.456)%
--(9.830,4.449)--(9.836,4.443)--(9.842,4.437)--(9.849,4.431)--(9.856,4.426)%
--(9.863,4.421)--(9.871,4.417)--(9.878,4.413)--(9.886,4.409)--(9.894,4.406)%
--(9.902,4.403)--(9.911,4.400)--(9.919,4.398)--(9.928,4.397)--(9.936,4.395)%
--(9.945,4.395)--(9.954,4.395)--(9.962,4.395)--(9.971,4.395)--(9.979,4.397)%
--(9.988,4.398)--(9.996,4.400)--(10.005,4.403)--(10.013,4.406)--(10.021,4.409)%
--(10.029,4.413)--(10.037,4.417)--(10.044,4.421)--(10.051,4.426)--(10.058,4.431)%
--(10.065,4.437)--(10.071,4.443)--(10.077,4.449)--(10.083,4.456)--(10.088,4.463)%
--(10.093,4.470)--(10.097,4.478)--(10.101,4.485)--(10.105,4.493)--(10.108,4.501)%
--(10.111,4.509)--(10.114,4.518)--(10.116,4.526)--(10.117,4.535)--(10.119,4.543)--(10.119,4.552)--cycle;
%
\gpfill{rgb color={0.000,0.000,0.000},opacity=0.15} (10.011,4.582)--(10.010,4.591)--(10.009,4.601)--(10.008,4.610)%
--(10.006,4.620)--(10.004,4.629)--(10.001,4.638)--(9.998,4.647)--(9.995,4.656)%
--(9.990,4.665)--(9.986,4.674)--(9.981,4.682)--(9.975,4.690)--(9.969,4.697)%
--(9.963,4.705)--(9.957,4.712)--(9.950,4.718)--(9.942,4.724)--(9.935,4.730)%
--(9.927,4.736)--(9.919,4.741)--(9.910,4.745)--(9.901,4.750)--(9.892,4.753)%
--(9.883,4.756)--(9.874,4.759)--(9.865,4.761)--(9.855,4.763)--(9.846,4.764)%
--(9.836,4.765)--(9.827,4.766)--(9.817,4.765)--(9.807,4.764)--(9.798,4.763)%
--(9.788,4.761)--(9.779,4.759)--(9.770,4.756)--(9.761,4.753)--(9.752,4.750)%
--(9.743,4.745)--(9.735,4.741)--(9.726,4.736)--(9.718,4.730)--(9.711,4.724)%
--(9.703,4.718)--(9.696,4.712)--(9.690,4.705)--(9.684,4.697)--(9.678,4.690)%
--(9.672,4.682)--(9.667,4.674)--(9.663,4.665)--(9.658,4.656)--(9.655,4.647)%
--(9.652,4.638)--(9.649,4.629)--(9.647,4.620)--(9.645,4.610)--(9.644,4.601)%
--(9.643,4.591)--(9.643,4.582)--(9.643,4.572)--(9.644,4.562)--(9.645,4.553)%
--(9.647,4.543)--(9.649,4.534)--(9.652,4.525)--(9.655,4.516)--(9.658,4.507)%
--(9.663,4.498)--(9.667,4.490)--(9.672,4.481)--(9.678,4.473)--(9.684,4.466)%
--(9.690,4.458)--(9.696,4.451)--(9.703,4.445)--(9.711,4.439)--(9.718,4.433)%
--(9.726,4.427)--(9.735,4.422)--(9.743,4.418)--(9.752,4.413)--(9.761,4.410)%
--(9.770,4.407)--(9.779,4.404)--(9.788,4.402)--(9.798,4.400)--(9.807,4.399)%
--(9.817,4.398)--(9.827,4.398)--(9.836,4.398)--(9.846,4.399)--(9.855,4.400)%
--(9.865,4.402)--(9.874,4.404)--(9.883,4.407)--(9.892,4.410)--(9.901,4.413)%
--(9.910,4.418)--(9.919,4.422)--(9.927,4.427)--(9.935,4.433)--(9.942,4.439)%
--(9.950,4.445)--(9.957,4.451)--(9.963,4.458)--(9.969,4.466)--(9.975,4.473)%
--(9.981,4.481)--(9.986,4.490)--(9.990,4.498)--(9.995,4.507)--(9.998,4.516)%
--(10.001,4.525)--(10.004,4.534)--(10.006,4.543)--(10.008,4.553)--(10.009,4.562)--(10.010,4.572)--cycle;
%
\gpfill{rgb color={0.000,0.000,0.000},opacity=0.15} (9.894,4.603)--(9.893,4.613)--(9.892,4.624)--(9.891,4.634)%
--(9.889,4.644)--(9.887,4.655)--(9.884,4.665)--(9.880,4.675)--(9.876,4.684)%
--(9.872,4.694)--(9.867,4.703)--(9.861,4.712)--(9.855,4.721)--(9.849,4.729)%
--(9.842,4.737)--(9.835,4.745)--(9.827,4.752)--(9.819,4.759)--(9.811,4.765)%
--(9.802,4.771)--(9.793,4.777)--(9.784,4.782)--(9.774,4.786)--(9.765,4.790)%
--(9.755,4.794)--(9.745,4.797)--(9.734,4.799)--(9.724,4.801)--(9.714,4.802)%
--(9.703,4.803)--(9.693,4.804)--(9.682,4.803)--(9.671,4.802)--(9.661,4.801)%
--(9.651,4.799)--(9.640,4.797)--(9.630,4.794)--(9.620,4.790)--(9.611,4.786)%
--(9.601,4.782)--(9.592,4.777)--(9.583,4.771)--(9.574,4.765)--(9.566,4.759)%
--(9.558,4.752)--(9.550,4.745)--(9.543,4.737)--(9.536,4.729)--(9.530,4.721)%
--(9.524,4.712)--(9.518,4.703)--(9.513,4.694)--(9.509,4.684)--(9.505,4.675)%
--(9.501,4.665)--(9.498,4.655)--(9.496,4.644)--(9.494,4.634)--(9.493,4.624)%
--(9.492,4.613)--(9.492,4.603)--(9.492,4.592)--(9.493,4.581)--(9.494,4.571)%
--(9.496,4.561)--(9.498,4.550)--(9.501,4.540)--(9.505,4.530)--(9.509,4.521)%
--(9.513,4.511)--(9.518,4.502)--(9.524,4.493)--(9.530,4.484)--(9.536,4.476)%
--(9.543,4.468)--(9.550,4.460)--(9.558,4.453)--(9.566,4.446)--(9.574,4.440)%
--(9.583,4.434)--(9.592,4.428)--(9.601,4.423)--(9.611,4.419)--(9.620,4.415)%
--(9.630,4.411)--(9.640,4.408)--(9.651,4.406)--(9.661,4.404)--(9.671,4.403)%
--(9.682,4.402)--(9.693,4.402)--(9.703,4.402)--(9.714,4.403)--(9.724,4.404)%
--(9.734,4.406)--(9.745,4.408)--(9.755,4.411)--(9.765,4.415)--(9.774,4.419)%
--(9.784,4.423)--(9.793,4.428)--(9.802,4.434)--(9.811,4.440)--(9.819,4.446)%
--(9.827,4.453)--(9.835,4.460)--(9.842,4.468)--(9.849,4.476)--(9.855,4.484)%
--(9.861,4.493)--(9.867,4.502)--(9.872,4.511)--(9.876,4.521)--(9.880,4.530)%
--(9.884,4.540)--(9.887,4.550)--(9.889,4.561)--(9.891,4.571)--(9.892,4.581)--(9.893,4.592)--cycle;
%
\gpfill{rgb color={0.000,0.000,0.000},opacity=0.15} (9.770,4.624)--(9.769,4.635)--(9.768,4.646)--(9.767,4.658)%
--(9.765,4.669)--(9.762,4.680)--(9.759,4.691)--(9.755,4.702)--(9.751,4.712)%
--(9.746,4.722)--(9.740,4.733)--(9.734,4.742)--(9.728,4.752)--(9.721,4.761)%
--(9.714,4.769)--(9.706,4.778)--(9.697,4.786)--(9.689,4.793)--(9.680,4.800)%
--(9.670,4.806)--(9.661,4.812)--(9.650,4.818)--(9.640,4.823)--(9.630,4.827)%
--(9.619,4.831)--(9.608,4.834)--(9.597,4.837)--(9.586,4.839)--(9.574,4.840)%
--(9.563,4.841)--(9.552,4.842)--(9.540,4.841)--(9.529,4.840)--(9.517,4.839)%
--(9.506,4.837)--(9.495,4.834)--(9.484,4.831)--(9.473,4.827)--(9.463,4.823)%
--(9.453,4.818)--(9.443,4.812)--(9.433,4.806)--(9.423,4.800)--(9.414,4.793)%
--(9.406,4.786)--(9.397,4.778)--(9.389,4.769)--(9.382,4.761)--(9.375,4.752)%
--(9.369,4.742)--(9.363,4.733)--(9.357,4.722)--(9.352,4.712)--(9.348,4.702)%
--(9.344,4.691)--(9.341,4.680)--(9.338,4.669)--(9.336,4.658)--(9.335,4.646)%
--(9.334,4.635)--(9.334,4.624)--(9.334,4.612)--(9.335,4.601)--(9.336,4.589)%
--(9.338,4.578)--(9.341,4.567)--(9.344,4.556)--(9.348,4.545)--(9.352,4.535)%
--(9.357,4.525)--(9.363,4.515)--(9.369,4.505)--(9.375,4.495)--(9.382,4.486)%
--(9.389,4.478)--(9.397,4.469)--(9.406,4.461)--(9.414,4.454)--(9.423,4.447)%
--(9.433,4.441)--(9.443,4.435)--(9.453,4.429)--(9.463,4.424)--(9.473,4.420)%
--(9.484,4.416)--(9.495,4.413)--(9.506,4.410)--(9.517,4.408)--(9.529,4.407)%
--(9.540,4.406)--(9.552,4.406)--(9.563,4.406)--(9.574,4.407)--(9.586,4.408)%
--(9.597,4.410)--(9.608,4.413)--(9.619,4.416)--(9.630,4.420)--(9.640,4.424)%
--(9.650,4.429)--(9.661,4.435)--(9.670,4.441)--(9.680,4.447)--(9.689,4.454)%
--(9.697,4.461)--(9.706,4.469)--(9.714,4.478)--(9.721,4.486)--(9.728,4.495)%
--(9.734,4.505)--(9.740,4.515)--(9.746,4.525)--(9.751,4.535)--(9.755,4.545)%
--(9.759,4.556)--(9.762,4.567)--(9.765,4.578)--(9.767,4.589)--(9.768,4.601)--(9.769,4.612)--cycle;
%
\gpfill{rgb color={0.000,0.000,0.000},opacity=0.15} (9.640,4.646)--(9.639,4.658)--(9.638,4.670)--(9.637,4.682)%
--(9.634,4.694)--(9.632,4.706)--(9.628,4.718)--(9.624,4.729)--(9.619,4.741)%
--(9.614,4.752)--(9.608,4.763)--(9.602,4.773)--(9.595,4.783)--(9.587,4.793)%
--(9.579,4.802)--(9.571,4.811)--(9.562,4.819)--(9.553,4.827)--(9.543,4.835)%
--(9.533,4.842)--(9.523,4.848)--(9.512,4.854)--(9.501,4.859)--(9.489,4.864)%
--(9.478,4.868)--(9.466,4.872)--(9.454,4.874)--(9.442,4.877)--(9.430,4.878)%
--(9.418,4.879)--(9.406,4.880)--(9.393,4.879)--(9.381,4.878)--(9.369,4.877)%
--(9.357,4.874)--(9.345,4.872)--(9.333,4.868)--(9.322,4.864)--(9.310,4.859)%
--(9.299,4.854)--(9.289,4.848)--(9.278,4.842)--(9.268,4.835)--(9.258,4.827)%
--(9.249,4.819)--(9.240,4.811)--(9.232,4.802)--(9.224,4.793)--(9.216,4.783)%
--(9.209,4.773)--(9.203,4.763)--(9.197,4.752)--(9.192,4.741)--(9.187,4.729)%
--(9.183,4.718)--(9.179,4.706)--(9.177,4.694)--(9.174,4.682)--(9.173,4.670)%
--(9.172,4.658)--(9.172,4.646)--(9.172,4.633)--(9.173,4.621)--(9.174,4.609)%
--(9.177,4.597)--(9.179,4.585)--(9.183,4.573)--(9.187,4.562)--(9.192,4.550)%
--(9.197,4.539)--(9.203,4.529)--(9.209,4.518)--(9.216,4.508)--(9.224,4.498)%
--(9.232,4.489)--(9.240,4.480)--(9.249,4.472)--(9.258,4.464)--(9.268,4.456)%
--(9.278,4.449)--(9.289,4.443)--(9.299,4.437)--(9.310,4.432)--(9.322,4.427)%
--(9.333,4.423)--(9.345,4.419)--(9.357,4.417)--(9.369,4.414)--(9.381,4.413)%
--(9.393,4.412)--(9.406,4.412)--(9.418,4.412)--(9.430,4.413)--(9.442,4.414)%
--(9.454,4.417)--(9.466,4.419)--(9.478,4.423)--(9.489,4.427)--(9.501,4.432)%
--(9.512,4.437)--(9.523,4.443)--(9.533,4.449)--(9.543,4.456)--(9.553,4.464)%
--(9.562,4.472)--(9.571,4.480)--(9.579,4.489)--(9.587,4.498)--(9.595,4.508)%
--(9.602,4.518)--(9.608,4.529)--(9.614,4.539)--(9.619,4.550)--(9.624,4.562)%
--(9.628,4.573)--(9.632,4.585)--(9.634,4.597)--(9.637,4.609)--(9.638,4.621)--(9.639,4.633)--cycle;
%
\gpfill{rgb color={0.000,0.000,0.000},opacity=0.15} (9.503,4.668)--(9.502,4.681)--(9.501,4.694)--(9.499,4.706)%
--(9.497,4.719)--(9.494,4.732)--(9.490,4.744)--(9.486,4.757)--(9.481,4.769)%
--(9.475,4.781)--(9.469,4.792)--(9.462,4.803)--(9.455,4.814)--(9.447,4.824)%
--(9.439,4.834)--(9.430,4.844)--(9.420,4.853)--(9.410,4.861)--(9.400,4.869)%
--(9.389,4.876)--(9.378,4.883)--(9.367,4.889)--(9.355,4.895)--(9.343,4.900)%
--(9.330,4.904)--(9.318,4.908)--(9.305,4.911)--(9.292,4.913)--(9.280,4.915)%
--(9.267,4.916)--(9.254,4.917)--(9.240,4.916)--(9.227,4.915)--(9.215,4.913)%
--(9.202,4.911)--(9.189,4.908)--(9.177,4.904)--(9.164,4.900)--(9.152,4.895)%
--(9.140,4.889)--(9.129,4.883)--(9.118,4.876)--(9.107,4.869)--(9.097,4.861)%
--(9.087,4.853)--(9.077,4.844)--(9.068,4.834)--(9.060,4.824)--(9.052,4.814)%
--(9.045,4.803)--(9.038,4.792)--(9.032,4.781)--(9.026,4.769)--(9.021,4.757)%
--(9.017,4.744)--(9.013,4.732)--(9.010,4.719)--(9.008,4.706)--(9.006,4.694)%
--(9.005,4.681)--(9.005,4.668)--(9.005,4.654)--(9.006,4.641)--(9.008,4.629)%
--(9.010,4.616)--(9.013,4.603)--(9.017,4.591)--(9.021,4.578)--(9.026,4.566)%
--(9.032,4.554)--(9.038,4.543)--(9.045,4.532)--(9.052,4.521)--(9.060,4.511)%
--(9.068,4.501)--(9.077,4.491)--(9.087,4.482)--(9.097,4.474)--(9.107,4.466)%
--(9.118,4.459)--(9.129,4.452)--(9.140,4.446)--(9.152,4.440)--(9.164,4.435)%
--(9.177,4.431)--(9.189,4.427)--(9.202,4.424)--(9.215,4.422)--(9.227,4.420)%
--(9.240,4.419)--(9.254,4.419)--(9.267,4.419)--(9.280,4.420)--(9.292,4.422)%
--(9.305,4.424)--(9.318,4.427)--(9.330,4.431)--(9.343,4.435)--(9.355,4.440)%
--(9.367,4.446)--(9.378,4.452)--(9.389,4.459)--(9.400,4.466)--(9.410,4.474)%
--(9.420,4.482)--(9.430,4.491)--(9.439,4.501)--(9.447,4.511)--(9.455,4.521)%
--(9.462,4.532)--(9.469,4.543)--(9.475,4.554)--(9.481,4.566)--(9.486,4.578)%
--(9.490,4.591)--(9.494,4.603)--(9.497,4.616)--(9.499,4.629)--(9.501,4.641)--(9.502,4.654)--cycle;
%
\gpfill{rgb color={0.000,0.000,0.000},opacity=0.15} (9.359,4.690)--(9.358,4.703)--(9.357,4.717)--(9.355,4.730)%
--(9.353,4.744)--(9.350,4.757)--(9.346,4.770)--(9.341,4.783)--(9.336,4.796)%
--(9.330,4.808)--(9.323,4.821)--(9.316,4.832)--(9.308,4.843)--(9.300,4.854)%
--(9.291,4.865)--(9.282,4.875)--(9.272,4.884)--(9.261,4.893)--(9.250,4.901)%
--(9.239,4.909)--(9.228,4.916)--(9.215,4.923)--(9.203,4.929)--(9.190,4.934)%
--(9.177,4.939)--(9.164,4.943)--(9.151,4.946)--(9.137,4.948)--(9.124,4.950)%
--(9.110,4.951)--(9.097,4.952)--(9.083,4.951)--(9.069,4.950)--(9.056,4.948)%
--(9.042,4.946)--(9.029,4.943)--(9.016,4.939)--(9.003,4.934)--(8.990,4.929)%
--(8.978,4.923)--(8.966,4.916)--(8.954,4.909)--(8.943,4.901)--(8.932,4.893)%
--(8.921,4.884)--(8.911,4.875)--(8.902,4.865)--(8.893,4.854)--(8.885,4.843)%
--(8.877,4.832)--(8.870,4.821)--(8.863,4.808)--(8.857,4.796)--(8.852,4.783)%
--(8.847,4.770)--(8.843,4.757)--(8.840,4.744)--(8.838,4.730)--(8.836,4.717)%
--(8.835,4.703)--(8.835,4.690)--(8.835,4.676)--(8.836,4.662)--(8.838,4.649)%
--(8.840,4.635)--(8.843,4.622)--(8.847,4.609)--(8.852,4.596)--(8.857,4.583)%
--(8.863,4.571)--(8.870,4.559)--(8.877,4.547)--(8.885,4.536)--(8.893,4.525)%
--(8.902,4.514)--(8.911,4.504)--(8.921,4.495)--(8.932,4.486)--(8.943,4.478)%
--(8.954,4.470)--(8.966,4.463)--(8.978,4.456)--(8.990,4.450)--(9.003,4.445)%
--(9.016,4.440)--(9.029,4.436)--(9.042,4.433)--(9.056,4.431)--(9.069,4.429)%
--(9.083,4.428)--(9.097,4.428)--(9.110,4.428)--(9.124,4.429)--(9.137,4.431)%
--(9.151,4.433)--(9.164,4.436)--(9.177,4.440)--(9.190,4.445)--(9.203,4.450)%
--(9.215,4.456)--(9.228,4.463)--(9.239,4.470)--(9.250,4.478)--(9.261,4.486)%
--(9.272,4.495)--(9.282,4.504)--(9.291,4.514)--(9.300,4.525)--(9.308,4.536)%
--(9.316,4.547)--(9.323,4.559)--(9.330,4.571)--(9.336,4.583)--(9.341,4.596)%
--(9.346,4.609)--(9.350,4.622)--(9.353,4.635)--(9.355,4.649)--(9.357,4.662)--(9.358,4.676)--cycle;
%
\gpfill{rgb color={0.000,0.000,0.000},opacity=0.15} (9.208,4.711)--(9.207,4.725)--(9.206,4.739)--(9.204,4.753)%
--(9.202,4.767)--(9.198,4.781)--(9.194,4.795)--(9.189,4.808)--(9.184,4.822)%
--(9.178,4.834)--(9.171,4.847)--(9.163,4.859)--(9.155,4.871)--(9.147,4.882)%
--(9.137,4.893)--(9.128,4.904)--(9.117,4.913)--(9.106,4.923)--(9.095,4.931)%
--(9.083,4.939)--(9.071,4.947)--(9.058,4.954)--(9.046,4.960)--(9.032,4.965)%
--(9.019,4.970)--(9.005,4.974)--(8.991,4.978)--(8.977,4.980)--(8.963,4.982)%
--(8.949,4.983)--(8.935,4.984)--(8.920,4.983)--(8.906,4.982)--(8.892,4.980)%
--(8.878,4.978)--(8.864,4.974)--(8.850,4.970)--(8.837,4.965)--(8.823,4.960)%
--(8.811,4.954)--(8.798,4.947)--(8.786,4.939)--(8.774,4.931)--(8.763,4.923)%
--(8.752,4.913)--(8.741,4.904)--(8.732,4.893)--(8.722,4.882)--(8.714,4.871)%
--(8.706,4.859)--(8.698,4.847)--(8.691,4.834)--(8.685,4.822)--(8.680,4.808)%
--(8.675,4.795)--(8.671,4.781)--(8.667,4.767)--(8.665,4.753)--(8.663,4.739)%
--(8.662,4.725)--(8.662,4.711)--(8.662,4.696)--(8.663,4.682)--(8.665,4.668)%
--(8.667,4.654)--(8.671,4.640)--(8.675,4.626)--(8.680,4.613)--(8.685,4.599)%
--(8.691,4.587)--(8.698,4.574)--(8.706,4.562)--(8.714,4.550)--(8.722,4.539)%
--(8.732,4.528)--(8.741,4.517)--(8.752,4.508)--(8.763,4.498)--(8.774,4.490)%
--(8.786,4.482)--(8.798,4.474)--(8.811,4.467)--(8.823,4.461)--(8.837,4.456)%
--(8.850,4.451)--(8.864,4.447)--(8.878,4.443)--(8.892,4.441)--(8.906,4.439)%
--(8.920,4.438)--(8.935,4.438)--(8.949,4.438)--(8.963,4.439)--(8.977,4.441)%
--(8.991,4.443)--(9.005,4.447)--(9.019,4.451)--(9.032,4.456)--(9.046,4.461)%
--(9.058,4.467)--(9.071,4.474)--(9.083,4.482)--(9.095,4.490)--(9.106,4.498)%
--(9.117,4.508)--(9.128,4.517)--(9.137,4.528)--(9.147,4.539)--(9.155,4.550)%
--(9.163,4.562)--(9.171,4.574)--(9.178,4.587)--(9.184,4.599)--(9.189,4.613)%
--(9.194,4.626)--(9.198,4.640)--(9.202,4.654)--(9.204,4.668)--(9.206,4.682)--(9.207,4.696)--cycle;
%
\gpfill{rgb color={0.000,0.000,0.000},opacity=0.15} (9.050,4.733)--(9.049,4.747)--(9.048,4.762)--(9.046,4.777)%
--(9.043,4.791)--(9.040,4.805)--(9.036,4.820)--(9.031,4.834)--(9.025,4.847)%
--(9.019,4.861)--(9.012,4.874)--(9.004,4.886)--(8.996,4.898)--(8.987,4.910)%
--(8.977,4.921)--(8.967,4.932)--(8.956,4.942)--(8.945,4.952)--(8.933,4.961)%
--(8.921,4.969)--(8.909,4.977)--(8.896,4.984)--(8.882,4.990)--(8.869,4.996)%
--(8.855,5.001)--(8.840,5.005)--(8.826,5.008)--(8.812,5.011)--(8.797,5.013)%
--(8.782,5.014)--(8.768,5.015)--(8.753,5.014)--(8.738,5.013)--(8.723,5.011)%
--(8.709,5.008)--(8.695,5.005)--(8.680,5.001)--(8.666,4.996)--(8.653,4.990)%
--(8.639,4.984)--(8.627,4.977)--(8.614,4.969)--(8.602,4.961)--(8.590,4.952)%
--(8.579,4.942)--(8.568,4.932)--(8.558,4.921)--(8.548,4.910)--(8.539,4.898)%
--(8.531,4.886)--(8.523,4.874)--(8.516,4.861)--(8.510,4.847)--(8.504,4.834)%
--(8.499,4.820)--(8.495,4.805)--(8.492,4.791)--(8.489,4.777)--(8.487,4.762)%
--(8.486,4.747)--(8.486,4.733)--(8.486,4.718)--(8.487,4.703)--(8.489,4.688)%
--(8.492,4.674)--(8.495,4.660)--(8.499,4.645)--(8.504,4.631)--(8.510,4.618)%
--(8.516,4.604)--(8.523,4.592)--(8.531,4.579)--(8.539,4.567)--(8.548,4.555)%
--(8.558,4.544)--(8.568,4.533)--(8.579,4.523)--(8.590,4.513)--(8.602,4.504)%
--(8.614,4.496)--(8.627,4.488)--(8.639,4.481)--(8.653,4.475)--(8.666,4.469)%
--(8.680,4.464)--(8.695,4.460)--(8.709,4.457)--(8.723,4.454)--(8.738,4.452)%
--(8.753,4.451)--(8.768,4.451)--(8.782,4.451)--(8.797,4.452)--(8.812,4.454)%
--(8.826,4.457)--(8.840,4.460)--(8.855,4.464)--(8.869,4.469)--(8.882,4.475)%
--(8.896,4.481)--(8.909,4.488)--(8.921,4.496)--(8.933,4.504)--(8.945,4.513)%
--(8.956,4.523)--(8.967,4.533)--(8.977,4.544)--(8.987,4.555)--(8.996,4.567)%
--(9.004,4.579)--(9.012,4.592)--(9.019,4.604)--(9.025,4.618)--(9.031,4.631)%
--(9.036,4.645)--(9.040,4.660)--(9.043,4.674)--(9.046,4.688)--(9.048,4.703)--(9.049,4.718)--cycle;
%
\gpfill{rgb color={0.000,0.000,0.000},opacity=0.15} (8.884,4.754)--(8.883,4.769)--(8.882,4.783)--(8.880,4.798)%
--(8.877,4.813)--(8.874,4.828)--(8.869,4.842)--(8.864,4.856)--(8.859,4.870)%
--(8.852,4.884)--(8.845,4.897)--(8.837,4.910)--(8.829,4.922)--(8.820,4.934)%
--(8.810,4.946)--(8.799,4.956)--(8.789,4.967)--(8.777,4.977)--(8.765,4.986)%
--(8.753,4.994)--(8.740,5.002)--(8.727,5.009)--(8.713,5.016)--(8.699,5.021)%
--(8.685,5.026)--(8.671,5.031)--(8.656,5.034)--(8.641,5.037)--(8.626,5.039)%
--(8.612,5.040)--(8.597,5.041)--(8.581,5.040)--(8.567,5.039)--(8.552,5.037)%
--(8.537,5.034)--(8.522,5.031)--(8.508,5.026)--(8.494,5.021)--(8.480,5.016)%
--(8.466,5.009)--(8.453,5.002)--(8.440,4.994)--(8.428,4.986)--(8.416,4.977)%
--(8.404,4.967)--(8.394,4.956)--(8.383,4.946)--(8.373,4.934)--(8.364,4.922)%
--(8.356,4.910)--(8.348,4.897)--(8.341,4.884)--(8.334,4.870)--(8.329,4.856)%
--(8.324,4.842)--(8.319,4.828)--(8.316,4.813)--(8.313,4.798)--(8.311,4.783)%
--(8.310,4.769)--(8.310,4.754)--(8.310,4.738)--(8.311,4.724)--(8.313,4.709)%
--(8.316,4.694)--(8.319,4.679)--(8.324,4.665)--(8.329,4.651)--(8.334,4.637)%
--(8.341,4.623)--(8.348,4.610)--(8.356,4.597)--(8.364,4.585)--(8.373,4.573)%
--(8.383,4.561)--(8.394,4.551)--(8.404,4.540)--(8.416,4.530)--(8.428,4.521)%
--(8.440,4.513)--(8.453,4.505)--(8.466,4.498)--(8.480,4.491)--(8.494,4.486)%
--(8.508,4.481)--(8.522,4.476)--(8.537,4.473)--(8.552,4.470)--(8.567,4.468)%
--(8.581,4.467)--(8.597,4.467)--(8.612,4.467)--(8.626,4.468)--(8.641,4.470)%
--(8.656,4.473)--(8.671,4.476)--(8.685,4.481)--(8.699,4.486)--(8.713,4.491)%
--(8.727,4.498)--(8.740,4.505)--(8.753,4.513)--(8.765,4.521)--(8.777,4.530)%
--(8.789,4.540)--(8.799,4.551)--(8.810,4.561)--(8.820,4.573)--(8.829,4.585)%
--(8.837,4.597)--(8.845,4.610)--(8.852,4.623)--(8.859,4.637)--(8.864,4.651)%
--(8.869,4.665)--(8.874,4.679)--(8.877,4.694)--(8.880,4.709)--(8.882,4.724)--(8.883,4.738)--cycle;
%
\gpfill{rgb color={0.000,0.000,0.000},opacity=0.15} (8.708,4.774)--(8.707,4.789)--(8.706,4.803)--(8.704,4.818)%
--(8.701,4.833)--(8.698,4.848)--(8.693,4.862)--(8.688,4.876)--(8.683,4.890)%
--(8.676,4.904)--(8.669,4.917)--(8.661,4.930)--(8.653,4.942)--(8.644,4.954)%
--(8.634,4.966)--(8.623,4.976)--(8.613,4.987)--(8.601,4.997)--(8.589,5.006)%
--(8.577,5.014)--(8.564,5.022)--(8.551,5.029)--(8.537,5.036)--(8.523,5.041)%
--(8.509,5.046)--(8.495,5.051)--(8.480,5.054)--(8.465,5.057)--(8.450,5.059)%
--(8.436,5.060)--(8.421,5.061)--(8.405,5.060)--(8.391,5.059)--(8.376,5.057)%
--(8.361,5.054)--(8.346,5.051)--(8.332,5.046)--(8.318,5.041)--(8.304,5.036)%
--(8.290,5.029)--(8.277,5.022)--(8.264,5.014)--(8.252,5.006)--(8.240,4.997)%
--(8.228,4.987)--(8.218,4.976)--(8.207,4.966)--(8.197,4.954)--(8.188,4.942)%
--(8.180,4.930)--(8.172,4.917)--(8.165,4.904)--(8.158,4.890)--(8.153,4.876)%
--(8.148,4.862)--(8.143,4.848)--(8.140,4.833)--(8.137,4.818)--(8.135,4.803)%
--(8.134,4.789)--(8.134,4.774)--(8.134,4.758)--(8.135,4.744)--(8.137,4.729)%
--(8.140,4.714)--(8.143,4.699)--(8.148,4.685)--(8.153,4.671)--(8.158,4.657)%
--(8.165,4.643)--(8.172,4.630)--(8.180,4.617)--(8.188,4.605)--(8.197,4.593)%
--(8.207,4.581)--(8.218,4.571)--(8.228,4.560)--(8.240,4.550)--(8.252,4.541)%
--(8.264,4.533)--(8.277,4.525)--(8.290,4.518)--(8.304,4.511)--(8.318,4.506)%
--(8.332,4.501)--(8.346,4.496)--(8.361,4.493)--(8.376,4.490)--(8.391,4.488)%
--(8.405,4.487)--(8.421,4.487)--(8.436,4.487)--(8.450,4.488)--(8.465,4.490)%
--(8.480,4.493)--(8.495,4.496)--(8.509,4.501)--(8.523,4.506)--(8.537,4.511)%
--(8.551,4.518)--(8.564,4.525)--(8.577,4.533)--(8.589,4.541)--(8.601,4.550)%
--(8.613,4.560)--(8.623,4.571)--(8.634,4.581)--(8.644,4.593)--(8.653,4.605)%
--(8.661,4.617)--(8.669,4.630)--(8.676,4.643)--(8.683,4.657)--(8.688,4.671)%
--(8.693,4.685)--(8.698,4.699)--(8.701,4.714)--(8.704,4.729)--(8.706,4.744)--(8.707,4.758)--cycle;
%
\gpfill{rgb color={0.000,0.000,0.000},opacity=0.15} (8.523,4.794)--(8.522,4.808)--(8.521,4.823)--(8.519,4.838)%
--(8.516,4.852)--(8.513,4.866)--(8.509,4.881)--(8.504,4.895)--(8.498,4.908)%
--(8.492,4.922)--(8.485,4.935)--(8.477,4.947)--(8.469,4.959)--(8.460,4.971)%
--(8.450,4.982)--(8.440,4.993)--(8.429,5.003)--(8.418,5.013)--(8.406,5.022)%
--(8.394,5.030)--(8.382,5.038)--(8.369,5.045)--(8.355,5.051)--(8.342,5.057)%
--(8.328,5.062)--(8.313,5.066)--(8.299,5.069)--(8.285,5.072)--(8.270,5.074)%
--(8.255,5.075)--(8.241,5.076)--(8.226,5.075)--(8.211,5.074)--(8.196,5.072)%
--(8.182,5.069)--(8.168,5.066)--(8.153,5.062)--(8.139,5.057)--(8.126,5.051)%
--(8.112,5.045)--(8.100,5.038)--(8.087,5.030)--(8.075,5.022)--(8.063,5.013)%
--(8.052,5.003)--(8.041,4.993)--(8.031,4.982)--(8.021,4.971)--(8.012,4.959)%
--(8.004,4.947)--(7.996,4.935)--(7.989,4.922)--(7.983,4.908)--(7.977,4.895)%
--(7.972,4.881)--(7.968,4.866)--(7.965,4.852)--(7.962,4.838)--(7.960,4.823)%
--(7.959,4.808)--(7.959,4.794)--(7.959,4.779)--(7.960,4.764)--(7.962,4.749)%
--(7.965,4.735)--(7.968,4.721)--(7.972,4.706)--(7.977,4.692)--(7.983,4.679)%
--(7.989,4.665)--(7.996,4.653)--(8.004,4.640)--(8.012,4.628)--(8.021,4.616)%
--(8.031,4.605)--(8.041,4.594)--(8.052,4.584)--(8.063,4.574)--(8.075,4.565)%
--(8.087,4.557)--(8.100,4.549)--(8.112,4.542)--(8.126,4.536)--(8.139,4.530)%
--(8.153,4.525)--(8.168,4.521)--(8.182,4.518)--(8.196,4.515)--(8.211,4.513)%
--(8.226,4.512)--(8.241,4.512)--(8.255,4.512)--(8.270,4.513)--(8.285,4.515)%
--(8.299,4.518)--(8.313,4.521)--(8.328,4.525)--(8.342,4.530)--(8.355,4.536)%
--(8.369,4.542)--(8.382,4.549)--(8.394,4.557)--(8.406,4.565)--(8.418,4.574)%
--(8.429,4.584)--(8.440,4.594)--(8.450,4.605)--(8.460,4.616)--(8.469,4.628)%
--(8.477,4.640)--(8.485,4.653)--(8.492,4.665)--(8.498,4.679)--(8.504,4.692)%
--(8.509,4.706)--(8.513,4.721)--(8.516,4.735)--(8.519,4.749)--(8.521,4.764)--(8.522,4.779)--cycle;
%
\gpfill{rgb color={0.000,0.000,0.000},opacity=0.15} (8.331,4.813)--(8.330,4.827)--(8.329,4.841)--(8.327,4.855)%
--(8.325,4.869)--(8.321,4.883)--(8.317,4.897)--(8.312,4.910)--(8.307,4.924)%
--(8.301,4.936)--(8.294,4.949)--(8.286,4.961)--(8.278,4.973)--(8.270,4.984)%
--(8.260,4.995)--(8.251,5.006)--(8.240,5.015)--(8.229,5.025)--(8.218,5.033)%
--(8.206,5.041)--(8.194,5.049)--(8.181,5.056)--(8.169,5.062)--(8.155,5.067)%
--(8.142,5.072)--(8.128,5.076)--(8.114,5.080)--(8.100,5.082)--(8.086,5.084)%
--(8.072,5.085)--(8.058,5.086)--(8.043,5.085)--(8.029,5.084)--(8.015,5.082)%
--(8.001,5.080)--(7.987,5.076)--(7.973,5.072)--(7.960,5.067)--(7.946,5.062)%
--(7.934,5.056)--(7.921,5.049)--(7.909,5.041)--(7.897,5.033)--(7.886,5.025)%
--(7.875,5.015)--(7.864,5.006)--(7.855,4.995)--(7.845,4.984)--(7.837,4.973)%
--(7.829,4.961)--(7.821,4.949)--(7.814,4.936)--(7.808,4.924)--(7.803,4.910)%
--(7.798,4.897)--(7.794,4.883)--(7.790,4.869)--(7.788,4.855)--(7.786,4.841)%
--(7.785,4.827)--(7.785,4.813)--(7.785,4.798)--(7.786,4.784)--(7.788,4.770)%
--(7.790,4.756)--(7.794,4.742)--(7.798,4.728)--(7.803,4.715)--(7.808,4.701)%
--(7.814,4.689)--(7.821,4.676)--(7.829,4.664)--(7.837,4.652)--(7.845,4.641)%
--(7.855,4.630)--(7.864,4.619)--(7.875,4.610)--(7.886,4.600)--(7.897,4.592)%
--(7.909,4.584)--(7.921,4.576)--(7.934,4.569)--(7.946,4.563)--(7.960,4.558)%
--(7.973,4.553)--(7.987,4.549)--(8.001,4.545)--(8.015,4.543)--(8.029,4.541)%
--(8.043,4.540)--(8.058,4.540)--(8.072,4.540)--(8.086,4.541)--(8.100,4.543)%
--(8.114,4.545)--(8.128,4.549)--(8.142,4.553)--(8.155,4.558)--(8.169,4.563)%
--(8.181,4.569)--(8.194,4.576)--(8.206,4.584)--(8.218,4.592)--(8.229,4.600)%
--(8.240,4.610)--(8.251,4.619)--(8.260,4.630)--(8.270,4.641)--(8.278,4.652)%
--(8.286,4.664)--(8.294,4.676)--(8.301,4.689)--(8.307,4.701)--(8.312,4.715)%
--(8.317,4.728)--(8.321,4.742)--(8.325,4.756)--(8.327,4.770)--(8.329,4.784)--(8.330,4.798)--cycle;
%
\gpfill{rgb color={0.000,0.000,0.000},opacity=0.15} (8.130,4.832)--(8.129,4.845)--(8.128,4.858)--(8.126,4.872)%
--(8.124,4.885)--(8.121,4.898)--(8.117,4.911)--(8.112,4.924)--(8.107,4.936)%
--(8.101,4.949)--(8.095,4.961)--(8.088,4.972)--(8.080,4.983)--(8.072,4.994)%
--(8.063,5.004)--(8.054,5.014)--(8.044,5.023)--(8.034,5.032)--(8.023,5.040)%
--(8.012,5.048)--(8.001,5.055)--(7.989,5.061)--(7.976,5.067)--(7.964,5.072)%
--(7.951,5.077)--(7.938,5.081)--(7.925,5.084)--(7.912,5.086)--(7.898,5.088)%
--(7.885,5.089)--(7.872,5.090)--(7.858,5.089)--(7.845,5.088)--(7.831,5.086)%
--(7.818,5.084)--(7.805,5.081)--(7.792,5.077)--(7.779,5.072)--(7.767,5.067)%
--(7.754,5.061)--(7.743,5.055)--(7.731,5.048)--(7.720,5.040)--(7.709,5.032)%
--(7.699,5.023)--(7.689,5.014)--(7.680,5.004)--(7.671,4.994)--(7.663,4.983)%
--(7.655,4.972)--(7.648,4.961)--(7.642,4.949)--(7.636,4.936)--(7.631,4.924)%
--(7.626,4.911)--(7.622,4.898)--(7.619,4.885)--(7.617,4.872)--(7.615,4.858)%
--(7.614,4.845)--(7.614,4.832)--(7.614,4.818)--(7.615,4.805)--(7.617,4.791)%
--(7.619,4.778)--(7.622,4.765)--(7.626,4.752)--(7.631,4.739)--(7.636,4.727)%
--(7.642,4.714)--(7.648,4.703)--(7.655,4.691)--(7.663,4.680)--(7.671,4.669)%
--(7.680,4.659)--(7.689,4.649)--(7.699,4.640)--(7.709,4.631)--(7.720,4.623)%
--(7.731,4.615)--(7.743,4.608)--(7.754,4.602)--(7.767,4.596)--(7.779,4.591)%
--(7.792,4.586)--(7.805,4.582)--(7.818,4.579)--(7.831,4.577)--(7.845,4.575)%
--(7.858,4.574)--(7.872,4.574)--(7.885,4.574)--(7.898,4.575)--(7.912,4.577)%
--(7.925,4.579)--(7.938,4.582)--(7.951,4.586)--(7.964,4.591)--(7.976,4.596)%
--(7.989,4.602)--(8.001,4.608)--(8.012,4.615)--(8.023,4.623)--(8.034,4.631)%
--(8.044,4.640)--(8.054,4.649)--(8.063,4.659)--(8.072,4.669)--(8.080,4.680)%
--(8.088,4.691)--(8.095,4.703)--(8.101,4.714)--(8.107,4.727)--(8.112,4.739)%
--(8.117,4.752)--(8.121,4.765)--(8.124,4.778)--(8.126,4.791)--(8.128,4.805)--(8.129,4.818)--cycle;
%
\gpfill{rgb color={0.000,0.000,0.000},opacity=0.15} (7.919,4.850)--(7.918,4.862)--(7.917,4.874)--(7.916,4.886)%
--(7.913,4.899)--(7.910,4.911)--(7.907,4.922)--(7.903,4.934)--(7.898,4.945)%
--(7.893,4.957)--(7.887,4.968)--(7.880,4.978)--(7.873,4.988)--(7.866,4.998)%
--(7.858,5.007)--(7.849,5.016)--(7.840,5.025)--(7.831,5.033)--(7.821,5.040)%
--(7.811,5.047)--(7.801,5.054)--(7.790,5.060)--(7.778,5.065)--(7.767,5.070)%
--(7.755,5.074)--(7.744,5.077)--(7.732,5.080)--(7.719,5.083)--(7.707,5.084)%
--(7.695,5.085)--(7.683,5.086)--(7.670,5.085)--(7.658,5.084)--(7.646,5.083)%
--(7.633,5.080)--(7.621,5.077)--(7.610,5.074)--(7.598,5.070)--(7.587,5.065)%
--(7.575,5.060)--(7.565,5.054)--(7.554,5.047)--(7.544,5.040)--(7.534,5.033)%
--(7.525,5.025)--(7.516,5.016)--(7.507,5.007)--(7.499,4.998)--(7.492,4.988)%
--(7.485,4.978)--(7.478,4.968)--(7.472,4.957)--(7.467,4.945)--(7.462,4.934)%
--(7.458,4.922)--(7.455,4.911)--(7.452,4.899)--(7.449,4.886)--(7.448,4.874)%
--(7.447,4.862)--(7.447,4.850)--(7.447,4.837)--(7.448,4.825)--(7.449,4.813)%
--(7.452,4.800)--(7.455,4.788)--(7.458,4.777)--(7.462,4.765)--(7.467,4.754)%
--(7.472,4.742)--(7.478,4.732)--(7.485,4.721)--(7.492,4.711)--(7.499,4.701)%
--(7.507,4.692)--(7.516,4.683)--(7.525,4.674)--(7.534,4.666)--(7.544,4.659)%
--(7.554,4.652)--(7.565,4.645)--(7.575,4.639)--(7.587,4.634)--(7.598,4.629)%
--(7.610,4.625)--(7.621,4.622)--(7.633,4.619)--(7.646,4.616)--(7.658,4.615)%
--(7.670,4.614)--(7.683,4.614)--(7.695,4.614)--(7.707,4.615)--(7.719,4.616)%
--(7.732,4.619)--(7.744,4.622)--(7.755,4.625)--(7.767,4.629)--(7.778,4.634)%
--(7.790,4.639)--(7.801,4.645)--(7.811,4.652)--(7.821,4.659)--(7.831,4.666)%
--(7.840,4.674)--(7.849,4.683)--(7.858,4.692)--(7.866,4.701)--(7.873,4.711)%
--(7.880,4.721)--(7.887,4.732)--(7.893,4.742)--(7.898,4.754)--(7.903,4.765)%
--(7.907,4.777)--(7.910,4.788)--(7.913,4.800)--(7.916,4.813)--(7.917,4.825)--(7.918,4.837)--cycle;
%
\gpfill{rgb color={0.000,0.000,0.000},opacity=0.15} (7.698,4.866)--(7.697,4.876)--(7.696,4.887)--(7.695,4.898)%
--(7.693,4.909)--(7.690,4.919)--(7.687,4.929)--(7.684,4.940)--(7.680,4.950)%
--(7.675,4.959)--(7.670,4.969)--(7.664,4.978)--(7.658,4.987)--(7.651,4.996)%
--(7.644,5.004)--(7.637,5.012)--(7.629,5.019)--(7.621,5.026)--(7.612,5.033)%
--(7.603,5.039)--(7.594,5.045)--(7.584,5.050)--(7.575,5.055)--(7.565,5.059)%
--(7.554,5.062)--(7.544,5.065)--(7.534,5.068)--(7.523,5.070)--(7.512,5.071)%
--(7.501,5.072)--(7.491,5.073)--(7.480,5.072)--(7.469,5.071)--(7.458,5.070)%
--(7.447,5.068)--(7.437,5.065)--(7.427,5.062)--(7.416,5.059)--(7.406,5.055)%
--(7.397,5.050)--(7.387,5.045)--(7.378,5.039)--(7.369,5.033)--(7.360,5.026)%
--(7.352,5.019)--(7.344,5.012)--(7.337,5.004)--(7.330,4.996)--(7.323,4.987)%
--(7.317,4.978)--(7.311,4.969)--(7.306,4.959)--(7.301,4.950)--(7.297,4.940)%
--(7.294,4.929)--(7.291,4.919)--(7.288,4.909)--(7.286,4.898)--(7.285,4.887)%
--(7.284,4.876)--(7.284,4.866)--(7.284,4.855)--(7.285,4.844)--(7.286,4.833)%
--(7.288,4.822)--(7.291,4.812)--(7.294,4.802)--(7.297,4.791)--(7.301,4.781)%
--(7.306,4.772)--(7.311,4.762)--(7.317,4.753)--(7.323,4.744)--(7.330,4.735)%
--(7.337,4.727)--(7.344,4.719)--(7.352,4.712)--(7.360,4.705)--(7.369,4.698)%
--(7.378,4.692)--(7.387,4.686)--(7.397,4.681)--(7.406,4.676)--(7.416,4.672)%
--(7.427,4.669)--(7.437,4.666)--(7.447,4.663)--(7.458,4.661)--(7.469,4.660)%
--(7.480,4.659)--(7.491,4.659)--(7.501,4.659)--(7.512,4.660)--(7.523,4.661)%
--(7.534,4.663)--(7.544,4.666)--(7.554,4.669)--(7.565,4.672)--(7.575,4.676)%
--(7.584,4.681)--(7.594,4.686)--(7.603,4.692)--(7.612,4.698)--(7.621,4.705)%
--(7.629,4.712)--(7.637,4.719)--(7.644,4.727)--(7.651,4.735)--(7.658,4.744)%
--(7.664,4.753)--(7.670,4.762)--(7.675,4.772)--(7.680,4.781)--(7.684,4.791)%
--(7.687,4.802)--(7.690,4.812)--(7.693,4.822)--(7.695,4.833)--(7.696,4.844)--(7.697,4.855)--cycle;
%
\gpfill{rgb color={0.000,0.000,0.000},opacity=0.15} (7.492,4.882)--(7.491,4.892)--(7.490,4.902)--(7.489,4.912)%
--(7.487,4.922)--(7.485,4.932)--(7.482,4.942)--(7.478,4.952)--(7.475,4.961)%
--(7.470,4.970)--(7.465,4.980)--(7.460,4.988)--(7.454,4.997)--(7.448,5.005)%
--(7.441,5.013)--(7.434,5.020)--(7.427,5.027)--(7.419,5.034)--(7.411,5.040)%
--(7.402,5.046)--(7.394,5.051)--(7.384,5.056)--(7.375,5.061)--(7.366,5.064)%
--(7.356,5.068)--(7.346,5.071)--(7.336,5.073)--(7.326,5.075)--(7.316,5.076)%
--(7.306,5.077)--(7.296,5.078)--(7.285,5.077)--(7.275,5.076)--(7.265,5.075)%
--(7.255,5.073)--(7.245,5.071)--(7.235,5.068)--(7.225,5.064)--(7.216,5.061)%
--(7.207,5.056)--(7.198,5.051)--(7.189,5.046)--(7.180,5.040)--(7.172,5.034)%
--(7.164,5.027)--(7.157,5.020)--(7.150,5.013)--(7.143,5.005)--(7.137,4.997)%
--(7.131,4.988)--(7.126,4.980)--(7.121,4.970)--(7.116,4.961)--(7.113,4.952)%
--(7.109,4.942)--(7.106,4.932)--(7.104,4.922)--(7.102,4.912)--(7.101,4.902)%
--(7.100,4.892)--(7.100,4.882)--(7.100,4.871)--(7.101,4.861)--(7.102,4.851)%
--(7.104,4.841)--(7.106,4.831)--(7.109,4.821)--(7.113,4.811)--(7.116,4.802)%
--(7.121,4.793)--(7.126,4.784)--(7.131,4.775)--(7.137,4.766)--(7.143,4.758)%
--(7.150,4.750)--(7.157,4.743)--(7.164,4.736)--(7.172,4.729)--(7.180,4.723)%
--(7.189,4.717)--(7.198,4.712)--(7.207,4.707)--(7.216,4.702)--(7.225,4.699)%
--(7.235,4.695)--(7.245,4.692)--(7.255,4.690)--(7.265,4.688)--(7.275,4.687)%
--(7.285,4.686)--(7.296,4.686)--(7.306,4.686)--(7.316,4.687)--(7.326,4.688)%
--(7.336,4.690)--(7.346,4.692)--(7.356,4.695)--(7.366,4.699)--(7.375,4.702)%
--(7.384,4.707)--(7.394,4.712)--(7.402,4.717)--(7.411,4.723)--(7.419,4.729)%
--(7.427,4.736)--(7.434,4.743)--(7.441,4.750)--(7.448,4.758)--(7.454,4.766)%
--(7.460,4.775)--(7.465,4.784)--(7.470,4.793)--(7.475,4.802)--(7.478,4.811)%
--(7.482,4.821)--(7.485,4.831)--(7.487,4.841)--(7.489,4.851)--(7.490,4.861)--(7.491,4.871)--cycle;
%
\gpfill{rgb color={0.000,0.000,0.000},opacity=0.15} (7.298,4.896)--(7.297,4.906)--(7.296,4.916)--(7.295,4.926)%
--(7.293,4.937)--(7.291,4.947)--(7.288,4.957)--(7.284,4.966)--(7.280,4.976)%
--(7.276,4.985)--(7.271,4.995)--(7.266,5.003)--(7.260,5.012)--(7.253,5.020)%
--(7.247,5.028)--(7.240,5.036)--(7.232,5.043)--(7.224,5.049)--(7.216,5.056)%
--(7.207,5.062)--(7.199,5.067)--(7.189,5.072)--(7.180,5.076)--(7.170,5.080)%
--(7.161,5.084)--(7.151,5.087)--(7.141,5.089)--(7.130,5.091)--(7.120,5.092)%
--(7.110,5.093)--(7.100,5.094)--(7.089,5.093)--(7.079,5.092)--(7.069,5.091)%
--(7.058,5.089)--(7.048,5.087)--(7.038,5.084)--(7.029,5.080)--(7.019,5.076)%
--(7.010,5.072)--(7.001,5.067)--(6.992,5.062)--(6.983,5.056)--(6.975,5.049)%
--(6.967,5.043)--(6.959,5.036)--(6.952,5.028)--(6.946,5.020)--(6.939,5.012)%
--(6.933,5.003)--(6.928,4.995)--(6.923,4.985)--(6.919,4.976)--(6.915,4.966)%
--(6.911,4.957)--(6.908,4.947)--(6.906,4.937)--(6.904,4.926)--(6.903,4.916)%
--(6.902,4.906)--(6.902,4.896)--(6.902,4.885)--(6.903,4.875)--(6.904,4.865)%
--(6.906,4.854)--(6.908,4.844)--(6.911,4.834)--(6.915,4.825)--(6.919,4.815)%
--(6.923,4.806)--(6.928,4.797)--(6.933,4.788)--(6.939,4.779)--(6.946,4.771)%
--(6.952,4.763)--(6.959,4.755)--(6.967,4.748)--(6.975,4.742)--(6.983,4.735)%
--(6.992,4.729)--(7.001,4.724)--(7.010,4.719)--(7.019,4.715)--(7.029,4.711)%
--(7.038,4.707)--(7.048,4.704)--(7.058,4.702)--(7.069,4.700)--(7.079,4.699)%
--(7.089,4.698)--(7.100,4.698)--(7.110,4.698)--(7.120,4.699)--(7.130,4.700)%
--(7.141,4.702)--(7.151,4.704)--(7.161,4.707)--(7.170,4.711)--(7.180,4.715)%
--(7.189,4.719)--(7.199,4.724)--(7.207,4.729)--(7.216,4.735)--(7.224,4.742)%
--(7.232,4.748)--(7.240,4.755)--(7.247,4.763)--(7.253,4.771)--(7.260,4.779)%
--(7.266,4.788)--(7.271,4.797)--(7.276,4.806)--(7.280,4.815)--(7.284,4.825)%
--(7.288,4.834)--(7.291,4.844)--(7.293,4.854)--(7.295,4.865)--(7.296,4.875)--(7.297,4.885)--cycle;
%
\gpfill{rgb color={0.000,0.000,0.000},opacity=0.15} (7.103,4.910)--(7.102,4.920)--(7.101,4.930)--(7.100,4.941)%
--(7.098,4.951)--(7.096,4.961)--(7.093,4.971)--(7.089,4.981)--(7.085,4.991)%
--(7.081,5.000)--(7.076,5.010)--(7.070,5.018)--(7.064,5.027)--(7.058,5.035)%
--(7.051,5.043)--(7.044,5.051)--(7.036,5.058)--(7.028,5.065)--(7.020,5.071)%
--(7.011,5.077)--(7.003,5.083)--(6.993,5.088)--(6.984,5.092)--(6.974,5.096)%
--(6.964,5.100)--(6.954,5.103)--(6.944,5.105)--(6.934,5.107)--(6.923,5.108)%
--(6.913,5.109)--(6.903,5.110)--(6.892,5.109)--(6.882,5.108)--(6.871,5.107)%
--(6.861,5.105)--(6.851,5.103)--(6.841,5.100)--(6.831,5.096)--(6.821,5.092)%
--(6.812,5.088)--(6.803,5.083)--(6.794,5.077)--(6.785,5.071)--(6.777,5.065)%
--(6.769,5.058)--(6.761,5.051)--(6.754,5.043)--(6.747,5.035)--(6.741,5.027)%
--(6.735,5.018)--(6.729,5.010)--(6.724,5.000)--(6.720,4.991)--(6.716,4.981)%
--(6.712,4.971)--(6.709,4.961)--(6.707,4.951)--(6.705,4.941)--(6.704,4.930)%
--(6.703,4.920)--(6.703,4.910)--(6.703,4.899)--(6.704,4.889)--(6.705,4.878)%
--(6.707,4.868)--(6.709,4.858)--(6.712,4.848)--(6.716,4.838)--(6.720,4.828)%
--(6.724,4.819)--(6.729,4.810)--(6.735,4.801)--(6.741,4.792)--(6.747,4.784)%
--(6.754,4.776)--(6.761,4.768)--(6.769,4.761)--(6.777,4.754)--(6.785,4.748)%
--(6.794,4.742)--(6.803,4.736)--(6.812,4.731)--(6.821,4.727)--(6.831,4.723)%
--(6.841,4.719)--(6.851,4.716)--(6.861,4.714)--(6.871,4.712)--(6.882,4.711)%
--(6.892,4.710)--(6.903,4.710)--(6.913,4.710)--(6.923,4.711)--(6.934,4.712)%
--(6.944,4.714)--(6.954,4.716)--(6.964,4.719)--(6.974,4.723)--(6.984,4.727)%
--(6.993,4.731)--(7.003,4.736)--(7.011,4.742)--(7.020,4.748)--(7.028,4.754)%
--(7.036,4.761)--(7.044,4.768)--(7.051,4.776)--(7.058,4.784)--(7.064,4.792)%
--(7.070,4.801)--(7.076,4.810)--(7.081,4.819)--(7.085,4.828)--(7.089,4.838)%
--(7.093,4.848)--(7.096,4.858)--(7.098,4.868)--(7.100,4.878)--(7.101,4.889)--(7.102,4.899)--cycle;
%
\gpfill{rgb color={0.000,0.000,0.000},opacity=0.15} (6.904,4.922)--(6.903,4.932)--(6.902,4.942)--(6.901,4.953)%
--(6.899,4.963)--(6.897,4.973)--(6.894,4.983)--(6.890,4.993)--(6.886,5.003)%
--(6.882,5.012)--(6.877,5.022)--(6.871,5.030)--(6.865,5.039)--(6.859,5.047)%
--(6.852,5.055)--(6.845,5.063)--(6.837,5.070)--(6.829,5.077)--(6.821,5.083)%
--(6.812,5.089)--(6.804,5.095)--(6.794,5.100)--(6.785,5.104)--(6.775,5.108)%
--(6.765,5.112)--(6.755,5.115)--(6.745,5.117)--(6.735,5.119)--(6.724,5.120)%
--(6.714,5.121)--(6.704,5.122)--(6.693,5.121)--(6.683,5.120)--(6.672,5.119)%
--(6.662,5.117)--(6.652,5.115)--(6.642,5.112)--(6.632,5.108)--(6.622,5.104)%
--(6.613,5.100)--(6.604,5.095)--(6.595,5.089)--(6.586,5.083)--(6.578,5.077)%
--(6.570,5.070)--(6.562,5.063)--(6.555,5.055)--(6.548,5.047)--(6.542,5.039)%
--(6.536,5.030)--(6.530,5.022)--(6.525,5.012)--(6.521,5.003)--(6.517,4.993)%
--(6.513,4.983)--(6.510,4.973)--(6.508,4.963)--(6.506,4.953)--(6.505,4.942)%
--(6.504,4.932)--(6.504,4.922)--(6.504,4.911)--(6.505,4.901)--(6.506,4.890)%
--(6.508,4.880)--(6.510,4.870)--(6.513,4.860)--(6.517,4.850)--(6.521,4.840)%
--(6.525,4.831)--(6.530,4.822)--(6.536,4.813)--(6.542,4.804)--(6.548,4.796)%
--(6.555,4.788)--(6.562,4.780)--(6.570,4.773)--(6.578,4.766)--(6.586,4.760)%
--(6.595,4.754)--(6.604,4.748)--(6.613,4.743)--(6.622,4.739)--(6.632,4.735)%
--(6.642,4.731)--(6.652,4.728)--(6.662,4.726)--(6.672,4.724)--(6.683,4.723)%
--(6.693,4.722)--(6.704,4.722)--(6.714,4.722)--(6.724,4.723)--(6.735,4.724)%
--(6.745,4.726)--(6.755,4.728)--(6.765,4.731)--(6.775,4.735)--(6.785,4.739)%
--(6.794,4.743)--(6.804,4.748)--(6.812,4.754)--(6.821,4.760)--(6.829,4.766)%
--(6.837,4.773)--(6.845,4.780)--(6.852,4.788)--(6.859,4.796)--(6.865,4.804)%
--(6.871,4.813)--(6.877,4.822)--(6.882,4.831)--(6.886,4.840)--(6.890,4.850)%
--(6.894,4.860)--(6.897,4.870)--(6.899,4.880)--(6.901,4.890)--(6.902,4.901)--(6.903,4.911)--cycle;
%
\gpfill{rgb color={0.000,0.000,0.000},opacity=0.15} (6.719,4.977)--(6.718,4.988)--(6.717,4.999)--(6.716,5.010)%
--(6.714,5.021)--(6.711,5.032)--(6.708,5.043)--(6.704,5.054)--(6.700,5.064)%
--(6.695,5.074)--(6.690,5.084)--(6.684,5.094)--(6.677,5.103)--(6.671,5.112)%
--(6.663,5.120)--(6.656,5.129)--(6.647,5.136)--(6.639,5.144)--(6.630,5.150)%
--(6.621,5.157)--(6.611,5.163)--(6.601,5.168)--(6.591,5.173)--(6.581,5.177)%
--(6.570,5.181)--(6.559,5.184)--(6.548,5.187)--(6.537,5.189)--(6.526,5.190)%
--(6.515,5.191)--(6.504,5.192)--(6.492,5.191)--(6.481,5.190)--(6.470,5.189)%
--(6.459,5.187)--(6.448,5.184)--(6.437,5.181)--(6.426,5.177)--(6.416,5.173)%
--(6.406,5.168)--(6.396,5.163)--(6.386,5.157)--(6.377,5.150)--(6.368,5.144)%
--(6.360,5.136)--(6.351,5.129)--(6.344,5.120)--(6.336,5.112)--(6.330,5.103)%
--(6.323,5.094)--(6.317,5.084)--(6.312,5.074)--(6.307,5.064)--(6.303,5.054)%
--(6.299,5.043)--(6.296,5.032)--(6.293,5.021)--(6.291,5.010)--(6.290,4.999)%
--(6.289,4.988)--(6.289,4.977)--(6.289,4.965)--(6.290,4.954)--(6.291,4.943)%
--(6.293,4.932)--(6.296,4.921)--(6.299,4.910)--(6.303,4.899)--(6.307,4.889)%
--(6.312,4.879)--(6.317,4.869)--(6.323,4.859)--(6.330,4.850)--(6.336,4.841)%
--(6.344,4.833)--(6.351,4.824)--(6.360,4.817)--(6.368,4.809)--(6.377,4.803)%
--(6.386,4.796)--(6.396,4.790)--(6.406,4.785)--(6.416,4.780)--(6.426,4.776)%
--(6.437,4.772)--(6.448,4.769)--(6.459,4.766)--(6.470,4.764)--(6.481,4.763)%
--(6.492,4.762)--(6.504,4.762)--(6.515,4.762)--(6.526,4.763)--(6.537,4.764)%
--(6.548,4.766)--(6.559,4.769)--(6.570,4.772)--(6.581,4.776)--(6.591,4.780)%
--(6.601,4.785)--(6.611,4.790)--(6.621,4.796)--(6.630,4.803)--(6.639,4.809)%
--(6.647,4.817)--(6.656,4.824)--(6.663,4.833)--(6.671,4.841)--(6.677,4.850)%
--(6.684,4.859)--(6.690,4.869)--(6.695,4.879)--(6.700,4.889)--(6.704,4.899)%
--(6.708,4.910)--(6.711,4.921)--(6.714,4.932)--(6.716,4.943)--(6.717,4.954)--(6.718,4.965)--cycle;
%
\gpfill{rgb color={0.000,0.000,0.000},opacity=0.15} (6.521,5.055)--(6.520,5.066)--(6.519,5.077)--(6.518,5.088)%
--(6.516,5.100)--(6.513,5.111)--(6.510,5.122)--(6.506,5.132)--(6.502,5.143)%
--(6.497,5.153)--(6.491,5.163)--(6.485,5.173)--(6.479,5.182)--(6.472,5.191)%
--(6.465,5.200)--(6.457,5.208)--(6.449,5.216)--(6.440,5.223)--(6.431,5.230)%
--(6.422,5.236)--(6.412,5.242)--(6.402,5.248)--(6.392,5.253)--(6.381,5.257)%
--(6.371,5.261)--(6.360,5.264)--(6.349,5.267)--(6.337,5.269)--(6.326,5.270)%
--(6.315,5.271)--(6.304,5.272)--(6.292,5.271)--(6.281,5.270)--(6.270,5.269)%
--(6.258,5.267)--(6.247,5.264)--(6.236,5.261)--(6.226,5.257)--(6.215,5.253)%
--(6.205,5.248)--(6.195,5.242)--(6.185,5.236)--(6.176,5.230)--(6.167,5.223)%
--(6.158,5.216)--(6.150,5.208)--(6.142,5.200)--(6.135,5.191)--(6.128,5.182)%
--(6.122,5.173)--(6.116,5.163)--(6.110,5.153)--(6.105,5.143)--(6.101,5.132)%
--(6.097,5.122)--(6.094,5.111)--(6.091,5.100)--(6.089,5.088)--(6.088,5.077)%
--(6.087,5.066)--(6.087,5.055)--(6.087,5.043)--(6.088,5.032)--(6.089,5.021)%
--(6.091,5.009)--(6.094,4.998)--(6.097,4.987)--(6.101,4.977)--(6.105,4.966)%
--(6.110,4.956)--(6.116,4.946)--(6.122,4.936)--(6.128,4.927)--(6.135,4.918)%
--(6.142,4.909)--(6.150,4.901)--(6.158,4.893)--(6.167,4.886)--(6.176,4.879)%
--(6.185,4.873)--(6.195,4.867)--(6.205,4.861)--(6.215,4.856)--(6.226,4.852)%
--(6.236,4.848)--(6.247,4.845)--(6.258,4.842)--(6.270,4.840)--(6.281,4.839)%
--(6.292,4.838)--(6.304,4.838)--(6.315,4.838)--(6.326,4.839)--(6.337,4.840)%
--(6.349,4.842)--(6.360,4.845)--(6.371,4.848)--(6.381,4.852)--(6.392,4.856)%
--(6.402,4.861)--(6.412,4.867)--(6.422,4.873)--(6.431,4.879)--(6.440,4.886)%
--(6.449,4.893)--(6.457,4.901)--(6.465,4.909)--(6.472,4.918)--(6.479,4.927)%
--(6.485,4.936)--(6.491,4.946)--(6.497,4.956)--(6.502,4.966)--(6.506,4.977)%
--(6.510,4.987)--(6.513,4.998)--(6.516,5.009)--(6.518,5.021)--(6.519,5.032)--(6.520,5.043)--cycle;
%
\gpfill{rgb color={0.000,0.000,0.000},opacity=0.15} (6.320,5.137)--(6.319,5.148)--(6.318,5.159)--(6.317,5.170)%
--(6.315,5.182)--(6.312,5.193)--(6.309,5.204)--(6.305,5.214)--(6.301,5.225)%
--(6.296,5.235)--(6.290,5.245)--(6.284,5.255)--(6.278,5.264)--(6.271,5.273)%
--(6.264,5.282)--(6.256,5.290)--(6.248,5.298)--(6.239,5.305)--(6.230,5.312)%
--(6.221,5.318)--(6.211,5.324)--(6.201,5.330)--(6.191,5.335)--(6.180,5.339)%
--(6.170,5.343)--(6.159,5.346)--(6.148,5.349)--(6.136,5.351)--(6.125,5.352)%
--(6.114,5.353)--(6.103,5.354)--(6.091,5.353)--(6.080,5.352)--(6.069,5.351)%
--(6.057,5.349)--(6.046,5.346)--(6.035,5.343)--(6.025,5.339)--(6.014,5.335)%
--(6.004,5.330)--(5.994,5.324)--(5.984,5.318)--(5.975,5.312)--(5.966,5.305)%
--(5.957,5.298)--(5.949,5.290)--(5.941,5.282)--(5.934,5.273)--(5.927,5.264)%
--(5.921,5.255)--(5.915,5.245)--(5.909,5.235)--(5.904,5.225)--(5.900,5.214)%
--(5.896,5.204)--(5.893,5.193)--(5.890,5.182)--(5.888,5.170)--(5.887,5.159)%
--(5.886,5.148)--(5.886,5.137)--(5.886,5.125)--(5.887,5.114)--(5.888,5.103)%
--(5.890,5.091)--(5.893,5.080)--(5.896,5.069)--(5.900,5.059)--(5.904,5.048)%
--(5.909,5.038)--(5.915,5.028)--(5.921,5.018)--(5.927,5.009)--(5.934,5.000)%
--(5.941,4.991)--(5.949,4.983)--(5.957,4.975)--(5.966,4.968)--(5.975,4.961)%
--(5.984,4.955)--(5.994,4.949)--(6.004,4.943)--(6.014,4.938)--(6.025,4.934)%
--(6.035,4.930)--(6.046,4.927)--(6.057,4.924)--(6.069,4.922)--(6.080,4.921)%
--(6.091,4.920)--(6.103,4.920)--(6.114,4.920)--(6.125,4.921)--(6.136,4.922)%
--(6.148,4.924)--(6.159,4.927)--(6.170,4.930)--(6.180,4.934)--(6.191,4.938)%
--(6.201,4.943)--(6.211,4.949)--(6.221,4.955)--(6.230,4.961)--(6.239,4.968)%
--(6.248,4.975)--(6.256,4.983)--(6.264,4.991)--(6.271,5.000)--(6.278,5.009)%
--(6.284,5.018)--(6.290,5.028)--(6.296,5.038)--(6.301,5.048)--(6.305,5.059)%
--(6.309,5.069)--(6.312,5.080)--(6.315,5.091)--(6.317,5.103)--(6.318,5.114)--(6.319,5.125)--cycle;
%
\gpfill{rgb color={0.000,0.000,0.000},opacity=0.15} (6.168,5.221)--(6.167,5.234)--(6.166,5.248)--(6.164,5.262)%
--(6.162,5.276)--(6.158,5.289)--(6.155,5.302)--(6.150,5.315)--(6.145,5.328)%
--(6.139,5.341)--(6.132,5.353)--(6.125,5.365)--(6.117,5.376)--(6.108,5.387)%
--(6.099,5.398)--(6.090,5.408)--(6.080,5.417)--(6.069,5.426)--(6.058,5.435)%
--(6.047,5.443)--(6.035,5.450)--(6.023,5.457)--(6.010,5.463)--(5.997,5.468)%
--(5.984,5.473)--(5.971,5.476)--(5.958,5.480)--(5.944,5.482)--(5.930,5.484)%
--(5.916,5.485)--(5.903,5.486)--(5.889,5.485)--(5.875,5.484)--(5.861,5.482)%
--(5.847,5.480)--(5.834,5.476)--(5.821,5.473)--(5.808,5.468)--(5.795,5.463)%
--(5.782,5.457)--(5.770,5.450)--(5.758,5.443)--(5.747,5.435)--(5.736,5.426)%
--(5.725,5.417)--(5.715,5.408)--(5.706,5.398)--(5.697,5.387)--(5.688,5.376)%
--(5.680,5.365)--(5.673,5.353)--(5.666,5.341)--(5.660,5.328)--(5.655,5.315)%
--(5.650,5.302)--(5.647,5.289)--(5.643,5.276)--(5.641,5.262)--(5.639,5.248)%
--(5.638,5.234)--(5.638,5.221)--(5.638,5.207)--(5.639,5.193)--(5.641,5.179)%
--(5.643,5.165)--(5.647,5.152)--(5.650,5.139)--(5.655,5.126)--(5.660,5.113)%
--(5.666,5.100)--(5.673,5.088)--(5.680,5.076)--(5.688,5.065)--(5.697,5.054)%
--(5.706,5.043)--(5.715,5.033)--(5.725,5.024)--(5.736,5.015)--(5.747,5.006)%
--(5.758,4.998)--(5.770,4.991)--(5.782,4.984)--(5.795,4.978)--(5.808,4.973)%
--(5.821,4.968)--(5.834,4.965)--(5.847,4.961)--(5.861,4.959)--(5.875,4.957)%
--(5.889,4.956)--(5.903,4.956)--(5.916,4.956)--(5.930,4.957)--(5.944,4.959)%
--(5.958,4.961)--(5.971,4.965)--(5.984,4.968)--(5.997,4.973)--(6.010,4.978)%
--(6.023,4.984)--(6.035,4.991)--(6.047,4.998)--(6.058,5.006)--(6.069,5.015)%
--(6.080,5.024)--(6.090,5.033)--(6.099,5.043)--(6.108,5.054)--(6.117,5.065)%
--(6.125,5.076)--(6.132,5.088)--(6.139,5.100)--(6.145,5.113)--(6.150,5.126)%
--(6.155,5.139)--(6.158,5.152)--(6.162,5.165)--(6.164,5.179)--(6.166,5.193)--(6.167,5.207)--cycle;
%
\gpfill{rgb color={0.000,0.000,0.000},opacity=0.15} (6.048,5.306)--(6.047,5.324)--(6.046,5.342)--(6.043,5.359)%
--(6.040,5.377)--(6.036,5.395)--(6.031,5.412)--(6.025,5.429)--(6.018,5.446)%
--(6.010,5.462)--(6.001,5.478)--(5.992,5.493)--(5.982,5.508)--(5.971,5.523)%
--(5.959,5.536)--(5.946,5.549)--(5.933,5.562)--(5.920,5.574)--(5.905,5.585)%
--(5.890,5.595)--(5.875,5.604)--(5.859,5.613)--(5.843,5.621)--(5.826,5.628)%
--(5.809,5.634)--(5.792,5.639)--(5.774,5.643)--(5.756,5.646)--(5.739,5.649)%
--(5.721,5.650)--(5.703,5.651)--(5.684,5.650)--(5.666,5.649)--(5.649,5.646)%
--(5.631,5.643)--(5.613,5.639)--(5.596,5.634)--(5.579,5.628)--(5.562,5.621)%
--(5.546,5.613)--(5.530,5.604)--(5.515,5.595)--(5.500,5.585)--(5.485,5.574)%
--(5.472,5.562)--(5.459,5.549)--(5.446,5.536)--(5.434,5.523)--(5.423,5.508)%
--(5.413,5.493)--(5.404,5.478)--(5.395,5.462)--(5.387,5.446)--(5.380,5.429)%
--(5.374,5.412)--(5.369,5.395)--(5.365,5.377)--(5.362,5.359)--(5.359,5.342)%
--(5.358,5.324)--(5.358,5.306)--(5.358,5.287)--(5.359,5.269)--(5.362,5.252)%
--(5.365,5.234)--(5.369,5.216)--(5.374,5.199)--(5.380,5.182)--(5.387,5.165)%
--(5.395,5.149)--(5.404,5.133)--(5.413,5.118)--(5.423,5.103)--(5.434,5.088)%
--(5.446,5.075)--(5.459,5.062)--(5.472,5.049)--(5.485,5.037)--(5.500,5.026)%
--(5.515,5.016)--(5.530,5.007)--(5.546,4.998)--(5.562,4.990)--(5.579,4.983)%
--(5.596,4.977)--(5.613,4.972)--(5.631,4.968)--(5.649,4.965)--(5.666,4.962)%
--(5.684,4.961)--(5.703,4.961)--(5.721,4.961)--(5.739,4.962)--(5.756,4.965)%
--(5.774,4.968)--(5.792,4.972)--(5.809,4.977)--(5.826,4.983)--(5.843,4.990)%
--(5.859,4.998)--(5.875,5.007)--(5.890,5.016)--(5.905,5.026)--(5.920,5.037)%
--(5.933,5.049)--(5.946,5.062)--(5.959,5.075)--(5.971,5.088)--(5.982,5.103)%
--(5.992,5.118)--(6.001,5.133)--(6.010,5.149)--(6.018,5.165)--(6.025,5.182)%
--(6.031,5.199)--(6.036,5.216)--(6.040,5.234)--(6.043,5.252)--(6.046,5.269)--(6.047,5.287)--cycle;
%
\gpfill{rgb color={0.000,0.000,0.000},opacity=0.15} (5.930,5.390)--(5.929,5.412)--(5.927,5.434)--(5.924,5.456)%
--(5.920,5.478)--(5.915,5.500)--(5.909,5.521)--(5.901,5.542)--(5.893,5.563)%
--(5.883,5.583)--(5.872,5.603)--(5.861,5.622)--(5.848,5.640)--(5.835,5.658)%
--(5.820,5.675)--(5.805,5.691)--(5.789,5.706)--(5.772,5.721)--(5.754,5.734)%
--(5.736,5.747)--(5.717,5.758)--(5.697,5.769)--(5.677,5.779)--(5.656,5.787)%
--(5.635,5.795)--(5.614,5.801)--(5.592,5.806)--(5.570,5.810)--(5.548,5.813)%
--(5.526,5.815)--(5.504,5.816)--(5.481,5.815)--(5.459,5.813)--(5.437,5.810)%
--(5.415,5.806)--(5.393,5.801)--(5.372,5.795)--(5.351,5.787)--(5.330,5.779)%
--(5.310,5.769)--(5.291,5.758)--(5.271,5.747)--(5.253,5.734)--(5.235,5.721)%
--(5.218,5.706)--(5.202,5.691)--(5.187,5.675)--(5.172,5.658)--(5.159,5.640)%
--(5.146,5.622)--(5.135,5.603)--(5.124,5.583)--(5.114,5.563)--(5.106,5.542)%
--(5.098,5.521)--(5.092,5.500)--(5.087,5.478)--(5.083,5.456)--(5.080,5.434)%
--(5.078,5.412)--(5.078,5.390)--(5.078,5.367)--(5.080,5.345)--(5.083,5.323)%
--(5.087,5.301)--(5.092,5.279)--(5.098,5.258)--(5.106,5.237)--(5.114,5.216)%
--(5.124,5.196)--(5.135,5.177)--(5.146,5.157)--(5.159,5.139)--(5.172,5.121)%
--(5.187,5.104)--(5.202,5.088)--(5.218,5.073)--(5.235,5.058)--(5.253,5.045)%
--(5.271,5.032)--(5.291,5.021)--(5.310,5.010)--(5.330,5.000)--(5.351,4.992)%
--(5.372,4.984)--(5.393,4.978)--(5.415,4.973)--(5.437,4.969)--(5.459,4.966)%
--(5.481,4.964)--(5.504,4.964)--(5.526,4.964)--(5.548,4.966)--(5.570,4.969)%
--(5.592,4.973)--(5.614,4.978)--(5.635,4.984)--(5.656,4.992)--(5.677,5.000)%
--(5.697,5.010)--(5.717,5.021)--(5.736,5.032)--(5.754,5.045)--(5.772,5.058)%
--(5.789,5.073)--(5.805,5.088)--(5.820,5.104)--(5.835,5.121)--(5.848,5.139)%
--(5.861,5.157)--(5.872,5.177)--(5.883,5.196)--(5.893,5.216)--(5.901,5.237)%
--(5.909,5.258)--(5.915,5.279)--(5.920,5.301)--(5.924,5.323)--(5.927,5.345)--(5.929,5.367)--cycle;
%
\gpfill{rgb color={0.000,0.000,0.000},opacity=0.15} (5.815,5.473)--(5.814,5.499)--(5.812,5.526)--(5.808,5.552)%
--(5.803,5.578)--(5.797,5.604)--(5.790,5.629)--(5.781,5.655)--(5.771,5.679)%
--(5.759,5.703)--(5.746,5.727)--(5.733,5.749)--(5.717,5.771)--(5.701,5.792)%
--(5.684,5.812)--(5.666,5.832)--(5.646,5.850)--(5.626,5.867)--(5.605,5.883)%
--(5.583,5.899)--(5.561,5.912)--(5.537,5.925)--(5.513,5.937)--(5.489,5.947)%
--(5.463,5.956)--(5.438,5.963)--(5.412,5.969)--(5.386,5.974)--(5.360,5.978)%
--(5.333,5.980)--(5.307,5.981)--(5.280,5.980)--(5.253,5.978)--(5.227,5.974)%
--(5.201,5.969)--(5.175,5.963)--(5.150,5.956)--(5.124,5.947)--(5.100,5.937)%
--(5.076,5.925)--(5.053,5.912)--(5.030,5.899)--(5.008,5.883)--(4.987,5.867)%
--(4.967,5.850)--(4.947,5.832)--(4.929,5.812)--(4.912,5.792)--(4.896,5.771)%
--(4.880,5.749)--(4.867,5.727)--(4.854,5.703)--(4.842,5.679)--(4.832,5.655)%
--(4.823,5.629)--(4.816,5.604)--(4.810,5.578)--(4.805,5.552)--(4.801,5.526)%
--(4.799,5.499)--(4.799,5.473)--(4.799,5.446)--(4.801,5.419)--(4.805,5.393)%
--(4.810,5.367)--(4.816,5.341)--(4.823,5.316)--(4.832,5.290)--(4.842,5.266)%
--(4.854,5.242)--(4.867,5.219)--(4.880,5.196)--(4.896,5.174)--(4.912,5.153)%
--(4.929,5.133)--(4.947,5.113)--(4.967,5.095)--(4.987,5.078)--(5.008,5.062)%
--(5.030,5.046)--(5.053,5.033)--(5.076,5.020)--(5.100,5.008)--(5.124,4.998)%
--(5.150,4.989)--(5.175,4.982)--(5.201,4.976)--(5.227,4.971)--(5.253,4.967)%
--(5.280,4.965)--(5.307,4.965)--(5.333,4.965)--(5.360,4.967)--(5.386,4.971)%
--(5.412,4.976)--(5.438,4.982)--(5.463,4.989)--(5.489,4.998)--(5.513,5.008)%
--(5.537,5.020)--(5.561,5.033)--(5.583,5.046)--(5.605,5.062)--(5.626,5.078)%
--(5.646,5.095)--(5.666,5.113)--(5.684,5.133)--(5.701,5.153)--(5.717,5.174)%
--(5.733,5.196)--(5.746,5.219)--(5.759,5.242)--(5.771,5.266)--(5.781,5.290)%
--(5.790,5.316)--(5.797,5.341)--(5.803,5.367)--(5.808,5.393)--(5.812,5.419)--(5.814,5.446)--cycle;
%
\gpfill{rgb color={0.000,0.000,0.000},opacity=0.15} (5.699,5.553)--(5.698,5.583)--(5.695,5.614)--(5.691,5.644)%
--(5.686,5.675)--(5.678,5.705)--(5.670,5.734)--(5.659,5.763)--(5.648,5.792)%
--(5.634,5.819)--(5.620,5.847)--(5.604,5.873)--(5.586,5.898)--(5.567,5.923)%
--(5.547,5.946)--(5.526,5.968)--(5.504,5.989)--(5.481,6.009)--(5.456,6.028)%
--(5.431,6.046)--(5.405,6.062)--(5.377,6.076)--(5.350,6.090)--(5.321,6.101)%
--(5.292,6.112)--(5.263,6.120)--(5.233,6.128)--(5.202,6.133)--(5.172,6.137)%
--(5.141,6.140)--(5.111,6.141)--(5.080,6.140)--(5.049,6.137)--(5.019,6.133)%
--(4.988,6.128)--(4.958,6.120)--(4.929,6.112)--(4.900,6.101)--(4.871,6.090)%
--(4.844,6.076)--(4.817,6.062)--(4.790,6.046)--(4.765,6.028)--(4.740,6.009)%
--(4.717,5.989)--(4.695,5.968)--(4.674,5.946)--(4.654,5.923)--(4.635,5.898)%
--(4.617,5.873)--(4.601,5.847)--(4.587,5.819)--(4.573,5.792)--(4.562,5.763)%
--(4.551,5.734)--(4.543,5.705)--(4.535,5.675)--(4.530,5.644)--(4.526,5.614)%
--(4.523,5.583)--(4.523,5.553)--(4.523,5.522)--(4.526,5.491)--(4.530,5.461)%
--(4.535,5.430)--(4.543,5.400)--(4.551,5.371)--(4.562,5.342)--(4.573,5.313)%
--(4.587,5.286)--(4.601,5.259)--(4.617,5.232)--(4.635,5.207)--(4.654,5.182)%
--(4.674,5.159)--(4.695,5.137)--(4.717,5.116)--(4.740,5.096)--(4.765,5.077)%
--(4.790,5.059)--(4.817,5.043)--(4.844,5.029)--(4.871,5.015)--(4.900,5.004)%
--(4.929,4.993)--(4.958,4.985)--(4.988,4.977)--(5.019,4.972)--(5.049,4.968)%
--(5.080,4.965)--(5.111,4.965)--(5.141,4.965)--(5.172,4.968)--(5.202,4.972)%
--(5.233,4.977)--(5.263,4.985)--(5.292,4.993)--(5.321,5.004)--(5.350,5.015)%
--(5.377,5.029)--(5.405,5.043)--(5.431,5.059)--(5.456,5.077)--(5.481,5.096)%
--(5.504,5.116)--(5.526,5.137)--(5.547,5.159)--(5.567,5.182)--(5.586,5.207)%
--(5.604,5.232)--(5.620,5.259)--(5.634,5.286)--(5.648,5.313)--(5.659,5.342)%
--(5.670,5.371)--(5.678,5.400)--(5.686,5.430)--(5.691,5.461)--(5.695,5.491)--(5.698,5.522)--cycle;
%
\gpfill{rgb color={0.000,0.000,0.000},opacity=0.15} (5.580,5.629)--(5.579,5.663)--(5.576,5.698)--(5.571,5.732)%
--(5.565,5.767)--(5.557,5.800)--(5.547,5.834)--(5.535,5.866)--(5.522,5.899)%
--(5.507,5.930)--(5.491,5.961)--(5.472,5.990)--(5.453,6.019)--(5.432,6.046)%
--(5.409,6.073)--(5.385,6.098)--(5.360,6.122)--(5.333,6.145)--(5.306,6.166)%
--(5.277,6.185)--(5.248,6.204)--(5.217,6.220)--(5.186,6.235)--(5.153,6.248)%
--(5.121,6.260)--(5.087,6.270)--(5.054,6.278)--(5.019,6.284)--(4.985,6.289)%
--(4.950,6.292)--(4.916,6.293)--(4.881,6.292)--(4.846,6.289)--(4.812,6.284)%
--(4.777,6.278)--(4.744,6.270)--(4.710,6.260)--(4.678,6.248)--(4.645,6.235)%
--(4.614,6.220)--(4.584,6.204)--(4.554,6.185)--(4.525,6.166)--(4.498,6.145)%
--(4.471,6.122)--(4.446,6.098)--(4.422,6.073)--(4.399,6.046)--(4.378,6.019)%
--(4.359,5.990)--(4.340,5.961)--(4.324,5.930)--(4.309,5.899)--(4.296,5.866)%
--(4.284,5.834)--(4.274,5.800)--(4.266,5.767)--(4.260,5.732)--(4.255,5.698)%
--(4.252,5.663)--(4.252,5.629)--(4.252,5.594)--(4.255,5.559)--(4.260,5.525)%
--(4.266,5.490)--(4.274,5.457)--(4.284,5.423)--(4.296,5.391)--(4.309,5.358)%
--(4.324,5.327)--(4.340,5.297)--(4.359,5.267)--(4.378,5.238)--(4.399,5.211)%
--(4.422,5.184)--(4.446,5.159)--(4.471,5.135)--(4.498,5.112)--(4.525,5.091)%
--(4.554,5.072)--(4.584,5.053)--(4.614,5.037)--(4.645,5.022)--(4.678,5.009)%
--(4.710,4.997)--(4.744,4.987)--(4.777,4.979)--(4.812,4.973)--(4.846,4.968)%
--(4.881,4.965)--(4.916,4.965)--(4.950,4.965)--(4.985,4.968)--(5.019,4.973)%
--(5.054,4.979)--(5.087,4.987)--(5.121,4.997)--(5.153,5.009)--(5.186,5.022)%
--(5.217,5.037)--(5.248,5.053)--(5.277,5.072)--(5.306,5.091)--(5.333,5.112)%
--(5.360,5.135)--(5.385,5.159)--(5.409,5.184)--(5.432,5.211)--(5.453,5.238)%
--(5.472,5.267)--(5.491,5.297)--(5.507,5.327)--(5.522,5.358)--(5.535,5.391)%
--(5.547,5.423)--(5.557,5.457)--(5.565,5.490)--(5.571,5.525)--(5.576,5.559)--(5.579,5.594)--cycle;
%
\gpfill{rgb color={0.000,0.000,0.000},opacity=0.15} (5.461,5.699)--(5.459,5.737)--(5.456,5.776)--(5.451,5.814)%
--(5.444,5.852)--(5.435,5.889)--(5.424,5.926)--(5.412,5.963)--(5.397,5.998)%
--(5.380,6.033)--(5.362,6.067)--(5.342,6.100)--(5.320,6.132)--(5.296,6.162)%
--(5.271,6.192)--(5.245,6.220)--(5.217,6.246)--(5.187,6.271)--(5.157,6.295)%
--(5.125,6.317)--(5.092,6.337)--(5.058,6.355)--(5.023,6.372)--(4.988,6.387)%
--(4.951,6.399)--(4.914,6.410)--(4.877,6.419)--(4.839,6.426)--(4.801,6.431)%
--(4.762,6.434)--(4.724,6.436)--(4.685,6.434)--(4.646,6.431)--(4.608,6.426)%
--(4.570,6.419)--(4.533,6.410)--(4.496,6.399)--(4.459,6.387)--(4.424,6.372)%
--(4.389,6.355)--(4.355,6.337)--(4.322,6.317)--(4.290,6.295)--(4.260,6.271)%
--(4.230,6.246)--(4.202,6.220)--(4.176,6.192)--(4.151,6.162)--(4.127,6.132)%
--(4.105,6.100)--(4.085,6.067)--(4.067,6.033)--(4.050,5.998)--(4.035,5.963)%
--(4.023,5.926)--(4.012,5.889)--(4.003,5.852)--(3.996,5.814)--(3.991,5.776)%
--(3.988,5.737)--(3.987,5.699)--(3.988,5.660)--(3.991,5.621)--(3.996,5.583)%
--(4.003,5.545)--(4.012,5.508)--(4.023,5.471)--(4.035,5.434)--(4.050,5.399)%
--(4.067,5.364)--(4.085,5.330)--(4.105,5.297)--(4.127,5.265)--(4.151,5.235)%
--(4.176,5.205)--(4.202,5.177)--(4.230,5.151)--(4.260,5.126)--(4.290,5.102)%
--(4.322,5.080)--(4.355,5.060)--(4.389,5.042)--(4.424,5.025)--(4.459,5.010)%
--(4.496,4.998)--(4.533,4.987)--(4.570,4.978)--(4.608,4.971)--(4.646,4.966)%
--(4.685,4.963)--(4.724,4.962)--(4.762,4.963)--(4.801,4.966)--(4.839,4.971)%
--(4.877,4.978)--(4.914,4.987)--(4.951,4.998)--(4.988,5.010)--(5.023,5.025)%
--(5.058,5.042)--(5.092,5.060)--(5.125,5.080)--(5.157,5.102)--(5.187,5.126)%
--(5.217,5.151)--(5.245,5.177)--(5.271,5.205)--(5.296,5.235)--(5.320,5.265)%
--(5.342,5.297)--(5.362,5.330)--(5.380,5.364)--(5.397,5.399)--(5.412,5.434)%
--(5.424,5.471)--(5.435,5.508)--(5.444,5.545)--(5.451,5.583)--(5.456,5.621)--(5.459,5.660)--cycle;
%
\gpfill{rgb color={0.000,0.000,0.000},opacity=0.15} (5.341,5.763)--(5.339,5.805)--(5.336,5.847)--(5.331,5.889)%
--(5.323,5.930)--(5.313,5.971)--(5.301,6.012)--(5.287,6.051)--(5.271,6.090)%
--(5.253,6.128)--(5.233,6.166)--(5.210,6.201)--(5.187,6.236)--(5.161,6.270)%
--(5.133,6.302)--(5.104,6.332)--(5.074,6.361)--(5.042,6.389)--(5.008,6.415)%
--(4.973,6.438)--(4.938,6.461)--(4.900,6.481)--(4.862,6.499)--(4.823,6.515)%
--(4.784,6.529)--(4.743,6.541)--(4.702,6.551)--(4.661,6.559)--(4.619,6.564)%
--(4.577,6.567)--(4.535,6.569)--(4.492,6.567)--(4.450,6.564)--(4.408,6.559)%
--(4.367,6.551)--(4.326,6.541)--(4.285,6.529)--(4.246,6.515)--(4.207,6.499)%
--(4.169,6.481)--(4.132,6.461)--(4.096,6.438)--(4.061,6.415)--(4.027,6.389)%
--(3.995,6.361)--(3.965,6.332)--(3.936,6.302)--(3.908,6.270)--(3.882,6.236)%
--(3.859,6.201)--(3.836,6.166)--(3.816,6.128)--(3.798,6.090)--(3.782,6.051)%
--(3.768,6.012)--(3.756,5.971)--(3.746,5.930)--(3.738,5.889)--(3.733,5.847)%
--(3.730,5.805)--(3.729,5.763)--(3.730,5.720)--(3.733,5.678)--(3.738,5.636)%
--(3.746,5.595)--(3.756,5.554)--(3.768,5.513)--(3.782,5.474)--(3.798,5.435)%
--(3.816,5.397)--(3.836,5.360)--(3.859,5.324)--(3.882,5.289)--(3.908,5.255)%
--(3.936,5.223)--(3.965,5.193)--(3.995,5.164)--(4.027,5.136)--(4.061,5.110)%
--(4.096,5.087)--(4.132,5.064)--(4.169,5.044)--(4.207,5.026)--(4.246,5.010)%
--(4.285,4.996)--(4.326,4.984)--(4.367,4.974)--(4.408,4.966)--(4.450,4.961)%
--(4.492,4.958)--(4.535,4.957)--(4.577,4.958)--(4.619,4.961)--(4.661,4.966)%
--(4.702,4.974)--(4.743,4.984)--(4.784,4.996)--(4.823,5.010)--(4.862,5.026)%
--(4.900,5.044)--(4.938,5.064)--(4.973,5.087)--(5.008,5.110)--(5.042,5.136)%
--(5.074,5.164)--(5.104,5.193)--(5.133,5.223)--(5.161,5.255)--(5.187,5.289)%
--(5.210,5.324)--(5.233,5.360)--(5.253,5.397)--(5.271,5.435)--(5.287,5.474)%
--(5.301,5.513)--(5.313,5.554)--(5.323,5.595)--(5.331,5.636)--(5.336,5.678)--(5.339,5.720)--cycle;
%
\gpfill{rgb color={0.000,0.000,0.000},opacity=0.15} (5.217,5.818)--(5.215,5.863)--(5.212,5.908)--(5.206,5.953)%
--(5.198,5.998)--(5.187,6.042)--(5.174,6.086)--(5.159,6.129)--(5.141,6.171)%
--(5.122,6.212)--(5.100,6.252)--(5.076,6.290)--(5.051,6.328)--(5.023,6.364)%
--(4.994,6.398)--(4.962,6.431)--(4.929,6.463)--(4.895,6.492)--(4.859,6.520)%
--(4.821,6.545)--(4.783,6.569)--(4.743,6.591)--(4.702,6.610)--(4.660,6.628)%
--(4.617,6.643)--(4.573,6.656)--(4.529,6.667)--(4.484,6.675)--(4.439,6.681)%
--(4.394,6.684)--(4.349,6.686)--(4.303,6.684)--(4.258,6.681)--(4.213,6.675)%
--(4.168,6.667)--(4.124,6.656)--(4.080,6.643)--(4.037,6.628)--(3.995,6.610)%
--(3.954,6.591)--(3.915,6.569)--(3.876,6.545)--(3.838,6.520)--(3.802,6.492)%
--(3.768,6.463)--(3.735,6.431)--(3.703,6.398)--(3.674,6.364)--(3.646,6.328)%
--(3.621,6.290)--(3.597,6.252)--(3.575,6.212)--(3.556,6.171)--(3.538,6.129)%
--(3.523,6.086)--(3.510,6.042)--(3.499,5.998)--(3.491,5.953)--(3.485,5.908)%
--(3.482,5.863)--(3.481,5.818)--(3.482,5.772)--(3.485,5.727)--(3.491,5.682)%
--(3.499,5.637)--(3.510,5.593)--(3.523,5.549)--(3.538,5.506)--(3.556,5.464)%
--(3.575,5.423)--(3.597,5.384)--(3.621,5.345)--(3.646,5.307)--(3.674,5.271)%
--(3.703,5.237)--(3.735,5.204)--(3.768,5.172)--(3.802,5.143)--(3.838,5.115)%
--(3.876,5.090)--(3.914,5.066)--(3.954,5.044)--(3.995,5.025)--(4.037,5.007)%
--(4.080,4.992)--(4.124,4.979)--(4.168,4.968)--(4.213,4.960)--(4.258,4.954)%
--(4.303,4.951)--(4.349,4.950)--(4.394,4.951)--(4.439,4.954)--(4.484,4.960)%
--(4.529,4.968)--(4.573,4.979)--(4.617,4.992)--(4.660,5.007)--(4.702,5.025)%
--(4.743,5.044)--(4.783,5.066)--(4.821,5.090)--(4.859,5.115)--(4.895,5.143)%
--(4.929,5.172)--(4.962,5.204)--(4.994,5.237)--(5.023,5.271)--(5.051,5.307)%
--(5.076,5.345)--(5.100,5.384)--(5.122,5.423)--(5.141,5.464)--(5.159,5.506)%
--(5.174,5.549)--(5.187,5.593)--(5.198,5.637)--(5.206,5.682)--(5.212,5.727)--(5.215,5.772)--cycle;
%
\gpfill{rgb color={0.000,0.000,0.000},opacity=0.15} (5.089,5.865)--(5.087,5.913)--(5.083,5.961)--(5.077,6.009)%
--(5.068,6.056)--(5.057,6.103)--(5.043,6.150)--(5.027,6.195)--(5.009,6.240)%
--(4.988,6.284)--(4.965,6.326)--(4.940,6.367)--(4.912,6.407)--(4.883,6.445)%
--(4.851,6.482)--(4.818,6.517)--(4.783,6.550)--(4.746,6.582)--(4.708,6.611)%
--(4.668,6.639)--(4.627,6.664)--(4.585,6.687)--(4.541,6.708)--(4.496,6.726)%
--(4.451,6.742)--(4.404,6.756)--(4.357,6.767)--(4.310,6.776)--(4.262,6.782)%
--(4.214,6.786)--(4.166,6.788)--(4.117,6.786)--(4.069,6.782)--(4.021,6.776)%
--(3.974,6.767)--(3.927,6.756)--(3.880,6.742)--(3.835,6.726)--(3.790,6.708)%
--(3.746,6.687)--(3.704,6.664)--(3.663,6.639)--(3.623,6.611)--(3.585,6.582)%
--(3.548,6.550)--(3.513,6.517)--(3.480,6.482)--(3.448,6.445)--(3.419,6.407)%
--(3.391,6.367)--(3.366,6.326)--(3.343,6.284)--(3.322,6.240)--(3.304,6.195)%
--(3.288,6.150)--(3.274,6.103)--(3.263,6.056)--(3.254,6.009)--(3.248,5.961)%
--(3.244,5.913)--(3.243,5.865)--(3.244,5.816)--(3.248,5.768)--(3.254,5.720)%
--(3.263,5.673)--(3.274,5.626)--(3.288,5.579)--(3.304,5.534)--(3.322,5.489)%
--(3.343,5.445)--(3.366,5.403)--(3.391,5.362)--(3.419,5.322)--(3.448,5.284)%
--(3.480,5.247)--(3.513,5.212)--(3.548,5.179)--(3.585,5.147)--(3.623,5.118)%
--(3.663,5.090)--(3.704,5.065)--(3.746,5.042)--(3.790,5.021)--(3.835,5.003)%
--(3.880,4.987)--(3.927,4.973)--(3.974,4.962)--(4.021,4.953)--(4.069,4.947)%
--(4.117,4.943)--(4.166,4.942)--(4.214,4.943)--(4.262,4.947)--(4.310,4.953)%
--(4.357,4.962)--(4.404,4.973)--(4.451,4.987)--(4.496,5.003)--(4.541,5.021)%
--(4.585,5.042)--(4.627,5.065)--(4.668,5.090)--(4.708,5.118)--(4.746,5.147)%
--(4.783,5.179)--(4.818,5.212)--(4.851,5.247)--(4.883,5.284)--(4.912,5.322)%
--(4.940,5.362)--(4.965,5.403)--(4.988,5.445)--(5.009,5.489)--(5.027,5.534)%
--(5.043,5.579)--(5.057,5.626)--(5.068,5.673)--(5.077,5.720)--(5.083,5.768)--(5.087,5.816)--cycle;
%
\gpfill{rgb color={0.000,0.000,0.000},opacity=0.15} (4.956,5.903)--(4.954,5.953)--(4.950,6.004)--(4.944,6.054)%
--(4.934,6.104)--(4.922,6.154)--(4.908,6.202)--(4.891,6.250)--(4.872,6.297)%
--(4.850,6.343)--(4.826,6.388)--(4.799,6.431)--(4.770,6.473)--(4.739,6.513)%
--(4.706,6.552)--(4.671,6.588)--(4.635,6.623)--(4.596,6.656)--(4.556,6.687)%
--(4.514,6.716)--(4.471,6.743)--(4.426,6.767)--(4.380,6.789)--(4.333,6.808)%
--(4.285,6.825)--(4.237,6.839)--(4.187,6.851)--(4.137,6.861)--(4.087,6.867)%
--(4.036,6.871)--(3.986,6.873)--(3.935,6.871)--(3.884,6.867)--(3.834,6.861)%
--(3.784,6.851)--(3.734,6.839)--(3.686,6.825)--(3.638,6.808)--(3.591,6.789)%
--(3.545,6.767)--(3.501,6.743)--(3.457,6.716)--(3.415,6.687)--(3.375,6.656)%
--(3.336,6.623)--(3.300,6.588)--(3.265,6.552)--(3.232,6.513)--(3.201,6.473)%
--(3.172,6.431)--(3.145,6.388)--(3.121,6.343)--(3.099,6.297)--(3.080,6.250)%
--(3.063,6.202)--(3.049,6.154)--(3.037,6.104)--(3.027,6.054)--(3.021,6.004)%
--(3.017,5.953)--(3.016,5.903)--(3.017,5.852)--(3.021,5.801)--(3.027,5.751)%
--(3.037,5.701)--(3.049,5.651)--(3.063,5.603)--(3.080,5.555)--(3.099,5.508)%
--(3.121,5.462)--(3.145,5.418)--(3.172,5.374)--(3.201,5.332)--(3.232,5.292)%
--(3.265,5.253)--(3.300,5.217)--(3.336,5.182)--(3.375,5.149)--(3.415,5.118)%
--(3.457,5.089)--(3.500,5.062)--(3.545,5.038)--(3.591,5.016)--(3.638,4.997)%
--(3.686,4.980)--(3.734,4.966)--(3.784,4.954)--(3.834,4.944)--(3.884,4.938)%
--(3.935,4.934)--(3.986,4.933)--(4.036,4.934)--(4.087,4.938)--(4.137,4.944)%
--(4.187,4.954)--(4.237,4.966)--(4.285,4.980)--(4.333,4.997)--(4.380,5.016)%
--(4.426,5.038)--(4.471,5.062)--(4.514,5.089)--(4.556,5.118)--(4.596,5.149)%
--(4.635,5.182)--(4.671,5.217)--(4.706,5.253)--(4.739,5.292)--(4.770,5.332)%
--(4.799,5.374)--(4.826,5.418)--(4.850,5.462)--(4.872,5.508)--(4.891,5.555)%
--(4.908,5.603)--(4.922,5.651)--(4.934,5.701)--(4.944,5.751)--(4.950,5.801)--(4.954,5.852)--cycle;
%
\gpfill{rgb color={0.000,0.000,0.000},opacity=0.15} (4.819,5.931)--(4.817,5.983)--(4.813,6.036)--(4.806,6.088)%
--(4.796,6.140)--(4.784,6.192)--(4.769,6.242)--(4.751,6.292)--(4.731,6.341)%
--(4.709,6.389)--(4.683,6.435)--(4.656,6.480)--(4.626,6.524)--(4.594,6.565)%
--(4.559,6.606)--(4.523,6.644)--(4.485,6.680)--(4.444,6.715)--(4.403,6.747)%
--(4.359,6.777)--(4.314,6.804)--(4.268,6.830)--(4.220,6.852)--(4.171,6.872)%
--(4.121,6.890)--(4.071,6.905)--(4.019,6.917)--(3.967,6.927)--(3.915,6.934)%
--(3.862,6.938)--(3.810,6.940)--(3.757,6.938)--(3.704,6.934)--(3.652,6.927)%
--(3.600,6.917)--(3.548,6.905)--(3.498,6.890)--(3.448,6.872)--(3.399,6.852)%
--(3.351,6.830)--(3.305,6.804)--(3.260,6.777)--(3.216,6.747)--(3.175,6.715)%
--(3.134,6.680)--(3.096,6.644)--(3.060,6.606)--(3.025,6.565)--(2.993,6.524)%
--(2.963,6.480)--(2.936,6.435)--(2.910,6.389)--(2.888,6.341)--(2.868,6.292)%
--(2.850,6.242)--(2.835,6.192)--(2.823,6.140)--(2.813,6.088)--(2.806,6.036)%
--(2.802,5.983)--(2.801,5.931)--(2.802,5.878)--(2.806,5.825)--(2.813,5.773)%
--(2.823,5.721)--(2.835,5.669)--(2.850,5.619)--(2.868,5.569)--(2.888,5.520)%
--(2.910,5.472)--(2.936,5.426)--(2.963,5.381)--(2.993,5.337)--(3.025,5.296)%
--(3.060,5.255)--(3.096,5.217)--(3.134,5.181)--(3.175,5.146)--(3.216,5.114)%
--(3.260,5.084)--(3.305,5.057)--(3.351,5.031)--(3.399,5.009)--(3.448,4.989)%
--(3.498,4.971)--(3.548,4.956)--(3.600,4.944)--(3.652,4.934)--(3.704,4.927)%
--(3.757,4.923)--(3.810,4.922)--(3.862,4.923)--(3.915,4.927)--(3.967,4.934)%
--(4.019,4.944)--(4.071,4.956)--(4.121,4.971)--(4.171,4.989)--(4.220,5.009)%
--(4.268,5.031)--(4.314,5.057)--(4.359,5.084)--(4.403,5.114)--(4.444,5.146)%
--(4.485,5.181)--(4.523,5.217)--(4.559,5.255)--(4.594,5.296)--(4.626,5.337)%
--(4.656,5.381)--(4.683,5.426)--(4.709,5.472)--(4.731,5.520)--(4.751,5.569)%
--(4.769,5.619)--(4.784,5.669)--(4.796,5.721)--(4.806,5.773)--(4.813,5.825)--(4.817,5.878)--cycle;
%
\gpfill{rgb color={0.000,0.000,0.000},opacity=0.15} (4.678,5.948)--(4.676,6.002)--(4.672,6.056)--(4.665,6.110)%
--(4.655,6.164)--(4.642,6.216)--(4.627,6.269)--(4.608,6.320)--(4.588,6.370)%
--(4.564,6.419)--(4.538,6.467)--(4.510,6.513)--(4.479,6.558)--(4.446,6.601)%
--(4.411,6.643)--(4.373,6.682)--(4.334,6.720)--(4.292,6.755)--(4.249,6.788)%
--(4.204,6.819)--(4.158,6.847)--(4.110,6.873)--(4.061,6.897)--(4.011,6.917)%
--(3.960,6.936)--(3.907,6.951)--(3.855,6.964)--(3.801,6.974)--(3.747,6.981)%
--(3.693,6.985)--(3.639,6.987)--(3.584,6.985)--(3.530,6.981)--(3.476,6.974)%
--(3.422,6.964)--(3.370,6.951)--(3.317,6.936)--(3.266,6.917)--(3.216,6.897)%
--(3.167,6.873)--(3.119,6.847)--(3.073,6.819)--(3.028,6.788)--(2.985,6.755)%
--(2.943,6.720)--(2.904,6.682)--(2.866,6.643)--(2.831,6.601)--(2.798,6.558)%
--(2.767,6.513)--(2.739,6.467)--(2.713,6.419)--(2.689,6.370)--(2.669,6.320)%
--(2.650,6.269)--(2.635,6.216)--(2.622,6.164)--(2.612,6.110)--(2.605,6.056)%
--(2.601,6.002)--(2.600,5.948)--(2.601,5.893)--(2.605,5.839)--(2.612,5.785)%
--(2.622,5.731)--(2.635,5.679)--(2.650,5.626)--(2.669,5.575)--(2.689,5.525)%
--(2.713,5.476)--(2.739,5.428)--(2.767,5.382)--(2.798,5.337)--(2.831,5.294)%
--(2.866,5.252)--(2.904,5.213)--(2.943,5.175)--(2.985,5.140)--(3.028,5.107)%
--(3.073,5.076)--(3.119,5.048)--(3.167,5.022)--(3.216,4.998)--(3.266,4.978)%
--(3.317,4.959)--(3.370,4.944)--(3.422,4.931)--(3.476,4.921)--(3.530,4.914)%
--(3.584,4.910)--(3.639,4.909)--(3.693,4.910)--(3.747,4.914)--(3.801,4.921)%
--(3.855,4.931)--(3.907,4.944)--(3.960,4.959)--(4.011,4.978)--(4.061,4.998)%
--(4.110,5.022)--(4.158,5.048)--(4.204,5.076)--(4.249,5.107)--(4.292,5.140)%
--(4.334,5.175)--(4.373,5.213)--(4.411,5.252)--(4.446,5.294)--(4.479,5.337)%
--(4.510,5.382)--(4.538,5.428)--(4.564,5.476)--(4.588,5.525)--(4.608,5.575)%
--(4.627,5.626)--(4.642,5.679)--(4.655,5.731)--(4.665,5.785)--(4.672,5.839)--(4.676,5.893)--cycle;
%
\gpfill{rgb color={0.000,0.000,0.000},opacity=0.15} (4.531,5.955)--(4.529,6.010)--(4.525,6.065)--(4.517,6.120)%
--(4.507,6.175)--(4.494,6.229)--(4.479,6.282)--(4.460,6.334)--(4.439,6.385)%
--(4.415,6.435)--(4.389,6.484)--(4.360,6.531)--(4.328,6.577)--(4.294,6.621)%
--(4.258,6.663)--(4.220,6.703)--(4.180,6.741)--(4.138,6.777)--(4.094,6.811)%
--(4.048,6.843)--(4.001,6.872)--(3.952,6.898)--(3.902,6.922)--(3.851,6.943)%
--(3.799,6.962)--(3.746,6.977)--(3.692,6.990)--(3.637,7.000)--(3.582,7.008)%
--(3.527,7.012)--(3.472,7.014)--(3.416,7.012)--(3.361,7.008)--(3.306,7.000)%
--(3.251,6.990)--(3.197,6.977)--(3.144,6.962)--(3.092,6.943)--(3.041,6.922)%
--(2.991,6.898)--(2.942,6.872)--(2.895,6.843)--(2.849,6.811)--(2.805,6.777)%
--(2.763,6.741)--(2.723,6.703)--(2.685,6.663)--(2.649,6.621)--(2.615,6.577)%
--(2.583,6.531)--(2.554,6.484)--(2.528,6.435)--(2.504,6.385)--(2.483,6.334)%
--(2.464,6.282)--(2.449,6.229)--(2.436,6.175)--(2.426,6.120)--(2.418,6.065)%
--(2.414,6.010)--(2.413,5.955)--(2.414,5.899)--(2.418,5.844)--(2.426,5.789)%
--(2.436,5.734)--(2.449,5.680)--(2.464,5.627)--(2.483,5.575)--(2.504,5.524)%
--(2.528,5.474)--(2.554,5.425)--(2.583,5.378)--(2.615,5.332)--(2.649,5.288)%
--(2.685,5.246)--(2.723,5.206)--(2.763,5.168)--(2.805,5.132)--(2.849,5.098)%
--(2.895,5.066)--(2.942,5.037)--(2.991,5.011)--(3.041,4.987)--(3.092,4.966)%
--(3.144,4.947)--(3.197,4.932)--(3.251,4.919)--(3.306,4.909)--(3.361,4.901)%
--(3.416,4.897)--(3.472,4.896)--(3.527,4.897)--(3.582,4.901)--(3.637,4.909)%
--(3.692,4.919)--(3.746,4.932)--(3.799,4.947)--(3.851,4.966)--(3.902,4.987)%
--(3.952,5.011)--(4.001,5.037)--(4.048,5.066)--(4.094,5.098)--(4.138,5.132)%
--(4.180,5.168)--(4.220,5.206)--(4.258,5.246)--(4.294,5.288)--(4.328,5.332)%
--(4.360,5.378)--(4.389,5.425)--(4.415,5.474)--(4.439,5.524)--(4.460,5.575)%
--(4.479,5.627)--(4.494,5.680)--(4.507,5.734)--(4.517,5.789)--(4.525,5.844)--(4.529,5.899)--cycle;
%
\gpfill{rgb color={0.000,0.000,0.000},opacity=0.15} (4.380,5.952)--(4.378,6.007)--(4.374,6.063)--(4.366,6.119)%
--(4.356,6.174)--(4.343,6.228)--(4.327,6.282)--(4.308,6.335)--(4.287,6.387)%
--(4.263,6.437)--(4.236,6.487)--(4.207,6.534)--(4.175,6.580)--(4.141,6.625)%
--(4.105,6.667)--(4.066,6.708)--(4.025,6.747)--(3.983,6.783)--(3.938,6.817)%
--(3.892,6.849)--(3.845,6.878)--(3.795,6.905)--(3.745,6.929)--(3.693,6.950)%
--(3.640,6.969)--(3.586,6.985)--(3.532,6.998)--(3.477,7.008)--(3.421,7.016)%
--(3.365,7.020)--(3.310,7.022)--(3.254,7.020)--(3.198,7.016)--(3.142,7.008)%
--(3.087,6.998)--(3.033,6.985)--(2.979,6.969)--(2.926,6.950)--(2.874,6.929)%
--(2.824,6.905)--(2.775,6.878)--(2.727,6.849)--(2.681,6.817)--(2.636,6.783)%
--(2.594,6.747)--(2.553,6.708)--(2.514,6.667)--(2.478,6.625)--(2.444,6.580)%
--(2.412,6.534)--(2.383,6.487)--(2.356,6.437)--(2.332,6.387)--(2.311,6.335)%
--(2.292,6.282)--(2.276,6.228)--(2.263,6.174)--(2.253,6.119)--(2.245,6.063)%
--(2.241,6.007)--(2.240,5.952)--(2.241,5.896)--(2.245,5.840)--(2.253,5.784)%
--(2.263,5.729)--(2.276,5.675)--(2.292,5.621)--(2.311,5.568)--(2.332,5.516)%
--(2.356,5.466)--(2.383,5.417)--(2.412,5.369)--(2.444,5.323)--(2.478,5.278)%
--(2.514,5.236)--(2.553,5.195)--(2.594,5.156)--(2.636,5.120)--(2.681,5.086)%
--(2.727,5.054)--(2.774,5.025)--(2.824,4.998)--(2.874,4.974)--(2.926,4.953)%
--(2.979,4.934)--(3.033,4.918)--(3.087,4.905)--(3.142,4.895)--(3.198,4.887)%
--(3.254,4.883)--(3.310,4.882)--(3.365,4.883)--(3.421,4.887)--(3.477,4.895)%
--(3.532,4.905)--(3.586,4.918)--(3.640,4.934)--(3.693,4.953)--(3.745,4.974)%
--(3.795,4.998)--(3.845,5.025)--(3.892,5.054)--(3.938,5.086)--(3.983,5.120)%
--(4.025,5.156)--(4.066,5.195)--(4.105,5.236)--(4.141,5.278)--(4.175,5.323)%
--(4.207,5.369)--(4.236,5.417)--(4.263,5.466)--(4.287,5.516)--(4.308,5.568)%
--(4.327,5.621)--(4.343,5.675)--(4.356,5.729)--(4.366,5.784)--(4.374,5.840)--(4.378,5.896)--cycle;
%
\gpfill{rgb color={0.000,0.000,0.000},opacity=0.15} (4.224,5.938)--(4.222,5.994)--(4.218,6.049)--(4.210,6.105)%
--(4.200,6.160)--(4.187,6.215)--(4.171,6.268)--(4.152,6.321)--(4.131,6.373)%
--(4.107,6.424)--(4.080,6.473)--(4.051,6.521)--(4.019,6.567)--(3.985,6.612)%
--(3.948,6.654)--(3.910,6.695)--(3.869,6.733)--(3.827,6.770)--(3.782,6.804)%
--(3.736,6.836)--(3.688,6.865)--(3.639,6.892)--(3.588,6.916)--(3.536,6.937)%
--(3.483,6.956)--(3.430,6.972)--(3.375,6.985)--(3.320,6.995)--(3.264,7.003)%
--(3.209,7.007)--(3.153,7.009)--(3.096,7.007)--(3.041,7.003)--(2.985,6.995)%
--(2.930,6.985)--(2.875,6.972)--(2.822,6.956)--(2.769,6.937)--(2.717,6.916)%
--(2.666,6.892)--(2.617,6.865)--(2.569,6.836)--(2.523,6.804)--(2.478,6.770)%
--(2.436,6.733)--(2.395,6.695)--(2.357,6.654)--(2.320,6.612)--(2.286,6.567)%
--(2.254,6.521)--(2.225,6.473)--(2.198,6.424)--(2.174,6.373)--(2.153,6.321)%
--(2.134,6.268)--(2.118,6.215)--(2.105,6.160)--(2.095,6.105)--(2.087,6.049)%
--(2.083,5.994)--(2.082,5.938)--(2.083,5.881)--(2.087,5.826)--(2.095,5.770)%
--(2.105,5.715)--(2.118,5.660)--(2.134,5.607)--(2.153,5.554)--(2.174,5.502)%
--(2.198,5.451)--(2.225,5.402)--(2.254,5.354)--(2.286,5.308)--(2.320,5.263)%
--(2.357,5.221)--(2.395,5.180)--(2.436,5.142)--(2.478,5.105)--(2.523,5.071)%
--(2.569,5.039)--(2.617,5.010)--(2.666,4.983)--(2.717,4.959)--(2.769,4.938)%
--(2.822,4.919)--(2.875,4.903)--(2.930,4.890)--(2.985,4.880)--(3.041,4.872)%
--(3.096,4.868)--(3.153,4.867)--(3.209,4.868)--(3.264,4.872)--(3.320,4.880)%
--(3.375,4.890)--(3.430,4.903)--(3.483,4.919)--(3.536,4.938)--(3.588,4.959)%
--(3.639,4.983)--(3.688,5.010)--(3.736,5.039)--(3.782,5.071)--(3.827,5.105)%
--(3.869,5.142)--(3.910,5.180)--(3.948,5.221)--(3.985,5.263)--(4.019,5.308)%
--(4.051,5.354)--(4.080,5.402)--(4.107,5.451)--(4.131,5.502)--(4.152,5.554)%
--(4.171,5.607)--(4.187,5.660)--(4.200,5.715)--(4.210,5.770)--(4.218,5.826)--(4.222,5.881)--cycle;
%
\gpfill{rgb color={0.000,0.000,0.000},opacity=0.15} (4.064,5.915)--(4.062,5.970)--(4.058,6.026)--(4.050,6.081)%
--(4.040,6.136)--(4.027,6.190)--(4.011,6.243)--(3.993,6.295)--(3.972,6.347)%
--(3.948,6.397)--(3.921,6.446)--(3.892,6.493)--(3.860,6.539)--(3.827,6.583)%
--(3.790,6.626)--(3.752,6.666)--(3.712,6.704)--(3.669,6.741)--(3.625,6.774)%
--(3.579,6.806)--(3.532,6.835)--(3.483,6.862)--(3.433,6.886)--(3.381,6.907)%
--(3.329,6.925)--(3.276,6.941)--(3.222,6.954)--(3.167,6.964)--(3.112,6.972)%
--(3.056,6.976)--(3.001,6.978)--(2.945,6.976)--(2.889,6.972)--(2.834,6.964)%
--(2.779,6.954)--(2.725,6.941)--(2.672,6.925)--(2.620,6.907)--(2.568,6.886)%
--(2.518,6.862)--(2.469,6.835)--(2.422,6.806)--(2.376,6.774)--(2.332,6.741)%
--(2.289,6.704)--(2.249,6.666)--(2.211,6.626)--(2.174,6.583)--(2.141,6.539)%
--(2.109,6.493)--(2.080,6.446)--(2.053,6.397)--(2.029,6.347)--(2.008,6.295)%
--(1.990,6.243)--(1.974,6.190)--(1.961,6.136)--(1.951,6.081)--(1.943,6.026)%
--(1.939,5.970)--(1.938,5.915)--(1.939,5.859)--(1.943,5.803)--(1.951,5.748)%
--(1.961,5.693)--(1.974,5.639)--(1.990,5.586)--(2.008,5.534)--(2.029,5.482)%
--(2.053,5.432)--(2.080,5.383)--(2.109,5.336)--(2.141,5.290)--(2.174,5.246)%
--(2.211,5.203)--(2.249,5.163)--(2.289,5.125)--(2.332,5.088)--(2.376,5.055)%
--(2.422,5.023)--(2.469,4.994)--(2.518,4.967)--(2.568,4.943)--(2.620,4.922)%
--(2.672,4.904)--(2.725,4.888)--(2.779,4.875)--(2.834,4.865)--(2.889,4.857)%
--(2.945,4.853)--(3.001,4.852)--(3.056,4.853)--(3.112,4.857)--(3.167,4.865)%
--(3.222,4.875)--(3.276,4.888)--(3.329,4.904)--(3.381,4.922)--(3.433,4.943)%
--(3.483,4.967)--(3.532,4.994)--(3.579,5.023)--(3.625,5.055)--(3.669,5.088)%
--(3.712,5.125)--(3.752,5.163)--(3.790,5.203)--(3.827,5.246)--(3.860,5.290)%
--(3.892,5.336)--(3.921,5.383)--(3.948,5.432)--(3.972,5.482)--(3.993,5.534)%
--(4.011,5.586)--(4.027,5.639)--(4.040,5.693)--(4.050,5.748)--(4.058,5.803)--(4.062,5.859)--cycle;
%
\gpfill{rgb color={0.000,0.000,0.000},opacity=0.15} (3.902,5.882)--(3.900,5.936)--(3.896,5.991)--(3.889,6.045)%
--(3.879,6.099)--(3.866,6.152)--(3.850,6.205)--(3.832,6.257)--(3.811,6.307)%
--(3.787,6.357)--(3.761,6.405)--(3.733,6.452)--(3.702,6.497)--(3.668,6.540)%
--(3.633,6.582)--(3.595,6.622)--(3.555,6.660)--(3.513,6.695)--(3.470,6.729)%
--(3.425,6.760)--(3.378,6.788)--(3.330,6.814)--(3.280,6.838)--(3.230,6.859)%
--(3.178,6.877)--(3.125,6.893)--(3.072,6.906)--(3.018,6.916)--(2.964,6.923)%
--(2.909,6.927)--(2.855,6.929)--(2.800,6.927)--(2.745,6.923)--(2.691,6.916)%
--(2.637,6.906)--(2.584,6.893)--(2.531,6.877)--(2.479,6.859)--(2.429,6.838)%
--(2.379,6.814)--(2.331,6.788)--(2.284,6.760)--(2.239,6.729)--(2.196,6.695)%
--(2.154,6.660)--(2.114,6.622)--(2.076,6.582)--(2.041,6.540)--(2.007,6.497)%
--(1.976,6.452)--(1.948,6.405)--(1.922,6.357)--(1.898,6.307)--(1.877,6.257)%
--(1.859,6.205)--(1.843,6.152)--(1.830,6.099)--(1.820,6.045)--(1.813,5.991)%
--(1.809,5.936)--(1.808,5.882)--(1.809,5.827)--(1.813,5.772)--(1.820,5.718)%
--(1.830,5.664)--(1.843,5.611)--(1.859,5.558)--(1.877,5.506)--(1.898,5.456)%
--(1.922,5.406)--(1.948,5.358)--(1.976,5.311)--(2.007,5.266)--(2.041,5.223)%
--(2.076,5.181)--(2.114,5.141)--(2.154,5.103)--(2.196,5.068)--(2.239,5.034)%
--(2.284,5.003)--(2.331,4.975)--(2.379,4.949)--(2.429,4.925)--(2.479,4.904)%
--(2.531,4.886)--(2.584,4.870)--(2.637,4.857)--(2.691,4.847)--(2.745,4.840)%
--(2.800,4.836)--(2.855,4.835)--(2.909,4.836)--(2.964,4.840)--(3.018,4.847)%
--(3.072,4.857)--(3.125,4.870)--(3.178,4.886)--(3.230,4.904)--(3.280,4.925)%
--(3.330,4.949)--(3.378,4.975)--(3.425,5.003)--(3.470,5.034)--(3.513,5.068)%
--(3.555,5.103)--(3.595,5.141)--(3.633,5.181)--(3.668,5.223)--(3.702,5.266)%
--(3.733,5.311)--(3.761,5.358)--(3.787,5.406)--(3.811,5.456)--(3.832,5.506)%
--(3.850,5.558)--(3.866,5.611)--(3.879,5.664)--(3.889,5.718)--(3.896,5.772)--(3.900,5.827)--cycle;
%
\gpfill{rgb color={0.000,0.000,0.000},opacity=0.15} (3.737,5.840)--(3.735,5.893)--(3.731,5.946)--(3.724,6.000)%
--(3.714,6.052)--(3.702,6.104)--(3.686,6.156)--(3.669,6.206)--(3.648,6.256)%
--(3.625,6.304)--(3.599,6.351)--(3.571,6.397)--(3.541,6.441)--(3.509,6.483)%
--(3.474,6.524)--(3.437,6.563)--(3.398,6.600)--(3.357,6.635)--(3.315,6.667)%
--(3.271,6.697)--(3.225,6.725)--(3.178,6.751)--(3.130,6.774)--(3.080,6.795)%
--(3.030,6.812)--(2.978,6.828)--(2.926,6.840)--(2.874,6.850)--(2.820,6.857)%
--(2.767,6.861)--(2.714,6.863)--(2.660,6.861)--(2.607,6.857)--(2.553,6.850)%
--(2.501,6.840)--(2.449,6.828)--(2.397,6.812)--(2.347,6.795)--(2.297,6.774)%
--(2.249,6.751)--(2.202,6.725)--(2.156,6.697)--(2.112,6.667)--(2.070,6.635)%
--(2.029,6.600)--(1.990,6.563)--(1.953,6.524)--(1.918,6.483)--(1.886,6.441)%
--(1.856,6.397)--(1.828,6.351)--(1.802,6.304)--(1.779,6.256)--(1.758,6.206)%
--(1.741,6.156)--(1.725,6.104)--(1.713,6.052)--(1.703,6.000)--(1.696,5.946)%
--(1.692,5.893)--(1.691,5.840)--(1.692,5.786)--(1.696,5.733)--(1.703,5.679)%
--(1.713,5.627)--(1.725,5.575)--(1.741,5.523)--(1.758,5.473)--(1.779,5.423)%
--(1.802,5.375)--(1.828,5.328)--(1.856,5.282)--(1.886,5.238)--(1.918,5.196)%
--(1.953,5.155)--(1.990,5.116)--(2.029,5.079)--(2.070,5.044)--(2.112,5.012)%
--(2.156,4.982)--(2.202,4.954)--(2.249,4.928)--(2.297,4.905)--(2.347,4.884)%
--(2.397,4.867)--(2.449,4.851)--(2.501,4.839)--(2.553,4.829)--(2.607,4.822)%
--(2.660,4.818)--(2.714,4.817)--(2.767,4.818)--(2.820,4.822)--(2.874,4.829)%
--(2.926,4.839)--(2.978,4.851)--(3.030,4.867)--(3.080,4.884)--(3.130,4.905)%
--(3.178,4.928)--(3.225,4.954)--(3.271,4.982)--(3.315,5.012)--(3.357,5.044)%
--(3.398,5.079)--(3.437,5.116)--(3.474,5.155)--(3.509,5.196)--(3.541,5.238)%
--(3.571,5.282)--(3.599,5.328)--(3.625,5.375)--(3.648,5.423)--(3.669,5.473)%
--(3.686,5.523)--(3.702,5.575)--(3.714,5.627)--(3.724,5.679)--(3.731,5.733)--(3.735,5.786)--cycle;
%
\gpfill{rgb color={0.000,0.000,0.000},opacity=0.15} (3.573,5.790)--(3.571,5.841)--(3.567,5.893)--(3.560,5.945)%
--(3.551,5.996)--(3.539,6.047)--(3.524,6.096)--(3.507,6.145)--(3.487,6.193)%
--(3.464,6.240)--(3.439,6.286)--(3.412,6.330)--(3.383,6.373)--(3.351,6.414)%
--(3.317,6.454)--(3.282,6.492)--(3.244,6.527)--(3.204,6.561)--(3.163,6.593)%
--(3.120,6.622)--(3.076,6.649)--(3.030,6.674)--(2.983,6.697)--(2.935,6.717)%
--(2.886,6.734)--(2.837,6.749)--(2.786,6.761)--(2.735,6.770)--(2.683,6.777)%
--(2.631,6.781)--(2.580,6.783)--(2.528,6.781)--(2.476,6.777)--(2.424,6.770)%
--(2.373,6.761)--(2.322,6.749)--(2.273,6.734)--(2.224,6.717)--(2.176,6.697)%
--(2.129,6.674)--(2.083,6.649)--(2.039,6.622)--(1.996,6.593)--(1.955,6.561)%
--(1.915,6.527)--(1.877,6.492)--(1.842,6.454)--(1.808,6.414)--(1.776,6.373)%
--(1.747,6.330)--(1.720,6.286)--(1.695,6.240)--(1.672,6.193)--(1.652,6.145)%
--(1.635,6.096)--(1.620,6.047)--(1.608,5.996)--(1.599,5.945)--(1.592,5.893)%
--(1.588,5.841)--(1.587,5.790)--(1.588,5.738)--(1.592,5.686)--(1.599,5.634)%
--(1.608,5.583)--(1.620,5.532)--(1.635,5.483)--(1.652,5.434)--(1.672,5.386)%
--(1.695,5.339)--(1.720,5.293)--(1.747,5.249)--(1.776,5.206)--(1.808,5.165)%
--(1.842,5.125)--(1.877,5.087)--(1.915,5.052)--(1.955,5.018)--(1.996,4.986)%
--(2.039,4.957)--(2.083,4.930)--(2.129,4.905)--(2.176,4.882)--(2.224,4.862)%
--(2.273,4.845)--(2.322,4.830)--(2.373,4.818)--(2.424,4.809)--(2.476,4.802)%
--(2.528,4.798)--(2.580,4.797)--(2.631,4.798)--(2.683,4.802)--(2.735,4.809)%
--(2.786,4.818)--(2.837,4.830)--(2.886,4.845)--(2.935,4.862)--(2.983,4.882)%
--(3.030,4.905)--(3.076,4.930)--(3.120,4.957)--(3.163,4.986)--(3.204,5.018)%
--(3.244,5.052)--(3.282,5.087)--(3.317,5.125)--(3.351,5.165)--(3.383,5.206)%
--(3.412,5.249)--(3.439,5.293)--(3.464,5.339)--(3.487,5.386)--(3.507,5.434)%
--(3.524,5.483)--(3.539,5.532)--(3.551,5.583)--(3.560,5.634)--(3.567,5.686)--(3.571,5.738)--cycle;
%
\gpfill{rgb color={0.000,0.000,0.000},opacity=0.15} (3.409,5.732)--(3.407,5.782)--(3.403,5.831)--(3.397,5.881)%
--(3.388,5.930)--(3.376,5.979)--(3.362,6.027)--(3.345,6.074)--(3.326,6.120)%
--(3.304,6.166)--(3.280,6.210)--(3.254,6.252)--(3.226,6.293)--(3.195,6.333)%
--(3.163,6.371)--(3.128,6.407)--(3.092,6.442)--(3.054,6.474)--(3.014,6.505)%
--(2.973,6.533)--(2.931,6.559)--(2.887,6.583)--(2.841,6.605)--(2.795,6.624)%
--(2.748,6.641)--(2.700,6.655)--(2.651,6.667)--(2.602,6.676)--(2.552,6.682)%
--(2.503,6.686)--(2.453,6.688)--(2.402,6.686)--(2.353,6.682)--(2.303,6.676)%
--(2.254,6.667)--(2.205,6.655)--(2.157,6.641)--(2.110,6.624)--(2.064,6.605)%
--(2.018,6.583)--(1.975,6.559)--(1.932,6.533)--(1.891,6.505)--(1.851,6.474)%
--(1.813,6.442)--(1.777,6.407)--(1.742,6.371)--(1.710,6.333)--(1.679,6.293)%
--(1.651,6.252)--(1.625,6.210)--(1.601,6.166)--(1.579,6.120)--(1.560,6.074)%
--(1.543,6.027)--(1.529,5.979)--(1.517,5.930)--(1.508,5.881)--(1.502,5.831)%
--(1.498,5.782)--(1.497,5.732)--(1.498,5.681)--(1.502,5.632)--(1.508,5.582)%
--(1.517,5.533)--(1.529,5.484)--(1.543,5.436)--(1.560,5.389)--(1.579,5.343)%
--(1.601,5.297)--(1.625,5.254)--(1.651,5.211)--(1.679,5.170)--(1.710,5.130)%
--(1.742,5.092)--(1.777,5.056)--(1.813,5.021)--(1.851,4.989)--(1.891,4.958)%
--(1.932,4.930)--(1.974,4.904)--(2.018,4.880)--(2.064,4.858)--(2.110,4.839)%
--(2.157,4.822)--(2.205,4.808)--(2.254,4.796)--(2.303,4.787)--(2.353,4.781)%
--(2.402,4.777)--(2.453,4.776)--(2.503,4.777)--(2.552,4.781)--(2.602,4.787)%
--(2.651,4.796)--(2.700,4.808)--(2.748,4.822)--(2.795,4.839)--(2.841,4.858)%
--(2.887,4.880)--(2.931,4.904)--(2.973,4.930)--(3.014,4.958)--(3.054,4.989)%
--(3.092,5.021)--(3.128,5.056)--(3.163,5.092)--(3.195,5.130)--(3.226,5.170)%
--(3.254,5.211)--(3.280,5.254)--(3.304,5.297)--(3.326,5.343)--(3.345,5.389)%
--(3.362,5.436)--(3.376,5.484)--(3.388,5.533)--(3.397,5.582)--(3.403,5.632)--(3.407,5.681)--cycle;
%
\gpfill{rgb color={0.000,0.000,0.000},opacity=0.15} (3.243,5.668)--(3.241,5.715)--(3.238,5.763)--(3.231,5.810)%
--(3.223,5.857)--(3.211,5.904)--(3.198,5.949)--(3.182,5.994)--(3.164,6.038)%
--(3.143,6.082)--(3.120,6.124)--(3.095,6.164)--(3.068,6.204)--(3.039,6.241)%
--(3.008,6.278)--(2.975,6.312)--(2.941,6.345)--(2.904,6.376)--(2.867,6.405)%
--(2.827,6.432)--(2.787,6.457)--(2.745,6.480)--(2.701,6.501)--(2.657,6.519)%
--(2.612,6.535)--(2.567,6.548)--(2.520,6.560)--(2.473,6.568)--(2.426,6.575)%
--(2.378,6.578)--(2.331,6.580)--(2.283,6.578)--(2.235,6.575)--(2.188,6.568)%
--(2.141,6.560)--(2.094,6.548)--(2.049,6.535)--(2.004,6.519)--(1.960,6.501)%
--(1.916,6.480)--(1.875,6.457)--(1.834,6.432)--(1.794,6.405)--(1.757,6.376)%
--(1.720,6.345)--(1.686,6.312)--(1.653,6.278)--(1.622,6.241)--(1.593,6.204)%
--(1.566,6.164)--(1.541,6.124)--(1.518,6.082)--(1.497,6.038)--(1.479,5.994)%
--(1.463,5.949)--(1.450,5.904)--(1.438,5.857)--(1.430,5.810)--(1.423,5.763)%
--(1.420,5.715)--(1.419,5.668)--(1.420,5.620)--(1.423,5.572)--(1.430,5.525)%
--(1.438,5.478)--(1.450,5.431)--(1.463,5.386)--(1.479,5.341)--(1.497,5.297)%
--(1.518,5.253)--(1.541,5.212)--(1.566,5.171)--(1.593,5.131)--(1.622,5.094)%
--(1.653,5.057)--(1.686,5.023)--(1.720,4.990)--(1.757,4.959)--(1.794,4.930)%
--(1.834,4.903)--(1.874,4.878)--(1.916,4.855)--(1.960,4.834)--(2.004,4.816)%
--(2.049,4.800)--(2.094,4.787)--(2.141,4.775)--(2.188,4.767)--(2.235,4.760)%
--(2.283,4.757)--(2.331,4.756)--(2.378,4.757)--(2.426,4.760)--(2.473,4.767)%
--(2.520,4.775)--(2.567,4.787)--(2.612,4.800)--(2.657,4.816)--(2.701,4.834)%
--(2.745,4.855)--(2.787,4.878)--(2.827,4.903)--(2.867,4.930)--(2.904,4.959)%
--(2.941,4.990)--(2.975,5.023)--(3.008,5.057)--(3.039,5.094)--(3.068,5.131)%
--(3.095,5.171)--(3.120,5.212)--(3.143,5.253)--(3.164,5.297)--(3.182,5.341)%
--(3.198,5.386)--(3.211,5.431)--(3.223,5.478)--(3.231,5.525)--(3.238,5.572)--(3.241,5.620)--cycle;
%
\gpfill{rgb color={0.000,0.000,0.000},opacity=0.15} (3.081,5.598)--(3.079,5.643)--(3.076,5.688)--(3.070,5.733)%
--(3.062,5.777)--(3.051,5.821)--(3.038,5.864)--(3.023,5.907)--(3.006,5.949)%
--(2.986,5.990)--(2.965,6.030)--(2.941,6.068)--(2.915,6.105)--(2.888,6.141)%
--(2.859,6.176)--(2.827,6.208)--(2.795,6.240)--(2.760,6.269)--(2.724,6.296)%
--(2.687,6.322)--(2.649,6.346)--(2.609,6.367)--(2.568,6.387)--(2.526,6.404)%
--(2.483,6.419)--(2.440,6.432)--(2.396,6.443)--(2.352,6.451)--(2.307,6.457)%
--(2.262,6.460)--(2.217,6.462)--(2.171,6.460)--(2.126,6.457)--(2.081,6.451)%
--(2.037,6.443)--(1.993,6.432)--(1.950,6.419)--(1.907,6.404)--(1.865,6.387)%
--(1.824,6.367)--(1.785,6.346)--(1.746,6.322)--(1.709,6.296)--(1.673,6.269)%
--(1.638,6.240)--(1.606,6.208)--(1.574,6.176)--(1.545,6.141)--(1.518,6.105)%
--(1.492,6.068)--(1.468,6.030)--(1.447,5.990)--(1.427,5.949)--(1.410,5.907)%
--(1.395,5.864)--(1.382,5.821)--(1.371,5.777)--(1.363,5.733)--(1.357,5.688)%
--(1.354,5.643)--(1.353,5.598)--(1.354,5.552)--(1.357,5.507)--(1.363,5.462)%
--(1.371,5.418)--(1.382,5.374)--(1.395,5.331)--(1.410,5.288)--(1.427,5.246)%
--(1.447,5.205)--(1.468,5.166)--(1.492,5.127)--(1.518,5.090)--(1.545,5.054)%
--(1.574,5.019)--(1.606,4.987)--(1.638,4.955)--(1.673,4.926)--(1.709,4.899)%
--(1.746,4.873)--(1.784,4.849)--(1.824,4.828)--(1.865,4.808)--(1.907,4.791)%
--(1.950,4.776)--(1.993,4.763)--(2.037,4.752)--(2.081,4.744)--(2.126,4.738)%
--(2.171,4.735)--(2.217,4.734)--(2.262,4.735)--(2.307,4.738)--(2.352,4.744)%
--(2.396,4.752)--(2.440,4.763)--(2.483,4.776)--(2.526,4.791)--(2.568,4.808)%
--(2.609,4.828)--(2.649,4.849)--(2.687,4.873)--(2.724,4.899)--(2.760,4.926)%
--(2.795,4.955)--(2.827,4.987)--(2.859,5.019)--(2.888,5.054)--(2.915,5.090)%
--(2.941,5.127)--(2.965,5.166)--(2.986,5.205)--(3.006,5.246)--(3.023,5.288)%
--(3.038,5.331)--(3.051,5.374)--(3.062,5.418)--(3.070,5.462)--(3.076,5.507)--(3.079,5.552)--cycle;
%
\gpfill{rgb color={0.000,0.000,0.000},opacity=0.15} (2.922,5.523)--(2.920,5.565)--(2.917,5.607)--(2.912,5.650)%
--(2.904,5.691)--(2.894,5.733)--(2.882,5.773)--(2.868,5.813)--(2.851,5.853)%
--(2.833,5.891)--(2.813,5.929)--(2.791,5.965)--(2.766,6.000)--(2.741,6.034)%
--(2.713,6.066)--(2.684,6.097)--(2.653,6.126)--(2.621,6.154)--(2.587,6.179)%
--(2.552,6.204)--(2.516,6.226)--(2.478,6.246)--(2.440,6.264)--(2.400,6.281)%
--(2.360,6.295)--(2.320,6.307)--(2.278,6.317)--(2.237,6.325)--(2.194,6.330)%
--(2.152,6.333)--(2.110,6.335)--(2.067,6.333)--(2.025,6.330)--(1.982,6.325)%
--(1.941,6.317)--(1.899,6.307)--(1.859,6.295)--(1.819,6.281)--(1.779,6.264)%
--(1.741,6.246)--(1.704,6.226)--(1.667,6.204)--(1.632,6.179)--(1.598,6.154)%
--(1.566,6.126)--(1.535,6.097)--(1.506,6.066)--(1.478,6.034)--(1.453,6.000)%
--(1.428,5.965)--(1.406,5.929)--(1.386,5.891)--(1.368,5.853)--(1.351,5.813)%
--(1.337,5.773)--(1.325,5.733)--(1.315,5.691)--(1.307,5.650)--(1.302,5.607)%
--(1.299,5.565)--(1.298,5.523)--(1.299,5.480)--(1.302,5.438)--(1.307,5.395)%
--(1.315,5.354)--(1.325,5.312)--(1.337,5.272)--(1.351,5.232)--(1.368,5.192)%
--(1.386,5.154)--(1.406,5.117)--(1.428,5.080)--(1.453,5.045)--(1.478,5.011)%
--(1.506,4.979)--(1.535,4.948)--(1.566,4.919)--(1.598,4.891)--(1.632,4.866)%
--(1.667,4.841)--(1.703,4.819)--(1.741,4.799)--(1.779,4.781)--(1.819,4.764)%
--(1.859,4.750)--(1.899,4.738)--(1.941,4.728)--(1.982,4.720)--(2.025,4.715)%
--(2.067,4.712)--(2.110,4.711)--(2.152,4.712)--(2.194,4.715)--(2.237,4.720)%
--(2.278,4.728)--(2.320,4.738)--(2.360,4.750)--(2.400,4.764)--(2.440,4.781)%
--(2.478,4.799)--(2.516,4.819)--(2.552,4.841)--(2.587,4.866)--(2.621,4.891)%
--(2.653,4.919)--(2.684,4.948)--(2.713,4.979)--(2.741,5.011)--(2.766,5.045)%
--(2.791,5.080)--(2.813,5.117)--(2.833,5.154)--(2.851,5.192)--(2.868,5.232)%
--(2.882,5.272)--(2.894,5.312)--(2.904,5.354)--(2.912,5.395)--(2.917,5.438)--(2.920,5.480)--cycle;
%
\gpfill{rgb color={0.000,0.000,0.000},opacity=0.15} (2.762,5.443)--(2.760,5.482)--(2.757,5.521)--(2.752,5.560)%
--(2.745,5.599)--(2.736,5.637)--(2.725,5.675)--(2.711,5.712)--(2.696,5.749)%
--(2.679,5.784)--(2.661,5.819)--(2.640,5.853)--(2.618,5.885)--(2.594,5.916)%
--(2.568,5.946)--(2.541,5.975)--(2.512,6.002)--(2.482,6.028)--(2.451,6.052)%
--(2.419,6.074)--(2.385,6.095)--(2.350,6.113)--(2.315,6.130)--(2.278,6.145)%
--(2.241,6.159)--(2.203,6.170)--(2.165,6.179)--(2.126,6.186)--(2.087,6.191)%
--(2.048,6.194)--(2.009,6.196)--(1.969,6.194)--(1.930,6.191)--(1.891,6.186)%
--(1.852,6.179)--(1.814,6.170)--(1.776,6.159)--(1.739,6.145)--(1.702,6.130)%
--(1.667,6.113)--(1.632,6.095)--(1.598,6.074)--(1.566,6.052)--(1.535,6.028)%
--(1.505,6.002)--(1.476,5.975)--(1.449,5.946)--(1.423,5.916)--(1.399,5.885)%
--(1.377,5.853)--(1.356,5.819)--(1.338,5.784)--(1.321,5.749)--(1.306,5.712)%
--(1.292,5.675)--(1.281,5.637)--(1.272,5.599)--(1.265,5.560)--(1.260,5.521)%
--(1.257,5.482)--(1.256,5.443)--(1.257,5.403)--(1.260,5.364)--(1.265,5.325)%
--(1.272,5.286)--(1.281,5.248)--(1.292,5.210)--(1.306,5.173)--(1.321,5.136)%
--(1.338,5.101)--(1.356,5.066)--(1.377,5.032)--(1.399,5.000)--(1.423,4.969)%
--(1.449,4.939)--(1.476,4.910)--(1.505,4.883)--(1.535,4.857)--(1.566,4.833)%
--(1.598,4.811)--(1.632,4.790)--(1.667,4.772)--(1.702,4.755)--(1.739,4.740)%
--(1.776,4.726)--(1.814,4.715)--(1.852,4.706)--(1.891,4.699)--(1.930,4.694)%
--(1.969,4.691)--(2.008,4.690)--(2.048,4.691)--(2.087,4.694)--(2.126,4.699)%
--(2.165,4.706)--(2.203,4.715)--(2.241,4.726)--(2.278,4.740)--(2.315,4.755)%
--(2.350,4.772)--(2.385,4.790)--(2.419,4.811)--(2.451,4.833)--(2.482,4.857)%
--(2.512,4.883)--(2.541,4.910)--(2.568,4.939)--(2.594,4.969)--(2.618,5.000)%
--(2.640,5.032)--(2.661,5.066)--(2.679,5.101)--(2.696,5.136)--(2.711,5.173)%
--(2.725,5.210)--(2.736,5.248)--(2.745,5.286)--(2.752,5.325)--(2.757,5.364)--(2.760,5.403)--cycle;
%
\gpfill{rgb color={0.000,0.000,0.000},opacity=0.15} (2.608,5.361)--(2.607,5.397)--(2.604,5.433)--(2.599,5.469)%
--(2.592,5.504)--(2.584,5.540)--(2.574,5.574)--(2.562,5.608)--(2.548,5.642)%
--(2.532,5.675)--(2.515,5.707)--(2.496,5.737)--(2.475,5.767)--(2.453,5.796)%
--(2.430,5.824)--(2.405,5.850)--(2.379,5.875)--(2.351,5.898)--(2.322,5.920)%
--(2.292,5.941)--(2.262,5.960)--(2.230,5.977)--(2.197,5.993)--(2.163,6.007)%
--(2.129,6.019)--(2.095,6.029)--(2.059,6.037)--(2.024,6.044)--(1.988,6.049)%
--(1.952,6.052)--(1.916,6.053)--(1.879,6.052)--(1.843,6.049)--(1.807,6.044)%
--(1.772,6.037)--(1.736,6.029)--(1.702,6.019)--(1.668,6.007)--(1.634,5.993)%
--(1.601,5.977)--(1.570,5.960)--(1.539,5.941)--(1.509,5.920)--(1.480,5.898)%
--(1.452,5.875)--(1.426,5.850)--(1.401,5.824)--(1.378,5.796)--(1.356,5.767)%
--(1.335,5.737)--(1.316,5.707)--(1.299,5.675)--(1.283,5.642)--(1.269,5.608)%
--(1.257,5.574)--(1.247,5.540)--(1.239,5.504)--(1.232,5.469)--(1.227,5.433)%
--(1.224,5.397)--(1.224,5.361)--(1.224,5.324)--(1.227,5.288)--(1.232,5.252)%
--(1.239,5.217)--(1.247,5.181)--(1.257,5.147)--(1.269,5.113)--(1.283,5.079)%
--(1.299,5.046)--(1.316,5.015)--(1.335,4.984)--(1.356,4.954)--(1.378,4.925)%
--(1.401,4.897)--(1.426,4.871)--(1.452,4.846)--(1.480,4.823)--(1.509,4.801)%
--(1.539,4.780)--(1.569,4.761)--(1.601,4.744)--(1.634,4.728)--(1.668,4.714)%
--(1.702,4.702)--(1.736,4.692)--(1.772,4.684)--(1.807,4.677)--(1.843,4.672)%
--(1.879,4.669)--(1.915,4.669)--(1.952,4.669)--(1.988,4.672)--(2.024,4.677)%
--(2.059,4.684)--(2.095,4.692)--(2.129,4.702)--(2.163,4.714)--(2.197,4.728)%
--(2.230,4.744)--(2.262,4.761)--(2.292,4.780)--(2.322,4.801)--(2.351,4.823)%
--(2.379,4.846)--(2.405,4.871)--(2.430,4.897)--(2.453,4.925)--(2.475,4.954)%
--(2.496,4.984)--(2.515,5.015)--(2.532,5.046)--(2.548,5.079)--(2.562,5.113)%
--(2.574,5.147)--(2.584,5.181)--(2.592,5.217)--(2.599,5.252)--(2.604,5.288)--(2.607,5.324)--cycle;
%
\gpfill{rgb color={0.000,0.000,0.000},opacity=0.15} (2.460,5.275)--(2.459,5.307)--(2.456,5.340)--(2.452,5.373)%
--(2.446,5.405)--(2.438,5.437)--(2.429,5.469)--(2.418,5.500)--(2.405,5.530)%
--(2.391,5.560)--(2.375,5.589)--(2.358,5.617)--(2.339,5.644)--(2.319,5.670)%
--(2.298,5.695)--(2.275,5.719)--(2.251,5.742)--(2.226,5.763)--(2.200,5.783)%
--(2.173,5.802)--(2.145,5.819)--(2.116,5.835)--(2.086,5.849)--(2.056,5.862)%
--(2.025,5.873)--(1.993,5.882)--(1.961,5.890)--(1.929,5.896)--(1.896,5.900)%
--(1.863,5.903)--(1.831,5.904)--(1.798,5.903)--(1.765,5.900)--(1.732,5.896)%
--(1.700,5.890)--(1.668,5.882)--(1.636,5.873)--(1.605,5.862)--(1.575,5.849)%
--(1.545,5.835)--(1.516,5.819)--(1.488,5.802)--(1.461,5.783)--(1.435,5.763)%
--(1.410,5.742)--(1.386,5.719)--(1.363,5.695)--(1.342,5.670)--(1.322,5.644)%
--(1.303,5.617)--(1.286,5.589)--(1.270,5.560)--(1.256,5.530)--(1.243,5.500)%
--(1.232,5.469)--(1.223,5.437)--(1.215,5.405)--(1.209,5.373)--(1.205,5.340)%
--(1.202,5.307)--(1.202,5.275)--(1.202,5.242)--(1.205,5.209)--(1.209,5.176)%
--(1.215,5.144)--(1.223,5.112)--(1.232,5.080)--(1.243,5.049)--(1.256,5.019)%
--(1.270,4.989)--(1.286,4.960)--(1.303,4.932)--(1.322,4.905)--(1.342,4.879)%
--(1.363,4.854)--(1.386,4.830)--(1.410,4.807)--(1.435,4.786)--(1.461,4.766)%
--(1.488,4.747)--(1.516,4.730)--(1.545,4.714)--(1.575,4.700)--(1.605,4.687)%
--(1.636,4.676)--(1.668,4.667)--(1.700,4.659)--(1.732,4.653)--(1.765,4.649)%
--(1.798,4.646)--(1.830,4.646)--(1.863,4.646)--(1.896,4.649)--(1.929,4.653)%
--(1.961,4.659)--(1.993,4.667)--(2.025,4.676)--(2.056,4.687)--(2.086,4.700)%
--(2.116,4.714)--(2.145,4.730)--(2.173,4.747)--(2.200,4.766)--(2.226,4.786)%
--(2.251,4.807)--(2.275,4.830)--(2.298,4.854)--(2.319,4.879)--(2.339,4.905)%
--(2.358,4.932)--(2.375,4.960)--(2.391,4.989)--(2.405,5.019)--(2.418,5.049)%
--(2.429,5.080)--(2.438,5.112)--(2.446,5.144)--(2.452,5.176)--(2.456,5.209)--(2.459,5.242)--cycle;
%
\gpfill{rgb color={0.000,0.000,0.000},opacity=0.15} (2.316,5.187)--(2.315,5.216)--(2.312,5.245)--(2.309,5.275)%
--(2.303,5.304)--(2.296,5.332)--(2.288,5.360)--(2.278,5.388)--(2.267,5.415)%
--(2.254,5.442)--(2.240,5.468)--(2.225,5.493)--(2.208,5.517)--(2.190,5.541)%
--(2.171,5.563)--(2.151,5.585)--(2.129,5.605)--(2.107,5.624)--(2.083,5.642)%
--(2.059,5.659)--(2.034,5.674)--(2.008,5.688)--(1.981,5.701)--(1.954,5.712)%
--(1.926,5.722)--(1.898,5.730)--(1.870,5.737)--(1.841,5.743)--(1.811,5.746)%
--(1.782,5.749)--(1.753,5.750)--(1.723,5.749)--(1.694,5.746)--(1.664,5.743)%
--(1.635,5.737)--(1.607,5.730)--(1.579,5.722)--(1.551,5.712)--(1.524,5.701)%
--(1.497,5.688)--(1.471,5.674)--(1.446,5.659)--(1.422,5.642)--(1.398,5.624)%
--(1.376,5.605)--(1.354,5.585)--(1.334,5.563)--(1.315,5.541)--(1.297,5.517)%
--(1.280,5.493)--(1.265,5.468)--(1.251,5.442)--(1.238,5.415)--(1.227,5.388)%
--(1.217,5.360)--(1.209,5.332)--(1.202,5.304)--(1.196,5.275)--(1.193,5.245)%
--(1.190,5.216)--(1.190,5.187)--(1.190,5.157)--(1.193,5.128)--(1.196,5.098)%
--(1.202,5.069)--(1.209,5.041)--(1.217,5.013)--(1.227,4.985)--(1.238,4.958)%
--(1.251,4.931)--(1.265,4.905)--(1.280,4.880)--(1.297,4.856)--(1.315,4.832)%
--(1.334,4.810)--(1.354,4.788)--(1.376,4.768)--(1.398,4.749)--(1.422,4.731)%
--(1.446,4.714)--(1.471,4.699)--(1.497,4.685)--(1.524,4.672)--(1.551,4.661)%
--(1.579,4.651)--(1.607,4.643)--(1.635,4.636)--(1.664,4.630)--(1.694,4.627)%
--(1.723,4.624)--(1.753,4.624)--(1.782,4.624)--(1.811,4.627)--(1.841,4.630)%
--(1.870,4.636)--(1.898,4.643)--(1.926,4.651)--(1.954,4.661)--(1.981,4.672)%
--(2.008,4.685)--(2.034,4.699)--(2.059,4.714)--(2.083,4.731)--(2.107,4.749)%
--(2.129,4.768)--(2.151,4.788)--(2.171,4.810)--(2.190,4.832)--(2.208,4.856)%
--(2.225,4.880)--(2.240,4.905)--(2.254,4.931)--(2.267,4.958)--(2.278,4.985)%
--(2.288,5.013)--(2.296,5.041)--(2.303,5.069)--(2.309,5.098)--(2.312,5.128)--(2.315,5.157)--cycle;
%
\gpfill{rgb color={0.000,0.000,0.000},opacity=0.15} (2.180,5.098)--(2.179,5.124)--(2.177,5.149)--(2.173,5.175)%
--(2.169,5.201)--(2.163,5.226)--(2.155,5.251)--(2.146,5.276)--(2.137,5.300)%
--(2.125,5.323)--(2.113,5.346)--(2.099,5.368)--(2.085,5.390)--(2.069,5.410)%
--(2.052,5.430)--(2.034,5.449)--(2.015,5.467)--(1.995,5.484)--(1.975,5.500)%
--(1.953,5.514)--(1.931,5.528)--(1.908,5.540)--(1.885,5.552)--(1.861,5.561)%
--(1.836,5.570)--(1.811,5.578)--(1.786,5.584)--(1.760,5.588)--(1.734,5.592)%
--(1.709,5.594)--(1.683,5.595)--(1.656,5.594)--(1.631,5.592)--(1.605,5.588)%
--(1.579,5.584)--(1.554,5.578)--(1.529,5.570)--(1.504,5.561)--(1.480,5.552)%
--(1.457,5.540)--(1.434,5.528)--(1.412,5.514)--(1.390,5.500)--(1.370,5.484)%
--(1.350,5.467)--(1.331,5.449)--(1.313,5.430)--(1.296,5.410)--(1.280,5.390)%
--(1.266,5.368)--(1.252,5.346)--(1.240,5.323)--(1.228,5.300)--(1.219,5.276)%
--(1.210,5.251)--(1.202,5.226)--(1.196,5.201)--(1.192,5.175)--(1.188,5.149)%
--(1.186,5.124)--(1.186,5.098)--(1.186,5.071)--(1.188,5.046)--(1.192,5.020)%
--(1.196,4.994)--(1.202,4.969)--(1.210,4.944)--(1.219,4.919)--(1.228,4.895)%
--(1.240,4.872)--(1.252,4.849)--(1.266,4.827)--(1.280,4.805)--(1.296,4.785)%
--(1.313,4.765)--(1.331,4.746)--(1.350,4.728)--(1.370,4.711)--(1.390,4.695)%
--(1.412,4.681)--(1.434,4.667)--(1.457,4.655)--(1.480,4.643)--(1.504,4.634)%
--(1.529,4.625)--(1.554,4.617)--(1.579,4.611)--(1.605,4.607)--(1.631,4.603)%
--(1.656,4.601)--(1.683,4.601)--(1.709,4.601)--(1.734,4.603)--(1.760,4.607)%
--(1.786,4.611)--(1.811,4.617)--(1.836,4.625)--(1.861,4.634)--(1.885,4.643)%
--(1.908,4.655)--(1.931,4.667)--(1.953,4.681)--(1.975,4.695)--(1.995,4.711)%
--(2.015,4.728)--(2.034,4.746)--(2.052,4.765)--(2.069,4.785)--(2.085,4.805)%
--(2.099,4.827)--(2.113,4.849)--(2.125,4.872)--(2.137,4.895)--(2.146,4.919)%
--(2.155,4.944)--(2.163,4.969)--(2.169,4.994)--(2.173,5.020)--(2.177,5.046)--(2.179,5.071)--cycle;
%
\gpfill{rgb color={0.000,0.000,0.000},opacity=0.15} (2.052,5.008)--(2.051,5.030)--(2.049,5.052)--(2.046,5.075)%
--(2.042,5.097)--(2.037,5.119)--(2.030,5.140)--(2.023,5.162)--(2.014,5.182)%
--(2.005,5.203)--(1.994,5.223)--(1.982,5.242)--(1.969,5.260)--(1.956,5.278)%
--(1.941,5.295)--(1.926,5.312)--(1.909,5.327)--(1.892,5.342)--(1.874,5.355)%
--(1.856,5.368)--(1.837,5.380)--(1.817,5.391)--(1.796,5.400)--(1.776,5.409)%
--(1.754,5.416)--(1.733,5.423)--(1.711,5.428)--(1.689,5.432)--(1.666,5.435)%
--(1.644,5.437)--(1.622,5.438)--(1.599,5.437)--(1.577,5.435)--(1.554,5.432)%
--(1.532,5.428)--(1.510,5.423)--(1.489,5.416)--(1.467,5.409)--(1.447,5.400)%
--(1.426,5.391)--(1.407,5.380)--(1.387,5.368)--(1.369,5.355)--(1.351,5.342)%
--(1.334,5.327)--(1.317,5.312)--(1.302,5.295)--(1.287,5.278)--(1.274,5.260)%
--(1.261,5.242)--(1.249,5.223)--(1.238,5.203)--(1.229,5.182)--(1.220,5.162)%
--(1.213,5.140)--(1.206,5.119)--(1.201,5.097)--(1.197,5.075)--(1.194,5.052)%
--(1.192,5.030)--(1.192,5.008)--(1.192,4.985)--(1.194,4.963)--(1.197,4.940)%
--(1.201,4.918)--(1.206,4.896)--(1.213,4.875)--(1.220,4.853)--(1.229,4.833)%
--(1.238,4.812)--(1.249,4.793)--(1.261,4.773)--(1.274,4.755)--(1.287,4.737)%
--(1.302,4.720)--(1.317,4.703)--(1.334,4.688)--(1.351,4.673)--(1.369,4.660)%
--(1.387,4.647)--(1.406,4.635)--(1.426,4.624)--(1.447,4.615)--(1.467,4.606)%
--(1.489,4.599)--(1.510,4.592)--(1.532,4.587)--(1.554,4.583)--(1.577,4.580)%
--(1.599,4.578)--(1.622,4.578)--(1.644,4.578)--(1.666,4.580)--(1.689,4.583)%
--(1.711,4.587)--(1.733,4.592)--(1.754,4.599)--(1.776,4.606)--(1.796,4.615)%
--(1.817,4.624)--(1.837,4.635)--(1.856,4.647)--(1.874,4.660)--(1.892,4.673)%
--(1.909,4.688)--(1.926,4.703)--(1.941,4.720)--(1.956,4.737)--(1.969,4.755)%
--(1.982,4.773)--(1.994,4.793)--(2.005,4.812)--(2.014,4.833)--(2.023,4.853)%
--(2.030,4.875)--(2.037,4.896)--(2.042,4.918)--(2.046,4.940)--(2.049,4.963)--(2.051,4.985)--cycle;
%
\gpfill{rgb color={0.000,0.000,0.000},opacity=0.15} (1.931,4.918)--(1.930,4.936)--(1.929,4.955)--(1.926,4.974)%
--(1.923,4.993)--(1.918,5.011)--(1.913,5.030)--(1.906,5.048)--(1.899,5.065)%
--(1.891,5.082)--(1.882,5.099)--(1.872,5.115)--(1.861,5.131)--(1.850,5.146)%
--(1.837,5.160)--(1.824,5.174)--(1.810,5.187)--(1.796,5.200)--(1.781,5.211)%
--(1.765,5.222)--(1.749,5.232)--(1.732,5.241)--(1.715,5.249)--(1.698,5.256)%
--(1.680,5.263)--(1.661,5.268)--(1.643,5.273)--(1.624,5.276)--(1.605,5.279)%
--(1.586,5.280)--(1.568,5.281)--(1.549,5.280)--(1.530,5.279)--(1.511,5.276)%
--(1.492,5.273)--(1.474,5.268)--(1.455,5.263)--(1.437,5.256)--(1.420,5.249)%
--(1.403,5.241)--(1.386,5.232)--(1.370,5.222)--(1.354,5.211)--(1.339,5.200)%
--(1.325,5.187)--(1.311,5.174)--(1.298,5.160)--(1.285,5.146)--(1.274,5.131)%
--(1.263,5.115)--(1.253,5.099)--(1.244,5.082)--(1.236,5.065)--(1.229,5.048)%
--(1.222,5.030)--(1.217,5.011)--(1.212,4.993)--(1.209,4.974)--(1.206,4.955)%
--(1.205,4.936)--(1.205,4.918)--(1.205,4.899)--(1.206,4.880)--(1.209,4.861)%
--(1.212,4.842)--(1.217,4.824)--(1.222,4.805)--(1.229,4.787)--(1.236,4.770)%
--(1.244,4.753)--(1.253,4.736)--(1.263,4.720)--(1.274,4.704)--(1.285,4.689)%
--(1.298,4.675)--(1.311,4.661)--(1.325,4.648)--(1.339,4.635)--(1.354,4.624)%
--(1.370,4.613)--(1.386,4.603)--(1.403,4.594)--(1.420,4.586)--(1.437,4.579)%
--(1.455,4.572)--(1.474,4.567)--(1.492,4.562)--(1.511,4.559)--(1.530,4.556)%
--(1.549,4.555)--(1.568,4.555)--(1.586,4.555)--(1.605,4.556)--(1.624,4.559)%
--(1.643,4.562)--(1.661,4.567)--(1.680,4.572)--(1.698,4.579)--(1.715,4.586)%
--(1.732,4.594)--(1.749,4.603)--(1.765,4.613)--(1.781,4.624)--(1.796,4.635)%
--(1.810,4.648)--(1.824,4.661)--(1.837,4.675)--(1.850,4.689)--(1.861,4.704)%
--(1.872,4.720)--(1.882,4.736)--(1.891,4.753)--(1.899,4.770)--(1.906,4.787)%
--(1.913,4.805)--(1.918,4.824)--(1.923,4.842)--(1.926,4.861)--(1.929,4.880)--(1.930,4.899)--cycle;
%
\gpfill{rgb color={0.000,0.000,0.000},opacity=0.15} (1.817,4.827)--(1.816,4.842)--(1.815,4.857)--(1.813,4.873)%
--(1.810,4.888)--(1.806,4.903)--(1.802,4.918)--(1.797,4.932)--(1.791,4.946)%
--(1.784,4.960)--(1.777,4.974)--(1.769,4.987)--(1.760,5.000)--(1.751,5.012)%
--(1.741,5.024)--(1.730,5.035)--(1.719,5.046)--(1.707,5.056)--(1.695,5.065)%
--(1.682,5.074)--(1.669,5.082)--(1.655,5.089)--(1.641,5.096)--(1.627,5.102)%
--(1.613,5.107)--(1.598,5.111)--(1.583,5.115)--(1.568,5.118)--(1.552,5.120)%
--(1.537,5.121)--(1.522,5.122)--(1.506,5.121)--(1.491,5.120)--(1.475,5.118)%
--(1.460,5.115)--(1.445,5.111)--(1.430,5.107)--(1.416,5.102)--(1.402,5.096)%
--(1.388,5.089)--(1.374,5.082)--(1.361,5.074)--(1.348,5.065)--(1.336,5.056)%
--(1.324,5.046)--(1.313,5.035)--(1.302,5.024)--(1.292,5.012)--(1.283,5.000)%
--(1.274,4.987)--(1.266,4.974)--(1.259,4.960)--(1.252,4.946)--(1.246,4.932)%
--(1.241,4.918)--(1.237,4.903)--(1.233,4.888)--(1.230,4.873)--(1.228,4.857)%
--(1.227,4.842)--(1.227,4.827)--(1.227,4.811)--(1.228,4.796)--(1.230,4.780)%
--(1.233,4.765)--(1.237,4.750)--(1.241,4.735)--(1.246,4.721)--(1.252,4.707)%
--(1.259,4.693)--(1.266,4.679)--(1.274,4.666)--(1.283,4.653)--(1.292,4.641)%
--(1.302,4.629)--(1.313,4.618)--(1.324,4.607)--(1.336,4.597)--(1.348,4.588)%
--(1.361,4.579)--(1.374,4.571)--(1.388,4.564)--(1.402,4.557)--(1.416,4.551)%
--(1.430,4.546)--(1.445,4.542)--(1.460,4.538)--(1.475,4.535)--(1.491,4.533)%
--(1.506,4.532)--(1.522,4.532)--(1.537,4.532)--(1.552,4.533)--(1.568,4.535)%
--(1.583,4.538)--(1.598,4.542)--(1.613,4.546)--(1.627,4.551)--(1.641,4.557)%
--(1.655,4.564)--(1.669,4.571)--(1.682,4.579)--(1.695,4.588)--(1.707,4.597)%
--(1.719,4.607)--(1.730,4.618)--(1.741,4.629)--(1.751,4.641)--(1.760,4.653)%
--(1.769,4.666)--(1.777,4.679)--(1.784,4.693)--(1.791,4.707)--(1.797,4.721)%
--(1.802,4.735)--(1.806,4.750)--(1.810,4.765)--(1.813,4.780)--(1.815,4.796)--(1.816,4.811)--cycle;
%
\gpfill{rgb color={0.000,0.000,0.000},opacity=0.15} (1.711,4.736)--(1.710,4.747)--(1.709,4.759)--(1.708,4.771)%
--(1.706,4.783)--(1.703,4.794)--(1.699,4.806)--(1.695,4.817)--(1.691,4.828)%
--(1.686,4.839)--(1.680,4.849)--(1.674,4.859)--(1.667,4.869)--(1.660,4.878)%
--(1.652,4.887)--(1.644,4.896)--(1.635,4.904)--(1.626,4.912)--(1.617,4.919)%
--(1.607,4.926)--(1.597,4.932)--(1.587,4.938)--(1.576,4.943)--(1.565,4.947)%
--(1.554,4.951)--(1.542,4.955)--(1.531,4.958)--(1.519,4.960)--(1.507,4.961)%
--(1.495,4.962)--(1.484,4.963)--(1.472,4.962)--(1.460,4.961)--(1.448,4.960)%
--(1.436,4.958)--(1.425,4.955)--(1.413,4.951)--(1.402,4.947)--(1.391,4.943)%
--(1.380,4.938)--(1.370,4.932)--(1.360,4.926)--(1.350,4.919)--(1.341,4.912)%
--(1.332,4.904)--(1.323,4.896)--(1.315,4.887)--(1.307,4.878)--(1.300,4.869)%
--(1.293,4.859)--(1.287,4.849)--(1.281,4.839)--(1.276,4.828)--(1.272,4.817)%
--(1.268,4.806)--(1.264,4.794)--(1.261,4.783)--(1.259,4.771)--(1.258,4.759)%
--(1.257,4.747)--(1.257,4.736)--(1.257,4.724)--(1.258,4.712)--(1.259,4.700)%
--(1.261,4.688)--(1.264,4.677)--(1.268,4.665)--(1.272,4.654)--(1.276,4.643)%
--(1.281,4.632)--(1.287,4.622)--(1.293,4.612)--(1.300,4.602)--(1.307,4.593)%
--(1.315,4.584)--(1.323,4.575)--(1.332,4.567)--(1.341,4.559)--(1.350,4.552)%
--(1.360,4.545)--(1.370,4.539)--(1.380,4.533)--(1.391,4.528)--(1.402,4.524)%
--(1.413,4.520)--(1.425,4.516)--(1.436,4.513)--(1.448,4.511)--(1.460,4.510)%
--(1.472,4.509)--(1.484,4.509)--(1.495,4.509)--(1.507,4.510)--(1.519,4.511)%
--(1.531,4.513)--(1.542,4.516)--(1.554,4.520)--(1.565,4.524)--(1.576,4.528)%
--(1.587,4.533)--(1.597,4.539)--(1.607,4.545)--(1.617,4.552)--(1.626,4.559)%
--(1.635,4.567)--(1.644,4.575)--(1.652,4.584)--(1.660,4.593)--(1.667,4.602)%
--(1.674,4.612)--(1.680,4.622)--(1.686,4.632)--(1.691,4.643)--(1.695,4.654)%
--(1.699,4.665)--(1.703,4.677)--(1.706,4.688)--(1.708,4.700)--(1.709,4.712)--(1.710,4.724)--cycle;
%
\gpfill{rgb color={0.000,0.000,0.000},opacity=0.15} (1.615,4.646)--(1.614,4.654)--(1.614,4.662)--(1.613,4.671)%
--(1.611,4.679)--(1.609,4.687)--(1.607,4.695)--(1.604,4.703)--(1.601,4.711)%
--(1.597,4.718)--(1.593,4.726)--(1.589,4.733)--(1.584,4.740)--(1.579,4.746)%
--(1.573,4.753)--(1.568,4.759)--(1.562,4.764)--(1.555,4.770)--(1.549,4.775)%
--(1.542,4.780)--(1.535,4.784)--(1.527,4.788)--(1.520,4.792)--(1.512,4.795)%
--(1.504,4.798)--(1.496,4.800)--(1.488,4.802)--(1.480,4.804)--(1.471,4.805)%
--(1.463,4.805)--(1.455,4.806)--(1.446,4.805)--(1.438,4.805)--(1.429,4.804)%
--(1.421,4.802)--(1.413,4.800)--(1.405,4.798)--(1.397,4.795)--(1.389,4.792)%
--(1.382,4.788)--(1.375,4.784)--(1.367,4.780)--(1.360,4.775)--(1.354,4.770)%
--(1.347,4.764)--(1.341,4.759)--(1.336,4.753)--(1.330,4.746)--(1.325,4.740)%
--(1.320,4.733)--(1.316,4.726)--(1.312,4.718)--(1.308,4.711)--(1.305,4.703)%
--(1.302,4.695)--(1.300,4.687)--(1.298,4.679)--(1.296,4.671)--(1.295,4.662)%
--(1.295,4.654)--(1.295,4.646)--(1.295,4.637)--(1.295,4.629)--(1.296,4.620)%
--(1.298,4.612)--(1.300,4.604)--(1.302,4.596)--(1.305,4.588)--(1.308,4.580)%
--(1.312,4.573)--(1.316,4.566)--(1.320,4.558)--(1.325,4.551)--(1.330,4.545)%
--(1.336,4.538)--(1.341,4.532)--(1.347,4.527)--(1.354,4.521)--(1.360,4.516)%
--(1.367,4.511)--(1.375,4.507)--(1.382,4.503)--(1.389,4.499)--(1.397,4.496)%
--(1.405,4.493)--(1.413,4.491)--(1.421,4.489)--(1.429,4.487)--(1.438,4.486)%
--(1.446,4.486)--(1.455,4.486)--(1.463,4.486)--(1.471,4.486)--(1.480,4.487)%
--(1.488,4.489)--(1.496,4.491)--(1.504,4.493)--(1.512,4.496)--(1.520,4.499)%
--(1.527,4.503)--(1.535,4.507)--(1.542,4.511)--(1.549,4.516)--(1.555,4.521)%
--(1.562,4.527)--(1.568,4.532)--(1.573,4.538)--(1.579,4.545)--(1.584,4.551)%
--(1.589,4.558)--(1.593,4.566)--(1.597,4.573)--(1.601,4.580)--(1.604,4.588)%
--(1.607,4.596)--(1.609,4.604)--(1.611,4.612)--(1.613,4.620)--(1.614,4.629)--(1.614,4.637)--cycle;
%
\gpfill{rgb color={0.000,0.000,0.000},opacity=0.15} (1.529,4.555)--(1.528,4.559)--(1.528,4.564)--(1.527,4.569)%
--(1.526,4.574)--(1.525,4.579)--(1.524,4.584)--(1.522,4.589)--(1.520,4.593)%
--(1.518,4.598)--(1.516,4.602)--(1.513,4.606)--(1.510,4.610)--(1.507,4.614)%
--(1.504,4.618)--(1.501,4.622)--(1.497,4.625)--(1.493,4.628)--(1.489,4.631)%
--(1.485,4.634)--(1.481,4.637)--(1.477,4.639)--(1.472,4.641)--(1.468,4.643)%
--(1.463,4.645)--(1.458,4.646)--(1.453,4.647)--(1.448,4.648)--(1.443,4.649)%
--(1.438,4.649)--(1.434,4.650)--(1.429,4.649)--(1.424,4.649)--(1.419,4.648)%
--(1.414,4.647)--(1.409,4.646)--(1.404,4.645)--(1.399,4.643)--(1.395,4.641)%
--(1.390,4.639)--(1.386,4.637)--(1.382,4.634)--(1.378,4.631)--(1.374,4.628)%
--(1.370,4.625)--(1.366,4.622)--(1.363,4.618)--(1.360,4.614)--(1.357,4.610)%
--(1.354,4.606)--(1.351,4.602)--(1.349,4.598)--(1.347,4.593)--(1.345,4.589)%
--(1.343,4.584)--(1.342,4.579)--(1.341,4.574)--(1.340,4.569)--(1.339,4.564)%
--(1.339,4.559)--(1.339,4.555)--(1.339,4.550)--(1.339,4.545)--(1.340,4.540)%
--(1.341,4.535)--(1.342,4.530)--(1.343,4.525)--(1.345,4.520)--(1.347,4.516)%
--(1.349,4.511)--(1.351,4.507)--(1.354,4.503)--(1.357,4.499)--(1.360,4.495)%
--(1.363,4.491)--(1.366,4.487)--(1.370,4.484)--(1.374,4.481)--(1.378,4.478)%
--(1.382,4.475)--(1.386,4.472)--(1.390,4.470)--(1.395,4.468)--(1.399,4.466)%
--(1.404,4.464)--(1.409,4.463)--(1.414,4.462)--(1.419,4.461)--(1.424,4.460)%
--(1.429,4.460)--(1.434,4.460)--(1.438,4.460)--(1.443,4.460)--(1.448,4.461)%
--(1.453,4.462)--(1.458,4.463)--(1.463,4.464)--(1.468,4.466)--(1.472,4.468)%
--(1.477,4.470)--(1.481,4.472)--(1.485,4.475)--(1.489,4.478)--(1.493,4.481)%
--(1.497,4.484)--(1.501,4.487)--(1.504,4.491)--(1.507,4.495)--(1.510,4.499)%
--(1.513,4.503)--(1.516,4.507)--(1.518,4.511)--(1.520,4.516)--(1.522,4.520)%
--(1.524,4.525)--(1.525,4.530)--(1.526,4.535)--(1.527,4.540)--(1.528,4.545)--(1.528,4.550)--cycle;
%
\gpfill{rgb color={0.000,0.000,0.000},opacity=0.15} (1.512,4.465)--(1.511,4.469)--(1.511,4.474)--(1.510,4.479)%
--(1.510,4.483)--(1.508,4.488)--(1.507,4.493)--(1.505,4.497)--(1.504,4.502)%
--(1.502,4.506)--(1.499,4.510)--(1.497,4.514)--(1.494,4.518)--(1.491,4.522)%
--(1.488,4.525)--(1.485,4.529)--(1.481,4.532)--(1.478,4.535)--(1.474,4.538)%
--(1.470,4.541)--(1.466,4.543)--(1.462,4.546)--(1.458,4.548)--(1.453,4.549)%
--(1.449,4.551)--(1.444,4.552)--(1.439,4.554)--(1.435,4.554)--(1.430,4.555)%
--(1.425,4.555)--(1.421,4.556)--(1.416,4.555)--(1.411,4.555)--(1.406,4.554)%
--(1.402,4.554)--(1.397,4.552)--(1.392,4.551)--(1.388,4.549)--(1.383,4.548)%
--(1.379,4.546)--(1.375,4.543)--(1.371,4.541)--(1.367,4.538)--(1.363,4.535)%
--(1.360,4.532)--(1.356,4.529)--(1.353,4.525)--(1.350,4.522)--(1.347,4.518)%
--(1.344,4.514)--(1.342,4.510)--(1.339,4.506)--(1.337,4.502)--(1.336,4.497)%
--(1.334,4.493)--(1.333,4.488)--(1.331,4.483)--(1.331,4.479)--(1.330,4.474)%
--(1.330,4.469)--(1.330,4.465)--(1.330,4.460)--(1.330,4.455)--(1.331,4.450)%
--(1.331,4.446)--(1.333,4.441)--(1.334,4.436)--(1.336,4.432)--(1.337,4.427)%
--(1.339,4.423)--(1.342,4.419)--(1.344,4.415)--(1.347,4.411)--(1.350,4.407)%
--(1.353,4.404)--(1.356,4.400)--(1.360,4.397)--(1.363,4.394)--(1.367,4.391)%
--(1.371,4.388)--(1.375,4.386)--(1.379,4.383)--(1.383,4.381)--(1.388,4.380)%
--(1.392,4.378)--(1.397,4.377)--(1.402,4.375)--(1.406,4.375)--(1.411,4.374)%
--(1.416,4.374)--(1.421,4.374)--(1.425,4.374)--(1.430,4.374)--(1.435,4.375)%
--(1.439,4.375)--(1.444,4.377)--(1.449,4.378)--(1.453,4.380)--(1.458,4.381)%
--(1.462,4.383)--(1.466,4.386)--(1.470,4.388)--(1.474,4.391)--(1.478,4.394)%
--(1.481,4.397)--(1.485,4.400)--(1.488,4.404)--(1.491,4.407)--(1.494,4.411)%
--(1.497,4.415)--(1.499,4.419)--(1.502,4.423)--(1.504,4.427)--(1.505,4.432)%
--(1.507,4.436)--(1.508,4.441)--(1.510,4.446)--(1.510,4.450)--(1.511,4.455)--(1.511,4.460)--cycle;
%
\gpfill{rgb color={0.000,0.000,0.000},opacity=0.15} (1.451,4.375)--(1.450,4.376)--(1.450,4.378)--(1.450,4.380)%
--(1.450,4.382)--(1.449,4.383)--(1.449,4.385)--(1.448,4.387)--(1.448,4.388)%
--(1.447,4.390)--(1.446,4.392)--(1.445,4.393)--(1.444,4.394)--(1.443,4.396)%
--(1.442,4.397)--(1.441,4.399)--(1.439,4.400)--(1.438,4.401)--(1.436,4.402)%
--(1.435,4.403)--(1.434,4.404)--(1.432,4.405)--(1.430,4.406)--(1.429,4.406)%
--(1.427,4.407)--(1.425,4.407)--(1.424,4.408)--(1.422,4.408)--(1.420,4.408)%
--(1.418,4.408)--(1.417,4.409)--(1.415,4.408)--(1.413,4.408)--(1.411,4.408)%
--(1.409,4.408)--(1.408,4.407)--(1.406,4.407)--(1.404,4.406)--(1.403,4.406)%
--(1.401,4.405)--(1.400,4.404)--(1.398,4.403)--(1.397,4.402)--(1.395,4.401)%
--(1.394,4.400)--(1.392,4.399)--(1.391,4.397)--(1.390,4.396)--(1.389,4.394)%
--(1.388,4.393)--(1.387,4.392)--(1.386,4.390)--(1.385,4.388)--(1.385,4.387)%
--(1.384,4.385)--(1.384,4.383)--(1.383,4.382)--(1.383,4.380)--(1.383,4.378)%
--(1.383,4.376)--(1.383,4.375)--(1.383,4.373)--(1.383,4.371)--(1.383,4.369)%
--(1.383,4.367)--(1.384,4.366)--(1.384,4.364)--(1.385,4.362)--(1.385,4.361)%
--(1.386,4.359)--(1.387,4.358)--(1.388,4.356)--(1.389,4.355)--(1.390,4.353)%
--(1.391,4.352)--(1.392,4.350)--(1.394,4.349)--(1.395,4.348)--(1.397,4.347)%
--(1.398,4.346)--(1.400,4.345)--(1.401,4.344)--(1.403,4.343)--(1.404,4.343)%
--(1.406,4.342)--(1.408,4.342)--(1.409,4.341)--(1.411,4.341)--(1.413,4.341)%
--(1.415,4.341)--(1.417,4.341)--(1.418,4.341)--(1.420,4.341)--(1.422,4.341)%
--(1.424,4.341)--(1.425,4.342)--(1.427,4.342)--(1.429,4.343)--(1.430,4.343)%
--(1.432,4.344)--(1.434,4.345)--(1.435,4.346)--(1.436,4.347)--(1.438,4.348)%
--(1.439,4.349)--(1.441,4.350)--(1.442,4.352)--(1.443,4.353)--(1.444,4.355)%
--(1.445,4.356)--(1.446,4.358)--(1.447,4.359)--(1.448,4.361)--(1.448,4.362)%
--(1.449,4.364)--(1.449,4.366)--(1.450,4.367)--(1.450,4.369)--(1.450,4.371)--(1.450,4.373)--cycle;
%
\gpfill{rgb color={0.000,0.000,0.000},opacity=0.15} (1.511,4.284)--(1.510,4.288)--(1.510,4.293)--(1.509,4.298)%
--(1.509,4.302)--(1.507,4.307)--(1.506,4.311)--(1.505,4.316)--(1.503,4.320)%
--(1.501,4.324)--(1.498,4.329)--(1.496,4.333)--(1.493,4.336)--(1.490,4.340)%
--(1.487,4.344)--(1.484,4.347)--(1.481,4.350)--(1.477,4.353)--(1.473,4.356)%
--(1.470,4.359)--(1.466,4.361)--(1.461,4.364)--(1.457,4.366)--(1.453,4.368)%
--(1.448,4.369)--(1.444,4.370)--(1.439,4.372)--(1.435,4.372)--(1.430,4.373)%
--(1.425,4.373)--(1.421,4.374)--(1.416,4.373)--(1.411,4.373)--(1.406,4.372)%
--(1.402,4.372)--(1.397,4.370)--(1.393,4.369)--(1.388,4.368)--(1.384,4.366)%
--(1.380,4.364)--(1.376,4.361)--(1.371,4.359)--(1.368,4.356)--(1.364,4.353)%
--(1.360,4.350)--(1.357,4.347)--(1.354,4.344)--(1.351,4.340)--(1.348,4.336)%
--(1.345,4.333)--(1.343,4.329)--(1.340,4.324)--(1.338,4.320)--(1.336,4.316)%
--(1.335,4.311)--(1.334,4.307)--(1.332,4.302)--(1.332,4.298)--(1.331,4.293)%
--(1.331,4.288)--(1.331,4.284)--(1.331,4.279)--(1.331,4.274)--(1.332,4.269)%
--(1.332,4.265)--(1.334,4.260)--(1.335,4.256)--(1.336,4.251)--(1.338,4.247)%
--(1.340,4.243)--(1.343,4.239)--(1.345,4.234)--(1.348,4.231)--(1.351,4.227)%
--(1.354,4.223)--(1.357,4.220)--(1.360,4.217)--(1.364,4.214)--(1.368,4.211)%
--(1.371,4.208)--(1.376,4.206)--(1.380,4.203)--(1.384,4.201)--(1.388,4.199)%
--(1.393,4.198)--(1.397,4.197)--(1.402,4.195)--(1.406,4.195)--(1.411,4.194)%
--(1.416,4.194)--(1.421,4.194)--(1.425,4.194)--(1.430,4.194)--(1.435,4.195)%
--(1.439,4.195)--(1.444,4.197)--(1.448,4.198)--(1.453,4.199)--(1.457,4.201)%
--(1.461,4.203)--(1.466,4.206)--(1.470,4.208)--(1.473,4.211)--(1.477,4.214)%
--(1.481,4.217)--(1.484,4.220)--(1.487,4.223)--(1.490,4.227)--(1.493,4.231)%
--(1.496,4.234)--(1.498,4.239)--(1.501,4.243)--(1.503,4.247)--(1.505,4.251)%
--(1.506,4.256)--(1.507,4.260)--(1.509,4.265)--(1.509,4.269)--(1.510,4.274)--(1.510,4.279)--cycle;
%
\gpfill{rgb color={0.000,0.000,0.000},opacity=0.15} (1.524,4.195)--(1.523,4.199)--(1.523,4.204)--(1.522,4.209)%
--(1.522,4.213)--(1.520,4.218)--(1.519,4.222)--(1.518,4.227)--(1.516,4.231)%
--(1.514,4.235)--(1.511,4.240)--(1.509,4.244)--(1.506,4.247)--(1.503,4.251)%
--(1.500,4.255)--(1.497,4.258)--(1.494,4.261)--(1.490,4.264)--(1.486,4.267)%
--(1.483,4.270)--(1.479,4.272)--(1.474,4.275)--(1.470,4.277)--(1.466,4.279)%
--(1.461,4.280)--(1.457,4.281)--(1.452,4.283)--(1.448,4.283)--(1.443,4.284)%
--(1.438,4.284)--(1.434,4.285)--(1.429,4.284)--(1.424,4.284)--(1.419,4.283)%
--(1.415,4.283)--(1.410,4.281)--(1.406,4.280)--(1.401,4.279)--(1.397,4.277)%
--(1.393,4.275)--(1.389,4.272)--(1.384,4.270)--(1.381,4.267)--(1.377,4.264)%
--(1.373,4.261)--(1.370,4.258)--(1.367,4.255)--(1.364,4.251)--(1.361,4.247)%
--(1.358,4.244)--(1.356,4.240)--(1.353,4.235)--(1.351,4.231)--(1.349,4.227)%
--(1.348,4.222)--(1.347,4.218)--(1.345,4.213)--(1.345,4.209)--(1.344,4.204)%
--(1.344,4.199)--(1.344,4.195)--(1.344,4.190)--(1.344,4.185)--(1.345,4.180)%
--(1.345,4.176)--(1.347,4.171)--(1.348,4.167)--(1.349,4.162)--(1.351,4.158)%
--(1.353,4.154)--(1.356,4.150)--(1.358,4.145)--(1.361,4.142)--(1.364,4.138)%
--(1.367,4.134)--(1.370,4.131)--(1.373,4.128)--(1.377,4.125)--(1.381,4.122)%
--(1.384,4.119)--(1.389,4.117)--(1.393,4.114)--(1.397,4.112)--(1.401,4.110)%
--(1.406,4.109)--(1.410,4.108)--(1.415,4.106)--(1.419,4.106)--(1.424,4.105)%
--(1.429,4.105)--(1.434,4.105)--(1.438,4.105)--(1.443,4.105)--(1.448,4.106)%
--(1.452,4.106)--(1.457,4.108)--(1.461,4.109)--(1.466,4.110)--(1.470,4.112)%
--(1.474,4.114)--(1.479,4.117)--(1.483,4.119)--(1.486,4.122)--(1.490,4.125)%
--(1.494,4.128)--(1.497,4.131)--(1.500,4.134)--(1.503,4.138)--(1.506,4.142)%
--(1.509,4.145)--(1.511,4.150)--(1.514,4.154)--(1.516,4.158)--(1.518,4.162)%
--(1.519,4.167)--(1.520,4.171)--(1.522,4.176)--(1.522,4.180)--(1.523,4.185)--(1.523,4.190)--cycle;
%
\gpfill{rgb color={0.000,0.000,0.000},opacity=0.15} (1.574,4.108)--(1.573,4.114)--(1.573,4.120)--(1.572,4.126)%
--(1.571,4.132)--(1.569,4.138)--(1.568,4.144)--(1.566,4.150)--(1.563,4.156)%
--(1.561,4.162)--(1.558,4.167)--(1.554,4.172)--(1.551,4.177)--(1.547,4.182)%
--(1.543,4.187)--(1.539,4.192)--(1.534,4.196)--(1.529,4.200)--(1.524,4.204)%
--(1.519,4.207)--(1.514,4.211)--(1.509,4.214)--(1.503,4.216)--(1.497,4.219)%
--(1.491,4.221)--(1.485,4.222)--(1.479,4.224)--(1.473,4.225)--(1.467,4.226)%
--(1.461,4.226)--(1.455,4.227)--(1.448,4.226)--(1.442,4.226)--(1.436,4.225)%
--(1.430,4.224)--(1.424,4.222)--(1.418,4.221)--(1.412,4.219)--(1.406,4.216)%
--(1.400,4.214)--(1.395,4.211)--(1.390,4.207)--(1.385,4.204)--(1.380,4.200)%
--(1.375,4.196)--(1.370,4.192)--(1.366,4.187)--(1.362,4.182)--(1.358,4.177)%
--(1.355,4.172)--(1.351,4.167)--(1.348,4.162)--(1.346,4.156)--(1.343,4.150)%
--(1.341,4.144)--(1.340,4.138)--(1.338,4.132)--(1.337,4.126)--(1.336,4.120)%
--(1.336,4.114)--(1.336,4.108)--(1.336,4.101)--(1.336,4.095)--(1.337,4.089)%
--(1.338,4.083)--(1.340,4.077)--(1.341,4.071)--(1.343,4.065)--(1.346,4.059)%
--(1.348,4.053)--(1.351,4.048)--(1.355,4.043)--(1.358,4.038)--(1.362,4.033)%
--(1.366,4.028)--(1.370,4.023)--(1.375,4.019)--(1.380,4.015)--(1.385,4.011)%
--(1.390,4.008)--(1.395,4.004)--(1.400,4.001)--(1.406,3.999)--(1.412,3.996)%
--(1.418,3.994)--(1.424,3.993)--(1.430,3.991)--(1.436,3.990)--(1.442,3.989)%
--(1.448,3.989)--(1.455,3.989)--(1.461,3.989)--(1.467,3.989)--(1.473,3.990)%
--(1.479,3.991)--(1.485,3.993)--(1.491,3.994)--(1.497,3.996)--(1.503,3.999)%
--(1.509,4.001)--(1.514,4.004)--(1.519,4.008)--(1.524,4.011)--(1.529,4.015)%
--(1.534,4.019)--(1.539,4.023)--(1.543,4.028)--(1.547,4.033)--(1.551,4.038)%
--(1.554,4.043)--(1.558,4.048)--(1.561,4.053)--(1.563,4.059)--(1.566,4.065)%
--(1.568,4.071)--(1.569,4.077)--(1.571,4.083)--(1.572,4.089)--(1.573,4.095)--(1.573,4.101)--cycle;
%
\gpfill{rgb color={0.000,0.000,0.000},opacity=0.15} (1.656,4.024)--(1.655,4.033)--(1.655,4.041)--(1.653,4.050)%
--(1.652,4.059)--(1.650,4.068)--(1.647,4.077)--(1.644,4.085)--(1.641,4.093)%
--(1.637,4.102)--(1.632,4.110)--(1.628,4.117)--(1.623,4.125)--(1.617,4.132)%
--(1.611,4.139)--(1.605,4.145)--(1.599,4.151)--(1.592,4.157)--(1.585,4.163)%
--(1.577,4.168)--(1.570,4.172)--(1.562,4.177)--(1.553,4.181)--(1.545,4.184)%
--(1.537,4.187)--(1.528,4.190)--(1.519,4.192)--(1.510,4.193)--(1.501,4.195)%
--(1.493,4.195)--(1.484,4.196)--(1.474,4.195)--(1.466,4.195)--(1.457,4.193)%
--(1.448,4.192)--(1.439,4.190)--(1.430,4.187)--(1.422,4.184)--(1.414,4.181)%
--(1.405,4.177)--(1.398,4.172)--(1.390,4.168)--(1.382,4.163)--(1.375,4.157)%
--(1.368,4.151)--(1.362,4.145)--(1.356,4.139)--(1.350,4.132)--(1.344,4.125)%
--(1.339,4.117)--(1.335,4.110)--(1.330,4.102)--(1.326,4.093)--(1.323,4.085)%
--(1.320,4.077)--(1.317,4.068)--(1.315,4.059)--(1.314,4.050)--(1.312,4.041)%
--(1.312,4.033)--(1.312,4.024)--(1.312,4.014)--(1.312,4.006)--(1.314,3.997)%
--(1.315,3.988)--(1.317,3.979)--(1.320,3.970)--(1.323,3.962)--(1.326,3.954)%
--(1.330,3.945)--(1.335,3.938)--(1.339,3.930)--(1.344,3.922)--(1.350,3.915)%
--(1.356,3.908)--(1.362,3.902)--(1.368,3.896)--(1.375,3.890)--(1.382,3.884)%
--(1.390,3.879)--(1.398,3.875)--(1.405,3.870)--(1.414,3.866)--(1.422,3.863)%
--(1.430,3.860)--(1.439,3.857)--(1.448,3.855)--(1.457,3.854)--(1.466,3.852)%
--(1.474,3.852)--(1.484,3.852)--(1.493,3.852)--(1.501,3.852)--(1.510,3.854)%
--(1.519,3.855)--(1.528,3.857)--(1.537,3.860)--(1.545,3.863)--(1.553,3.866)%
--(1.562,3.870)--(1.570,3.875)--(1.577,3.879)--(1.585,3.884)--(1.592,3.890)%
--(1.599,3.896)--(1.605,3.902)--(1.611,3.908)--(1.617,3.915)--(1.623,3.922)%
--(1.628,3.930)--(1.632,3.938)--(1.637,3.945)--(1.641,3.954)--(1.644,3.962)%
--(1.647,3.970)--(1.650,3.979)--(1.652,3.988)--(1.653,3.997)--(1.655,4.006)--(1.655,4.014)--cycle;
%
\gpfill{rgb color={0.000,0.000,0.000},opacity=0.15} (1.746,3.944)--(1.745,3.955)--(1.744,3.967)--(1.743,3.979)%
--(1.741,3.990)--(1.738,4.001)--(1.735,4.013)--(1.731,4.024)--(1.726,4.035)%
--(1.721,4.045)--(1.715,4.056)--(1.709,4.065)--(1.703,4.075)--(1.696,4.084)%
--(1.688,4.093)--(1.680,4.102)--(1.671,4.110)--(1.662,4.118)--(1.653,4.125)%
--(1.643,4.131)--(1.634,4.137)--(1.623,4.143)--(1.613,4.148)--(1.602,4.153)%
--(1.591,4.157)--(1.579,4.160)--(1.568,4.163)--(1.557,4.165)--(1.545,4.166)%
--(1.533,4.167)--(1.522,4.168)--(1.510,4.167)--(1.498,4.166)--(1.486,4.165)%
--(1.475,4.163)--(1.464,4.160)--(1.452,4.157)--(1.441,4.153)--(1.430,4.148)%
--(1.420,4.143)--(1.410,4.137)--(1.400,4.131)--(1.390,4.125)--(1.381,4.118)%
--(1.372,4.110)--(1.363,4.102)--(1.355,4.093)--(1.347,4.084)--(1.340,4.075)%
--(1.334,4.065)--(1.328,4.056)--(1.322,4.045)--(1.317,4.035)--(1.312,4.024)%
--(1.308,4.013)--(1.305,4.001)--(1.302,3.990)--(1.300,3.979)--(1.299,3.967)%
--(1.298,3.955)--(1.298,3.944)--(1.298,3.932)--(1.299,3.920)--(1.300,3.908)%
--(1.302,3.897)--(1.305,3.886)--(1.308,3.874)--(1.312,3.863)--(1.317,3.852)%
--(1.322,3.842)--(1.328,3.832)--(1.334,3.822)--(1.340,3.812)--(1.347,3.803)%
--(1.355,3.794)--(1.363,3.785)--(1.372,3.777)--(1.381,3.769)--(1.390,3.762)%
--(1.400,3.756)--(1.410,3.750)--(1.420,3.744)--(1.430,3.739)--(1.441,3.734)%
--(1.452,3.730)--(1.464,3.727)--(1.475,3.724)--(1.486,3.722)--(1.498,3.721)%
--(1.510,3.720)--(1.522,3.720)--(1.533,3.720)--(1.545,3.721)--(1.557,3.722)%
--(1.568,3.724)--(1.579,3.727)--(1.591,3.730)--(1.602,3.734)--(1.613,3.739)%
--(1.623,3.744)--(1.634,3.750)--(1.643,3.756)--(1.653,3.762)--(1.662,3.769)%
--(1.671,3.777)--(1.680,3.785)--(1.688,3.794)--(1.696,3.803)--(1.703,3.812)%
--(1.709,3.822)--(1.715,3.832)--(1.721,3.842)--(1.726,3.852)--(1.731,3.863)%
--(1.735,3.874)--(1.738,3.886)--(1.741,3.897)--(1.743,3.908)--(1.744,3.920)--(1.745,3.932)--cycle;
%
\gpfill{rgb color={0.000,0.000,0.000},opacity=0.15} (1.840,3.868)--(1.839,3.882)--(1.838,3.896)--(1.836,3.910)%
--(1.834,3.924)--(1.830,3.938)--(1.826,3.952)--(1.821,3.965)--(1.816,3.978)%
--(1.810,3.991)--(1.803,4.004)--(1.796,4.016)--(1.788,4.027)--(1.779,4.039)%
--(1.770,4.050)--(1.760,4.060)--(1.750,4.070)--(1.739,4.079)--(1.727,4.088)%
--(1.716,4.096)--(1.704,4.103)--(1.691,4.110)--(1.678,4.116)--(1.665,4.121)%
--(1.652,4.126)--(1.638,4.130)--(1.624,4.134)--(1.610,4.136)--(1.596,4.138)%
--(1.582,4.139)--(1.568,4.140)--(1.553,4.139)--(1.539,4.138)--(1.525,4.136)%
--(1.511,4.134)--(1.497,4.130)--(1.483,4.126)--(1.470,4.121)--(1.457,4.116)%
--(1.444,4.110)--(1.432,4.103)--(1.419,4.096)--(1.408,4.088)--(1.396,4.079)%
--(1.385,4.070)--(1.375,4.060)--(1.365,4.050)--(1.356,4.039)--(1.347,4.027)%
--(1.339,4.016)--(1.332,4.004)--(1.325,3.991)--(1.319,3.978)--(1.314,3.965)%
--(1.309,3.952)--(1.305,3.938)--(1.301,3.924)--(1.299,3.910)--(1.297,3.896)%
--(1.296,3.882)--(1.296,3.868)--(1.296,3.853)--(1.297,3.839)--(1.299,3.825)%
--(1.301,3.811)--(1.305,3.797)--(1.309,3.783)--(1.314,3.770)--(1.319,3.757)%
--(1.325,3.744)--(1.332,3.732)--(1.339,3.719)--(1.347,3.708)--(1.356,3.696)%
--(1.365,3.685)--(1.375,3.675)--(1.385,3.665)--(1.396,3.656)--(1.408,3.647)%
--(1.419,3.639)--(1.432,3.632)--(1.444,3.625)--(1.457,3.619)--(1.470,3.614)%
--(1.483,3.609)--(1.497,3.605)--(1.511,3.601)--(1.525,3.599)--(1.539,3.597)%
--(1.553,3.596)--(1.568,3.596)--(1.582,3.596)--(1.596,3.597)--(1.610,3.599)%
--(1.624,3.601)--(1.638,3.605)--(1.652,3.609)--(1.665,3.614)--(1.678,3.619)%
--(1.691,3.625)--(1.704,3.632)--(1.716,3.639)--(1.727,3.647)--(1.739,3.656)%
--(1.750,3.665)--(1.760,3.675)--(1.770,3.685)--(1.779,3.696)--(1.788,3.708)%
--(1.796,3.719)--(1.803,3.732)--(1.810,3.744)--(1.816,3.757)--(1.821,3.770)%
--(1.826,3.783)--(1.830,3.797)--(1.834,3.811)--(1.836,3.825)--(1.838,3.839)--(1.839,3.853)--cycle;
%
\gpfill{rgb color={0.000,0.000,0.000},opacity=0.15} (1.938,3.798)--(1.937,3.814)--(1.936,3.831)--(1.934,3.847)%
--(1.931,3.863)--(1.927,3.879)--(1.922,3.895)--(1.917,3.911)--(1.910,3.926)%
--(1.903,3.941)--(1.895,3.956)--(1.887,3.970)--(1.877,3.983)--(1.867,3.996)%
--(1.856,4.009)--(1.845,4.021)--(1.833,4.032)--(1.820,4.043)--(1.807,4.053)%
--(1.794,4.063)--(1.780,4.071)--(1.765,4.079)--(1.750,4.086)--(1.735,4.093)%
--(1.719,4.098)--(1.703,4.103)--(1.687,4.107)--(1.671,4.110)--(1.655,4.112)%
--(1.638,4.113)--(1.622,4.114)--(1.605,4.113)--(1.588,4.112)--(1.572,4.110)%
--(1.556,4.107)--(1.540,4.103)--(1.524,4.098)--(1.508,4.093)--(1.493,4.086)%
--(1.478,4.079)--(1.464,4.071)--(1.449,4.063)--(1.436,4.053)--(1.423,4.043)%
--(1.410,4.032)--(1.398,4.021)--(1.387,4.009)--(1.376,3.996)--(1.366,3.983)%
--(1.356,3.970)--(1.348,3.956)--(1.340,3.941)--(1.333,3.926)--(1.326,3.911)%
--(1.321,3.895)--(1.316,3.879)--(1.312,3.863)--(1.309,3.847)--(1.307,3.831)%
--(1.306,3.814)--(1.306,3.798)--(1.306,3.781)--(1.307,3.764)--(1.309,3.748)%
--(1.312,3.732)--(1.316,3.716)--(1.321,3.700)--(1.326,3.684)--(1.333,3.669)%
--(1.340,3.654)--(1.348,3.640)--(1.356,3.625)--(1.366,3.612)--(1.376,3.599)%
--(1.387,3.586)--(1.398,3.574)--(1.410,3.563)--(1.423,3.552)--(1.436,3.542)%
--(1.449,3.532)--(1.463,3.524)--(1.478,3.516)--(1.493,3.509)--(1.508,3.502)%
--(1.524,3.497)--(1.540,3.492)--(1.556,3.488)--(1.572,3.485)--(1.588,3.483)%
--(1.605,3.482)--(1.622,3.482)--(1.638,3.482)--(1.655,3.483)--(1.671,3.485)%
--(1.687,3.488)--(1.703,3.492)--(1.719,3.497)--(1.735,3.502)--(1.750,3.509)%
--(1.765,3.516)--(1.780,3.524)--(1.794,3.532)--(1.807,3.542)--(1.820,3.552)%
--(1.833,3.563)--(1.845,3.574)--(1.856,3.586)--(1.867,3.599)--(1.877,3.612)%
--(1.887,3.625)--(1.895,3.640)--(1.903,3.654)--(1.910,3.669)--(1.917,3.684)%
--(1.922,3.700)--(1.927,3.716)--(1.931,3.732)--(1.934,3.748)--(1.936,3.764)--(1.937,3.781)--cycle;
%
\gpfill{rgb color={0.000,0.000,0.000},opacity=0.15} (2.038,3.733)--(2.037,3.751)--(2.036,3.770)--(2.033,3.788)%
--(2.030,3.806)--(2.025,3.824)--(2.020,3.842)--(2.014,3.860)--(2.007,3.877)%
--(1.999,3.894)--(1.990,3.910)--(1.980,3.926)--(1.970,3.941)--(1.958,3.956)%
--(1.946,3.970)--(1.934,3.984)--(1.920,3.996)--(1.906,4.008)--(1.891,4.020)%
--(1.876,4.030)--(1.860,4.040)--(1.844,4.049)--(1.827,4.057)--(1.810,4.064)%
--(1.792,4.070)--(1.774,4.075)--(1.756,4.080)--(1.738,4.083)--(1.720,4.086)%
--(1.701,4.087)--(1.683,4.088)--(1.664,4.087)--(1.645,4.086)--(1.627,4.083)%
--(1.609,4.080)--(1.591,4.075)--(1.573,4.070)--(1.555,4.064)--(1.538,4.057)%
--(1.521,4.049)--(1.505,4.040)--(1.489,4.030)--(1.474,4.020)--(1.459,4.008)%
--(1.445,3.996)--(1.431,3.984)--(1.419,3.970)--(1.407,3.956)--(1.395,3.941)%
--(1.385,3.926)--(1.375,3.910)--(1.366,3.894)--(1.358,3.877)--(1.351,3.860)%
--(1.345,3.842)--(1.340,3.824)--(1.335,3.806)--(1.332,3.788)--(1.329,3.770)%
--(1.328,3.751)--(1.328,3.733)--(1.328,3.714)--(1.329,3.695)--(1.332,3.677)%
--(1.335,3.659)--(1.340,3.641)--(1.345,3.623)--(1.351,3.605)--(1.358,3.588)%
--(1.366,3.571)--(1.375,3.555)--(1.385,3.539)--(1.395,3.524)--(1.407,3.509)%
--(1.419,3.495)--(1.431,3.481)--(1.445,3.469)--(1.459,3.457)--(1.474,3.445)%
--(1.489,3.435)--(1.505,3.425)--(1.521,3.416)--(1.538,3.408)--(1.555,3.401)%
--(1.573,3.395)--(1.591,3.390)--(1.609,3.385)--(1.627,3.382)--(1.645,3.379)%
--(1.664,3.378)--(1.683,3.378)--(1.701,3.378)--(1.720,3.379)--(1.738,3.382)%
--(1.756,3.385)--(1.774,3.390)--(1.792,3.395)--(1.810,3.401)--(1.827,3.408)%
--(1.844,3.416)--(1.860,3.425)--(1.876,3.435)--(1.891,3.445)--(1.906,3.457)%
--(1.920,3.469)--(1.934,3.481)--(1.946,3.495)--(1.958,3.509)--(1.970,3.524)%
--(1.980,3.539)--(1.990,3.555)--(1.999,3.571)--(2.007,3.588)--(2.014,3.605)%
--(2.020,3.623)--(2.025,3.641)--(2.030,3.659)--(2.033,3.677)--(2.036,3.695)--(2.037,3.714)--cycle;
%
\gpfill{rgb color={0.000,0.000,0.000},opacity=0.15} (2.143,3.674)--(2.142,3.694)--(2.140,3.714)--(2.138,3.735)%
--(2.134,3.755)--(2.129,3.774)--(2.123,3.794)--(2.117,3.813)--(2.109,3.832)%
--(2.100,3.851)--(2.090,3.869)--(2.080,3.886)--(2.068,3.903)--(2.056,3.919)%
--(2.042,3.934)--(2.028,3.949)--(2.013,3.963)--(1.998,3.977)--(1.982,3.989)%
--(1.965,4.001)--(1.948,4.011)--(1.930,4.021)--(1.911,4.030)--(1.892,4.038)%
--(1.873,4.044)--(1.853,4.050)--(1.834,4.055)--(1.814,4.059)--(1.793,4.061)%
--(1.773,4.063)--(1.753,4.064)--(1.732,4.063)--(1.712,4.061)--(1.691,4.059)%
--(1.671,4.055)--(1.652,4.050)--(1.632,4.044)--(1.613,4.038)--(1.594,4.030)%
--(1.575,4.021)--(1.558,4.011)--(1.540,4.001)--(1.523,3.989)--(1.507,3.977)%
--(1.492,3.963)--(1.477,3.949)--(1.463,3.934)--(1.449,3.919)--(1.437,3.903)%
--(1.425,3.886)--(1.415,3.869)--(1.405,3.851)--(1.396,3.832)--(1.388,3.813)%
--(1.382,3.794)--(1.376,3.774)--(1.371,3.755)--(1.367,3.735)--(1.365,3.714)%
--(1.363,3.694)--(1.363,3.674)--(1.363,3.653)--(1.365,3.633)--(1.367,3.612)%
--(1.371,3.592)--(1.376,3.573)--(1.382,3.553)--(1.388,3.534)--(1.396,3.515)%
--(1.405,3.496)--(1.415,3.479)--(1.425,3.461)--(1.437,3.444)--(1.449,3.428)%
--(1.463,3.413)--(1.477,3.398)--(1.492,3.384)--(1.507,3.370)--(1.523,3.358)%
--(1.540,3.346)--(1.557,3.336)--(1.575,3.326)--(1.594,3.317)--(1.613,3.309)%
--(1.632,3.303)--(1.652,3.297)--(1.671,3.292)--(1.691,3.288)--(1.712,3.286)%
--(1.732,3.284)--(1.753,3.284)--(1.773,3.284)--(1.793,3.286)--(1.814,3.288)%
--(1.834,3.292)--(1.853,3.297)--(1.873,3.303)--(1.892,3.309)--(1.911,3.317)%
--(1.930,3.326)--(1.948,3.336)--(1.965,3.346)--(1.982,3.358)--(1.998,3.370)%
--(2.013,3.384)--(2.028,3.398)--(2.042,3.413)--(2.056,3.428)--(2.068,3.444)%
--(2.080,3.461)--(2.090,3.479)--(2.100,3.496)--(2.109,3.515)--(2.117,3.534)%
--(2.123,3.553)--(2.129,3.573)--(2.134,3.592)--(2.138,3.612)--(2.140,3.633)--(2.142,3.653)--cycle;
%
\gpfill{rgb color={0.000,0.000,0.000},opacity=0.15} (2.247,3.622)--(2.246,3.643)--(2.244,3.665)--(2.241,3.687)%
--(2.237,3.708)--(2.232,3.729)--(2.226,3.750)--(2.219,3.771)--(2.211,3.791)%
--(2.201,3.810)--(2.191,3.830)--(2.179,3.848)--(2.167,3.866)--(2.154,3.883)%
--(2.140,3.900)--(2.125,3.916)--(2.109,3.931)--(2.092,3.945)--(2.075,3.958)%
--(2.057,3.970)--(2.039,3.982)--(2.019,3.992)--(2.000,4.002)--(1.980,4.010)%
--(1.959,4.017)--(1.938,4.023)--(1.917,4.028)--(1.896,4.032)--(1.874,4.035)%
--(1.852,4.037)--(1.831,4.038)--(1.809,4.037)--(1.787,4.035)--(1.765,4.032)%
--(1.744,4.028)--(1.723,4.023)--(1.702,4.017)--(1.681,4.010)--(1.661,4.002)%
--(1.642,3.992)--(1.623,3.982)--(1.604,3.970)--(1.586,3.958)--(1.569,3.945)%
--(1.552,3.931)--(1.536,3.916)--(1.521,3.900)--(1.507,3.883)--(1.494,3.866)%
--(1.482,3.848)--(1.470,3.830)--(1.460,3.810)--(1.450,3.791)--(1.442,3.771)%
--(1.435,3.750)--(1.429,3.729)--(1.424,3.708)--(1.420,3.687)--(1.417,3.665)%
--(1.415,3.643)--(1.415,3.622)--(1.415,3.600)--(1.417,3.578)--(1.420,3.556)%
--(1.424,3.535)--(1.429,3.514)--(1.435,3.493)--(1.442,3.472)--(1.450,3.452)%
--(1.460,3.433)--(1.470,3.414)--(1.482,3.395)--(1.494,3.377)--(1.507,3.360)%
--(1.521,3.343)--(1.536,3.327)--(1.552,3.312)--(1.569,3.298)--(1.586,3.285)%
--(1.604,3.273)--(1.622,3.261)--(1.642,3.251)--(1.661,3.241)--(1.681,3.233)%
--(1.702,3.226)--(1.723,3.220)--(1.744,3.215)--(1.765,3.211)--(1.787,3.208)%
--(1.809,3.206)--(1.831,3.206)--(1.852,3.206)--(1.874,3.208)--(1.896,3.211)%
--(1.917,3.215)--(1.938,3.220)--(1.959,3.226)--(1.980,3.233)--(2.000,3.241)%
--(2.019,3.251)--(2.039,3.261)--(2.057,3.273)--(2.075,3.285)--(2.092,3.298)%
--(2.109,3.312)--(2.125,3.327)--(2.140,3.343)--(2.154,3.360)--(2.167,3.377)%
--(2.179,3.395)--(2.191,3.414)--(2.201,3.433)--(2.211,3.452)--(2.219,3.472)%
--(2.226,3.493)--(2.232,3.514)--(2.237,3.535)--(2.241,3.556)--(2.244,3.578)--(2.246,3.600)--cycle;
%
\gpfill{rgb color={0.000,0.000,0.000},opacity=0.15} (2.353,3.577)--(2.352,3.599)--(2.350,3.622)--(2.347,3.645)%
--(2.343,3.667)--(2.338,3.690)--(2.331,3.712)--(2.323,3.733)--(2.315,3.754)%
--(2.305,3.775)--(2.294,3.795)--(2.282,3.815)--(2.269,3.833)--(2.255,3.852)%
--(2.240,3.869)--(2.225,3.886)--(2.208,3.901)--(2.191,3.916)--(2.172,3.930)%
--(2.154,3.943)--(2.134,3.955)--(2.114,3.966)--(2.093,3.976)--(2.072,3.984)%
--(2.051,3.992)--(2.029,3.999)--(2.006,4.004)--(1.984,4.008)--(1.961,4.011)%
--(1.938,4.013)--(1.916,4.014)--(1.893,4.013)--(1.870,4.011)--(1.847,4.008)%
--(1.825,4.004)--(1.802,3.999)--(1.780,3.992)--(1.759,3.984)--(1.738,3.976)%
--(1.717,3.966)--(1.697,3.955)--(1.677,3.943)--(1.659,3.930)--(1.640,3.916)%
--(1.623,3.901)--(1.606,3.886)--(1.591,3.869)--(1.576,3.852)--(1.562,3.833)%
--(1.549,3.815)--(1.537,3.795)--(1.526,3.775)--(1.516,3.754)--(1.508,3.733)%
--(1.500,3.712)--(1.493,3.690)--(1.488,3.667)--(1.484,3.645)--(1.481,3.622)%
--(1.479,3.599)--(1.479,3.577)--(1.479,3.554)--(1.481,3.531)--(1.484,3.508)%
--(1.488,3.486)--(1.493,3.463)--(1.500,3.441)--(1.508,3.420)--(1.516,3.399)%
--(1.526,3.378)--(1.537,3.358)--(1.549,3.338)--(1.562,3.320)--(1.576,3.301)%
--(1.591,3.284)--(1.606,3.267)--(1.623,3.252)--(1.640,3.237)--(1.659,3.223)%
--(1.677,3.210)--(1.697,3.198)--(1.717,3.187)--(1.738,3.177)--(1.759,3.169)%
--(1.780,3.161)--(1.802,3.154)--(1.825,3.149)--(1.847,3.145)--(1.870,3.142)%
--(1.893,3.140)--(1.916,3.140)--(1.938,3.140)--(1.961,3.142)--(1.984,3.145)%
--(2.006,3.149)--(2.029,3.154)--(2.051,3.161)--(2.072,3.169)--(2.093,3.177)%
--(2.114,3.187)--(2.134,3.198)--(2.154,3.210)--(2.172,3.223)--(2.191,3.237)%
--(2.208,3.252)--(2.225,3.267)--(2.240,3.284)--(2.255,3.301)--(2.269,3.320)%
--(2.282,3.338)--(2.294,3.358)--(2.305,3.378)--(2.315,3.399)--(2.323,3.420)%
--(2.331,3.441)--(2.338,3.463)--(2.343,3.486)--(2.347,3.508)--(2.350,3.531)--(2.352,3.554)--cycle;
%
\gpfill{rgb color={0.000,0.000,0.000},opacity=0.15} (2.462,3.538)--(2.461,3.561)--(2.459,3.585)--(2.456,3.608)%
--(2.452,3.632)--(2.446,3.655)--(2.439,3.677)--(2.431,3.700)--(2.422,3.722)%
--(2.412,3.743)--(2.401,3.764)--(2.388,3.784)--(2.375,3.804)--(2.361,3.823)%
--(2.345,3.841)--(2.329,3.858)--(2.312,3.874)--(2.294,3.890)--(2.275,3.904)%
--(2.255,3.917)--(2.235,3.930)--(2.214,3.941)--(2.193,3.951)--(2.171,3.960)%
--(2.148,3.968)--(2.126,3.975)--(2.103,3.981)--(2.079,3.985)--(2.056,3.988)%
--(2.032,3.990)--(2.009,3.991)--(1.985,3.990)--(1.961,3.988)--(1.938,3.985)%
--(1.914,3.981)--(1.891,3.975)--(1.869,3.968)--(1.846,3.960)--(1.824,3.951)%
--(1.803,3.941)--(1.782,3.930)--(1.762,3.917)--(1.742,3.904)--(1.723,3.890)%
--(1.705,3.874)--(1.688,3.858)--(1.672,3.841)--(1.656,3.823)--(1.642,3.804)%
--(1.629,3.784)--(1.616,3.764)--(1.605,3.743)--(1.595,3.722)--(1.586,3.700)%
--(1.578,3.677)--(1.571,3.655)--(1.565,3.632)--(1.561,3.608)--(1.558,3.585)%
--(1.556,3.561)--(1.556,3.538)--(1.556,3.514)--(1.558,3.490)--(1.561,3.467)%
--(1.565,3.443)--(1.571,3.420)--(1.578,3.398)--(1.586,3.375)--(1.595,3.353)%
--(1.605,3.332)--(1.616,3.311)--(1.629,3.291)--(1.642,3.271)--(1.656,3.252)%
--(1.672,3.234)--(1.688,3.217)--(1.705,3.201)--(1.723,3.185)--(1.742,3.171)%
--(1.762,3.158)--(1.782,3.145)--(1.803,3.134)--(1.824,3.124)--(1.846,3.115)%
--(1.869,3.107)--(1.891,3.100)--(1.914,3.094)--(1.938,3.090)--(1.961,3.087)%
--(1.985,3.085)--(2.009,3.085)--(2.032,3.085)--(2.056,3.087)--(2.079,3.090)%
--(2.103,3.094)--(2.126,3.100)--(2.148,3.107)--(2.171,3.115)--(2.193,3.124)%
--(2.214,3.134)--(2.235,3.145)--(2.255,3.158)--(2.275,3.171)--(2.294,3.185)%
--(2.312,3.201)--(2.329,3.217)--(2.345,3.234)--(2.361,3.252)--(2.375,3.271)%
--(2.388,3.291)--(2.401,3.311)--(2.412,3.332)--(2.422,3.353)--(2.431,3.375)%
--(2.439,3.398)--(2.446,3.420)--(2.452,3.443)--(2.456,3.467)--(2.459,3.490)--(2.461,3.514)--cycle;
%
\gpfill{rgb color={0.000,0.000,0.000},opacity=0.15} (2.575,3.505)--(2.574,3.529)--(2.572,3.553)--(2.569,3.577)%
--(2.564,3.601)--(2.559,3.625)--(2.552,3.648)--(2.544,3.671)--(2.534,3.694)%
--(2.524,3.716)--(2.512,3.737)--(2.499,3.758)--(2.486,3.778)--(2.471,3.797)%
--(2.455,3.816)--(2.438,3.833)--(2.421,3.850)--(2.402,3.866)--(2.383,3.881)%
--(2.363,3.894)--(2.342,3.907)--(2.321,3.919)--(2.299,3.929)--(2.276,3.939)%
--(2.253,3.947)--(2.230,3.954)--(2.206,3.959)--(2.182,3.964)--(2.158,3.967)%
--(2.134,3.969)--(2.110,3.970)--(2.085,3.969)--(2.061,3.967)--(2.037,3.964)%
--(2.013,3.959)--(1.989,3.954)--(1.966,3.947)--(1.943,3.939)--(1.920,3.929)%
--(1.898,3.919)--(1.877,3.907)--(1.856,3.894)--(1.836,3.881)--(1.817,3.866)%
--(1.798,3.850)--(1.781,3.833)--(1.764,3.816)--(1.748,3.797)--(1.733,3.778)%
--(1.720,3.758)--(1.707,3.737)--(1.695,3.716)--(1.685,3.694)--(1.675,3.671)%
--(1.667,3.648)--(1.660,3.625)--(1.655,3.601)--(1.650,3.577)--(1.647,3.553)%
--(1.645,3.529)--(1.645,3.505)--(1.645,3.480)--(1.647,3.456)--(1.650,3.432)%
--(1.655,3.408)--(1.660,3.384)--(1.667,3.361)--(1.675,3.338)--(1.685,3.315)%
--(1.695,3.293)--(1.707,3.272)--(1.720,3.251)--(1.733,3.231)--(1.748,3.212)%
--(1.764,3.193)--(1.781,3.176)--(1.798,3.159)--(1.817,3.143)--(1.836,3.128)%
--(1.856,3.115)--(1.877,3.102)--(1.898,3.090)--(1.920,3.080)--(1.943,3.070)%
--(1.966,3.062)--(1.989,3.055)--(2.013,3.050)--(2.037,3.045)--(2.061,3.042)%
--(2.085,3.040)--(2.110,3.040)--(2.134,3.040)--(2.158,3.042)--(2.182,3.045)%
--(2.206,3.050)--(2.230,3.055)--(2.253,3.062)--(2.276,3.070)--(2.299,3.080)%
--(2.321,3.090)--(2.342,3.102)--(2.363,3.115)--(2.383,3.128)--(2.402,3.143)%
--(2.421,3.159)--(2.438,3.176)--(2.455,3.193)--(2.471,3.212)--(2.486,3.231)%
--(2.499,3.251)--(2.512,3.272)--(2.524,3.293)--(2.534,3.315)--(2.544,3.338)%
--(2.552,3.361)--(2.559,3.384)--(2.564,3.408)--(2.569,3.432)--(2.572,3.456)--(2.574,3.480)--cycle;
%
\gpfill{rgb color={0.000,0.000,0.000},opacity=0.15} (2.688,3.479)--(2.687,3.503)--(2.685,3.528)--(2.682,3.552)%
--(2.677,3.576)--(2.671,3.600)--(2.664,3.624)--(2.656,3.647)--(2.647,3.670)%
--(2.636,3.692)--(2.624,3.714)--(2.612,3.735)--(2.598,3.755)--(2.583,3.775)%
--(2.567,3.794)--(2.550,3.812)--(2.532,3.829)--(2.513,3.845)--(2.493,3.860)%
--(2.473,3.874)--(2.452,3.886)--(2.430,3.898)--(2.408,3.909)--(2.385,3.918)%
--(2.362,3.926)--(2.338,3.933)--(2.314,3.939)--(2.290,3.944)--(2.266,3.947)%
--(2.241,3.949)--(2.217,3.950)--(2.192,3.949)--(2.167,3.947)--(2.143,3.944)%
--(2.119,3.939)--(2.095,3.933)--(2.071,3.926)--(2.048,3.918)--(2.025,3.909)%
--(2.003,3.898)--(1.981,3.886)--(1.960,3.874)--(1.940,3.860)--(1.920,3.845)%
--(1.901,3.829)--(1.883,3.812)--(1.866,3.794)--(1.850,3.775)--(1.835,3.755)%
--(1.821,3.735)--(1.809,3.714)--(1.797,3.692)--(1.786,3.670)--(1.777,3.647)%
--(1.769,3.624)--(1.762,3.600)--(1.756,3.576)--(1.751,3.552)--(1.748,3.528)%
--(1.746,3.503)--(1.746,3.479)--(1.746,3.454)--(1.748,3.429)--(1.751,3.405)%
--(1.756,3.381)--(1.762,3.357)--(1.769,3.333)--(1.777,3.310)--(1.786,3.287)%
--(1.797,3.265)--(1.809,3.243)--(1.821,3.222)--(1.835,3.202)--(1.850,3.182)%
--(1.866,3.163)--(1.883,3.145)--(1.901,3.128)--(1.920,3.112)--(1.940,3.097)%
--(1.960,3.083)--(1.981,3.071)--(2.003,3.059)--(2.025,3.048)--(2.048,3.039)%
--(2.071,3.031)--(2.095,3.024)--(2.119,3.018)--(2.143,3.013)--(2.167,3.010)%
--(2.192,3.008)--(2.217,3.008)--(2.241,3.008)--(2.266,3.010)--(2.290,3.013)%
--(2.314,3.018)--(2.338,3.024)--(2.362,3.031)--(2.385,3.039)--(2.408,3.048)%
--(2.430,3.059)--(2.452,3.071)--(2.473,3.083)--(2.493,3.097)--(2.513,3.112)%
--(2.532,3.128)--(2.550,3.145)--(2.567,3.163)--(2.583,3.182)--(2.598,3.202)%
--(2.612,3.222)--(2.624,3.243)--(2.636,3.265)--(2.647,3.287)--(2.656,3.310)%
--(2.664,3.333)--(2.671,3.357)--(2.677,3.381)--(2.682,3.405)--(2.685,3.429)--(2.687,3.454)--cycle;
%
\gpfill{rgb color={0.000,0.000,0.000},opacity=0.15} (2.804,3.459)--(2.803,3.483)--(2.801,3.508)--(2.798,3.532)%
--(2.793,3.557)--(2.787,3.581)--(2.780,3.605)--(2.772,3.628)--(2.763,3.651)%
--(2.752,3.673)--(2.740,3.695)--(2.727,3.716)--(2.713,3.737)--(2.698,3.756)%
--(2.682,3.775)--(2.665,3.793)--(2.647,3.810)--(2.628,3.826)--(2.609,3.841)%
--(2.588,3.855)--(2.567,3.868)--(2.545,3.880)--(2.523,3.891)--(2.500,3.900)%
--(2.477,3.908)--(2.453,3.915)--(2.429,3.921)--(2.404,3.926)--(2.380,3.929)%
--(2.355,3.931)--(2.331,3.932)--(2.306,3.931)--(2.281,3.929)--(2.257,3.926)%
--(2.232,3.921)--(2.208,3.915)--(2.184,3.908)--(2.161,3.900)--(2.138,3.891)%
--(2.116,3.880)--(2.094,3.868)--(2.073,3.855)--(2.052,3.841)--(2.033,3.826)%
--(2.014,3.810)--(1.996,3.793)--(1.979,3.775)--(1.963,3.756)--(1.948,3.737)%
--(1.934,3.716)--(1.921,3.695)--(1.909,3.673)--(1.898,3.651)--(1.889,3.628)%
--(1.881,3.605)--(1.874,3.581)--(1.868,3.557)--(1.863,3.532)--(1.860,3.508)%
--(1.858,3.483)--(1.858,3.459)--(1.858,3.434)--(1.860,3.409)--(1.863,3.385)%
--(1.868,3.360)--(1.874,3.336)--(1.881,3.312)--(1.889,3.289)--(1.898,3.266)%
--(1.909,3.244)--(1.921,3.222)--(1.934,3.201)--(1.948,3.180)--(1.963,3.161)%
--(1.979,3.142)--(1.996,3.124)--(2.014,3.107)--(2.033,3.091)--(2.052,3.076)%
--(2.073,3.062)--(2.094,3.049)--(2.116,3.037)--(2.138,3.026)--(2.161,3.017)%
--(2.184,3.009)--(2.208,3.002)--(2.232,2.996)--(2.257,2.991)--(2.281,2.988)%
--(2.306,2.986)--(2.331,2.986)--(2.355,2.986)--(2.380,2.988)--(2.404,2.991)%
--(2.429,2.996)--(2.453,3.002)--(2.477,3.009)--(2.500,3.017)--(2.523,3.026)%
--(2.545,3.037)--(2.567,3.049)--(2.588,3.062)--(2.609,3.076)--(2.628,3.091)%
--(2.647,3.107)--(2.665,3.124)--(2.682,3.142)--(2.698,3.161)--(2.713,3.180)%
--(2.727,3.201)--(2.740,3.222)--(2.752,3.244)--(2.763,3.266)--(2.772,3.289)%
--(2.780,3.312)--(2.787,3.336)--(2.793,3.360)--(2.798,3.385)--(2.801,3.409)--(2.803,3.434)--cycle;
%
\gpfill{rgb color={0.000,0.000,0.000},opacity=0.15} (2.925,3.445)--(2.924,3.469)--(2.922,3.494)--(2.919,3.518)%
--(2.914,3.543)--(2.908,3.567)--(2.901,3.590)--(2.893,3.614)--(2.884,3.636)%
--(2.873,3.659)--(2.861,3.681)--(2.848,3.702)--(2.834,3.722)--(2.819,3.742)%
--(2.803,3.760)--(2.786,3.778)--(2.768,3.795)--(2.750,3.811)--(2.730,3.826)%
--(2.710,3.840)--(2.689,3.853)--(2.667,3.865)--(2.644,3.876)--(2.622,3.885)%
--(2.598,3.893)--(2.575,3.900)--(2.551,3.906)--(2.526,3.911)--(2.502,3.914)%
--(2.477,3.916)--(2.453,3.917)--(2.428,3.916)--(2.403,3.914)--(2.379,3.911)%
--(2.354,3.906)--(2.330,3.900)--(2.307,3.893)--(2.283,3.885)--(2.261,3.876)%
--(2.238,3.865)--(2.217,3.853)--(2.195,3.840)--(2.175,3.826)--(2.155,3.811)%
--(2.137,3.795)--(2.119,3.778)--(2.102,3.760)--(2.086,3.742)--(2.071,3.722)%
--(2.057,3.702)--(2.044,3.681)--(2.032,3.659)--(2.021,3.636)--(2.012,3.614)%
--(2.004,3.590)--(1.997,3.567)--(1.991,3.543)--(1.986,3.518)--(1.983,3.494)%
--(1.981,3.469)--(1.981,3.445)--(1.981,3.420)--(1.983,3.395)--(1.986,3.371)%
--(1.991,3.346)--(1.997,3.322)--(2.004,3.299)--(2.012,3.275)--(2.021,3.253)%
--(2.032,3.230)--(2.044,3.209)--(2.057,3.187)--(2.071,3.167)--(2.086,3.147)%
--(2.102,3.129)--(2.119,3.111)--(2.137,3.094)--(2.155,3.078)--(2.175,3.063)%
--(2.195,3.049)--(2.217,3.036)--(2.238,3.024)--(2.261,3.013)--(2.283,3.004)%
--(2.307,2.996)--(2.330,2.989)--(2.354,2.983)--(2.379,2.978)--(2.403,2.975)%
--(2.428,2.973)--(2.453,2.973)--(2.477,2.973)--(2.502,2.975)--(2.526,2.978)%
--(2.551,2.983)--(2.575,2.989)--(2.598,2.996)--(2.622,3.004)--(2.644,3.013)%
--(2.667,3.024)--(2.689,3.036)--(2.710,3.049)--(2.730,3.063)--(2.750,3.078)%
--(2.768,3.094)--(2.786,3.111)--(2.803,3.129)--(2.819,3.147)--(2.834,3.167)%
--(2.848,3.187)--(2.861,3.209)--(2.873,3.230)--(2.884,3.253)--(2.893,3.275)%
--(2.901,3.299)--(2.908,3.322)--(2.914,3.346)--(2.919,3.371)--(2.922,3.395)--(2.924,3.420)--cycle;
%
\gpfill{rgb color={0.000,0.000,0.000},opacity=0.15} (3.043,3.435)--(3.042,3.459)--(3.040,3.483)--(3.037,3.507)%
--(3.032,3.531)--(3.027,3.554)--(3.020,3.578)--(3.012,3.600)--(3.002,3.623)%
--(2.992,3.645)--(2.980,3.666)--(2.968,3.687)--(2.954,3.707)--(2.939,3.726)%
--(2.924,3.744)--(2.907,3.762)--(2.889,3.779)--(2.871,3.794)--(2.852,3.809)%
--(2.832,3.823)--(2.811,3.835)--(2.790,3.847)--(2.768,3.857)--(2.745,3.867)%
--(2.723,3.875)--(2.699,3.882)--(2.676,3.887)--(2.652,3.892)--(2.628,3.895)%
--(2.604,3.897)--(2.580,3.898)--(2.555,3.897)--(2.531,3.895)--(2.507,3.892)%
--(2.483,3.887)--(2.460,3.882)--(2.436,3.875)--(2.414,3.867)--(2.391,3.857)%
--(2.369,3.847)--(2.348,3.835)--(2.327,3.823)--(2.307,3.809)--(2.288,3.794)%
--(2.270,3.779)--(2.252,3.762)--(2.235,3.744)--(2.220,3.726)--(2.205,3.707)%
--(2.191,3.687)--(2.179,3.666)--(2.167,3.645)--(2.157,3.623)--(2.147,3.600)%
--(2.139,3.578)--(2.132,3.554)--(2.127,3.531)--(2.122,3.507)--(2.119,3.483)%
--(2.117,3.459)--(2.117,3.435)--(2.117,3.410)--(2.119,3.386)--(2.122,3.362)%
--(2.127,3.338)--(2.132,3.315)--(2.139,3.291)--(2.147,3.269)--(2.157,3.246)%
--(2.167,3.224)--(2.179,3.203)--(2.191,3.182)--(2.205,3.162)--(2.220,3.143)%
--(2.235,3.125)--(2.252,3.107)--(2.270,3.090)--(2.288,3.075)--(2.307,3.060)%
--(2.327,3.046)--(2.348,3.034)--(2.369,3.022)--(2.391,3.012)--(2.414,3.002)%
--(2.436,2.994)--(2.460,2.987)--(2.483,2.982)--(2.507,2.977)--(2.531,2.974)%
--(2.555,2.972)--(2.580,2.972)--(2.604,2.972)--(2.628,2.974)--(2.652,2.977)%
--(2.676,2.982)--(2.699,2.987)--(2.723,2.994)--(2.745,3.002)--(2.768,3.012)%
--(2.790,3.022)--(2.811,3.034)--(2.832,3.046)--(2.852,3.060)--(2.871,3.075)%
--(2.889,3.090)--(2.907,3.107)--(2.924,3.125)--(2.939,3.143)--(2.954,3.162)%
--(2.968,3.182)--(2.980,3.203)--(2.992,3.224)--(3.002,3.246)--(3.012,3.269)%
--(3.020,3.291)--(3.027,3.315)--(3.032,3.338)--(3.037,3.362)--(3.040,3.386)--(3.042,3.410)--cycle;
%
\gpfill{rgb color={0.000,0.000,0.000},opacity=0.15} (3.165,3.430)--(3.164,3.453)--(3.162,3.477)--(3.159,3.500)%
--(3.155,3.523)--(3.149,3.546)--(3.142,3.569)--(3.135,3.591)--(3.126,3.613)%
--(3.115,3.634)--(3.104,3.655)--(3.092,3.675)--(3.078,3.695)--(3.064,3.713)%
--(3.049,3.731)--(3.032,3.748)--(3.015,3.765)--(2.997,3.780)--(2.979,3.794)%
--(2.959,3.808)--(2.939,3.820)--(2.918,3.831)--(2.897,3.842)--(2.875,3.851)%
--(2.853,3.858)--(2.830,3.865)--(2.807,3.871)--(2.784,3.875)--(2.761,3.878)%
--(2.737,3.880)--(2.714,3.881)--(2.690,3.880)--(2.666,3.878)--(2.643,3.875)%
--(2.620,3.871)--(2.597,3.865)--(2.574,3.858)--(2.552,3.851)--(2.530,3.842)%
--(2.509,3.831)--(2.488,3.820)--(2.468,3.808)--(2.448,3.794)--(2.430,3.780)%
--(2.412,3.765)--(2.395,3.748)--(2.378,3.731)--(2.363,3.713)--(2.349,3.695)%
--(2.335,3.675)--(2.323,3.655)--(2.312,3.634)--(2.301,3.613)--(2.292,3.591)%
--(2.285,3.569)--(2.278,3.546)--(2.272,3.523)--(2.268,3.500)--(2.265,3.477)%
--(2.263,3.453)--(2.263,3.430)--(2.263,3.406)--(2.265,3.382)--(2.268,3.359)%
--(2.272,3.336)--(2.278,3.313)--(2.285,3.290)--(2.292,3.268)--(2.301,3.246)%
--(2.312,3.225)--(2.323,3.204)--(2.335,3.184)--(2.349,3.164)--(2.363,3.146)%
--(2.378,3.128)--(2.395,3.111)--(2.412,3.094)--(2.430,3.079)--(2.448,3.065)%
--(2.468,3.051)--(2.488,3.039)--(2.509,3.028)--(2.530,3.017)--(2.552,3.008)%
--(2.574,3.001)--(2.597,2.994)--(2.620,2.988)--(2.643,2.984)--(2.666,2.981)%
--(2.690,2.979)--(2.714,2.979)--(2.737,2.979)--(2.761,2.981)--(2.784,2.984)%
--(2.807,2.988)--(2.830,2.994)--(2.853,3.001)--(2.875,3.008)--(2.897,3.017)%
--(2.918,3.028)--(2.939,3.039)--(2.959,3.051)--(2.979,3.065)--(2.997,3.079)%
--(3.015,3.094)--(3.032,3.111)--(3.049,3.128)--(3.064,3.146)--(3.078,3.164)%
--(3.092,3.184)--(3.104,3.204)--(3.115,3.225)--(3.126,3.246)--(3.135,3.268)%
--(3.142,3.290)--(3.149,3.313)--(3.155,3.336)--(3.159,3.359)--(3.162,3.382)--(3.164,3.406)--cycle;
%
\gpfill{rgb color={0.000,0.000,0.000},opacity=0.15} (3.293,3.429)--(3.292,3.451)--(3.290,3.474)--(3.287,3.497)%
--(3.283,3.520)--(3.278,3.542)--(3.271,3.564)--(3.263,3.585)--(3.255,3.607)%
--(3.245,3.627)--(3.234,3.648)--(3.222,3.667)--(3.209,3.686)--(3.195,3.704)%
--(3.180,3.722)--(3.164,3.738)--(3.148,3.754)--(3.130,3.769)--(3.112,3.783)%
--(3.093,3.796)--(3.074,3.808)--(3.053,3.819)--(3.033,3.829)--(3.011,3.837)%
--(2.990,3.845)--(2.968,3.852)--(2.946,3.857)--(2.923,3.861)--(2.900,3.864)%
--(2.877,3.866)--(2.855,3.867)--(2.832,3.866)--(2.809,3.864)--(2.786,3.861)%
--(2.763,3.857)--(2.741,3.852)--(2.719,3.845)--(2.698,3.837)--(2.676,3.829)%
--(2.656,3.819)--(2.636,3.808)--(2.616,3.796)--(2.597,3.783)--(2.579,3.769)%
--(2.561,3.754)--(2.545,3.738)--(2.529,3.722)--(2.514,3.704)--(2.500,3.686)%
--(2.487,3.667)--(2.475,3.648)--(2.464,3.627)--(2.454,3.607)--(2.446,3.585)%
--(2.438,3.564)--(2.431,3.542)--(2.426,3.520)--(2.422,3.497)--(2.419,3.474)%
--(2.417,3.451)--(2.417,3.429)--(2.417,3.406)--(2.419,3.383)--(2.422,3.360)%
--(2.426,3.337)--(2.431,3.315)--(2.438,3.293)--(2.446,3.272)--(2.454,3.250)%
--(2.464,3.230)--(2.475,3.210)--(2.487,3.190)--(2.500,3.171)--(2.514,3.153)%
--(2.529,3.135)--(2.545,3.119)--(2.561,3.103)--(2.579,3.088)--(2.597,3.074)%
--(2.616,3.061)--(2.636,3.049)--(2.656,3.038)--(2.676,3.028)--(2.698,3.020)%
--(2.719,3.012)--(2.741,3.005)--(2.763,3.000)--(2.786,2.996)--(2.809,2.993)%
--(2.832,2.991)--(2.855,2.991)--(2.877,2.991)--(2.900,2.993)--(2.923,2.996)%
--(2.946,3.000)--(2.968,3.005)--(2.990,3.012)--(3.011,3.020)--(3.033,3.028)%
--(3.053,3.038)--(3.074,3.049)--(3.093,3.061)--(3.112,3.074)--(3.130,3.088)%
--(3.148,3.103)--(3.164,3.119)--(3.180,3.135)--(3.195,3.153)--(3.209,3.171)%
--(3.222,3.190)--(3.234,3.210)--(3.245,3.230)--(3.255,3.250)--(3.263,3.272)%
--(3.271,3.293)--(3.278,3.315)--(3.283,3.337)--(3.287,3.360)--(3.290,3.383)--(3.292,3.406)--cycle;
%
\gpfill{rgb color={0.000,0.000,0.000},opacity=0.15} (3.423,3.432)--(3.422,3.454)--(3.420,3.476)--(3.417,3.498)%
--(3.413,3.519)--(3.408,3.541)--(3.402,3.562)--(3.394,3.583)--(3.386,3.603)%
--(3.377,3.623)--(3.366,3.643)--(3.354,3.661)--(3.342,3.680)--(3.328,3.697)%
--(3.314,3.714)--(3.299,3.730)--(3.283,3.745)--(3.266,3.759)--(3.249,3.773)%
--(3.230,3.785)--(3.212,3.797)--(3.192,3.808)--(3.172,3.817)--(3.152,3.825)%
--(3.131,3.833)--(3.110,3.839)--(3.088,3.844)--(3.067,3.848)--(3.045,3.851)%
--(3.023,3.853)--(3.001,3.854)--(2.978,3.853)--(2.956,3.851)--(2.934,3.848)%
--(2.913,3.844)--(2.891,3.839)--(2.870,3.833)--(2.849,3.825)--(2.829,3.817)%
--(2.809,3.808)--(2.790,3.797)--(2.771,3.785)--(2.752,3.773)--(2.735,3.759)%
--(2.718,3.745)--(2.702,3.730)--(2.687,3.714)--(2.673,3.697)--(2.659,3.680)%
--(2.647,3.661)--(2.635,3.643)--(2.624,3.623)--(2.615,3.603)--(2.607,3.583)%
--(2.599,3.562)--(2.593,3.541)--(2.588,3.519)--(2.584,3.498)--(2.581,3.476)%
--(2.579,3.454)--(2.579,3.432)--(2.579,3.409)--(2.581,3.387)--(2.584,3.365)%
--(2.588,3.344)--(2.593,3.322)--(2.599,3.301)--(2.607,3.280)--(2.615,3.260)%
--(2.624,3.240)--(2.635,3.221)--(2.647,3.202)--(2.659,3.183)--(2.673,3.166)%
--(2.687,3.149)--(2.702,3.133)--(2.718,3.118)--(2.735,3.104)--(2.752,3.090)%
--(2.771,3.078)--(2.790,3.066)--(2.809,3.055)--(2.829,3.046)--(2.849,3.038)%
--(2.870,3.030)--(2.891,3.024)--(2.913,3.019)--(2.934,3.015)--(2.956,3.012)%
--(2.978,3.010)--(3.001,3.010)--(3.023,3.010)--(3.045,3.012)--(3.067,3.015)%
--(3.088,3.019)--(3.110,3.024)--(3.131,3.030)--(3.152,3.038)--(3.172,3.046)%
--(3.192,3.055)--(3.212,3.066)--(3.230,3.078)--(3.249,3.090)--(3.266,3.104)%
--(3.283,3.118)--(3.299,3.133)--(3.314,3.149)--(3.328,3.166)--(3.342,3.183)%
--(3.354,3.202)--(3.366,3.221)--(3.377,3.240)--(3.386,3.260)--(3.394,3.280)%
--(3.402,3.301)--(3.408,3.322)--(3.413,3.344)--(3.417,3.365)--(3.420,3.387)--(3.422,3.409)--cycle;
%
\gpfill{rgb color={0.000,0.000,0.000},opacity=0.15} (3.558,3.438)--(3.557,3.459)--(3.555,3.480)--(3.553,3.501)%
--(3.549,3.522)--(3.544,3.542)--(3.538,3.563)--(3.531,3.583)--(3.522,3.602)%
--(3.513,3.621)--(3.503,3.640)--(3.492,3.658)--(3.480,3.676)--(3.467,3.692)%
--(3.453,3.708)--(3.439,3.724)--(3.423,3.738)--(3.407,3.752)--(3.391,3.765)%
--(3.373,3.777)--(3.355,3.788)--(3.336,3.798)--(3.317,3.807)--(3.298,3.816)%
--(3.278,3.823)--(3.257,3.829)--(3.237,3.834)--(3.216,3.838)--(3.195,3.840)%
--(3.174,3.842)--(3.153,3.843)--(3.131,3.842)--(3.110,3.840)--(3.089,3.838)%
--(3.068,3.834)--(3.048,3.829)--(3.027,3.823)--(3.007,3.816)--(2.988,3.807)%
--(2.969,3.798)--(2.950,3.788)--(2.932,3.777)--(2.914,3.765)--(2.898,3.752)%
--(2.882,3.738)--(2.866,3.724)--(2.852,3.708)--(2.838,3.692)--(2.825,3.676)%
--(2.813,3.658)--(2.802,3.640)--(2.792,3.621)--(2.783,3.602)--(2.774,3.583)%
--(2.767,3.563)--(2.761,3.542)--(2.756,3.522)--(2.752,3.501)--(2.750,3.480)%
--(2.748,3.459)--(2.748,3.438)--(2.748,3.416)--(2.750,3.395)--(2.752,3.374)%
--(2.756,3.353)--(2.761,3.333)--(2.767,3.312)--(2.774,3.292)--(2.783,3.273)%
--(2.792,3.254)--(2.802,3.235)--(2.813,3.217)--(2.825,3.199)--(2.838,3.183)%
--(2.852,3.167)--(2.866,3.151)--(2.882,3.137)--(2.898,3.123)--(2.914,3.110)%
--(2.932,3.098)--(2.950,3.087)--(2.969,3.077)--(2.988,3.068)--(3.007,3.059)%
--(3.027,3.052)--(3.048,3.046)--(3.068,3.041)--(3.089,3.037)--(3.110,3.035)%
--(3.131,3.033)--(3.153,3.033)--(3.174,3.033)--(3.195,3.035)--(3.216,3.037)%
--(3.237,3.041)--(3.257,3.046)--(3.278,3.052)--(3.298,3.059)--(3.317,3.068)%
--(3.336,3.077)--(3.355,3.087)--(3.373,3.098)--(3.391,3.110)--(3.407,3.123)%
--(3.423,3.137)--(3.439,3.151)--(3.453,3.167)--(3.467,3.183)--(3.480,3.199)%
--(3.492,3.217)--(3.503,3.235)--(3.513,3.254)--(3.522,3.273)--(3.531,3.292)%
--(3.538,3.312)--(3.544,3.333)--(3.549,3.353)--(3.553,3.374)--(3.555,3.395)--(3.557,3.416)--cycle;
%
\gpfill{rgb color={0.000,0.000,0.000},opacity=0.15} (3.696,3.446)--(3.695,3.466)--(3.693,3.486)--(3.691,3.506)%
--(3.687,3.526)--(3.682,3.545)--(3.677,3.565)--(3.670,3.584)--(3.662,3.603)%
--(3.653,3.621)--(3.644,3.639)--(3.633,3.656)--(3.622,3.672)--(3.609,3.688)%
--(3.596,3.704)--(3.582,3.718)--(3.568,3.732)--(3.552,3.745)--(3.536,3.758)%
--(3.520,3.769)--(3.503,3.780)--(3.485,3.789)--(3.467,3.798)--(3.448,3.806)%
--(3.429,3.813)--(3.409,3.818)--(3.390,3.823)--(3.370,3.827)--(3.350,3.829)%
--(3.330,3.831)--(3.310,3.832)--(3.289,3.831)--(3.269,3.829)--(3.249,3.827)%
--(3.229,3.823)--(3.210,3.818)--(3.190,3.813)--(3.171,3.806)--(3.152,3.798)%
--(3.134,3.789)--(3.117,3.780)--(3.099,3.769)--(3.083,3.758)--(3.067,3.745)%
--(3.051,3.732)--(3.037,3.718)--(3.023,3.704)--(3.010,3.688)--(2.997,3.672)%
--(2.986,3.656)--(2.975,3.639)--(2.966,3.621)--(2.957,3.603)--(2.949,3.584)%
--(2.942,3.565)--(2.937,3.545)--(2.932,3.526)--(2.928,3.506)--(2.926,3.486)%
--(2.924,3.466)--(2.924,3.446)--(2.924,3.425)--(2.926,3.405)--(2.928,3.385)%
--(2.932,3.365)--(2.937,3.346)--(2.942,3.326)--(2.949,3.307)--(2.957,3.288)%
--(2.966,3.270)--(2.975,3.253)--(2.986,3.235)--(2.997,3.219)--(3.010,3.203)%
--(3.023,3.187)--(3.037,3.173)--(3.051,3.159)--(3.067,3.146)--(3.083,3.133)%
--(3.099,3.122)--(3.117,3.111)--(3.134,3.102)--(3.152,3.093)--(3.171,3.085)%
--(3.190,3.078)--(3.210,3.073)--(3.229,3.068)--(3.249,3.064)--(3.269,3.062)%
--(3.289,3.060)--(3.310,3.060)--(3.330,3.060)--(3.350,3.062)--(3.370,3.064)%
--(3.390,3.068)--(3.409,3.073)--(3.429,3.078)--(3.448,3.085)--(3.467,3.093)%
--(3.485,3.102)--(3.503,3.111)--(3.520,3.122)--(3.536,3.133)--(3.552,3.146)%
--(3.568,3.159)--(3.582,3.173)--(3.596,3.187)--(3.609,3.203)--(3.622,3.219)%
--(3.633,3.235)--(3.644,3.253)--(3.653,3.270)--(3.662,3.288)--(3.670,3.307)%
--(3.677,3.326)--(3.682,3.346)--(3.687,3.365)--(3.691,3.385)--(3.693,3.405)--(3.695,3.425)--cycle;
%
\gpfill{rgb color={0.000,0.000,0.000},opacity=0.15} (3.840,3.456)--(3.839,3.475)--(3.837,3.494)--(3.835,3.513)%
--(3.831,3.532)--(3.827,3.551)--(3.821,3.569)--(3.815,3.587)--(3.808,3.605)%
--(3.799,3.623)--(3.790,3.640)--(3.780,3.656)--(3.769,3.672)--(3.757,3.687)%
--(3.745,3.702)--(3.732,3.716)--(3.718,3.729)--(3.703,3.741)--(3.688,3.753)%
--(3.672,3.764)--(3.656,3.774)--(3.639,3.783)--(3.621,3.792)--(3.603,3.799)%
--(3.585,3.805)--(3.567,3.811)--(3.548,3.815)--(3.529,3.819)--(3.510,3.821)%
--(3.491,3.823)--(3.472,3.824)--(3.452,3.823)--(3.433,3.821)--(3.414,3.819)%
--(3.395,3.815)--(3.376,3.811)--(3.358,3.805)--(3.340,3.799)--(3.322,3.792)%
--(3.304,3.783)--(3.288,3.774)--(3.271,3.764)--(3.255,3.753)--(3.240,3.741)%
--(3.225,3.729)--(3.211,3.716)--(3.198,3.702)--(3.186,3.687)--(3.174,3.672)%
--(3.163,3.656)--(3.153,3.640)--(3.144,3.623)--(3.135,3.605)--(3.128,3.587)%
--(3.122,3.569)--(3.116,3.551)--(3.112,3.532)--(3.108,3.513)--(3.106,3.494)%
--(3.104,3.475)--(3.104,3.456)--(3.104,3.436)--(3.106,3.417)--(3.108,3.398)%
--(3.112,3.379)--(3.116,3.360)--(3.122,3.342)--(3.128,3.324)--(3.135,3.306)%
--(3.144,3.288)--(3.153,3.272)--(3.163,3.255)--(3.174,3.239)--(3.186,3.224)%
--(3.198,3.209)--(3.211,3.195)--(3.225,3.182)--(3.240,3.170)--(3.255,3.158)%
--(3.271,3.147)--(3.288,3.137)--(3.304,3.128)--(3.322,3.119)--(3.340,3.112)%
--(3.358,3.106)--(3.376,3.100)--(3.395,3.096)--(3.414,3.092)--(3.433,3.090)%
--(3.452,3.088)--(3.472,3.088)--(3.491,3.088)--(3.510,3.090)--(3.529,3.092)%
--(3.548,3.096)--(3.567,3.100)--(3.585,3.106)--(3.603,3.112)--(3.621,3.119)%
--(3.639,3.128)--(3.656,3.137)--(3.672,3.147)--(3.688,3.158)--(3.703,3.170)%
--(3.718,3.182)--(3.732,3.195)--(3.745,3.209)--(3.757,3.224)--(3.769,3.239)%
--(3.780,3.255)--(3.790,3.272)--(3.799,3.288)--(3.808,3.306)--(3.815,3.324)%
--(3.821,3.342)--(3.827,3.360)--(3.831,3.379)--(3.835,3.398)--(3.837,3.417)--(3.839,3.436)--cycle;
%
\gpfill{rgb color={0.000,0.000,0.000},opacity=0.15} (3.988,3.468)--(3.987,3.486)--(3.986,3.504)--(3.983,3.522)%
--(3.980,3.540)--(3.976,3.558)--(3.970,3.575)--(3.964,3.593)--(3.957,3.609)%
--(3.949,3.626)--(3.941,3.642)--(3.931,3.658)--(3.921,3.673)--(3.910,3.687)%
--(3.898,3.701)--(3.885,3.714)--(3.872,3.727)--(3.858,3.739)--(3.844,3.750)%
--(3.829,3.760)--(3.813,3.770)--(3.797,3.778)--(3.780,3.786)--(3.764,3.793)%
--(3.746,3.799)--(3.729,3.805)--(3.711,3.809)--(3.693,3.812)--(3.675,3.815)%
--(3.657,3.816)--(3.639,3.817)--(3.620,3.816)--(3.602,3.815)--(3.584,3.812)%
--(3.566,3.809)--(3.548,3.805)--(3.531,3.799)--(3.513,3.793)--(3.497,3.786)%
--(3.480,3.778)--(3.464,3.770)--(3.448,3.760)--(3.433,3.750)--(3.419,3.739)%
--(3.405,3.727)--(3.392,3.714)--(3.379,3.701)--(3.367,3.687)--(3.356,3.673)%
--(3.346,3.658)--(3.336,3.642)--(3.328,3.626)--(3.320,3.609)--(3.313,3.593)%
--(3.307,3.575)--(3.301,3.558)--(3.297,3.540)--(3.294,3.522)--(3.291,3.504)%
--(3.290,3.486)--(3.290,3.468)--(3.290,3.449)--(3.291,3.431)--(3.294,3.413)%
--(3.297,3.395)--(3.301,3.377)--(3.307,3.360)--(3.313,3.342)--(3.320,3.326)%
--(3.328,3.309)--(3.336,3.293)--(3.346,3.277)--(3.356,3.262)--(3.367,3.248)%
--(3.379,3.234)--(3.392,3.221)--(3.405,3.208)--(3.419,3.196)--(3.433,3.185)%
--(3.448,3.175)--(3.464,3.165)--(3.480,3.157)--(3.497,3.149)--(3.513,3.142)%
--(3.531,3.136)--(3.548,3.130)--(3.566,3.126)--(3.584,3.123)--(3.602,3.120)%
--(3.620,3.119)--(3.639,3.119)--(3.657,3.119)--(3.675,3.120)--(3.693,3.123)%
--(3.711,3.126)--(3.729,3.130)--(3.746,3.136)--(3.764,3.142)--(3.780,3.149)%
--(3.797,3.157)--(3.813,3.165)--(3.829,3.175)--(3.844,3.185)--(3.858,3.196)%
--(3.872,3.208)--(3.885,3.221)--(3.898,3.234)--(3.910,3.248)--(3.921,3.262)%
--(3.931,3.277)--(3.941,3.293)--(3.949,3.309)--(3.957,3.326)--(3.964,3.342)%
--(3.970,3.360)--(3.976,3.377)--(3.980,3.395)--(3.983,3.413)--(3.986,3.431)--(3.987,3.449)--cycle;
%
\gpfill{rgb color={0.000,0.000,0.000},opacity=0.15} (4.141,3.481)--(4.140,3.498)--(4.139,3.515)--(4.136,3.532)%
--(4.133,3.549)--(4.129,3.566)--(4.124,3.583)--(4.119,3.599)--(4.112,3.615)%
--(4.104,3.631)--(4.096,3.646)--(4.087,3.661)--(4.077,3.675)--(4.067,3.689)%
--(4.055,3.702)--(4.044,3.715)--(4.031,3.726)--(4.018,3.738)--(4.004,3.748)%
--(3.990,3.758)--(3.975,3.767)--(3.960,3.775)--(3.944,3.783)--(3.928,3.790)%
--(3.912,3.795)--(3.895,3.800)--(3.878,3.804)--(3.861,3.807)--(3.844,3.810)%
--(3.827,3.811)--(3.810,3.812)--(3.792,3.811)--(3.775,3.810)--(3.758,3.807)%
--(3.741,3.804)--(3.724,3.800)--(3.707,3.795)--(3.691,3.790)--(3.675,3.783)%
--(3.659,3.775)--(3.644,3.767)--(3.629,3.758)--(3.615,3.748)--(3.601,3.738)%
--(3.588,3.726)--(3.575,3.715)--(3.564,3.702)--(3.552,3.689)--(3.542,3.675)%
--(3.532,3.661)--(3.523,3.646)--(3.515,3.631)--(3.507,3.615)--(3.500,3.599)%
--(3.495,3.583)--(3.490,3.566)--(3.486,3.549)--(3.483,3.532)--(3.480,3.515)%
--(3.479,3.498)--(3.479,3.481)--(3.479,3.463)--(3.480,3.446)--(3.483,3.429)%
--(3.486,3.412)--(3.490,3.395)--(3.495,3.378)--(3.500,3.362)--(3.507,3.346)%
--(3.515,3.330)--(3.523,3.315)--(3.532,3.300)--(3.542,3.286)--(3.552,3.272)%
--(3.564,3.259)--(3.575,3.246)--(3.588,3.235)--(3.601,3.223)--(3.615,3.213)%
--(3.629,3.203)--(3.644,3.194)--(3.659,3.186)--(3.675,3.178)--(3.691,3.171)%
--(3.707,3.166)--(3.724,3.161)--(3.741,3.157)--(3.758,3.154)--(3.775,3.151)%
--(3.792,3.150)--(3.810,3.150)--(3.827,3.150)--(3.844,3.151)--(3.861,3.154)%
--(3.878,3.157)--(3.895,3.161)--(3.912,3.166)--(3.928,3.171)--(3.944,3.178)%
--(3.960,3.186)--(3.975,3.194)--(3.990,3.203)--(4.004,3.213)--(4.018,3.223)%
--(4.031,3.235)--(4.044,3.246)--(4.055,3.259)--(4.067,3.272)--(4.077,3.286)%
--(4.087,3.300)--(4.096,3.315)--(4.104,3.330)--(4.112,3.346)--(4.119,3.362)%
--(4.124,3.378)--(4.129,3.395)--(4.133,3.412)--(4.136,3.429)--(4.139,3.446)--(4.140,3.463)--cycle;
%
\gpfill{rgb color={0.000,0.000,0.000},opacity=0.15} (4.300,3.495)--(4.299,3.511)--(4.298,3.527)--(4.296,3.544)%
--(4.293,3.560)--(4.289,3.576)--(4.284,3.592)--(4.279,3.607)--(4.272,3.622)%
--(4.265,3.637)--(4.257,3.652)--(4.249,3.666)--(4.240,3.679)--(4.230,3.692)%
--(4.219,3.705)--(4.208,3.717)--(4.196,3.728)--(4.183,3.739)--(4.170,3.749)%
--(4.157,3.758)--(4.143,3.766)--(4.128,3.774)--(4.113,3.781)--(4.098,3.788)%
--(4.083,3.793)--(4.067,3.798)--(4.051,3.802)--(4.035,3.805)--(4.018,3.807)%
--(4.002,3.808)--(3.986,3.809)--(3.969,3.808)--(3.953,3.807)--(3.936,3.805)%
--(3.920,3.802)--(3.904,3.798)--(3.888,3.793)--(3.873,3.788)--(3.858,3.781)%
--(3.843,3.774)--(3.829,3.766)--(3.814,3.758)--(3.801,3.749)--(3.788,3.739)%
--(3.775,3.728)--(3.763,3.717)--(3.752,3.705)--(3.741,3.692)--(3.731,3.679)%
--(3.722,3.666)--(3.714,3.652)--(3.706,3.637)--(3.699,3.622)--(3.692,3.607)%
--(3.687,3.592)--(3.682,3.576)--(3.678,3.560)--(3.675,3.544)--(3.673,3.527)%
--(3.672,3.511)--(3.672,3.495)--(3.672,3.478)--(3.673,3.462)--(3.675,3.445)%
--(3.678,3.429)--(3.682,3.413)--(3.687,3.397)--(3.692,3.382)--(3.699,3.367)%
--(3.706,3.352)--(3.714,3.338)--(3.722,3.323)--(3.731,3.310)--(3.741,3.297)%
--(3.752,3.284)--(3.763,3.272)--(3.775,3.261)--(3.788,3.250)--(3.801,3.240)%
--(3.814,3.231)--(3.829,3.223)--(3.843,3.215)--(3.858,3.208)--(3.873,3.201)%
--(3.888,3.196)--(3.904,3.191)--(3.920,3.187)--(3.936,3.184)--(3.953,3.182)%
--(3.969,3.181)--(3.986,3.181)--(4.002,3.181)--(4.018,3.182)--(4.035,3.184)%
--(4.051,3.187)--(4.067,3.191)--(4.083,3.196)--(4.098,3.201)--(4.113,3.208)%
--(4.128,3.215)--(4.143,3.223)--(4.157,3.231)--(4.170,3.240)--(4.183,3.250)%
--(4.196,3.261)--(4.208,3.272)--(4.219,3.284)--(4.230,3.297)--(4.240,3.310)%
--(4.249,3.323)--(4.257,3.338)--(4.265,3.352)--(4.272,3.367)--(4.279,3.382)%
--(4.284,3.397)--(4.289,3.413)--(4.293,3.429)--(4.296,3.445)--(4.298,3.462)--(4.299,3.478)--cycle;
%
\gpfill{rgb color={0.000,0.000,0.000},opacity=0.15} (4.464,3.509)--(4.463,3.524)--(4.462,3.540)--(4.460,3.555)%
--(4.457,3.570)--(4.453,3.586)--(4.449,3.601)--(4.444,3.615)--(4.438,3.630)%
--(4.431,3.644)--(4.424,3.658)--(4.415,3.671)--(4.407,3.684)--(4.397,3.696)%
--(4.387,3.708)--(4.376,3.719)--(4.365,3.730)--(4.353,3.740)--(4.341,3.750)%
--(4.328,3.758)--(4.315,3.767)--(4.301,3.774)--(4.287,3.781)--(4.272,3.787)%
--(4.258,3.792)--(4.243,3.796)--(4.227,3.800)--(4.212,3.803)--(4.197,3.805)%
--(4.181,3.806)--(4.166,3.807)--(4.150,3.806)--(4.134,3.805)--(4.119,3.803)%
--(4.104,3.800)--(4.088,3.796)--(4.073,3.792)--(4.059,3.787)--(4.044,3.781)%
--(4.030,3.774)--(4.017,3.767)--(4.003,3.758)--(3.990,3.750)--(3.978,3.740)%
--(3.966,3.730)--(3.955,3.719)--(3.944,3.708)--(3.934,3.696)--(3.924,3.684)%
--(3.916,3.671)--(3.907,3.658)--(3.900,3.644)--(3.893,3.630)--(3.887,3.615)%
--(3.882,3.601)--(3.878,3.586)--(3.874,3.570)--(3.871,3.555)--(3.869,3.540)%
--(3.868,3.524)--(3.868,3.509)--(3.868,3.493)--(3.869,3.477)--(3.871,3.462)%
--(3.874,3.447)--(3.878,3.431)--(3.882,3.416)--(3.887,3.402)--(3.893,3.387)%
--(3.900,3.373)--(3.907,3.360)--(3.916,3.346)--(3.924,3.333)--(3.934,3.321)%
--(3.944,3.309)--(3.955,3.298)--(3.966,3.287)--(3.978,3.277)--(3.990,3.267)%
--(4.003,3.259)--(4.017,3.250)--(4.030,3.243)--(4.044,3.236)--(4.059,3.230)%
--(4.073,3.225)--(4.088,3.221)--(4.104,3.217)--(4.119,3.214)--(4.134,3.212)%
--(4.150,3.211)--(4.166,3.211)--(4.181,3.211)--(4.197,3.212)--(4.212,3.214)%
--(4.227,3.217)--(4.243,3.221)--(4.258,3.225)--(4.272,3.230)--(4.287,3.236)%
--(4.301,3.243)--(4.315,3.250)--(4.328,3.259)--(4.341,3.267)--(4.353,3.277)%
--(4.365,3.287)--(4.376,3.298)--(4.387,3.309)--(4.397,3.321)--(4.407,3.333)%
--(4.415,3.346)--(4.424,3.360)--(4.431,3.373)--(4.438,3.387)--(4.444,3.402)%
--(4.449,3.416)--(4.453,3.431)--(4.457,3.447)--(4.460,3.462)--(4.462,3.477)--(4.463,3.493)--cycle;
%
\gpfill{rgb color={0.000,0.000,0.000},opacity=0.15} (4.631,3.524)--(4.630,3.538)--(4.629,3.553)--(4.627,3.568)%
--(4.624,3.582)--(4.621,3.596)--(4.617,3.611)--(4.612,3.625)--(4.606,3.638)%
--(4.600,3.652)--(4.593,3.665)--(4.585,3.677)--(4.577,3.689)--(4.568,3.701)%
--(4.558,3.712)--(4.548,3.723)--(4.537,3.733)--(4.526,3.743)--(4.514,3.752)%
--(4.502,3.760)--(4.490,3.768)--(4.477,3.775)--(4.463,3.781)--(4.450,3.787)%
--(4.436,3.792)--(4.421,3.796)--(4.407,3.799)--(4.393,3.802)--(4.378,3.804)%
--(4.363,3.805)--(4.349,3.806)--(4.334,3.805)--(4.319,3.804)--(4.304,3.802)%
--(4.290,3.799)--(4.276,3.796)--(4.261,3.792)--(4.247,3.787)--(4.234,3.781)%
--(4.220,3.775)--(4.208,3.768)--(4.195,3.760)--(4.183,3.752)--(4.171,3.743)%
--(4.160,3.733)--(4.149,3.723)--(4.139,3.712)--(4.129,3.701)--(4.120,3.689)%
--(4.112,3.677)--(4.104,3.665)--(4.097,3.652)--(4.091,3.638)--(4.085,3.625)%
--(4.080,3.611)--(4.076,3.596)--(4.073,3.582)--(4.070,3.568)--(4.068,3.553)%
--(4.067,3.538)--(4.067,3.524)--(4.067,3.509)--(4.068,3.494)--(4.070,3.479)%
--(4.073,3.465)--(4.076,3.451)--(4.080,3.436)--(4.085,3.422)--(4.091,3.409)%
--(4.097,3.395)--(4.104,3.383)--(4.112,3.370)--(4.120,3.358)--(4.129,3.346)%
--(4.139,3.335)--(4.149,3.324)--(4.160,3.314)--(4.171,3.304)--(4.183,3.295)%
--(4.195,3.287)--(4.208,3.279)--(4.220,3.272)--(4.234,3.266)--(4.247,3.260)%
--(4.261,3.255)--(4.276,3.251)--(4.290,3.248)--(4.304,3.245)--(4.319,3.243)%
--(4.334,3.242)--(4.349,3.242)--(4.363,3.242)--(4.378,3.243)--(4.393,3.245)%
--(4.407,3.248)--(4.421,3.251)--(4.436,3.255)--(4.450,3.260)--(4.463,3.266)%
--(4.477,3.272)--(4.490,3.279)--(4.502,3.287)--(4.514,3.295)--(4.526,3.304)%
--(4.537,3.314)--(4.548,3.324)--(4.558,3.335)--(4.568,3.346)--(4.577,3.358)%
--(4.585,3.370)--(4.593,3.383)--(4.600,3.395)--(4.606,3.409)--(4.612,3.422)%
--(4.617,3.436)--(4.621,3.451)--(4.624,3.465)--(4.627,3.479)--(4.629,3.494)--(4.630,3.509)--cycle;
%
\gpfill{rgb color={0.000,0.000,0.000},opacity=0.15} (4.803,3.539)--(4.802,3.553)--(4.801,3.567)--(4.799,3.580)%
--(4.797,3.594)--(4.793,3.608)--(4.789,3.621)--(4.785,3.635)--(4.779,3.648)%
--(4.773,3.660)--(4.767,3.673)--(4.759,3.684)--(4.751,3.696)--(4.743,3.707)%
--(4.734,3.718)--(4.724,3.728)--(4.714,3.738)--(4.703,3.747)--(4.692,3.755)%
--(4.680,3.763)--(4.669,3.771)--(4.656,3.777)--(4.644,3.783)--(4.631,3.789)%
--(4.617,3.793)--(4.604,3.797)--(4.590,3.801)--(4.576,3.803)--(4.563,3.805)%
--(4.549,3.806)--(4.535,3.807)--(4.520,3.806)--(4.506,3.805)--(4.493,3.803)%
--(4.479,3.801)--(4.465,3.797)--(4.452,3.793)--(4.438,3.789)--(4.425,3.783)%
--(4.413,3.777)--(4.401,3.771)--(4.389,3.763)--(4.377,3.755)--(4.366,3.747)%
--(4.355,3.738)--(4.345,3.728)--(4.335,3.718)--(4.326,3.707)--(4.318,3.696)%
--(4.310,3.684)--(4.302,3.673)--(4.296,3.660)--(4.290,3.648)--(4.284,3.635)%
--(4.280,3.621)--(4.276,3.608)--(4.272,3.594)--(4.270,3.580)--(4.268,3.567)%
--(4.267,3.553)--(4.267,3.539)--(4.267,3.524)--(4.268,3.510)--(4.270,3.497)%
--(4.272,3.483)--(4.276,3.469)--(4.280,3.456)--(4.284,3.442)--(4.290,3.429)%
--(4.296,3.417)--(4.302,3.405)--(4.310,3.393)--(4.318,3.381)--(4.326,3.370)%
--(4.335,3.359)--(4.345,3.349)--(4.355,3.339)--(4.366,3.330)--(4.377,3.322)%
--(4.389,3.314)--(4.401,3.306)--(4.413,3.300)--(4.425,3.294)--(4.438,3.288)%
--(4.452,3.284)--(4.465,3.280)--(4.479,3.276)--(4.493,3.274)--(4.506,3.272)%
--(4.520,3.271)--(4.535,3.271)--(4.549,3.271)--(4.563,3.272)--(4.576,3.274)%
--(4.590,3.276)--(4.604,3.280)--(4.617,3.284)--(4.631,3.288)--(4.644,3.294)%
--(4.656,3.300)--(4.669,3.306)--(4.680,3.314)--(4.692,3.322)--(4.703,3.330)%
--(4.714,3.339)--(4.724,3.349)--(4.734,3.359)--(4.743,3.370)--(4.751,3.381)%
--(4.759,3.393)--(4.767,3.405)--(4.773,3.417)--(4.779,3.429)--(4.785,3.442)%
--(4.789,3.456)--(4.793,3.469)--(4.797,3.483)--(4.799,3.497)--(4.801,3.510)--(4.802,3.524)--cycle;
%
\gpfill{rgb color={0.000,0.000,0.000},opacity=0.15} (4.979,3.554)--(4.978,3.567)--(4.977,3.580)--(4.975,3.593)%
--(4.973,3.607)--(4.970,3.619)--(4.966,3.632)--(4.962,3.645)--(4.956,3.657)%
--(4.951,3.669)--(4.944,3.681)--(4.937,3.692)--(4.930,3.703)--(4.922,3.714)%
--(4.913,3.724)--(4.904,3.734)--(4.894,3.743)--(4.884,3.752)--(4.873,3.760)%
--(4.862,3.767)--(4.851,3.774)--(4.839,3.781)--(4.827,3.786)--(4.815,3.792)%
--(4.802,3.796)--(4.789,3.800)--(4.777,3.803)--(4.763,3.805)--(4.750,3.807)%
--(4.737,3.808)--(4.724,3.809)--(4.710,3.808)--(4.697,3.807)--(4.684,3.805)%
--(4.670,3.803)--(4.658,3.800)--(4.645,3.796)--(4.632,3.792)--(4.620,3.786)%
--(4.608,3.781)--(4.596,3.774)--(4.585,3.767)--(4.574,3.760)--(4.563,3.752)%
--(4.553,3.743)--(4.543,3.734)--(4.534,3.724)--(4.525,3.714)--(4.517,3.703)%
--(4.510,3.692)--(4.503,3.681)--(4.496,3.669)--(4.491,3.657)--(4.485,3.645)%
--(4.481,3.632)--(4.477,3.619)--(4.474,3.607)--(4.472,3.593)--(4.470,3.580)%
--(4.469,3.567)--(4.469,3.554)--(4.469,3.540)--(4.470,3.527)--(4.472,3.514)%
--(4.474,3.500)--(4.477,3.488)--(4.481,3.475)--(4.485,3.462)--(4.491,3.450)%
--(4.496,3.438)--(4.503,3.426)--(4.510,3.415)--(4.517,3.404)--(4.525,3.393)%
--(4.534,3.383)--(4.543,3.373)--(4.553,3.364)--(4.563,3.355)--(4.574,3.347)%
--(4.585,3.340)--(4.596,3.333)--(4.608,3.326)--(4.620,3.321)--(4.632,3.315)%
--(4.645,3.311)--(4.658,3.307)--(4.670,3.304)--(4.684,3.302)--(4.697,3.300)%
--(4.710,3.299)--(4.724,3.299)--(4.737,3.299)--(4.750,3.300)--(4.763,3.302)%
--(4.777,3.304)--(4.789,3.307)--(4.802,3.311)--(4.815,3.315)--(4.827,3.321)%
--(4.839,3.326)--(4.851,3.333)--(4.862,3.340)--(4.873,3.347)--(4.884,3.355)%
--(4.894,3.364)--(4.904,3.373)--(4.913,3.383)--(4.922,3.393)--(4.930,3.404)%
--(4.937,3.415)--(4.944,3.426)--(4.951,3.438)--(4.956,3.450)--(4.962,3.462)%
--(4.966,3.475)--(4.970,3.488)--(4.973,3.500)--(4.975,3.514)--(4.977,3.527)--(4.978,3.540)--cycle;
%
\gpfill{rgb color={0.000,0.000,0.000},opacity=0.15} (5.160,3.570)--(5.159,3.582)--(5.158,3.595)--(5.156,3.608)%
--(5.154,3.620)--(5.151,3.633)--(5.148,3.645)--(5.143,3.657)--(5.138,3.669)%
--(5.133,3.680)--(5.127,3.692)--(5.120,3.702)--(5.113,3.713)--(5.105,3.723)%
--(5.097,3.733)--(5.088,3.742)--(5.079,3.751)--(5.069,3.759)--(5.059,3.767)%
--(5.048,3.774)--(5.038,3.781)--(5.026,3.787)--(5.015,3.792)--(5.003,3.797)%
--(4.991,3.802)--(4.979,3.805)--(4.966,3.808)--(4.954,3.810)--(4.941,3.812)%
--(4.928,3.813)--(4.916,3.814)--(4.903,3.813)--(4.890,3.812)--(4.877,3.810)%
--(4.865,3.808)--(4.852,3.805)--(4.840,3.802)--(4.828,3.797)--(4.816,3.792)%
--(4.805,3.787)--(4.794,3.781)--(4.783,3.774)--(4.772,3.767)--(4.762,3.759)%
--(4.752,3.751)--(4.743,3.742)--(4.734,3.733)--(4.726,3.723)--(4.718,3.713)%
--(4.711,3.702)--(4.704,3.692)--(4.698,3.680)--(4.693,3.669)--(4.688,3.657)%
--(4.683,3.645)--(4.680,3.633)--(4.677,3.620)--(4.675,3.608)--(4.673,3.595)%
--(4.672,3.582)--(4.672,3.570)--(4.672,3.557)--(4.673,3.544)--(4.675,3.531)%
--(4.677,3.519)--(4.680,3.506)--(4.683,3.494)--(4.688,3.482)--(4.693,3.470)%
--(4.698,3.459)--(4.704,3.448)--(4.711,3.437)--(4.718,3.426)--(4.726,3.416)%
--(4.734,3.406)--(4.743,3.397)--(4.752,3.388)--(4.762,3.380)--(4.772,3.372)%
--(4.783,3.365)--(4.794,3.358)--(4.805,3.352)--(4.816,3.347)--(4.828,3.342)%
--(4.840,3.337)--(4.852,3.334)--(4.865,3.331)--(4.877,3.329)--(4.890,3.327)%
--(4.903,3.326)--(4.916,3.326)--(4.928,3.326)--(4.941,3.327)--(4.954,3.329)%
--(4.966,3.331)--(4.979,3.334)--(4.991,3.337)--(5.003,3.342)--(5.015,3.347)%
--(5.026,3.352)--(5.038,3.358)--(5.048,3.365)--(5.059,3.372)--(5.069,3.380)%
--(5.079,3.388)--(5.088,3.397)--(5.097,3.406)--(5.105,3.416)--(5.113,3.426)%
--(5.120,3.437)--(5.127,3.448)--(5.133,3.459)--(5.138,3.470)--(5.143,3.482)%
--(5.148,3.494)--(5.151,3.506)--(5.154,3.519)--(5.156,3.531)--(5.158,3.544)--(5.159,3.557)--cycle;
%
\gpfill{rgb color={0.000,0.000,0.000},opacity=0.15} (5.345,3.585)--(5.344,3.597)--(5.343,3.609)--(5.342,3.621)%
--(5.339,3.633)--(5.337,3.645)--(5.333,3.657)--(5.329,3.668)--(5.324,3.680)%
--(5.319,3.691)--(5.313,3.702)--(5.307,3.712)--(5.300,3.722)--(5.292,3.732)%
--(5.284,3.741)--(5.276,3.750)--(5.267,3.758)--(5.258,3.766)--(5.248,3.774)%
--(5.238,3.781)--(5.228,3.787)--(5.217,3.793)--(5.206,3.798)--(5.194,3.803)%
--(5.183,3.807)--(5.171,3.811)--(5.159,3.813)--(5.147,3.816)--(5.135,3.817)%
--(5.123,3.818)--(5.111,3.819)--(5.098,3.818)--(5.086,3.817)--(5.074,3.816)%
--(5.062,3.813)--(5.050,3.811)--(5.038,3.807)--(5.027,3.803)--(5.015,3.798)%
--(5.004,3.793)--(4.994,3.787)--(4.983,3.781)--(4.973,3.774)--(4.963,3.766)%
--(4.954,3.758)--(4.945,3.750)--(4.937,3.741)--(4.929,3.732)--(4.921,3.722)%
--(4.914,3.712)--(4.908,3.702)--(4.902,3.691)--(4.897,3.680)--(4.892,3.668)%
--(4.888,3.657)--(4.884,3.645)--(4.882,3.633)--(4.879,3.621)--(4.878,3.609)%
--(4.877,3.597)--(4.877,3.585)--(4.877,3.572)--(4.878,3.560)--(4.879,3.548)%
--(4.882,3.536)--(4.884,3.524)--(4.888,3.512)--(4.892,3.501)--(4.897,3.489)%
--(4.902,3.478)--(4.908,3.468)--(4.914,3.457)--(4.921,3.447)--(4.929,3.437)%
--(4.937,3.428)--(4.945,3.419)--(4.954,3.411)--(4.963,3.403)--(4.973,3.395)%
--(4.983,3.388)--(4.994,3.382)--(5.004,3.376)--(5.015,3.371)--(5.027,3.366)%
--(5.038,3.362)--(5.050,3.358)--(5.062,3.356)--(5.074,3.353)--(5.086,3.352)%
--(5.098,3.351)--(5.111,3.351)--(5.123,3.351)--(5.135,3.352)--(5.147,3.353)%
--(5.159,3.356)--(5.171,3.358)--(5.183,3.362)--(5.194,3.366)--(5.206,3.371)%
--(5.217,3.376)--(5.228,3.382)--(5.238,3.388)--(5.248,3.395)--(5.258,3.403)%
--(5.267,3.411)--(5.276,3.419)--(5.284,3.428)--(5.292,3.437)--(5.300,3.447)%
--(5.307,3.457)--(5.313,3.468)--(5.319,3.478)--(5.324,3.489)--(5.329,3.501)%
--(5.333,3.512)--(5.337,3.524)--(5.339,3.536)--(5.342,3.548)--(5.343,3.560)--(5.344,3.572)--cycle;
%
\gpfill{rgb color={0.000,0.000,0.000},opacity=0.15} (5.533,3.601)--(5.532,3.612)--(5.531,3.624)--(5.530,3.636)%
--(5.528,3.647)--(5.525,3.659)--(5.521,3.670)--(5.517,3.681)--(5.513,3.692)%
--(5.508,3.703)--(5.502,3.714)--(5.496,3.724)--(5.489,3.733)--(5.482,3.743)%
--(5.474,3.752)--(5.466,3.760)--(5.458,3.768)--(5.449,3.776)--(5.439,3.783)%
--(5.430,3.790)--(5.420,3.796)--(5.409,3.802)--(5.398,3.807)--(5.387,3.811)%
--(5.376,3.815)--(5.365,3.819)--(5.353,3.822)--(5.342,3.824)--(5.330,3.825)%
--(5.318,3.826)--(5.307,3.827)--(5.295,3.826)--(5.283,3.825)--(5.271,3.824)%
--(5.260,3.822)--(5.248,3.819)--(5.237,3.815)--(5.226,3.811)--(5.215,3.807)%
--(5.204,3.802)--(5.194,3.796)--(5.183,3.790)--(5.174,3.783)--(5.164,3.776)%
--(5.155,3.768)--(5.147,3.760)--(5.139,3.752)--(5.131,3.743)--(5.124,3.733)%
--(5.117,3.724)--(5.111,3.714)--(5.105,3.703)--(5.100,3.692)--(5.096,3.681)%
--(5.092,3.670)--(5.088,3.659)--(5.085,3.647)--(5.083,3.636)--(5.082,3.624)%
--(5.081,3.612)--(5.081,3.601)--(5.081,3.589)--(5.082,3.577)--(5.083,3.565)%
--(5.085,3.554)--(5.088,3.542)--(5.092,3.531)--(5.096,3.520)--(5.100,3.509)%
--(5.105,3.498)--(5.111,3.488)--(5.117,3.477)--(5.124,3.468)--(5.131,3.458)%
--(5.139,3.449)--(5.147,3.441)--(5.155,3.433)--(5.164,3.425)--(5.174,3.418)%
--(5.183,3.411)--(5.194,3.405)--(5.204,3.399)--(5.215,3.394)--(5.226,3.390)%
--(5.237,3.386)--(5.248,3.382)--(5.260,3.379)--(5.271,3.377)--(5.283,3.376)%
--(5.295,3.375)--(5.307,3.375)--(5.318,3.375)--(5.330,3.376)--(5.342,3.377)%
--(5.353,3.379)--(5.365,3.382)--(5.376,3.386)--(5.387,3.390)--(5.398,3.394)%
--(5.409,3.399)--(5.420,3.405)--(5.430,3.411)--(5.439,3.418)--(5.449,3.425)%
--(5.458,3.433)--(5.466,3.441)--(5.474,3.449)--(5.482,3.458)--(5.489,3.468)%
--(5.496,3.477)--(5.502,3.488)--(5.508,3.498)--(5.513,3.509)--(5.517,3.520)%
--(5.521,3.531)--(5.525,3.542)--(5.528,3.554)--(5.530,3.565)--(5.531,3.577)--(5.532,3.589)--cycle;
%
\gpfill{rgb color={0.000,0.000,0.000},opacity=0.15} (5.722,3.617)--(5.721,3.628)--(5.720,3.639)--(5.719,3.651)%
--(5.717,3.662)--(5.714,3.673)--(5.711,3.684)--(5.707,3.695)--(5.703,3.705)%
--(5.698,3.715)--(5.692,3.726)--(5.686,3.735)--(5.680,3.745)--(5.673,3.754)%
--(5.666,3.762)--(5.658,3.771)--(5.649,3.779)--(5.641,3.786)--(5.632,3.793)%
--(5.622,3.799)--(5.613,3.805)--(5.602,3.811)--(5.592,3.816)--(5.582,3.820)%
--(5.571,3.824)--(5.560,3.827)--(5.549,3.830)--(5.538,3.832)--(5.526,3.833)%
--(5.515,3.834)--(5.504,3.835)--(5.492,3.834)--(5.481,3.833)--(5.469,3.832)%
--(5.458,3.830)--(5.447,3.827)--(5.436,3.824)--(5.425,3.820)--(5.415,3.816)%
--(5.405,3.811)--(5.395,3.805)--(5.385,3.799)--(5.375,3.793)--(5.366,3.786)%
--(5.358,3.779)--(5.349,3.771)--(5.341,3.762)--(5.334,3.754)--(5.327,3.745)%
--(5.321,3.735)--(5.315,3.726)--(5.309,3.715)--(5.304,3.705)--(5.300,3.695)%
--(5.296,3.684)--(5.293,3.673)--(5.290,3.662)--(5.288,3.651)--(5.287,3.639)%
--(5.286,3.628)--(5.286,3.617)--(5.286,3.605)--(5.287,3.594)--(5.288,3.582)%
--(5.290,3.571)--(5.293,3.560)--(5.296,3.549)--(5.300,3.538)--(5.304,3.528)%
--(5.309,3.518)--(5.315,3.508)--(5.321,3.498)--(5.327,3.488)--(5.334,3.479)%
--(5.341,3.471)--(5.349,3.462)--(5.358,3.454)--(5.366,3.447)--(5.375,3.440)%
--(5.385,3.434)--(5.395,3.428)--(5.405,3.422)--(5.415,3.417)--(5.425,3.413)%
--(5.436,3.409)--(5.447,3.406)--(5.458,3.403)--(5.469,3.401)--(5.481,3.400)%
--(5.492,3.399)--(5.504,3.399)--(5.515,3.399)--(5.526,3.400)--(5.538,3.401)%
--(5.549,3.403)--(5.560,3.406)--(5.571,3.409)--(5.582,3.413)--(5.592,3.417)%
--(5.602,3.422)--(5.613,3.428)--(5.622,3.434)--(5.632,3.440)--(5.641,3.447)%
--(5.649,3.454)--(5.658,3.462)--(5.666,3.471)--(5.673,3.479)--(5.680,3.488)%
--(5.686,3.498)--(5.692,3.508)--(5.698,3.518)--(5.703,3.528)--(5.707,3.538)%
--(5.711,3.549)--(5.714,3.560)--(5.717,3.571)--(5.719,3.582)--(5.720,3.594)--(5.721,3.605)--cycle;
%
\gpfill{rgb color={0.000,0.000,0.000},opacity=0.15} (5.915,3.633)--(5.914,3.644)--(5.913,3.655)--(5.912,3.666)%
--(5.910,3.677)--(5.907,3.687)--(5.904,3.698)--(5.900,3.708)--(5.896,3.719)%
--(5.891,3.729)--(5.886,3.739)--(5.880,3.748)--(5.874,3.757)--(5.867,3.766)%
--(5.860,3.774)--(5.852,3.782)--(5.844,3.790)--(5.836,3.797)--(5.827,3.804)%
--(5.818,3.810)--(5.809,3.816)--(5.799,3.821)--(5.789,3.826)--(5.778,3.830)%
--(5.768,3.834)--(5.757,3.837)--(5.747,3.840)--(5.736,3.842)--(5.725,3.843)%
--(5.714,3.844)--(5.703,3.845)--(5.691,3.844)--(5.680,3.843)--(5.669,3.842)%
--(5.658,3.840)--(5.648,3.837)--(5.637,3.834)--(5.627,3.830)--(5.616,3.826)%
--(5.606,3.821)--(5.597,3.816)--(5.587,3.810)--(5.578,3.804)--(5.569,3.797)%
--(5.561,3.790)--(5.553,3.782)--(5.545,3.774)--(5.538,3.766)--(5.531,3.757)%
--(5.525,3.748)--(5.519,3.739)--(5.514,3.729)--(5.509,3.719)--(5.505,3.708)%
--(5.501,3.698)--(5.498,3.687)--(5.495,3.677)--(5.493,3.666)--(5.492,3.655)%
--(5.491,3.644)--(5.491,3.633)--(5.491,3.621)--(5.492,3.610)--(5.493,3.599)%
--(5.495,3.588)--(5.498,3.578)--(5.501,3.567)--(5.505,3.557)--(5.509,3.546)%
--(5.514,3.536)--(5.519,3.527)--(5.525,3.517)--(5.531,3.508)--(5.538,3.499)%
--(5.545,3.491)--(5.553,3.483)--(5.561,3.475)--(5.569,3.468)--(5.578,3.461)%
--(5.587,3.455)--(5.597,3.449)--(5.606,3.444)--(5.616,3.439)--(5.627,3.435)%
--(5.637,3.431)--(5.648,3.428)--(5.658,3.425)--(5.669,3.423)--(5.680,3.422)%
--(5.691,3.421)--(5.703,3.421)--(5.714,3.421)--(5.725,3.422)--(5.736,3.423)%
--(5.747,3.425)--(5.757,3.428)--(5.768,3.431)--(5.778,3.435)--(5.789,3.439)%
--(5.799,3.444)--(5.809,3.449)--(5.818,3.455)--(5.827,3.461)--(5.836,3.468)%
--(5.844,3.475)--(5.852,3.483)--(5.860,3.491)--(5.867,3.499)--(5.874,3.508)%
--(5.880,3.517)--(5.886,3.527)--(5.891,3.536)--(5.896,3.546)--(5.900,3.557)%
--(5.904,3.567)--(5.907,3.578)--(5.910,3.588)--(5.912,3.599)--(5.913,3.610)--(5.914,3.621)--cycle;
%
\gpfill{rgb color={0.000,0.000,0.000},opacity=0.15} (6.110,3.649)--(6.109,3.659)--(6.108,3.670)--(6.107,3.681)%
--(6.105,3.692)--(6.102,3.702)--(6.099,3.712)--(6.096,3.723)--(6.092,3.733)%
--(6.087,3.742)--(6.082,3.752)--(6.076,3.761)--(6.070,3.770)--(6.063,3.779)%
--(6.056,3.787)--(6.049,3.795)--(6.041,3.802)--(6.033,3.809)--(6.024,3.816)%
--(6.015,3.822)--(6.006,3.828)--(5.996,3.833)--(5.987,3.838)--(5.977,3.842)%
--(5.966,3.845)--(5.956,3.848)--(5.946,3.851)--(5.935,3.853)--(5.924,3.854)%
--(5.913,3.855)--(5.903,3.856)--(5.892,3.855)--(5.881,3.854)--(5.870,3.853)%
--(5.859,3.851)--(5.849,3.848)--(5.839,3.845)--(5.828,3.842)--(5.818,3.838)%
--(5.809,3.833)--(5.799,3.828)--(5.790,3.822)--(5.781,3.816)--(5.772,3.809)%
--(5.764,3.802)--(5.756,3.795)--(5.749,3.787)--(5.742,3.779)--(5.735,3.770)%
--(5.729,3.761)--(5.723,3.752)--(5.718,3.742)--(5.713,3.733)--(5.709,3.723)%
--(5.706,3.712)--(5.703,3.702)--(5.700,3.692)--(5.698,3.681)--(5.697,3.670)%
--(5.696,3.659)--(5.696,3.649)--(5.696,3.638)--(5.697,3.627)--(5.698,3.616)%
--(5.700,3.605)--(5.703,3.595)--(5.706,3.585)--(5.709,3.574)--(5.713,3.564)%
--(5.718,3.555)--(5.723,3.545)--(5.729,3.536)--(5.735,3.527)--(5.742,3.518)%
--(5.749,3.510)--(5.756,3.502)--(5.764,3.495)--(5.772,3.488)--(5.781,3.481)%
--(5.790,3.475)--(5.799,3.469)--(5.809,3.464)--(5.818,3.459)--(5.828,3.455)%
--(5.839,3.452)--(5.849,3.449)--(5.859,3.446)--(5.870,3.444)--(5.881,3.443)%
--(5.892,3.442)--(5.903,3.442)--(5.913,3.442)--(5.924,3.443)--(5.935,3.444)%
--(5.946,3.446)--(5.956,3.449)--(5.966,3.452)--(5.977,3.455)--(5.987,3.459)%
--(5.996,3.464)--(6.006,3.469)--(6.015,3.475)--(6.024,3.481)--(6.033,3.488)%
--(6.041,3.495)--(6.049,3.502)--(6.056,3.510)--(6.063,3.518)--(6.070,3.527)%
--(6.076,3.536)--(6.082,3.545)--(6.087,3.555)--(6.092,3.564)--(6.096,3.574)%
--(6.099,3.585)--(6.102,3.595)--(6.105,3.605)--(6.107,3.616)--(6.108,3.627)--(6.109,3.638)--cycle;
%
\gpfill{rgb color={0.000,0.000,0.000},opacity=0.15} (6.305,3.665)--(6.304,3.675)--(6.303,3.686)--(6.302,3.696)%
--(6.300,3.706)--(6.298,3.717)--(6.295,3.727)--(6.291,3.737)--(6.287,3.747)%
--(6.282,3.756)--(6.277,3.766)--(6.272,3.775)--(6.266,3.783)--(6.259,3.792)%
--(6.253,3.800)--(6.245,3.807)--(6.238,3.815)--(6.230,3.821)--(6.221,3.828)%
--(6.213,3.834)--(6.204,3.839)--(6.194,3.844)--(6.185,3.849)--(6.175,3.853)%
--(6.165,3.857)--(6.155,3.860)--(6.144,3.862)--(6.134,3.864)--(6.124,3.865)%
--(6.113,3.866)--(6.103,3.867)--(6.092,3.866)--(6.081,3.865)--(6.071,3.864)%
--(6.061,3.862)--(6.050,3.860)--(6.040,3.857)--(6.030,3.853)--(6.020,3.849)%
--(6.011,3.844)--(6.002,3.839)--(5.992,3.834)--(5.984,3.828)--(5.975,3.821)%
--(5.967,3.815)--(5.960,3.807)--(5.952,3.800)--(5.946,3.792)--(5.939,3.783)%
--(5.933,3.775)--(5.928,3.766)--(5.923,3.756)--(5.918,3.747)--(5.914,3.737)%
--(5.910,3.727)--(5.907,3.717)--(5.905,3.706)--(5.903,3.696)--(5.902,3.686)%
--(5.901,3.675)--(5.901,3.665)--(5.901,3.654)--(5.902,3.643)--(5.903,3.633)%
--(5.905,3.623)--(5.907,3.612)--(5.910,3.602)--(5.914,3.592)--(5.918,3.582)%
--(5.923,3.573)--(5.928,3.564)--(5.933,3.554)--(5.939,3.546)--(5.946,3.537)%
--(5.952,3.529)--(5.960,3.522)--(5.967,3.514)--(5.975,3.508)--(5.984,3.501)%
--(5.992,3.495)--(6.002,3.490)--(6.011,3.485)--(6.020,3.480)--(6.030,3.476)%
--(6.040,3.472)--(6.050,3.469)--(6.061,3.467)--(6.071,3.465)--(6.081,3.464)%
--(6.092,3.463)--(6.103,3.463)--(6.113,3.463)--(6.124,3.464)--(6.134,3.465)%
--(6.144,3.467)--(6.155,3.469)--(6.165,3.472)--(6.175,3.476)--(6.185,3.480)%
--(6.194,3.485)--(6.204,3.490)--(6.213,3.495)--(6.221,3.501)--(6.230,3.508)%
--(6.238,3.514)--(6.245,3.522)--(6.253,3.529)--(6.259,3.537)--(6.266,3.546)%
--(6.272,3.554)--(6.277,3.564)--(6.282,3.573)--(6.287,3.582)--(6.291,3.592)%
--(6.295,3.602)--(6.298,3.612)--(6.300,3.623)--(6.302,3.633)--(6.303,3.643)--(6.304,3.654)--cycle;
%
\gpfill{rgb color={0.000,0.000,0.000},opacity=0.15} (6.505,3.682)--(6.504,3.692)--(6.503,3.703)--(6.502,3.713)%
--(6.500,3.723)--(6.498,3.734)--(6.495,3.744)--(6.491,3.754)--(6.487,3.763)%
--(6.483,3.773)--(6.478,3.782)--(6.472,3.791)--(6.466,3.800)--(6.460,3.808)%
--(6.453,3.816)--(6.446,3.824)--(6.438,3.831)--(6.430,3.838)--(6.422,3.844)%
--(6.413,3.850)--(6.404,3.856)--(6.395,3.861)--(6.385,3.865)--(6.376,3.869)%
--(6.366,3.873)--(6.356,3.876)--(6.345,3.878)--(6.335,3.880)--(6.325,3.881)%
--(6.314,3.882)--(6.304,3.883)--(6.293,3.882)--(6.282,3.881)--(6.272,3.880)%
--(6.262,3.878)--(6.251,3.876)--(6.241,3.873)--(6.231,3.869)--(6.222,3.865)%
--(6.212,3.861)--(6.203,3.856)--(6.194,3.850)--(6.185,3.844)--(6.177,3.838)%
--(6.169,3.831)--(6.161,3.824)--(6.154,3.816)--(6.147,3.808)--(6.141,3.800)%
--(6.135,3.791)--(6.129,3.782)--(6.124,3.773)--(6.120,3.763)--(6.116,3.754)%
--(6.112,3.744)--(6.109,3.734)--(6.107,3.723)--(6.105,3.713)--(6.104,3.703)%
--(6.103,3.692)--(6.103,3.682)--(6.103,3.671)--(6.104,3.660)--(6.105,3.650)%
--(6.107,3.640)--(6.109,3.629)--(6.112,3.619)--(6.116,3.609)--(6.120,3.600)%
--(6.124,3.590)--(6.129,3.581)--(6.135,3.572)--(6.141,3.563)--(6.147,3.555)%
--(6.154,3.547)--(6.161,3.539)--(6.169,3.532)--(6.177,3.525)--(6.185,3.519)%
--(6.194,3.513)--(6.203,3.507)--(6.212,3.502)--(6.222,3.498)--(6.231,3.494)%
--(6.241,3.490)--(6.251,3.487)--(6.262,3.485)--(6.272,3.483)--(6.282,3.482)%
--(6.293,3.481)--(6.304,3.481)--(6.314,3.481)--(6.325,3.482)--(6.335,3.483)%
--(6.345,3.485)--(6.356,3.487)--(6.366,3.490)--(6.376,3.494)--(6.385,3.498)%
--(6.395,3.502)--(6.404,3.507)--(6.413,3.513)--(6.422,3.519)--(6.430,3.525)%
--(6.438,3.532)--(6.446,3.539)--(6.453,3.547)--(6.460,3.555)--(6.466,3.563)%
--(6.472,3.572)--(6.478,3.581)--(6.483,3.590)--(6.487,3.600)--(6.491,3.609)%
--(6.495,3.619)--(6.498,3.629)--(6.500,3.640)--(6.502,3.650)--(6.503,3.660)--(6.504,3.671)--cycle;
%
\gpfill{rgb color={0.000,0.000,0.000},opacity=0.15} (6.705,3.699)--(6.704,3.709)--(6.703,3.720)--(6.702,3.730)%
--(6.700,3.740)--(6.698,3.751)--(6.695,3.761)--(6.691,3.771)--(6.687,3.780)%
--(6.683,3.790)--(6.678,3.799)--(6.672,3.808)--(6.666,3.817)--(6.660,3.825)%
--(6.653,3.833)--(6.646,3.841)--(6.638,3.848)--(6.630,3.855)--(6.622,3.861)%
--(6.613,3.867)--(6.604,3.873)--(6.595,3.878)--(6.585,3.882)--(6.576,3.886)%
--(6.566,3.890)--(6.556,3.893)--(6.545,3.895)--(6.535,3.897)--(6.525,3.898)%
--(6.514,3.899)--(6.504,3.900)--(6.493,3.899)--(6.482,3.898)--(6.472,3.897)%
--(6.462,3.895)--(6.451,3.893)--(6.441,3.890)--(6.431,3.886)--(6.422,3.882)%
--(6.412,3.878)--(6.403,3.873)--(6.394,3.867)--(6.385,3.861)--(6.377,3.855)%
--(6.369,3.848)--(6.361,3.841)--(6.354,3.833)--(6.347,3.825)--(6.341,3.817)%
--(6.335,3.808)--(6.329,3.799)--(6.324,3.790)--(6.320,3.780)--(6.316,3.771)%
--(6.312,3.761)--(6.309,3.751)--(6.307,3.740)--(6.305,3.730)--(6.304,3.720)%
--(6.303,3.709)--(6.303,3.699)--(6.303,3.688)--(6.304,3.677)--(6.305,3.667)%
--(6.307,3.657)--(6.309,3.646)--(6.312,3.636)--(6.316,3.626)--(6.320,3.617)%
--(6.324,3.607)--(6.329,3.598)--(6.335,3.589)--(6.341,3.580)--(6.347,3.572)%
--(6.354,3.564)--(6.361,3.556)--(6.369,3.549)--(6.377,3.542)--(6.385,3.536)%
--(6.394,3.530)--(6.403,3.524)--(6.412,3.519)--(6.422,3.515)--(6.431,3.511)%
--(6.441,3.507)--(6.451,3.504)--(6.462,3.502)--(6.472,3.500)--(6.482,3.499)%
--(6.493,3.498)--(6.504,3.498)--(6.514,3.498)--(6.525,3.499)--(6.535,3.500)%
--(6.545,3.502)--(6.556,3.504)--(6.566,3.507)--(6.576,3.511)--(6.585,3.515)%
--(6.595,3.519)--(6.604,3.524)--(6.613,3.530)--(6.622,3.536)--(6.630,3.542)%
--(6.638,3.549)--(6.646,3.556)--(6.653,3.564)--(6.660,3.572)--(6.666,3.580)%
--(6.672,3.589)--(6.678,3.598)--(6.683,3.607)--(6.687,3.617)--(6.691,3.626)%
--(6.695,3.636)--(6.698,3.646)--(6.700,3.657)--(6.702,3.667)--(6.703,3.677)--(6.704,3.688)--cycle;
%
\gpfill{rgb color={0.000,0.000,0.000},opacity=0.15} (6.905,3.716)--(6.904,3.726)--(6.903,3.737)--(6.902,3.747)%
--(6.900,3.757)--(6.898,3.768)--(6.895,3.778)--(6.891,3.788)--(6.887,3.797)%
--(6.883,3.807)--(6.878,3.816)--(6.872,3.825)--(6.866,3.834)--(6.860,3.842)%
--(6.853,3.850)--(6.846,3.858)--(6.838,3.865)--(6.830,3.872)--(6.822,3.878)%
--(6.813,3.884)--(6.804,3.890)--(6.795,3.895)--(6.785,3.899)--(6.776,3.903)%
--(6.766,3.907)--(6.756,3.910)--(6.745,3.912)--(6.735,3.914)--(6.725,3.915)%
--(6.714,3.916)--(6.704,3.917)--(6.693,3.916)--(6.682,3.915)--(6.672,3.914)%
--(6.662,3.912)--(6.651,3.910)--(6.641,3.907)--(6.631,3.903)--(6.622,3.899)%
--(6.612,3.895)--(6.603,3.890)--(6.594,3.884)--(6.585,3.878)--(6.577,3.872)%
--(6.569,3.865)--(6.561,3.858)--(6.554,3.850)--(6.547,3.842)--(6.541,3.834)%
--(6.535,3.825)--(6.529,3.816)--(6.524,3.807)--(6.520,3.797)--(6.516,3.788)%
--(6.512,3.778)--(6.509,3.768)--(6.507,3.757)--(6.505,3.747)--(6.504,3.737)%
--(6.503,3.726)--(6.503,3.716)--(6.503,3.705)--(6.504,3.694)--(6.505,3.684)%
--(6.507,3.674)--(6.509,3.663)--(6.512,3.653)--(6.516,3.643)--(6.520,3.634)%
--(6.524,3.624)--(6.529,3.615)--(6.535,3.606)--(6.541,3.597)--(6.547,3.589)%
--(6.554,3.581)--(6.561,3.573)--(6.569,3.566)--(6.577,3.559)--(6.585,3.553)%
--(6.594,3.547)--(6.603,3.541)--(6.612,3.536)--(6.622,3.532)--(6.631,3.528)%
--(6.641,3.524)--(6.651,3.521)--(6.662,3.519)--(6.672,3.517)--(6.682,3.516)%
--(6.693,3.515)--(6.704,3.515)--(6.714,3.515)--(6.725,3.516)--(6.735,3.517)%
--(6.745,3.519)--(6.756,3.521)--(6.766,3.524)--(6.776,3.528)--(6.785,3.532)%
--(6.795,3.536)--(6.804,3.541)--(6.813,3.547)--(6.822,3.553)--(6.830,3.559)%
--(6.838,3.566)--(6.846,3.573)--(6.853,3.581)--(6.860,3.589)--(6.866,3.597)%
--(6.872,3.606)--(6.878,3.615)--(6.883,3.624)--(6.887,3.634)--(6.891,3.643)%
--(6.895,3.653)--(6.898,3.663)--(6.900,3.674)--(6.902,3.684)--(6.903,3.694)--(6.904,3.705)--cycle;
%
\gpfill{rgb color={0.000,0.000,0.000},opacity=0.15} (7.103,3.734)--(7.102,3.744)--(7.101,3.754)--(7.100,3.765)%
--(7.098,3.775)--(7.096,3.785)--(7.093,3.795)--(7.089,3.805)--(7.085,3.815)%
--(7.081,3.824)--(7.076,3.834)--(7.070,3.842)--(7.064,3.851)--(7.058,3.859)%
--(7.051,3.867)--(7.044,3.875)--(7.036,3.882)--(7.028,3.889)--(7.020,3.895)%
--(7.011,3.901)--(7.003,3.907)--(6.993,3.912)--(6.984,3.916)--(6.974,3.920)%
--(6.964,3.924)--(6.954,3.927)--(6.944,3.929)--(6.934,3.931)--(6.923,3.932)%
--(6.913,3.933)--(6.903,3.934)--(6.892,3.933)--(6.882,3.932)--(6.871,3.931)%
--(6.861,3.929)--(6.851,3.927)--(6.841,3.924)--(6.831,3.920)--(6.821,3.916)%
--(6.812,3.912)--(6.803,3.907)--(6.794,3.901)--(6.785,3.895)--(6.777,3.889)%
--(6.769,3.882)--(6.761,3.875)--(6.754,3.867)--(6.747,3.859)--(6.741,3.851)%
--(6.735,3.842)--(6.729,3.834)--(6.724,3.824)--(6.720,3.815)--(6.716,3.805)%
--(6.712,3.795)--(6.709,3.785)--(6.707,3.775)--(6.705,3.765)--(6.704,3.754)%
--(6.703,3.744)--(6.703,3.734)--(6.703,3.723)--(6.704,3.713)--(6.705,3.702)%
--(6.707,3.692)--(6.709,3.682)--(6.712,3.672)--(6.716,3.662)--(6.720,3.652)%
--(6.724,3.643)--(6.729,3.634)--(6.735,3.625)--(6.741,3.616)--(6.747,3.608)%
--(6.754,3.600)--(6.761,3.592)--(6.769,3.585)--(6.777,3.578)--(6.785,3.572)%
--(6.794,3.566)--(6.803,3.560)--(6.812,3.555)--(6.821,3.551)--(6.831,3.547)%
--(6.841,3.543)--(6.851,3.540)--(6.861,3.538)--(6.871,3.536)--(6.882,3.535)%
--(6.892,3.534)--(6.903,3.534)--(6.913,3.534)--(6.923,3.535)--(6.934,3.536)%
--(6.944,3.538)--(6.954,3.540)--(6.964,3.543)--(6.974,3.547)--(6.984,3.551)%
--(6.993,3.555)--(7.003,3.560)--(7.011,3.566)--(7.020,3.572)--(7.028,3.578)%
--(7.036,3.585)--(7.044,3.592)--(7.051,3.600)--(7.058,3.608)--(7.064,3.616)%
--(7.070,3.625)--(7.076,3.634)--(7.081,3.643)--(7.085,3.652)--(7.089,3.662)%
--(7.093,3.672)--(7.096,3.682)--(7.098,3.692)--(7.100,3.702)--(7.101,3.713)--(7.102,3.723)--cycle;
%
\gpfill{rgb color={0.000,0.000,0.000},opacity=0.15} (7.298,3.753)--(7.297,3.763)--(7.296,3.773)--(7.295,3.783)%
--(7.293,3.794)--(7.291,3.804)--(7.288,3.814)--(7.284,3.823)--(7.280,3.833)%
--(7.276,3.842)--(7.271,3.852)--(7.266,3.860)--(7.260,3.869)--(7.253,3.877)%
--(7.247,3.885)--(7.240,3.893)--(7.232,3.900)--(7.224,3.906)--(7.216,3.913)%
--(7.207,3.919)--(7.199,3.924)--(7.189,3.929)--(7.180,3.933)--(7.170,3.937)%
--(7.161,3.941)--(7.151,3.944)--(7.141,3.946)--(7.130,3.948)--(7.120,3.949)%
--(7.110,3.950)--(7.100,3.951)--(7.089,3.950)--(7.079,3.949)--(7.069,3.948)%
--(7.058,3.946)--(7.048,3.944)--(7.038,3.941)--(7.029,3.937)--(7.019,3.933)%
--(7.010,3.929)--(7.001,3.924)--(6.992,3.919)--(6.983,3.913)--(6.975,3.906)%
--(6.967,3.900)--(6.959,3.893)--(6.952,3.885)--(6.946,3.877)--(6.939,3.869)%
--(6.933,3.860)--(6.928,3.852)--(6.923,3.842)--(6.919,3.833)--(6.915,3.823)%
--(6.911,3.814)--(6.908,3.804)--(6.906,3.794)--(6.904,3.783)--(6.903,3.773)%
--(6.902,3.763)--(6.902,3.753)--(6.902,3.742)--(6.903,3.732)--(6.904,3.722)%
--(6.906,3.711)--(6.908,3.701)--(6.911,3.691)--(6.915,3.682)--(6.919,3.672)%
--(6.923,3.663)--(6.928,3.654)--(6.933,3.645)--(6.939,3.636)--(6.946,3.628)%
--(6.952,3.620)--(6.959,3.612)--(6.967,3.605)--(6.975,3.599)--(6.983,3.592)%
--(6.992,3.586)--(7.001,3.581)--(7.010,3.576)--(7.019,3.572)--(7.029,3.568)%
--(7.038,3.564)--(7.048,3.561)--(7.058,3.559)--(7.069,3.557)--(7.079,3.556)%
--(7.089,3.555)--(7.100,3.555)--(7.110,3.555)--(7.120,3.556)--(7.130,3.557)%
--(7.141,3.559)--(7.151,3.561)--(7.161,3.564)--(7.170,3.568)--(7.180,3.572)%
--(7.189,3.576)--(7.199,3.581)--(7.207,3.586)--(7.216,3.592)--(7.224,3.599)%
--(7.232,3.605)--(7.240,3.612)--(7.247,3.620)--(7.253,3.628)--(7.260,3.636)%
--(7.266,3.645)--(7.271,3.654)--(7.276,3.663)--(7.280,3.672)--(7.284,3.682)%
--(7.288,3.691)--(7.291,3.701)--(7.293,3.711)--(7.295,3.722)--(7.296,3.732)--(7.297,3.742)--cycle;
%
\gpfill{rgb color={0.000,0.000,0.000},opacity=0.15} (7.493,3.772)--(7.492,3.782)--(7.491,3.792)--(7.490,3.802)%
--(7.488,3.812)--(7.486,3.822)--(7.483,3.832)--(7.479,3.842)--(7.475,3.852)%
--(7.471,3.861)--(7.466,3.870)--(7.461,3.879)--(7.455,3.887)--(7.449,3.895)%
--(7.442,3.903)--(7.435,3.911)--(7.427,3.918)--(7.419,3.925)--(7.411,3.931)%
--(7.403,3.937)--(7.394,3.942)--(7.385,3.947)--(7.376,3.951)--(7.366,3.955)%
--(7.356,3.959)--(7.346,3.962)--(7.336,3.964)--(7.326,3.966)--(7.316,3.967)%
--(7.306,3.968)--(7.296,3.969)--(7.285,3.968)--(7.275,3.967)--(7.265,3.966)%
--(7.255,3.964)--(7.245,3.962)--(7.235,3.959)--(7.225,3.955)--(7.215,3.951)%
--(7.206,3.947)--(7.197,3.942)--(7.188,3.937)--(7.180,3.931)--(7.172,3.925)%
--(7.164,3.918)--(7.156,3.911)--(7.149,3.903)--(7.142,3.895)--(7.136,3.887)%
--(7.130,3.879)--(7.125,3.870)--(7.120,3.861)--(7.116,3.852)--(7.112,3.842)%
--(7.108,3.832)--(7.105,3.822)--(7.103,3.812)--(7.101,3.802)--(7.100,3.792)%
--(7.099,3.782)--(7.099,3.772)--(7.099,3.761)--(7.100,3.751)--(7.101,3.741)%
--(7.103,3.731)--(7.105,3.721)--(7.108,3.711)--(7.112,3.701)--(7.116,3.691)%
--(7.120,3.682)--(7.125,3.673)--(7.130,3.664)--(7.136,3.656)--(7.142,3.648)%
--(7.149,3.640)--(7.156,3.632)--(7.164,3.625)--(7.172,3.618)--(7.180,3.612)%
--(7.188,3.606)--(7.197,3.601)--(7.206,3.596)--(7.215,3.592)--(7.225,3.588)%
--(7.235,3.584)--(7.245,3.581)--(7.255,3.579)--(7.265,3.577)--(7.275,3.576)%
--(7.285,3.575)--(7.296,3.575)--(7.306,3.575)--(7.316,3.576)--(7.326,3.577)%
--(7.336,3.579)--(7.346,3.581)--(7.356,3.584)--(7.366,3.588)--(7.376,3.592)%
--(7.385,3.596)--(7.394,3.601)--(7.403,3.606)--(7.411,3.612)--(7.419,3.618)%
--(7.427,3.625)--(7.435,3.632)--(7.442,3.640)--(7.449,3.648)--(7.455,3.656)%
--(7.461,3.664)--(7.466,3.673)--(7.471,3.682)--(7.475,3.691)--(7.479,3.701)%
--(7.483,3.711)--(7.486,3.721)--(7.488,3.731)--(7.490,3.741)--(7.491,3.751)--(7.492,3.761)--cycle;
%
\gpfill{rgb color={0.000,0.000,0.000},opacity=0.15} (7.687,3.792)--(7.686,3.802)--(7.685,3.812)--(7.684,3.822)%
--(7.682,3.832)--(7.680,3.842)--(7.677,3.852)--(7.673,3.862)--(7.670,3.871)%
--(7.665,3.880)--(7.660,3.890)--(7.655,3.898)--(7.649,3.907)--(7.643,3.915)%
--(7.636,3.923)--(7.629,3.930)--(7.622,3.937)--(7.614,3.944)--(7.606,3.950)%
--(7.597,3.956)--(7.589,3.961)--(7.579,3.966)--(7.570,3.971)--(7.561,3.974)%
--(7.551,3.978)--(7.541,3.981)--(7.531,3.983)--(7.521,3.985)--(7.511,3.986)%
--(7.501,3.987)--(7.491,3.988)--(7.480,3.987)--(7.470,3.986)--(7.460,3.985)%
--(7.450,3.983)--(7.440,3.981)--(7.430,3.978)--(7.420,3.974)--(7.411,3.971)%
--(7.402,3.966)--(7.393,3.961)--(7.384,3.956)--(7.375,3.950)--(7.367,3.944)%
--(7.359,3.937)--(7.352,3.930)--(7.345,3.923)--(7.338,3.915)--(7.332,3.907)%
--(7.326,3.898)--(7.321,3.890)--(7.316,3.880)--(7.311,3.871)--(7.308,3.862)%
--(7.304,3.852)--(7.301,3.842)--(7.299,3.832)--(7.297,3.822)--(7.296,3.812)%
--(7.295,3.802)--(7.295,3.792)--(7.295,3.781)--(7.296,3.771)--(7.297,3.761)%
--(7.299,3.751)--(7.301,3.741)--(7.304,3.731)--(7.308,3.721)--(7.311,3.712)%
--(7.316,3.703)--(7.321,3.694)--(7.326,3.685)--(7.332,3.676)--(7.338,3.668)%
--(7.345,3.660)--(7.352,3.653)--(7.359,3.646)--(7.367,3.639)--(7.375,3.633)%
--(7.384,3.627)--(7.393,3.622)--(7.402,3.617)--(7.411,3.612)--(7.420,3.609)%
--(7.430,3.605)--(7.440,3.602)--(7.450,3.600)--(7.460,3.598)--(7.470,3.597)%
--(7.480,3.596)--(7.491,3.596)--(7.501,3.596)--(7.511,3.597)--(7.521,3.598)%
--(7.531,3.600)--(7.541,3.602)--(7.551,3.605)--(7.561,3.609)--(7.570,3.612)%
--(7.579,3.617)--(7.589,3.622)--(7.597,3.627)--(7.606,3.633)--(7.614,3.639)%
--(7.622,3.646)--(7.629,3.653)--(7.636,3.660)--(7.643,3.668)--(7.649,3.676)%
--(7.655,3.685)--(7.660,3.694)--(7.665,3.703)--(7.670,3.712)--(7.673,3.721)%
--(7.677,3.731)--(7.680,3.741)--(7.682,3.751)--(7.684,3.761)--(7.685,3.771)--(7.686,3.781)--cycle;
%
\gpfill{rgb color={0.000,0.000,0.000},opacity=0.15} (7.876,3.812)--(7.875,3.822)--(7.874,3.832)--(7.873,3.842)%
--(7.871,3.852)--(7.869,3.861)--(7.866,3.871)--(7.863,3.881)--(7.859,3.890)%
--(7.854,3.899)--(7.850,3.908)--(7.844,3.917)--(7.839,3.925)--(7.832,3.933)%
--(7.826,3.941)--(7.819,3.948)--(7.812,3.955)--(7.804,3.961)--(7.796,3.968)%
--(7.788,3.973)--(7.779,3.979)--(7.770,3.983)--(7.761,3.988)--(7.752,3.992)%
--(7.742,3.995)--(7.732,3.998)--(7.723,4.000)--(7.713,4.002)--(7.703,4.003)%
--(7.693,4.004)--(7.683,4.005)--(7.672,4.004)--(7.662,4.003)--(7.652,4.002)%
--(7.642,4.000)--(7.633,3.998)--(7.623,3.995)--(7.613,3.992)--(7.604,3.988)%
--(7.595,3.983)--(7.586,3.979)--(7.577,3.973)--(7.569,3.968)--(7.561,3.961)%
--(7.553,3.955)--(7.546,3.948)--(7.539,3.941)--(7.533,3.933)--(7.526,3.925)%
--(7.521,3.917)--(7.515,3.908)--(7.511,3.899)--(7.506,3.890)--(7.502,3.881)%
--(7.499,3.871)--(7.496,3.861)--(7.494,3.852)--(7.492,3.842)--(7.491,3.832)%
--(7.490,3.822)--(7.490,3.812)--(7.490,3.801)--(7.491,3.791)--(7.492,3.781)%
--(7.494,3.771)--(7.496,3.762)--(7.499,3.752)--(7.502,3.742)--(7.506,3.733)%
--(7.511,3.724)--(7.515,3.715)--(7.521,3.706)--(7.526,3.698)--(7.533,3.690)%
--(7.539,3.682)--(7.546,3.675)--(7.553,3.668)--(7.561,3.662)--(7.569,3.655)%
--(7.577,3.650)--(7.586,3.644)--(7.595,3.640)--(7.604,3.635)--(7.613,3.631)%
--(7.623,3.628)--(7.633,3.625)--(7.642,3.623)--(7.652,3.621)--(7.662,3.620)%
--(7.672,3.619)--(7.683,3.619)--(7.693,3.619)--(7.703,3.620)--(7.713,3.621)%
--(7.723,3.623)--(7.732,3.625)--(7.742,3.628)--(7.752,3.631)--(7.761,3.635)%
--(7.770,3.640)--(7.779,3.644)--(7.788,3.650)--(7.796,3.655)--(7.804,3.662)%
--(7.812,3.668)--(7.819,3.675)--(7.826,3.682)--(7.832,3.690)--(7.839,3.698)%
--(7.844,3.706)--(7.850,3.715)--(7.854,3.724)--(7.859,3.733)--(7.863,3.742)%
--(7.866,3.752)--(7.869,3.762)--(7.871,3.771)--(7.873,3.781)--(7.874,3.791)--(7.875,3.801)--cycle;
%
\gpfill{rgb color={0.000,0.000,0.000},opacity=0.15} (8.063,3.833)--(8.062,3.842)--(8.061,3.852)--(8.060,3.862)%
--(8.058,3.872)--(8.056,3.882)--(8.053,3.892)--(8.050,3.901)--(8.046,3.910)%
--(8.042,3.919)--(8.037,3.928)--(8.032,3.937)--(8.026,3.945)--(8.020,3.953)%
--(8.013,3.960)--(8.007,3.968)--(7.999,3.974)--(7.992,3.981)--(7.984,3.987)%
--(7.976,3.993)--(7.967,3.998)--(7.958,4.003)--(7.949,4.007)--(7.940,4.011)%
--(7.931,4.014)--(7.921,4.017)--(7.911,4.019)--(7.901,4.021)--(7.891,4.022)%
--(7.881,4.023)--(7.872,4.024)--(7.862,4.023)--(7.852,4.022)--(7.842,4.021)%
--(7.832,4.019)--(7.822,4.017)--(7.812,4.014)--(7.803,4.011)--(7.794,4.007)%
--(7.785,4.003)--(7.776,3.998)--(7.767,3.993)--(7.759,3.987)--(7.751,3.981)%
--(7.744,3.974)--(7.736,3.968)--(7.730,3.960)--(7.723,3.953)--(7.717,3.945)%
--(7.711,3.937)--(7.706,3.928)--(7.701,3.919)--(7.697,3.910)--(7.693,3.901)%
--(7.690,3.892)--(7.687,3.882)--(7.685,3.872)--(7.683,3.862)--(7.682,3.852)%
--(7.681,3.842)--(7.681,3.833)--(7.681,3.823)--(7.682,3.813)--(7.683,3.803)%
--(7.685,3.793)--(7.687,3.783)--(7.690,3.773)--(7.693,3.764)--(7.697,3.755)%
--(7.701,3.746)--(7.706,3.737)--(7.711,3.728)--(7.717,3.720)--(7.723,3.712)%
--(7.730,3.705)--(7.736,3.697)--(7.744,3.691)--(7.751,3.684)--(7.759,3.678)%
--(7.767,3.672)--(7.776,3.667)--(7.785,3.662)--(7.794,3.658)--(7.803,3.654)%
--(7.812,3.651)--(7.822,3.648)--(7.832,3.646)--(7.842,3.644)--(7.852,3.643)%
--(7.862,3.642)--(7.872,3.642)--(7.881,3.642)--(7.891,3.643)--(7.901,3.644)%
--(7.911,3.646)--(7.921,3.648)--(7.931,3.651)--(7.940,3.654)--(7.949,3.658)%
--(7.958,3.662)--(7.967,3.667)--(7.976,3.672)--(7.984,3.678)--(7.992,3.684)%
--(7.999,3.691)--(8.007,3.697)--(8.013,3.705)--(8.020,3.712)--(8.026,3.720)%
--(8.032,3.728)--(8.037,3.737)--(8.042,3.746)--(8.046,3.755)--(8.050,3.764)%
--(8.053,3.773)--(8.056,3.783)--(8.058,3.793)--(8.060,3.803)--(8.061,3.813)--(8.062,3.823)--cycle;
%
\gpfill{rgb color={0.000,0.000,0.000},opacity=0.15} (8.245,3.855)--(8.244,3.864)--(8.243,3.874)--(8.242,3.884)%
--(8.240,3.893)--(8.238,3.903)--(8.235,3.912)--(8.232,3.922)--(8.228,3.931)%
--(8.224,3.939)--(8.219,3.948)--(8.214,3.956)--(8.209,3.964)--(8.203,3.972)%
--(8.196,3.980)--(8.190,3.987)--(8.183,3.993)--(8.175,4.000)--(8.167,4.006)%
--(8.159,4.011)--(8.151,4.016)--(8.142,4.021)--(8.134,4.025)--(8.125,4.029)%
--(8.115,4.032)--(8.106,4.035)--(8.096,4.037)--(8.087,4.039)--(8.077,4.040)%
--(8.067,4.041)--(8.058,4.042)--(8.048,4.041)--(8.038,4.040)--(8.028,4.039)%
--(8.019,4.037)--(8.009,4.035)--(8.000,4.032)--(7.990,4.029)--(7.981,4.025)%
--(7.973,4.021)--(7.964,4.016)--(7.956,4.011)--(7.948,4.006)--(7.940,4.000)%
--(7.932,3.993)--(7.925,3.987)--(7.919,3.980)--(7.912,3.972)--(7.906,3.964)%
--(7.901,3.956)--(7.896,3.948)--(7.891,3.939)--(7.887,3.931)--(7.883,3.922)%
--(7.880,3.912)--(7.877,3.903)--(7.875,3.893)--(7.873,3.884)--(7.872,3.874)%
--(7.871,3.864)--(7.871,3.855)--(7.871,3.845)--(7.872,3.835)--(7.873,3.825)%
--(7.875,3.816)--(7.877,3.806)--(7.880,3.797)--(7.883,3.787)--(7.887,3.778)%
--(7.891,3.770)--(7.896,3.761)--(7.901,3.753)--(7.906,3.745)--(7.912,3.737)%
--(7.919,3.729)--(7.925,3.722)--(7.932,3.716)--(7.940,3.709)--(7.948,3.703)%
--(7.956,3.698)--(7.964,3.693)--(7.973,3.688)--(7.981,3.684)--(7.990,3.680)%
--(8.000,3.677)--(8.009,3.674)--(8.019,3.672)--(8.028,3.670)--(8.038,3.669)%
--(8.048,3.668)--(8.058,3.668)--(8.067,3.668)--(8.077,3.669)--(8.087,3.670)%
--(8.096,3.672)--(8.106,3.674)--(8.115,3.677)--(8.125,3.680)--(8.134,3.684)%
--(8.142,3.688)--(8.151,3.693)--(8.159,3.698)--(8.167,3.703)--(8.175,3.709)%
--(8.183,3.716)--(8.190,3.722)--(8.196,3.729)--(8.203,3.737)--(8.209,3.745)%
--(8.214,3.753)--(8.219,3.761)--(8.224,3.770)--(8.228,3.778)--(8.232,3.787)%
--(8.235,3.797)--(8.238,3.806)--(8.240,3.816)--(8.242,3.825)--(8.243,3.835)--(8.244,3.845)--cycle;
%
\gpfill{rgb color={0.000,0.000,0.000},opacity=0.15} (8.425,3.877)--(8.424,3.886)--(8.423,3.896)--(8.422,3.905)%
--(8.420,3.915)--(8.418,3.924)--(8.415,3.933)--(8.412,3.942)--(8.409,3.951)%
--(8.404,3.960)--(8.400,3.969)--(8.395,3.977)--(8.389,3.985)--(8.383,3.992)%
--(8.377,4.000)--(8.371,4.007)--(8.364,4.013)--(8.356,4.019)--(8.349,4.025)%
--(8.341,4.031)--(8.333,4.036)--(8.324,4.040)--(8.315,4.045)--(8.306,4.048)%
--(8.297,4.051)--(8.288,4.054)--(8.279,4.056)--(8.269,4.058)--(8.260,4.059)%
--(8.250,4.060)--(8.241,4.061)--(8.231,4.060)--(8.221,4.059)--(8.212,4.058)%
--(8.202,4.056)--(8.193,4.054)--(8.184,4.051)--(8.175,4.048)--(8.166,4.045)%
--(8.157,4.040)--(8.149,4.036)--(8.140,4.031)--(8.132,4.025)--(8.125,4.019)%
--(8.117,4.013)--(8.110,4.007)--(8.104,4.000)--(8.098,3.992)--(8.092,3.985)%
--(8.086,3.977)--(8.081,3.969)--(8.077,3.960)--(8.072,3.951)--(8.069,3.942)%
--(8.066,3.933)--(8.063,3.924)--(8.061,3.915)--(8.059,3.905)--(8.058,3.896)%
--(8.057,3.886)--(8.057,3.877)--(8.057,3.867)--(8.058,3.857)--(8.059,3.848)%
--(8.061,3.838)--(8.063,3.829)--(8.066,3.820)--(8.069,3.811)--(8.072,3.802)%
--(8.077,3.793)--(8.081,3.785)--(8.086,3.776)--(8.092,3.768)--(8.098,3.761)%
--(8.104,3.753)--(8.110,3.746)--(8.117,3.740)--(8.125,3.734)--(8.132,3.728)%
--(8.140,3.722)--(8.149,3.717)--(8.157,3.713)--(8.166,3.708)--(8.175,3.705)%
--(8.184,3.702)--(8.193,3.699)--(8.202,3.697)--(8.212,3.695)--(8.221,3.694)%
--(8.231,3.693)--(8.241,3.693)--(8.250,3.693)--(8.260,3.694)--(8.269,3.695)%
--(8.279,3.697)--(8.288,3.699)--(8.297,3.702)--(8.306,3.705)--(8.315,3.708)%
--(8.324,3.713)--(8.333,3.717)--(8.341,3.722)--(8.349,3.728)--(8.356,3.734)%
--(8.364,3.740)--(8.371,3.746)--(8.377,3.753)--(8.383,3.761)--(8.389,3.768)%
--(8.395,3.776)--(8.400,3.785)--(8.404,3.793)--(8.409,3.802)--(8.412,3.811)%
--(8.415,3.820)--(8.418,3.829)--(8.420,3.838)--(8.422,3.848)--(8.423,3.857)--(8.424,3.867)--cycle;
%
\gpfill{rgb color={0.000,0.000,0.000},opacity=0.15} (8.602,3.900)--(8.601,3.909)--(8.601,3.918)--(8.599,3.928)%
--(8.598,3.937)--(8.595,3.946)--(8.593,3.955)--(8.589,3.964)--(8.586,3.973)%
--(8.582,3.982)--(8.577,3.990)--(8.572,3.998)--(8.567,4.006)--(8.561,4.013)%
--(8.555,4.021)--(8.548,4.027)--(8.542,4.034)--(8.534,4.040)--(8.527,4.046)%
--(8.519,4.051)--(8.511,4.056)--(8.503,4.061)--(8.494,4.065)--(8.485,4.068)%
--(8.476,4.072)--(8.467,4.074)--(8.458,4.077)--(8.449,4.078)--(8.439,4.080)%
--(8.430,4.080)--(8.421,4.081)--(8.411,4.080)--(8.402,4.080)--(8.392,4.078)%
--(8.383,4.077)--(8.374,4.074)--(8.365,4.072)--(8.356,4.068)--(8.347,4.065)%
--(8.338,4.061)--(8.330,4.056)--(8.322,4.051)--(8.314,4.046)--(8.307,4.040)%
--(8.299,4.034)--(8.293,4.027)--(8.286,4.021)--(8.280,4.013)--(8.274,4.006)%
--(8.269,3.998)--(8.264,3.990)--(8.259,3.982)--(8.255,3.973)--(8.252,3.964)%
--(8.248,3.955)--(8.246,3.946)--(8.243,3.937)--(8.242,3.928)--(8.240,3.918)%
--(8.240,3.909)--(8.240,3.900)--(8.240,3.890)--(8.240,3.881)--(8.242,3.871)%
--(8.243,3.862)--(8.246,3.853)--(8.248,3.844)--(8.252,3.835)--(8.255,3.826)%
--(8.259,3.817)--(8.264,3.809)--(8.269,3.801)--(8.274,3.793)--(8.280,3.786)%
--(8.286,3.778)--(8.293,3.772)--(8.299,3.765)--(8.307,3.759)--(8.314,3.753)%
--(8.322,3.748)--(8.330,3.743)--(8.338,3.738)--(8.347,3.734)--(8.356,3.731)%
--(8.365,3.727)--(8.374,3.725)--(8.383,3.722)--(8.392,3.721)--(8.402,3.719)%
--(8.411,3.719)--(8.421,3.719)--(8.430,3.719)--(8.439,3.719)--(8.449,3.721)%
--(8.458,3.722)--(8.467,3.725)--(8.476,3.727)--(8.485,3.731)--(8.494,3.734)%
--(8.503,3.738)--(8.511,3.743)--(8.519,3.748)--(8.527,3.753)--(8.534,3.759)%
--(8.542,3.765)--(8.548,3.772)--(8.555,3.778)--(8.561,3.786)--(8.567,3.793)%
--(8.572,3.801)--(8.577,3.809)--(8.582,3.817)--(8.586,3.826)--(8.589,3.835)%
--(8.593,3.844)--(8.595,3.853)--(8.598,3.862)--(8.599,3.871)--(8.601,3.881)--(8.601,3.890)--cycle;
%
\gpfill{rgb color={0.000,0.000,0.000},opacity=0.15} (8.775,3.923)--(8.774,3.932)--(8.774,3.941)--(8.772,3.950)%
--(8.771,3.960)--(8.768,3.969)--(8.766,3.978)--(8.763,3.986)--(8.759,3.995)%
--(8.755,4.003)--(8.751,4.012)--(8.746,4.019)--(8.741,4.027)--(8.735,4.035)%
--(8.729,4.042)--(8.722,4.048)--(8.716,4.055)--(8.709,4.061)--(8.701,4.067)%
--(8.693,4.072)--(8.686,4.077)--(8.677,4.081)--(8.669,4.085)--(8.660,4.089)%
--(8.652,4.092)--(8.643,4.094)--(8.634,4.097)--(8.624,4.098)--(8.615,4.100)%
--(8.606,4.100)--(8.597,4.101)--(8.587,4.100)--(8.578,4.100)--(8.569,4.098)%
--(8.559,4.097)--(8.550,4.094)--(8.541,4.092)--(8.533,4.089)--(8.524,4.085)%
--(8.516,4.081)--(8.508,4.077)--(8.500,4.072)--(8.492,4.067)--(8.484,4.061)%
--(8.477,4.055)--(8.471,4.048)--(8.464,4.042)--(8.458,4.035)--(8.452,4.027)%
--(8.447,4.019)--(8.442,4.012)--(8.438,4.003)--(8.434,3.995)--(8.430,3.986)%
--(8.427,3.978)--(8.425,3.969)--(8.422,3.960)--(8.421,3.950)--(8.419,3.941)%
--(8.419,3.932)--(8.419,3.923)--(8.419,3.913)--(8.419,3.904)--(8.421,3.895)%
--(8.422,3.885)--(8.425,3.876)--(8.427,3.867)--(8.430,3.859)--(8.434,3.850)%
--(8.438,3.842)--(8.442,3.834)--(8.447,3.826)--(8.452,3.818)--(8.458,3.810)%
--(8.464,3.803)--(8.471,3.797)--(8.477,3.790)--(8.484,3.784)--(8.492,3.778)%
--(8.500,3.773)--(8.508,3.768)--(8.516,3.764)--(8.524,3.760)--(8.533,3.756)%
--(8.541,3.753)--(8.550,3.751)--(8.559,3.748)--(8.569,3.747)--(8.578,3.745)%
--(8.587,3.745)--(8.597,3.745)--(8.606,3.745)--(8.615,3.745)--(8.624,3.747)%
--(8.634,3.748)--(8.643,3.751)--(8.652,3.753)--(8.660,3.756)--(8.669,3.760)%
--(8.677,3.764)--(8.686,3.768)--(8.693,3.773)--(8.701,3.778)--(8.709,3.784)%
--(8.716,3.790)--(8.722,3.797)--(8.729,3.803)--(8.735,3.810)--(8.741,3.818)%
--(8.746,3.826)--(8.751,3.834)--(8.755,3.842)--(8.759,3.850)--(8.763,3.859)%
--(8.766,3.867)--(8.768,3.876)--(8.771,3.885)--(8.772,3.895)--(8.774,3.904)--(8.774,3.913)--cycle;
%
\gpfill{rgb color={0.000,0.000,0.000},opacity=0.15} (8.941,3.947)--(8.940,3.956)--(8.940,3.965)--(8.938,3.974)%
--(8.937,3.982)--(8.935,3.991)--(8.932,4.000)--(8.929,4.008)--(8.926,4.017)%
--(8.922,4.025)--(8.917,4.033)--(8.913,4.041)--(8.907,4.048)--(8.902,4.055)%
--(8.896,4.062)--(8.890,4.069)--(8.883,4.075)--(8.876,4.081)--(8.869,4.086)%
--(8.862,4.092)--(8.854,4.096)--(8.846,4.101)--(8.838,4.105)--(8.829,4.108)%
--(8.821,4.111)--(8.812,4.114)--(8.803,4.116)--(8.795,4.117)--(8.786,4.119)%
--(8.777,4.119)--(8.768,4.120)--(8.758,4.119)--(8.749,4.119)--(8.740,4.117)%
--(8.732,4.116)--(8.723,4.114)--(8.714,4.111)--(8.706,4.108)--(8.697,4.105)%
--(8.689,4.101)--(8.681,4.096)--(8.673,4.092)--(8.666,4.086)--(8.659,4.081)%
--(8.652,4.075)--(8.645,4.069)--(8.639,4.062)--(8.633,4.055)--(8.628,4.048)%
--(8.622,4.041)--(8.618,4.033)--(8.613,4.025)--(8.609,4.017)--(8.606,4.008)%
--(8.603,4.000)--(8.600,3.991)--(8.598,3.982)--(8.597,3.974)--(8.595,3.965)%
--(8.595,3.956)--(8.595,3.947)--(8.595,3.937)--(8.595,3.928)--(8.597,3.919)%
--(8.598,3.911)--(8.600,3.902)--(8.603,3.893)--(8.606,3.885)--(8.609,3.876)%
--(8.613,3.868)--(8.618,3.860)--(8.622,3.852)--(8.628,3.845)--(8.633,3.838)%
--(8.639,3.831)--(8.645,3.824)--(8.652,3.818)--(8.659,3.812)--(8.666,3.807)%
--(8.673,3.801)--(8.681,3.797)--(8.689,3.792)--(8.697,3.788)--(8.706,3.785)%
--(8.714,3.782)--(8.723,3.779)--(8.732,3.777)--(8.740,3.776)--(8.749,3.774)%
--(8.758,3.774)--(8.768,3.774)--(8.777,3.774)--(8.786,3.774)--(8.795,3.776)%
--(8.803,3.777)--(8.812,3.779)--(8.821,3.782)--(8.829,3.785)--(8.838,3.788)%
--(8.846,3.792)--(8.854,3.797)--(8.862,3.801)--(8.869,3.807)--(8.876,3.812)%
--(8.883,3.818)--(8.890,3.824)--(8.896,3.831)--(8.902,3.838)--(8.907,3.845)%
--(8.913,3.852)--(8.917,3.860)--(8.922,3.868)--(8.926,3.876)--(8.929,3.885)%
--(8.932,3.893)--(8.935,3.902)--(8.937,3.911)--(8.938,3.919)--(8.940,3.928)--(8.940,3.937)--cycle;
%
\gpfill{rgb color={0.000,0.000,0.000},opacity=0.15} (9.104,3.971)--(9.103,3.979)--(9.103,3.988)--(9.101,3.997)%
--(9.100,4.006)--(9.098,4.014)--(9.095,4.023)--(9.092,4.031)--(9.089,4.039)%
--(9.085,4.047)--(9.081,4.055)--(9.076,4.063)--(9.071,4.070)--(9.066,4.077)%
--(9.060,4.084)--(9.054,4.090)--(9.048,4.096)--(9.041,4.102)--(9.034,4.107)%
--(9.027,4.112)--(9.019,4.117)--(9.011,4.121)--(9.003,4.125)--(8.995,4.128)%
--(8.987,4.131)--(8.978,4.134)--(8.970,4.136)--(8.961,4.137)--(8.952,4.139)%
--(8.943,4.139)--(8.935,4.140)--(8.926,4.139)--(8.917,4.139)--(8.908,4.137)%
--(8.899,4.136)--(8.891,4.134)--(8.882,4.131)--(8.874,4.128)--(8.866,4.125)%
--(8.858,4.121)--(8.850,4.117)--(8.842,4.112)--(8.835,4.107)--(8.828,4.102)%
--(8.821,4.096)--(8.815,4.090)--(8.809,4.084)--(8.803,4.077)--(8.798,4.070)%
--(8.793,4.063)--(8.788,4.055)--(8.784,4.047)--(8.780,4.039)--(8.777,4.031)%
--(8.774,4.023)--(8.771,4.014)--(8.769,4.006)--(8.768,3.997)--(8.766,3.988)%
--(8.766,3.979)--(8.766,3.971)--(8.766,3.962)--(8.766,3.953)--(8.768,3.944)%
--(8.769,3.935)--(8.771,3.927)--(8.774,3.918)--(8.777,3.910)--(8.780,3.902)%
--(8.784,3.894)--(8.788,3.886)--(8.793,3.878)--(8.798,3.871)--(8.803,3.864)%
--(8.809,3.857)--(8.815,3.851)--(8.821,3.845)--(8.828,3.839)--(8.835,3.834)%
--(8.842,3.829)--(8.850,3.824)--(8.858,3.820)--(8.866,3.816)--(8.874,3.813)%
--(8.882,3.810)--(8.891,3.807)--(8.899,3.805)--(8.908,3.804)--(8.917,3.802)%
--(8.926,3.802)--(8.935,3.802)--(8.943,3.802)--(8.952,3.802)--(8.961,3.804)%
--(8.970,3.805)--(8.978,3.807)--(8.987,3.810)--(8.995,3.813)--(9.003,3.816)%
--(9.011,3.820)--(9.019,3.824)--(9.027,3.829)--(9.034,3.834)--(9.041,3.839)%
--(9.048,3.845)--(9.054,3.851)--(9.060,3.857)--(9.066,3.864)--(9.071,3.871)%
--(9.076,3.878)--(9.081,3.886)--(9.085,3.894)--(9.089,3.902)--(9.092,3.910)%
--(9.095,3.918)--(9.098,3.927)--(9.100,3.935)--(9.101,3.944)--(9.103,3.953)--(9.103,3.962)--cycle;
%
\gpfill{rgb color={0.000,0.000,0.000},opacity=0.15} (9.261,3.996)--(9.260,4.004)--(9.260,4.013)--(9.258,4.021)%
--(9.257,4.030)--(9.255,4.038)--(9.252,4.046)--(9.250,4.054)--(9.246,4.062)%
--(9.243,4.070)--(9.239,4.078)--(9.234,4.085)--(9.229,4.092)--(9.224,4.099)%
--(9.218,4.105)--(9.212,4.111)--(9.206,4.117)--(9.200,4.123)--(9.193,4.128)%
--(9.186,4.133)--(9.179,4.138)--(9.171,4.142)--(9.163,4.145)--(9.155,4.149)%
--(9.147,4.151)--(9.139,4.154)--(9.131,4.156)--(9.122,4.157)--(9.114,4.159)%
--(9.105,4.159)--(9.097,4.160)--(9.088,4.159)--(9.079,4.159)--(9.071,4.157)%
--(9.062,4.156)--(9.054,4.154)--(9.046,4.151)--(9.038,4.149)--(9.030,4.145)%
--(9.022,4.142)--(9.015,4.138)--(9.007,4.133)--(9.000,4.128)--(8.993,4.123)%
--(8.987,4.117)--(8.981,4.111)--(8.975,4.105)--(8.969,4.099)--(8.964,4.092)%
--(8.959,4.085)--(8.954,4.078)--(8.950,4.070)--(8.947,4.062)--(8.943,4.054)%
--(8.941,4.046)--(8.938,4.038)--(8.936,4.030)--(8.935,4.021)--(8.933,4.013)%
--(8.933,4.004)--(8.933,3.996)--(8.933,3.987)--(8.933,3.978)--(8.935,3.970)%
--(8.936,3.961)--(8.938,3.953)--(8.941,3.945)--(8.943,3.937)--(8.947,3.929)%
--(8.950,3.921)--(8.954,3.914)--(8.959,3.906)--(8.964,3.899)--(8.969,3.892)%
--(8.975,3.886)--(8.981,3.880)--(8.987,3.874)--(8.993,3.868)--(9.000,3.863)%
--(9.007,3.858)--(9.015,3.853)--(9.022,3.849)--(9.030,3.846)--(9.038,3.842)%
--(9.046,3.840)--(9.054,3.837)--(9.062,3.835)--(9.071,3.834)--(9.079,3.832)%
--(9.088,3.832)--(9.097,3.832)--(9.105,3.832)--(9.114,3.832)--(9.122,3.834)%
--(9.131,3.835)--(9.139,3.837)--(9.147,3.840)--(9.155,3.842)--(9.163,3.846)%
--(9.171,3.849)--(9.179,3.853)--(9.186,3.858)--(9.193,3.863)--(9.200,3.868)%
--(9.206,3.874)--(9.212,3.880)--(9.218,3.886)--(9.224,3.892)--(9.229,3.899)%
--(9.234,3.906)--(9.239,3.914)--(9.243,3.921)--(9.246,3.929)--(9.250,3.937)%
--(9.252,3.945)--(9.255,3.953)--(9.257,3.961)--(9.258,3.970)--(9.260,3.978)--(9.260,3.987)--cycle;
%
\gpfill{rgb color={0.000,0.000,0.000},opacity=0.15} (9.413,4.021)--(9.412,4.029)--(9.412,4.037)--(9.411,4.045)%
--(9.409,4.054)--(9.407,4.062)--(9.405,4.070)--(9.402,4.077)--(9.399,4.085)%
--(9.395,4.093)--(9.391,4.100)--(9.387,4.107)--(9.382,4.114)--(9.377,4.121)%
--(9.372,4.127)--(9.366,4.133)--(9.360,4.139)--(9.354,4.144)--(9.347,4.149)%
--(9.340,4.154)--(9.333,4.158)--(9.326,4.162)--(9.318,4.166)--(9.310,4.169)%
--(9.303,4.172)--(9.295,4.174)--(9.287,4.176)--(9.278,4.178)--(9.270,4.179)%
--(9.262,4.179)--(9.254,4.180)--(9.245,4.179)--(9.237,4.179)--(9.229,4.178)%
--(9.220,4.176)--(9.212,4.174)--(9.204,4.172)--(9.197,4.169)--(9.189,4.166)%
--(9.181,4.162)--(9.174,4.158)--(9.167,4.154)--(9.160,4.149)--(9.153,4.144)%
--(9.147,4.139)--(9.141,4.133)--(9.135,4.127)--(9.130,4.121)--(9.125,4.114)%
--(9.120,4.107)--(9.116,4.100)--(9.112,4.093)--(9.108,4.085)--(9.105,4.077)%
--(9.102,4.070)--(9.100,4.062)--(9.098,4.054)--(9.096,4.045)--(9.095,4.037)%
--(9.095,4.029)--(9.095,4.021)--(9.095,4.012)--(9.095,4.004)--(9.096,3.996)%
--(9.098,3.987)--(9.100,3.979)--(9.102,3.971)--(9.105,3.964)--(9.108,3.956)%
--(9.112,3.948)--(9.116,3.941)--(9.120,3.934)--(9.125,3.927)--(9.130,3.920)%
--(9.135,3.914)--(9.141,3.908)--(9.147,3.902)--(9.153,3.897)--(9.160,3.892)%
--(9.167,3.887)--(9.174,3.883)--(9.181,3.879)--(9.189,3.875)--(9.197,3.872)%
--(9.204,3.869)--(9.212,3.867)--(9.220,3.865)--(9.229,3.863)--(9.237,3.862)%
--(9.245,3.862)--(9.254,3.862)--(9.262,3.862)--(9.270,3.862)--(9.278,3.863)%
--(9.287,3.865)--(9.295,3.867)--(9.303,3.869)--(9.310,3.872)--(9.318,3.875)%
--(9.326,3.879)--(9.333,3.883)--(9.340,3.887)--(9.347,3.892)--(9.354,3.897)%
--(9.360,3.902)--(9.366,3.908)--(9.372,3.914)--(9.377,3.920)--(9.382,3.927)%
--(9.387,3.934)--(9.391,3.941)--(9.395,3.948)--(9.399,3.956)--(9.402,3.964)%
--(9.405,3.971)--(9.407,3.979)--(9.409,3.987)--(9.411,3.996)--(9.412,4.004)--(9.412,4.012)--cycle;
%
\gpfill{rgb color={0.000,0.000,0.000},opacity=0.15} (9.560,4.046)--(9.559,4.054)--(9.559,4.062)--(9.558,4.070)%
--(9.556,4.078)--(9.554,4.085)--(9.552,4.093)--(9.549,4.101)--(9.546,4.108)%
--(9.543,4.115)--(9.539,4.123)--(9.535,4.129)--(9.530,4.136)--(9.525,4.142)%
--(9.520,4.149)--(9.514,4.154)--(9.509,4.160)--(9.502,4.165)--(9.496,4.170)%
--(9.489,4.175)--(9.483,4.179)--(9.475,4.183)--(9.468,4.186)--(9.461,4.189)%
--(9.453,4.192)--(9.445,4.194)--(9.438,4.196)--(9.430,4.198)--(9.422,4.199)%
--(9.414,4.199)--(9.406,4.200)--(9.397,4.199)--(9.389,4.199)--(9.381,4.198)%
--(9.373,4.196)--(9.366,4.194)--(9.358,4.192)--(9.350,4.189)--(9.343,4.186)%
--(9.336,4.183)--(9.329,4.179)--(9.322,4.175)--(9.315,4.170)--(9.309,4.165)%
--(9.302,4.160)--(9.297,4.154)--(9.291,4.149)--(9.286,4.142)--(9.281,4.136)%
--(9.276,4.129)--(9.272,4.123)--(9.268,4.115)--(9.265,4.108)--(9.262,4.101)%
--(9.259,4.093)--(9.257,4.085)--(9.255,4.078)--(9.253,4.070)--(9.252,4.062)%
--(9.252,4.054)--(9.252,4.046)--(9.252,4.037)--(9.252,4.029)--(9.253,4.021)%
--(9.255,4.013)--(9.257,4.006)--(9.259,3.998)--(9.262,3.990)--(9.265,3.983)%
--(9.268,3.976)--(9.272,3.969)--(9.276,3.962)--(9.281,3.955)--(9.286,3.949)%
--(9.291,3.942)--(9.297,3.937)--(9.302,3.931)--(9.309,3.926)--(9.315,3.921)%
--(9.322,3.916)--(9.329,3.912)--(9.336,3.908)--(9.343,3.905)--(9.350,3.902)%
--(9.358,3.899)--(9.366,3.897)--(9.373,3.895)--(9.381,3.893)--(9.389,3.892)%
--(9.397,3.892)--(9.406,3.892)--(9.414,3.892)--(9.422,3.892)--(9.430,3.893)%
--(9.438,3.895)--(9.445,3.897)--(9.453,3.899)--(9.461,3.902)--(9.468,3.905)%
--(9.475,3.908)--(9.483,3.912)--(9.489,3.916)--(9.496,3.921)--(9.502,3.926)%
--(9.509,3.931)--(9.514,3.937)--(9.520,3.942)--(9.525,3.949)--(9.530,3.955)%
--(9.535,3.962)--(9.539,3.969)--(9.543,3.976)--(9.546,3.983)--(9.549,3.990)%
--(9.552,3.998)--(9.554,4.006)--(9.556,4.013)--(9.558,4.021)--(9.559,4.029)--(9.559,4.037)--cycle;
%
\gpfill{rgb color={0.000,0.000,0.000},opacity=0.15} (9.700,4.072)--(9.699,4.079)--(9.699,4.087)--(9.698,4.095)%
--(9.696,4.102)--(9.694,4.110)--(9.692,4.117)--(9.690,4.125)--(9.687,4.132)%
--(9.683,4.139)--(9.680,4.146)--(9.676,4.152)--(9.671,4.158)--(9.667,4.165)%
--(9.661,4.171)--(9.656,4.176)--(9.651,4.181)--(9.645,4.187)--(9.638,4.191)%
--(9.632,4.196)--(9.626,4.200)--(9.619,4.203)--(9.612,4.207)--(9.605,4.210)%
--(9.597,4.212)--(9.590,4.214)--(9.582,4.216)--(9.575,4.218)--(9.567,4.219)%
--(9.559,4.219)--(9.552,4.220)--(9.544,4.219)--(9.536,4.219)--(9.528,4.218)%
--(9.521,4.216)--(9.513,4.214)--(9.506,4.212)--(9.498,4.210)--(9.491,4.207)%
--(9.484,4.203)--(9.478,4.200)--(9.471,4.196)--(9.465,4.191)--(9.458,4.187)%
--(9.452,4.181)--(9.447,4.176)--(9.442,4.171)--(9.436,4.165)--(9.432,4.158)%
--(9.427,4.152)--(9.423,4.146)--(9.420,4.139)--(9.416,4.132)--(9.413,4.125)%
--(9.411,4.117)--(9.409,4.110)--(9.407,4.102)--(9.405,4.095)--(9.404,4.087)%
--(9.404,4.079)--(9.404,4.072)--(9.404,4.064)--(9.404,4.056)--(9.405,4.048)%
--(9.407,4.041)--(9.409,4.033)--(9.411,4.026)--(9.413,4.018)--(9.416,4.011)%
--(9.420,4.004)--(9.423,3.998)--(9.427,3.991)--(9.432,3.985)--(9.436,3.978)%
--(9.442,3.972)--(9.447,3.967)--(9.452,3.962)--(9.458,3.956)--(9.465,3.952)%
--(9.471,3.947)--(9.478,3.943)--(9.484,3.940)--(9.491,3.936)--(9.498,3.933)%
--(9.506,3.931)--(9.513,3.929)--(9.521,3.927)--(9.528,3.925)--(9.536,3.924)%
--(9.544,3.924)--(9.552,3.924)--(9.559,3.924)--(9.567,3.924)--(9.575,3.925)%
--(9.582,3.927)--(9.590,3.929)--(9.597,3.931)--(9.605,3.933)--(9.612,3.936)%
--(9.619,3.940)--(9.626,3.943)--(9.632,3.947)--(9.638,3.952)--(9.645,3.956)%
--(9.651,3.962)--(9.656,3.967)--(9.661,3.972)--(9.667,3.978)--(9.671,3.985)%
--(9.676,3.991)--(9.680,3.998)--(9.683,4.004)--(9.687,4.011)--(9.690,4.018)%
--(9.692,4.026)--(9.694,4.033)--(9.696,4.041)--(9.698,4.048)--(9.699,4.056)--(9.699,4.064)--cycle;
%
\gpfill{rgb color={0.000,0.000,0.000},opacity=0.15} (9.836,4.097)--(9.835,4.104)--(9.835,4.111)--(9.834,4.119)%
--(9.832,4.126)--(9.831,4.134)--(9.829,4.141)--(9.826,4.148)--(9.823,4.155)%
--(9.820,4.161)--(9.816,4.168)--(9.812,4.174)--(9.808,4.181)--(9.804,4.186)%
--(9.799,4.192)--(9.794,4.198)--(9.788,4.203)--(9.782,4.208)--(9.777,4.212)%
--(9.770,4.216)--(9.764,4.220)--(9.757,4.224)--(9.751,4.227)--(9.744,4.230)%
--(9.737,4.233)--(9.730,4.235)--(9.722,4.236)--(9.715,4.238)--(9.707,4.239)%
--(9.700,4.239)--(9.693,4.240)--(9.685,4.239)--(9.678,4.239)--(9.670,4.238)%
--(9.663,4.236)--(9.655,4.235)--(9.648,4.233)--(9.641,4.230)--(9.634,4.227)%
--(9.628,4.224)--(9.621,4.220)--(9.615,4.216)--(9.608,4.212)--(9.603,4.208)%
--(9.597,4.203)--(9.591,4.198)--(9.586,4.192)--(9.581,4.186)--(9.577,4.181)%
--(9.573,4.174)--(9.569,4.168)--(9.565,4.161)--(9.562,4.155)--(9.559,4.148)%
--(9.556,4.141)--(9.554,4.134)--(9.553,4.126)--(9.551,4.119)--(9.550,4.111)%
--(9.550,4.104)--(9.550,4.097)--(9.550,4.089)--(9.550,4.082)--(9.551,4.074)%
--(9.553,4.067)--(9.554,4.059)--(9.556,4.052)--(9.559,4.045)--(9.562,4.038)%
--(9.565,4.032)--(9.569,4.025)--(9.573,4.019)--(9.577,4.012)--(9.581,4.007)%
--(9.586,4.001)--(9.591,3.995)--(9.597,3.990)--(9.603,3.985)--(9.608,3.981)%
--(9.615,3.977)--(9.621,3.973)--(9.628,3.969)--(9.634,3.966)--(9.641,3.963)%
--(9.648,3.960)--(9.655,3.958)--(9.663,3.957)--(9.670,3.955)--(9.678,3.954)%
--(9.685,3.954)--(9.693,3.954)--(9.700,3.954)--(9.707,3.954)--(9.715,3.955)%
--(9.722,3.957)--(9.730,3.958)--(9.737,3.960)--(9.744,3.963)--(9.751,3.966)%
--(9.757,3.969)--(9.764,3.973)--(9.770,3.977)--(9.777,3.981)--(9.782,3.985)%
--(9.788,3.990)--(9.794,3.995)--(9.799,4.001)--(9.804,4.007)--(9.808,4.012)%
--(9.812,4.019)--(9.816,4.025)--(9.820,4.032)--(9.823,4.038)--(9.826,4.045)%
--(9.829,4.052)--(9.831,4.059)--(9.832,4.067)--(9.834,4.074)--(9.835,4.082)--(9.835,4.089)--cycle;
%
\gpfill{rgb color={0.000,0.000,0.000},opacity=0.15} (9.964,4.122)--(9.963,4.129)--(9.963,4.136)--(9.962,4.143)%
--(9.961,4.150)--(9.959,4.157)--(9.957,4.164)--(9.954,4.171)--(9.952,4.177)%
--(9.949,4.184)--(9.945,4.190)--(9.941,4.196)--(9.937,4.202)--(9.933,4.208)%
--(9.928,4.213)--(9.923,4.218)--(9.918,4.223)--(9.913,4.228)--(9.907,4.232)%
--(9.901,4.236)--(9.895,4.240)--(9.889,4.244)--(9.882,4.247)--(9.876,4.249)%
--(9.869,4.252)--(9.862,4.254)--(9.855,4.256)--(9.848,4.257)--(9.841,4.258)%
--(9.834,4.258)--(9.827,4.259)--(9.819,4.258)--(9.812,4.258)--(9.805,4.257)%
--(9.798,4.256)--(9.791,4.254)--(9.784,4.252)--(9.777,4.249)--(9.771,4.247)%
--(9.764,4.244)--(9.758,4.240)--(9.752,4.236)--(9.746,4.232)--(9.740,4.228)%
--(9.735,4.223)--(9.730,4.218)--(9.725,4.213)--(9.720,4.208)--(9.716,4.202)%
--(9.712,4.196)--(9.708,4.190)--(9.704,4.184)--(9.701,4.177)--(9.699,4.171)%
--(9.696,4.164)--(9.694,4.157)--(9.692,4.150)--(9.691,4.143)--(9.690,4.136)%
--(9.690,4.129)--(9.690,4.122)--(9.690,4.114)--(9.690,4.107)--(9.691,4.100)%
--(9.692,4.093)--(9.694,4.086)--(9.696,4.079)--(9.699,4.072)--(9.701,4.066)%
--(9.704,4.059)--(9.708,4.053)--(9.712,4.047)--(9.716,4.041)--(9.720,4.035)%
--(9.725,4.030)--(9.730,4.025)--(9.735,4.020)--(9.740,4.015)--(9.746,4.011)%
--(9.752,4.007)--(9.758,4.003)--(9.764,3.999)--(9.771,3.996)--(9.777,3.994)%
--(9.784,3.991)--(9.791,3.989)--(9.798,3.987)--(9.805,3.986)--(9.812,3.985)%
--(9.819,3.985)--(9.827,3.985)--(9.834,3.985)--(9.841,3.985)--(9.848,3.986)%
--(9.855,3.987)--(9.862,3.989)--(9.869,3.991)--(9.876,3.994)--(9.882,3.996)%
--(9.889,3.999)--(9.895,4.003)--(9.901,4.007)--(9.907,4.011)--(9.913,4.015)%
--(9.918,4.020)--(9.923,4.025)--(9.928,4.030)--(9.933,4.035)--(9.937,4.041)%
--(9.941,4.047)--(9.945,4.053)--(9.949,4.059)--(9.952,4.066)--(9.954,4.072)%
--(9.957,4.079)--(9.959,4.086)--(9.961,4.093)--(9.962,4.100)--(9.963,4.107)--(9.963,4.114)--cycle;
%
\gpfill{rgb color={0.000,0.000,0.000},opacity=0.15} (10.084,4.147)--(10.083,4.153)--(10.083,4.160)--(10.082,4.167)%
--(10.081,4.174)--(10.079,4.180)--(10.077,4.187)--(10.075,4.193)--(10.072,4.199)%
--(10.069,4.206)--(10.066,4.212)--(10.063,4.217)--(10.059,4.223)--(10.055,4.228)%
--(10.050,4.233)--(10.045,4.238)--(10.040,4.243)--(10.035,4.248)--(10.030,4.252)%
--(10.024,4.256)--(10.019,4.259)--(10.013,4.262)--(10.006,4.265)--(10.000,4.268)%
--(9.994,4.270)--(9.987,4.272)--(9.981,4.274)--(9.974,4.275)--(9.967,4.276)%
--(9.960,4.276)--(9.954,4.277)--(9.947,4.276)--(9.940,4.276)--(9.933,4.275)%
--(9.926,4.274)--(9.920,4.272)--(9.913,4.270)--(9.907,4.268)--(9.901,4.265)%
--(9.894,4.262)--(9.889,4.259)--(9.883,4.256)--(9.877,4.252)--(9.872,4.248)%
--(9.867,4.243)--(9.862,4.238)--(9.857,4.233)--(9.852,4.228)--(9.848,4.223)%
--(9.844,4.217)--(9.841,4.212)--(9.838,4.206)--(9.835,4.199)--(9.832,4.193)%
--(9.830,4.187)--(9.828,4.180)--(9.826,4.174)--(9.825,4.167)--(9.824,4.160)%
--(9.824,4.153)--(9.824,4.147)--(9.824,4.140)--(9.824,4.133)--(9.825,4.126)%
--(9.826,4.119)--(9.828,4.113)--(9.830,4.106)--(9.832,4.100)--(9.835,4.094)%
--(9.838,4.087)--(9.841,4.082)--(9.844,4.076)--(9.848,4.070)--(9.852,4.065)%
--(9.857,4.060)--(9.862,4.055)--(9.867,4.050)--(9.872,4.045)--(9.877,4.041)%
--(9.883,4.037)--(9.889,4.034)--(9.894,4.031)--(9.901,4.028)--(9.907,4.025)%
--(9.913,4.023)--(9.920,4.021)--(9.926,4.019)--(9.933,4.018)--(9.940,4.017)%
--(9.947,4.017)--(9.954,4.017)--(9.960,4.017)--(9.967,4.017)--(9.974,4.018)%
--(9.981,4.019)--(9.987,4.021)--(9.994,4.023)--(10.000,4.025)--(10.006,4.028)%
--(10.013,4.031)--(10.019,4.034)--(10.024,4.037)--(10.030,4.041)--(10.035,4.045)%
--(10.040,4.050)--(10.045,4.055)--(10.050,4.060)--(10.055,4.065)--(10.059,4.070)%
--(10.063,4.076)--(10.066,4.082)--(10.069,4.087)--(10.072,4.094)--(10.075,4.100)%
--(10.077,4.106)--(10.079,4.113)--(10.081,4.119)--(10.082,4.126)--(10.083,4.133)--(10.083,4.140)--cycle;
%
\gpfill{rgb color={0.000,0.000,0.000},opacity=0.15} (10.200,4.171)--(10.199,4.177)--(10.199,4.183)--(10.198,4.190)%
--(10.197,4.196)--(10.195,4.203)--(10.193,4.209)--(10.191,4.215)--(10.189,4.221)%
--(10.186,4.227)--(10.183,4.233)--(10.179,4.238)--(10.176,4.243)--(10.172,4.249)%
--(10.168,4.253)--(10.163,4.258)--(10.158,4.263)--(10.154,4.267)--(10.148,4.271)%
--(10.143,4.274)--(10.138,4.278)--(10.132,4.281)--(10.126,4.284)--(10.120,4.286)%
--(10.114,4.288)--(10.108,4.290)--(10.101,4.292)--(10.095,4.293)--(10.088,4.294)%
--(10.082,4.294)--(10.076,4.295)--(10.069,4.294)--(10.063,4.294)--(10.056,4.293)%
--(10.050,4.292)--(10.043,4.290)--(10.037,4.288)--(10.031,4.286)--(10.025,4.284)%
--(10.019,4.281)--(10.014,4.278)--(10.008,4.274)--(10.003,4.271)--(9.997,4.267)%
--(9.993,4.263)--(9.988,4.258)--(9.983,4.253)--(9.979,4.249)--(9.975,4.243)%
--(9.972,4.238)--(9.968,4.233)--(9.965,4.227)--(9.962,4.221)--(9.960,4.215)%
--(9.958,4.209)--(9.956,4.203)--(9.954,4.196)--(9.953,4.190)--(9.952,4.183)%
--(9.952,4.177)--(9.952,4.171)--(9.952,4.164)--(9.952,4.158)--(9.953,4.151)%
--(9.954,4.145)--(9.956,4.138)--(9.958,4.132)--(9.960,4.126)--(9.962,4.120)%
--(9.965,4.114)--(9.968,4.109)--(9.972,4.103)--(9.975,4.098)--(9.979,4.092)%
--(9.983,4.088)--(9.988,4.083)--(9.993,4.078)--(9.997,4.074)--(10.003,4.070)%
--(10.008,4.067)--(10.014,4.063)--(10.019,4.060)--(10.025,4.057)--(10.031,4.055)%
--(10.037,4.053)--(10.043,4.051)--(10.050,4.049)--(10.056,4.048)--(10.063,4.047)%
--(10.069,4.047)--(10.076,4.047)--(10.082,4.047)--(10.088,4.047)--(10.095,4.048)%
--(10.101,4.049)--(10.108,4.051)--(10.114,4.053)--(10.120,4.055)--(10.126,4.057)%
--(10.132,4.060)--(10.138,4.063)--(10.143,4.067)--(10.148,4.070)--(10.154,4.074)%
--(10.158,4.078)--(10.163,4.083)--(10.168,4.088)--(10.172,4.092)--(10.176,4.098)%
--(10.179,4.103)--(10.183,4.109)--(10.186,4.114)--(10.189,4.120)--(10.191,4.126)%
--(10.193,4.132)--(10.195,4.138)--(10.197,4.145)--(10.198,4.151)--(10.199,4.158)--(10.199,4.164)--cycle;
%
\gpfill{rgb color={0.000,0.000,0.000},opacity=0.15} (10.307,4.194)--(10.306,4.200)--(10.306,4.206)--(10.305,4.212)%
--(10.304,4.218)--(10.303,4.224)--(10.301,4.230)--(10.299,4.235)--(10.296,4.241)%
--(10.294,4.247)--(10.291,4.252)--(10.288,4.257)--(10.284,4.262)--(10.280,4.267)%
--(10.276,4.272)--(10.272,4.276)--(10.268,4.280)--(10.263,4.284)--(10.258,4.288)%
--(10.253,4.292)--(10.248,4.295)--(10.243,4.298)--(10.237,4.300)--(10.231,4.303)%
--(10.226,4.305)--(10.220,4.307)--(10.214,4.308)--(10.208,4.309)--(10.202,4.310)%
--(10.196,4.310)--(10.190,4.311)--(10.183,4.310)--(10.177,4.310)--(10.171,4.309)%
--(10.165,4.308)--(10.159,4.307)--(10.153,4.305)--(10.148,4.303)--(10.142,4.300)%
--(10.136,4.298)--(10.131,4.295)--(10.126,4.292)--(10.121,4.288)--(10.116,4.284)%
--(10.111,4.280)--(10.107,4.276)--(10.103,4.272)--(10.099,4.267)--(10.095,4.262)%
--(10.091,4.257)--(10.088,4.252)--(10.085,4.247)--(10.083,4.241)--(10.080,4.235)%
--(10.078,4.230)--(10.076,4.224)--(10.075,4.218)--(10.074,4.212)--(10.073,4.206)%
--(10.073,4.200)--(10.073,4.194)--(10.073,4.187)--(10.073,4.181)--(10.074,4.175)%
--(10.075,4.169)--(10.076,4.163)--(10.078,4.157)--(10.080,4.152)--(10.083,4.146)%
--(10.085,4.140)--(10.088,4.135)--(10.091,4.130)--(10.095,4.125)--(10.099,4.120)%
--(10.103,4.115)--(10.107,4.111)--(10.111,4.107)--(10.116,4.103)--(10.121,4.099)%
--(10.126,4.095)--(10.131,4.092)--(10.136,4.089)--(10.142,4.087)--(10.148,4.084)%
--(10.153,4.082)--(10.159,4.080)--(10.165,4.079)--(10.171,4.078)--(10.177,4.077)%
--(10.183,4.077)--(10.190,4.077)--(10.196,4.077)--(10.202,4.077)--(10.208,4.078)%
--(10.214,4.079)--(10.220,4.080)--(10.226,4.082)--(10.231,4.084)--(10.237,4.087)%
--(10.243,4.089)--(10.248,4.092)--(10.253,4.095)--(10.258,4.099)--(10.263,4.103)%
--(10.268,4.107)--(10.272,4.111)--(10.276,4.115)--(10.280,4.120)--(10.284,4.125)%
--(10.288,4.130)--(10.291,4.135)--(10.294,4.140)--(10.296,4.146)--(10.299,4.152)%
--(10.301,4.157)--(10.303,4.163)--(10.304,4.169)--(10.305,4.175)--(10.306,4.181)--(10.306,4.187)--cycle;
%
\gpfill{rgb color={0.000,0.000,0.000},opacity=0.15} (10.406,4.217)--(10.405,4.222)--(10.405,4.228)--(10.404,4.234)%
--(10.403,4.239)--(10.402,4.245)--(10.400,4.250)--(10.398,4.256)--(10.396,4.261)%
--(10.394,4.266)--(10.391,4.271)--(10.388,4.276)--(10.385,4.281)--(10.381,4.285)%
--(10.378,4.289)--(10.374,4.294)--(10.369,4.298)--(10.365,4.301)--(10.361,4.305)%
--(10.356,4.308)--(10.351,4.311)--(10.346,4.314)--(10.341,4.316)--(10.336,4.318)%
--(10.330,4.320)--(10.325,4.322)--(10.319,4.323)--(10.314,4.324)--(10.308,4.325)%
--(10.302,4.325)--(10.297,4.326)--(10.291,4.325)--(10.285,4.325)--(10.279,4.324)%
--(10.274,4.323)--(10.268,4.322)--(10.263,4.320)--(10.257,4.318)--(10.252,4.316)%
--(10.247,4.314)--(10.242,4.311)--(10.237,4.308)--(10.232,4.305)--(10.228,4.301)%
--(10.224,4.298)--(10.219,4.294)--(10.215,4.289)--(10.212,4.285)--(10.208,4.281)%
--(10.205,4.276)--(10.202,4.271)--(10.199,4.266)--(10.197,4.261)--(10.195,4.256)%
--(10.193,4.250)--(10.191,4.245)--(10.190,4.239)--(10.189,4.234)--(10.188,4.228)%
--(10.188,4.222)--(10.188,4.217)--(10.188,4.211)--(10.188,4.205)--(10.189,4.199)%
--(10.190,4.194)--(10.191,4.188)--(10.193,4.183)--(10.195,4.177)--(10.197,4.172)%
--(10.199,4.167)--(10.202,4.162)--(10.205,4.157)--(10.208,4.152)--(10.212,4.148)%
--(10.215,4.144)--(10.219,4.139)--(10.224,4.135)--(10.228,4.132)--(10.232,4.128)%
--(10.237,4.125)--(10.242,4.122)--(10.247,4.119)--(10.252,4.117)--(10.257,4.115)%
--(10.263,4.113)--(10.268,4.111)--(10.274,4.110)--(10.279,4.109)--(10.285,4.108)%
--(10.291,4.108)--(10.297,4.108)--(10.302,4.108)--(10.308,4.108)--(10.314,4.109)%
--(10.319,4.110)--(10.325,4.111)--(10.330,4.113)--(10.336,4.115)--(10.341,4.117)%
--(10.346,4.119)--(10.351,4.122)--(10.356,4.125)--(10.361,4.128)--(10.365,4.132)%
--(10.369,4.135)--(10.374,4.139)--(10.378,4.144)--(10.381,4.148)--(10.385,4.152)%
--(10.388,4.157)--(10.391,4.162)--(10.394,4.167)--(10.396,4.172)--(10.398,4.177)%
--(10.400,4.183)--(10.402,4.188)--(10.403,4.194)--(10.404,4.199)--(10.405,4.205)--(10.405,4.211)--cycle;
%
\gpfill{rgb color={0.000,0.000,0.000},opacity=0.15} (10.501,4.238)--(10.500,4.243)--(10.500,4.248)--(10.499,4.254)%
--(10.498,4.259)--(10.497,4.264)--(10.495,4.269)--(10.494,4.274)--(10.492,4.279)%
--(10.489,4.284)--(10.487,4.289)--(10.484,4.294)--(10.481,4.298)--(10.478,4.302)%
--(10.474,4.306)--(10.470,4.310)--(10.466,4.314)--(10.462,4.318)--(10.458,4.321)%
--(10.454,4.324)--(10.449,4.327)--(10.444,4.329)--(10.439,4.332)--(10.434,4.334)%
--(10.429,4.335)--(10.424,4.337)--(10.419,4.338)--(10.414,4.339)--(10.408,4.340)%
--(10.403,4.340)--(10.398,4.341)--(10.392,4.340)--(10.387,4.340)--(10.381,4.339)%
--(10.376,4.338)--(10.371,4.337)--(10.366,4.335)--(10.361,4.334)--(10.356,4.332)%
--(10.351,4.329)--(10.346,4.327)--(10.341,4.324)--(10.337,4.321)--(10.333,4.318)%
--(10.329,4.314)--(10.325,4.310)--(10.321,4.306)--(10.317,4.302)--(10.314,4.298)%
--(10.311,4.294)--(10.308,4.289)--(10.306,4.284)--(10.303,4.279)--(10.301,4.274)%
--(10.300,4.269)--(10.298,4.264)--(10.297,4.259)--(10.296,4.254)--(10.295,4.248)%
--(10.295,4.243)--(10.295,4.238)--(10.295,4.232)--(10.295,4.227)--(10.296,4.221)%
--(10.297,4.216)--(10.298,4.211)--(10.300,4.206)--(10.301,4.201)--(10.303,4.196)%
--(10.306,4.191)--(10.308,4.186)--(10.311,4.181)--(10.314,4.177)--(10.317,4.173)%
--(10.321,4.169)--(10.325,4.165)--(10.329,4.161)--(10.333,4.157)--(10.337,4.154)%
--(10.341,4.151)--(10.346,4.148)--(10.351,4.146)--(10.356,4.143)--(10.361,4.141)%
--(10.366,4.140)--(10.371,4.138)--(10.376,4.137)--(10.381,4.136)--(10.387,4.135)%
--(10.392,4.135)--(10.398,4.135)--(10.403,4.135)--(10.408,4.135)--(10.414,4.136)%
--(10.419,4.137)--(10.424,4.138)--(10.429,4.140)--(10.434,4.141)--(10.439,4.143)%
--(10.444,4.146)--(10.449,4.148)--(10.454,4.151)--(10.458,4.154)--(10.462,4.157)%
--(10.466,4.161)--(10.470,4.165)--(10.474,4.169)--(10.478,4.173)--(10.481,4.177)%
--(10.484,4.181)--(10.487,4.186)--(10.489,4.191)--(10.492,4.196)--(10.494,4.201)%
--(10.495,4.206)--(10.497,4.211)--(10.498,4.216)--(10.499,4.221)--(10.500,4.227)--(10.500,4.232)--cycle;
%
\gpfill{rgb color={0.000,0.000,0.000},opacity=0.15} (10.586,4.259)--(10.585,4.263)--(10.585,4.268)--(10.584,4.273)%
--(10.583,4.278)--(10.582,4.283)--(10.581,4.288)--(10.579,4.293)--(10.577,4.297)%
--(10.575,4.302)--(10.573,4.306)--(10.570,4.310)--(10.567,4.314)--(10.564,4.318)%
--(10.561,4.322)--(10.558,4.326)--(10.554,4.329)--(10.550,4.332)--(10.546,4.335)%
--(10.542,4.338)--(10.538,4.341)--(10.534,4.343)--(10.529,4.345)--(10.525,4.347)%
--(10.520,4.349)--(10.515,4.350)--(10.510,4.351)--(10.505,4.352)--(10.500,4.353)%
--(10.495,4.353)--(10.491,4.354)--(10.486,4.353)--(10.481,4.353)--(10.476,4.352)%
--(10.471,4.351)--(10.466,4.350)--(10.461,4.349)--(10.456,4.347)--(10.452,4.345)%
--(10.447,4.343)--(10.443,4.341)--(10.439,4.338)--(10.435,4.335)--(10.431,4.332)%
--(10.427,4.329)--(10.423,4.326)--(10.420,4.322)--(10.417,4.318)--(10.414,4.314)%
--(10.411,4.310)--(10.408,4.306)--(10.406,4.302)--(10.404,4.297)--(10.402,4.293)%
--(10.400,4.288)--(10.399,4.283)--(10.398,4.278)--(10.397,4.273)--(10.396,4.268)%
--(10.396,4.263)--(10.396,4.259)--(10.396,4.254)--(10.396,4.249)--(10.397,4.244)%
--(10.398,4.239)--(10.399,4.234)--(10.400,4.229)--(10.402,4.224)--(10.404,4.220)%
--(10.406,4.215)--(10.408,4.211)--(10.411,4.207)--(10.414,4.203)--(10.417,4.199)%
--(10.420,4.195)--(10.423,4.191)--(10.427,4.188)--(10.431,4.185)--(10.435,4.182)%
--(10.439,4.179)--(10.443,4.176)--(10.447,4.174)--(10.452,4.172)--(10.456,4.170)%
--(10.461,4.168)--(10.466,4.167)--(10.471,4.166)--(10.476,4.165)--(10.481,4.164)%
--(10.486,4.164)--(10.491,4.164)--(10.495,4.164)--(10.500,4.164)--(10.505,4.165)%
--(10.510,4.166)--(10.515,4.167)--(10.520,4.168)--(10.525,4.170)--(10.529,4.172)%
--(10.534,4.174)--(10.538,4.176)--(10.542,4.179)--(10.546,4.182)--(10.550,4.185)%
--(10.554,4.188)--(10.558,4.191)--(10.561,4.195)--(10.564,4.199)--(10.567,4.203)%
--(10.570,4.207)--(10.573,4.211)--(10.575,4.215)--(10.577,4.220)--(10.579,4.224)%
--(10.581,4.229)--(10.582,4.234)--(10.583,4.239)--(10.584,4.244)--(10.585,4.249)--(10.585,4.254)--cycle;
%
\gpfill{rgb color={0.000,0.000,0.000},opacity=0.15} (10.664,4.278)--(10.663,4.282)--(10.663,4.287)--(10.662,4.291)%
--(10.662,4.296)--(10.661,4.300)--(10.659,4.305)--(10.658,4.309)--(10.656,4.313)%
--(10.654,4.317)--(10.652,4.322)--(10.649,4.325)--(10.647,4.329)--(10.644,4.333)%
--(10.641,4.336)--(10.638,4.340)--(10.634,4.343)--(10.631,4.346)--(10.627,4.349)%
--(10.623,4.351)--(10.620,4.354)--(10.615,4.356)--(10.611,4.358)--(10.607,4.360)%
--(10.603,4.361)--(10.598,4.363)--(10.594,4.364)--(10.589,4.364)--(10.585,4.365)%
--(10.580,4.365)--(10.576,4.366)--(10.571,4.365)--(10.566,4.365)--(10.562,4.364)%
--(10.557,4.364)--(10.553,4.363)--(10.548,4.361)--(10.544,4.360)--(10.540,4.358)%
--(10.536,4.356)--(10.532,4.354)--(10.528,4.351)--(10.524,4.349)--(10.520,4.346)%
--(10.517,4.343)--(10.513,4.340)--(10.510,4.336)--(10.507,4.333)--(10.504,4.329)%
--(10.502,4.325)--(10.499,4.322)--(10.497,4.317)--(10.495,4.313)--(10.493,4.309)%
--(10.492,4.305)--(10.490,4.300)--(10.489,4.296)--(10.489,4.291)--(10.488,4.287)%
--(10.488,4.282)--(10.488,4.278)--(10.488,4.273)--(10.488,4.268)--(10.489,4.264)%
--(10.489,4.259)--(10.490,4.255)--(10.492,4.250)--(10.493,4.246)--(10.495,4.242)%
--(10.497,4.238)--(10.499,4.234)--(10.502,4.230)--(10.504,4.226)--(10.507,4.222)%
--(10.510,4.219)--(10.513,4.215)--(10.517,4.212)--(10.520,4.209)--(10.524,4.206)%
--(10.528,4.204)--(10.532,4.201)--(10.536,4.199)--(10.540,4.197)--(10.544,4.195)%
--(10.548,4.194)--(10.553,4.192)--(10.557,4.191)--(10.562,4.191)--(10.566,4.190)%
--(10.571,4.190)--(10.576,4.190)--(10.580,4.190)--(10.585,4.190)--(10.589,4.191)%
--(10.594,4.191)--(10.598,4.192)--(10.603,4.194)--(10.607,4.195)--(10.611,4.197)%
--(10.615,4.199)--(10.620,4.201)--(10.623,4.204)--(10.627,4.206)--(10.631,4.209)%
--(10.634,4.212)--(10.638,4.215)--(10.641,4.219)--(10.644,4.222)--(10.647,4.226)%
--(10.649,4.230)--(10.652,4.234)--(10.654,4.238)--(10.656,4.242)--(10.658,4.246)%
--(10.659,4.250)--(10.661,4.255)--(10.662,4.259)--(10.662,4.264)--(10.663,4.268)--(10.663,4.273)--cycle;
%
\gpfill{rgb color={0.000,0.000,0.000},opacity=0.15} (10.734,4.296)--(10.733,4.300)--(10.733,4.304)--(10.733,4.308)%
--(10.732,4.312)--(10.731,4.316)--(10.730,4.320)--(10.728,4.324)--(10.727,4.328)%
--(10.725,4.332)--(10.723,4.336)--(10.721,4.339)--(10.718,4.343)--(10.716,4.346)%
--(10.713,4.349)--(10.710,4.352)--(10.707,4.355)--(10.704,4.358)--(10.701,4.360)%
--(10.697,4.363)--(10.694,4.365)--(10.690,4.367)--(10.686,4.369)--(10.682,4.370)%
--(10.678,4.372)--(10.674,4.373)--(10.670,4.374)--(10.666,4.375)--(10.662,4.375)%
--(10.658,4.375)--(10.654,4.376)--(10.649,4.375)--(10.645,4.375)--(10.641,4.375)%
--(10.637,4.374)--(10.633,4.373)--(10.629,4.372)--(10.625,4.370)--(10.621,4.369)%
--(10.617,4.367)--(10.614,4.365)--(10.610,4.363)--(10.606,4.360)--(10.603,4.358)%
--(10.600,4.355)--(10.597,4.352)--(10.594,4.349)--(10.591,4.346)--(10.589,4.343)%
--(10.586,4.339)--(10.584,4.336)--(10.582,4.332)--(10.580,4.328)--(10.579,4.324)%
--(10.577,4.320)--(10.576,4.316)--(10.575,4.312)--(10.574,4.308)--(10.574,4.304)%
--(10.574,4.300)--(10.574,4.296)--(10.574,4.291)--(10.574,4.287)--(10.574,4.283)%
--(10.575,4.279)--(10.576,4.275)--(10.577,4.271)--(10.579,4.267)--(10.580,4.263)%
--(10.582,4.259)--(10.584,4.256)--(10.586,4.252)--(10.589,4.248)--(10.591,4.245)%
--(10.594,4.242)--(10.597,4.239)--(10.600,4.236)--(10.603,4.233)--(10.606,4.231)%
--(10.610,4.228)--(10.614,4.226)--(10.617,4.224)--(10.621,4.222)--(10.625,4.221)%
--(10.629,4.219)--(10.633,4.218)--(10.637,4.217)--(10.641,4.216)--(10.645,4.216)%
--(10.649,4.216)--(10.654,4.216)--(10.658,4.216)--(10.662,4.216)--(10.666,4.216)%
--(10.670,4.217)--(10.674,4.218)--(10.678,4.219)--(10.682,4.221)--(10.686,4.222)%
--(10.690,4.224)--(10.694,4.226)--(10.697,4.228)--(10.701,4.231)--(10.704,4.233)%
--(10.707,4.236)--(10.710,4.239)--(10.713,4.242)--(10.716,4.245)--(10.718,4.248)%
--(10.721,4.252)--(10.723,4.256)--(10.725,4.259)--(10.727,4.263)--(10.728,4.267)%
--(10.730,4.271)--(10.731,4.275)--(10.732,4.279)--(10.733,4.283)--(10.733,4.287)--(10.733,4.291)--cycle;
%
\gpfill{rgb color={0.000,0.000,0.000},opacity=0.15} (10.796,4.312)--(10.795,4.315)--(10.795,4.319)--(10.795,4.323)%
--(10.794,4.326)--(10.793,4.330)--(10.792,4.334)--(10.791,4.337)--(10.789,4.341)%
--(10.788,4.344)--(10.786,4.348)--(10.784,4.351)--(10.782,4.354)--(10.779,4.357)%
--(10.777,4.360)--(10.774,4.362)--(10.772,4.365)--(10.769,4.367)--(10.766,4.370)%
--(10.763,4.372)--(10.760,4.374)--(10.756,4.376)--(10.753,4.377)--(10.749,4.379)%
--(10.746,4.380)--(10.742,4.381)--(10.738,4.382)--(10.735,4.383)--(10.731,4.383)%
--(10.727,4.383)--(10.724,4.384)--(10.720,4.383)--(10.716,4.383)--(10.712,4.383)%
--(10.709,4.382)--(10.705,4.381)--(10.701,4.380)--(10.698,4.379)--(10.694,4.377)%
--(10.691,4.376)--(10.688,4.374)--(10.684,4.372)--(10.681,4.370)--(10.678,4.367)%
--(10.675,4.365)--(10.673,4.362)--(10.670,4.360)--(10.668,4.357)--(10.665,4.354)%
--(10.663,4.351)--(10.661,4.348)--(10.659,4.344)--(10.658,4.341)--(10.656,4.337)%
--(10.655,4.334)--(10.654,4.330)--(10.653,4.326)--(10.652,4.323)--(10.652,4.319)%
--(10.652,4.315)--(10.652,4.312)--(10.652,4.308)--(10.652,4.304)--(10.652,4.300)%
--(10.653,4.297)--(10.654,4.293)--(10.655,4.289)--(10.656,4.286)--(10.658,4.282)%
--(10.659,4.279)--(10.661,4.276)--(10.663,4.272)--(10.665,4.269)--(10.668,4.266)%
--(10.670,4.263)--(10.673,4.261)--(10.675,4.258)--(10.678,4.256)--(10.681,4.253)%
--(10.684,4.251)--(10.688,4.249)--(10.691,4.247)--(10.694,4.246)--(10.698,4.244)%
--(10.701,4.243)--(10.705,4.242)--(10.709,4.241)--(10.712,4.240)--(10.716,4.240)%
--(10.720,4.240)--(10.724,4.240)--(10.727,4.240)--(10.731,4.240)--(10.735,4.240)%
--(10.738,4.241)--(10.742,4.242)--(10.746,4.243)--(10.749,4.244)--(10.753,4.246)%
--(10.756,4.247)--(10.760,4.249)--(10.763,4.251)--(10.766,4.253)--(10.769,4.256)%
--(10.772,4.258)--(10.774,4.261)--(10.777,4.263)--(10.779,4.266)--(10.782,4.269)%
--(10.784,4.272)--(10.786,4.276)--(10.788,4.279)--(10.789,4.282)--(10.791,4.286)%
--(10.792,4.289)--(10.793,4.293)--(10.794,4.297)--(10.795,4.300)--(10.795,4.304)--(10.795,4.308)--cycle;
%
\gpfill{rgb color={0.000,0.000,0.000},opacity=0.15} (10.848,4.326)--(10.847,4.329)--(10.847,4.332)--(10.847,4.335)%
--(10.846,4.339)--(10.845,4.342)--(10.844,4.345)--(10.843,4.348)--(10.842,4.351)%
--(10.841,4.354)--(10.839,4.357)--(10.837,4.360)--(10.835,4.363)--(10.833,4.365)%
--(10.831,4.368)--(10.829,4.370)--(10.827,4.372)--(10.824,4.374)--(10.822,4.376)%
--(10.819,4.378)--(10.816,4.380)--(10.813,4.382)--(10.810,4.383)--(10.807,4.384)%
--(10.804,4.385)--(10.801,4.386)--(10.798,4.387)--(10.794,4.388)--(10.791,4.388)%
--(10.788,4.388)--(10.785,4.389)--(10.781,4.388)--(10.778,4.388)--(10.775,4.388)%
--(10.771,4.387)--(10.768,4.386)--(10.765,4.385)--(10.762,4.384)--(10.759,4.383)%
--(10.756,4.382)--(10.753,4.380)--(10.750,4.378)--(10.747,4.376)--(10.745,4.374)%
--(10.742,4.372)--(10.740,4.370)--(10.738,4.368)--(10.736,4.365)--(10.734,4.363)%
--(10.732,4.360)--(10.730,4.357)--(10.728,4.354)--(10.727,4.351)--(10.726,4.348)%
--(10.725,4.345)--(10.724,4.342)--(10.723,4.339)--(10.722,4.335)--(10.722,4.332)%
--(10.722,4.329)--(10.722,4.326)--(10.722,4.322)--(10.722,4.319)--(10.722,4.316)%
--(10.723,4.312)--(10.724,4.309)--(10.725,4.306)--(10.726,4.303)--(10.727,4.300)%
--(10.728,4.297)--(10.730,4.294)--(10.732,4.291)--(10.734,4.288)--(10.736,4.286)%
--(10.738,4.283)--(10.740,4.281)--(10.742,4.279)--(10.745,4.277)--(10.747,4.275)%
--(10.750,4.273)--(10.753,4.271)--(10.756,4.269)--(10.759,4.268)--(10.762,4.267)%
--(10.765,4.266)--(10.768,4.265)--(10.771,4.264)--(10.775,4.263)--(10.778,4.263)%
--(10.781,4.263)--(10.785,4.263)--(10.788,4.263)--(10.791,4.263)--(10.794,4.263)%
--(10.798,4.264)--(10.801,4.265)--(10.804,4.266)--(10.807,4.267)--(10.810,4.268)%
--(10.813,4.269)--(10.816,4.271)--(10.819,4.273)--(10.822,4.275)--(10.824,4.277)%
--(10.827,4.279)--(10.829,4.281)--(10.831,4.283)--(10.833,4.286)--(10.835,4.288)%
--(10.837,4.291)--(10.839,4.294)--(10.841,4.297)--(10.842,4.300)--(10.843,4.303)%
--(10.844,4.306)--(10.845,4.309)--(10.846,4.312)--(10.847,4.316)--(10.847,4.319)--(10.847,4.322)--cycle;
%
\gpfill{rgb color={0.000,0.000,0.000},opacity=0.15} (10.894,4.339)--(10.893,4.341)--(10.893,4.344)--(10.893,4.347)%
--(10.892,4.350)--(10.892,4.353)--(10.891,4.355)--(10.890,4.358)--(10.889,4.361)%
--(10.888,4.363)--(10.886,4.366)--(10.885,4.368)--(10.883,4.371)--(10.881,4.373)%
--(10.879,4.375)--(10.877,4.377)--(10.875,4.379)--(10.873,4.381)--(10.871,4.383)%
--(10.868,4.385)--(10.866,4.386)--(10.863,4.388)--(10.861,4.389)--(10.858,4.390)%
--(10.855,4.391)--(10.853,4.392)--(10.850,4.392)--(10.847,4.393)--(10.844,4.393)%
--(10.841,4.393)--(10.839,4.394)--(10.836,4.393)--(10.833,4.393)--(10.830,4.393)%
--(10.827,4.392)--(10.824,4.392)--(10.822,4.391)--(10.819,4.390)--(10.816,4.389)%
--(10.814,4.388)--(10.811,4.386)--(10.809,4.385)--(10.806,4.383)--(10.804,4.381)%
--(10.802,4.379)--(10.800,4.377)--(10.798,4.375)--(10.796,4.373)--(10.794,4.371)%
--(10.792,4.368)--(10.791,4.366)--(10.789,4.363)--(10.788,4.361)--(10.787,4.358)%
--(10.786,4.355)--(10.785,4.353)--(10.785,4.350)--(10.784,4.347)--(10.784,4.344)%
--(10.784,4.341)--(10.784,4.339)--(10.784,4.336)--(10.784,4.333)--(10.784,4.330)%
--(10.785,4.327)--(10.785,4.324)--(10.786,4.322)--(10.787,4.319)--(10.788,4.316)%
--(10.789,4.314)--(10.791,4.311)--(10.792,4.309)--(10.794,4.306)--(10.796,4.304)%
--(10.798,4.302)--(10.800,4.300)--(10.802,4.298)--(10.804,4.296)--(10.806,4.294)%
--(10.809,4.292)--(10.811,4.291)--(10.814,4.289)--(10.816,4.288)--(10.819,4.287)%
--(10.822,4.286)--(10.824,4.285)--(10.827,4.285)--(10.830,4.284)--(10.833,4.284)%
--(10.836,4.284)--(10.839,4.284)--(10.841,4.284)--(10.844,4.284)--(10.847,4.284)%
--(10.850,4.285)--(10.853,4.285)--(10.855,4.286)--(10.858,4.287)--(10.861,4.288)%
--(10.863,4.289)--(10.866,4.291)--(10.868,4.292)--(10.871,4.294)--(10.873,4.296)%
--(10.875,4.298)--(10.877,4.300)--(10.879,4.302)--(10.881,4.304)--(10.883,4.306)%
--(10.885,4.309)--(10.886,4.311)--(10.888,4.314)--(10.889,4.316)--(10.890,4.319)%
--(10.891,4.322)--(10.892,4.324)--(10.892,4.327)--(10.893,4.330)--(10.893,4.333)--(10.893,4.336)--cycle;
%
\gpfill{rgb color={0.000,0.000,0.000},opacity=0.15} (10.932,4.350)--(10.931,4.352)--(10.931,4.354)--(10.931,4.357)%
--(10.930,4.359)--(10.930,4.362)--(10.929,4.364)--(10.928,4.366)--(10.927,4.369)%
--(10.926,4.371)--(10.925,4.373)--(10.924,4.375)--(10.923,4.377)--(10.921,4.379)%
--(10.919,4.381)--(10.918,4.383)--(10.916,4.384)--(10.914,4.386)--(10.912,4.388)%
--(10.910,4.389)--(10.908,4.390)--(10.906,4.391)--(10.904,4.392)--(10.901,4.393)%
--(10.899,4.394)--(10.897,4.395)--(10.894,4.395)--(10.892,4.396)--(10.889,4.396)%
--(10.887,4.396)--(10.885,4.397)--(10.882,4.396)--(10.880,4.396)--(10.877,4.396)%
--(10.875,4.395)--(10.872,4.395)--(10.870,4.394)--(10.868,4.393)--(10.865,4.392)%
--(10.863,4.391)--(10.861,4.390)--(10.859,4.389)--(10.857,4.388)--(10.855,4.386)%
--(10.853,4.384)--(10.851,4.383)--(10.850,4.381)--(10.848,4.379)--(10.846,4.377)%
--(10.845,4.375)--(10.844,4.373)--(10.843,4.371)--(10.842,4.369)--(10.841,4.366)%
--(10.840,4.364)--(10.839,4.362)--(10.839,4.359)--(10.838,4.357)--(10.838,4.354)%
--(10.838,4.352)--(10.838,4.350)--(10.838,4.347)--(10.838,4.345)--(10.838,4.342)%
--(10.839,4.340)--(10.839,4.337)--(10.840,4.335)--(10.841,4.333)--(10.842,4.330)%
--(10.843,4.328)--(10.844,4.326)--(10.845,4.324)--(10.846,4.322)--(10.848,4.320)%
--(10.850,4.318)--(10.851,4.316)--(10.853,4.315)--(10.855,4.313)--(10.857,4.311)%
--(10.859,4.310)--(10.861,4.309)--(10.863,4.308)--(10.865,4.307)--(10.868,4.306)%
--(10.870,4.305)--(10.872,4.304)--(10.875,4.304)--(10.877,4.303)--(10.880,4.303)%
--(10.882,4.303)--(10.885,4.303)--(10.887,4.303)--(10.889,4.303)--(10.892,4.303)%
--(10.894,4.304)--(10.897,4.304)--(10.899,4.305)--(10.901,4.306)--(10.904,4.307)%
--(10.906,4.308)--(10.908,4.309)--(10.910,4.310)--(10.912,4.311)--(10.914,4.313)%
--(10.916,4.315)--(10.918,4.316)--(10.919,4.318)--(10.921,4.320)--(10.923,4.322)%
--(10.924,4.324)--(10.925,4.326)--(10.926,4.328)--(10.927,4.330)--(10.928,4.333)%
--(10.929,4.335)--(10.930,4.337)--(10.930,4.340)--(10.931,4.342)--(10.931,4.345)--(10.931,4.347)--cycle;
%
\gpfill{rgb color={0.000,0.000,0.000},opacity=0.15} (10.962,4.358)--(10.961,4.360)--(10.961,4.362)--(10.961,4.364)%
--(10.961,4.366)--(10.960,4.368)--(10.960,4.370)--(10.959,4.371)--(10.958,4.373)%
--(10.957,4.375)--(10.956,4.377)--(10.955,4.379)--(10.954,4.380)--(10.953,4.382)%
--(10.951,4.384)--(10.950,4.385)--(10.949,4.386)--(10.947,4.388)--(10.945,4.389)%
--(10.944,4.390)--(10.942,4.391)--(10.940,4.392)--(10.938,4.393)--(10.936,4.394)%
--(10.935,4.395)--(10.933,4.395)--(10.931,4.396)--(10.929,4.396)--(10.927,4.396)%
--(10.925,4.396)--(10.923,4.397)--(10.920,4.396)--(10.918,4.396)--(10.916,4.396)%
--(10.914,4.396)--(10.912,4.395)--(10.910,4.395)--(10.909,4.394)--(10.907,4.393)%
--(10.905,4.392)--(10.903,4.391)--(10.901,4.390)--(10.900,4.389)--(10.898,4.388)%
--(10.896,4.386)--(10.895,4.385)--(10.894,4.384)--(10.892,4.382)--(10.891,4.380)%
--(10.890,4.379)--(10.889,4.377)--(10.888,4.375)--(10.887,4.373)--(10.886,4.371)%
--(10.885,4.370)--(10.885,4.368)--(10.884,4.366)--(10.884,4.364)--(10.884,4.362)%
--(10.884,4.360)--(10.884,4.358)--(10.884,4.355)--(10.884,4.353)--(10.884,4.351)%
--(10.884,4.349)--(10.885,4.347)--(10.885,4.345)--(10.886,4.344)--(10.887,4.342)%
--(10.888,4.340)--(10.889,4.338)--(10.890,4.336)--(10.891,4.335)--(10.892,4.333)%
--(10.894,4.331)--(10.895,4.330)--(10.896,4.329)--(10.898,4.327)--(10.900,4.326)%
--(10.901,4.325)--(10.903,4.324)--(10.905,4.323)--(10.907,4.322)--(10.909,4.321)%
--(10.910,4.320)--(10.912,4.320)--(10.914,4.319)--(10.916,4.319)--(10.918,4.319)%
--(10.920,4.319)--(10.923,4.319)--(10.925,4.319)--(10.927,4.319)--(10.929,4.319)%
--(10.931,4.319)--(10.933,4.320)--(10.935,4.320)--(10.936,4.321)--(10.938,4.322)%
--(10.940,4.323)--(10.942,4.324)--(10.944,4.325)--(10.945,4.326)--(10.947,4.327)%
--(10.949,4.329)--(10.950,4.330)--(10.951,4.331)--(10.953,4.333)--(10.954,4.335)%
--(10.955,4.336)--(10.956,4.338)--(10.957,4.340)--(10.958,4.342)--(10.959,4.344)%
--(10.960,4.345)--(10.960,4.347)--(10.961,4.349)--(10.961,4.351)--(10.961,4.353)--(10.961,4.355)--cycle;
%
\gpfill{rgb color={0.000,0.000,0.000},opacity=0.15} (10.982,4.365)--(10.981,4.366)--(10.981,4.368)--(10.981,4.369)%
--(10.981,4.371)--(10.980,4.372)--(10.980,4.374)--(10.980,4.375)--(10.979,4.377)%
--(10.978,4.378)--(10.977,4.380)--(10.977,4.381)--(10.976,4.382)--(10.975,4.383)%
--(10.974,4.385)--(10.973,4.386)--(10.972,4.387)--(10.970,4.388)--(10.969,4.389)%
--(10.968,4.390)--(10.967,4.390)--(10.965,4.391)--(10.964,4.392)--(10.962,4.393)%
--(10.961,4.393)--(10.959,4.393)--(10.958,4.394)--(10.956,4.394)--(10.955,4.394)%
--(10.953,4.394)--(10.952,4.395)--(10.950,4.394)--(10.948,4.394)--(10.947,4.394)%
--(10.945,4.394)--(10.944,4.393)--(10.942,4.393)--(10.941,4.393)--(10.939,4.392)%
--(10.938,4.391)--(10.937,4.390)--(10.935,4.390)--(10.934,4.389)--(10.933,4.388)%
--(10.931,4.387)--(10.930,4.386)--(10.929,4.385)--(10.928,4.383)--(10.927,4.382)%
--(10.926,4.381)--(10.926,4.380)--(10.925,4.378)--(10.924,4.377)--(10.923,4.375)%
--(10.923,4.374)--(10.923,4.372)--(10.922,4.371)--(10.922,4.369)--(10.922,4.368)%
--(10.922,4.366)--(10.922,4.365)--(10.922,4.363)--(10.922,4.361)--(10.922,4.360)%
--(10.922,4.358)--(10.923,4.357)--(10.923,4.355)--(10.923,4.354)--(10.924,4.352)%
--(10.925,4.351)--(10.926,4.350)--(10.926,4.348)--(10.927,4.347)--(10.928,4.346)%
--(10.929,4.344)--(10.930,4.343)--(10.931,4.342)--(10.933,4.341)--(10.934,4.340)%
--(10.935,4.339)--(10.937,4.339)--(10.938,4.338)--(10.939,4.337)--(10.941,4.336)%
--(10.942,4.336)--(10.944,4.336)--(10.945,4.335)--(10.947,4.335)--(10.948,4.335)%
--(10.950,4.335)--(10.952,4.335)--(10.953,4.335)--(10.955,4.335)--(10.956,4.335)%
--(10.958,4.335)--(10.959,4.336)--(10.961,4.336)--(10.962,4.336)--(10.964,4.337)%
--(10.965,4.338)--(10.967,4.339)--(10.968,4.339)--(10.969,4.340)--(10.970,4.341)%
--(10.972,4.342)--(10.973,4.343)--(10.974,4.344)--(10.975,4.346)--(10.976,4.347)%
--(10.977,4.348)--(10.977,4.350)--(10.978,4.351)--(10.979,4.352)--(10.980,4.354)%
--(10.980,4.355)--(10.980,4.357)--(10.981,4.358)--(10.981,4.360)--(10.981,4.361)--(10.981,4.363)--cycle;
%
\gpfill{rgb color={0.000,0.000,0.000},opacity=0.15} (10.995,4.370)--(10.994,4.371)--(10.994,4.372)--(10.994,4.373)%
--(10.994,4.374)--(10.994,4.375)--(10.993,4.376)--(10.993,4.377)--(10.993,4.378)%
--(10.992,4.379)--(10.992,4.381)--(10.991,4.381)--(10.990,4.382)--(10.990,4.383)%
--(10.989,4.384)--(10.988,4.385)--(10.987,4.386)--(10.986,4.387)--(10.985,4.387)%
--(10.984,4.388)--(10.984,4.389)--(10.982,4.389)--(10.981,4.390)--(10.980,4.390)%
--(10.979,4.390)--(10.978,4.391)--(10.977,4.391)--(10.976,4.391)--(10.975,4.391)%
--(10.974,4.391)--(10.973,4.392)--(10.971,4.391)--(10.970,4.391)--(10.969,4.391)%
--(10.968,4.391)--(10.967,4.391)--(10.966,4.390)--(10.965,4.390)--(10.964,4.390)%
--(10.963,4.389)--(10.962,4.389)--(10.961,4.388)--(10.960,4.387)--(10.959,4.387)%
--(10.958,4.386)--(10.957,4.385)--(10.956,4.384)--(10.955,4.383)--(10.955,4.382)%
--(10.954,4.381)--(10.953,4.381)--(10.953,4.379)--(10.952,4.378)--(10.952,4.377)%
--(10.952,4.376)--(10.951,4.375)--(10.951,4.374)--(10.951,4.373)--(10.951,4.372)%
--(10.951,4.371)--(10.951,4.370)--(10.951,4.368)--(10.951,4.367)--(10.951,4.366)%
--(10.951,4.365)--(10.951,4.364)--(10.952,4.363)--(10.952,4.362)--(10.952,4.361)%
--(10.953,4.360)--(10.953,4.359)--(10.954,4.358)--(10.955,4.357)--(10.955,4.356)%
--(10.956,4.355)--(10.957,4.354)--(10.958,4.353)--(10.959,4.352)--(10.960,4.352)%
--(10.961,4.351)--(10.962,4.350)--(10.963,4.350)--(10.964,4.349)--(10.965,4.349)%
--(10.966,4.349)--(10.967,4.348)--(10.968,4.348)--(10.969,4.348)--(10.970,4.348)%
--(10.971,4.348)--(10.973,4.348)--(10.974,4.348)--(10.975,4.348)--(10.976,4.348)%
--(10.977,4.348)--(10.978,4.348)--(10.979,4.349)--(10.980,4.349)--(10.981,4.349)%
--(10.982,4.350)--(10.984,4.350)--(10.984,4.351)--(10.985,4.352)--(10.986,4.352)%
--(10.987,4.353)--(10.988,4.354)--(10.989,4.355)--(10.990,4.356)--(10.990,4.357)%
--(10.991,4.358)--(10.992,4.359)--(10.992,4.360)--(10.993,4.361)--(10.993,4.362)%
--(10.993,4.363)--(10.994,4.364)--(10.994,4.365)--(10.994,4.366)--(10.994,4.367)--(10.994,4.368)--cycle;
%
\gpfill{rgb color={0.000,0.000,0.000},opacity=0.15} (10.999,4.373)--(10.998,4.373)--(10.998,4.374)--(10.998,4.375)%
--(10.998,4.375)--(10.998,4.376)--(10.998,4.377)--(10.998,4.377)--(10.997,4.378)%
--(10.997,4.378)--(10.997,4.379)--(10.996,4.380)--(10.996,4.380)--(10.996,4.381)%
--(10.995,4.381)--(10.995,4.382)--(10.994,4.382)--(10.994,4.383)--(10.993,4.383)%
--(10.993,4.383)--(10.992,4.384)--(10.991,4.384)--(10.991,4.384)--(10.990,4.385)%
--(10.990,4.385)--(10.989,4.385)--(10.988,4.385)--(10.988,4.385)--(10.987,4.385)%
--(10.986,4.385)--(10.986,4.386)--(10.985,4.385)--(10.984,4.385)--(10.983,4.385)%
--(10.983,4.385)--(10.982,4.385)--(10.981,4.385)--(10.981,4.385)--(10.980,4.384)%
--(10.980,4.384)--(10.979,4.384)--(10.978,4.383)--(10.978,4.383)--(10.977,4.383)%
--(10.977,4.382)--(10.976,4.382)--(10.976,4.381)--(10.975,4.381)--(10.975,4.380)%
--(10.975,4.380)--(10.974,4.379)--(10.974,4.378)--(10.974,4.378)--(10.973,4.377)%
--(10.973,4.377)--(10.973,4.376)--(10.973,4.375)--(10.973,4.375)--(10.973,4.374)%
--(10.973,4.373)--(10.973,4.373)--(10.973,4.372)--(10.973,4.371)--(10.973,4.370)%
--(10.973,4.370)--(10.973,4.369)--(10.973,4.368)--(10.973,4.368)--(10.974,4.367)%
--(10.974,4.367)--(10.974,4.366)--(10.975,4.365)--(10.975,4.365)--(10.975,4.364)%
--(10.976,4.364)--(10.976,4.363)--(10.977,4.363)--(10.977,4.362)--(10.978,4.362)%
--(10.978,4.362)--(10.979,4.361)--(10.980,4.361)--(10.980,4.361)--(10.981,4.360)%
--(10.981,4.360)--(10.982,4.360)--(10.983,4.360)--(10.983,4.360)--(10.984,4.360)%
--(10.985,4.360)--(10.986,4.360)--(10.986,4.360)--(10.987,4.360)--(10.988,4.360)%
--(10.988,4.360)--(10.989,4.360)--(10.990,4.360)--(10.990,4.360)--(10.991,4.361)%
--(10.991,4.361)--(10.992,4.361)--(10.993,4.362)--(10.993,4.362)--(10.994,4.362)%
--(10.994,4.363)--(10.995,4.363)--(10.995,4.364)--(10.996,4.364)--(10.996,4.365)%
--(10.996,4.365)--(10.997,4.366)--(10.997,4.367)--(10.997,4.367)--(10.998,4.368)%
--(10.998,4.368)--(10.998,4.369)--(10.998,4.370)--(10.998,4.370)--(10.998,4.371)--(10.998,4.372)--cycle;
\gpcolor{rgb color={0.000,0.000,0.000}}
\gpsetlinetype{gp lt border}
\gpsetdashtype{gp dt solid}
\gpsetlinewidth{3.00}
\draw[gp path] (10.990,4.375)--(10.988,4.375)--(10.984,4.376)--(10.976,4.377)--(10.966,4.379)%
--(10.952,4.382)--(10.935,4.385)--(10.916,4.389)--(10.893,4.394)--(10.868,4.399)--(10.839,4.404)%
--(10.808,4.410)--(10.774,4.417)--(10.737,4.424)--(10.697,4.431)--(10.654,4.439)--(10.608,4.448)%
--(10.559,4.457)--(10.508,4.466)--(10.454,4.476)--(10.398,4.486)--(10.338,4.496)--(10.277,4.507)%
--(10.212,4.518)--(10.145,4.530)--(10.076,4.541)--(10.004,4.553)--(9.929,4.566)--(9.853,4.578)%
--(9.774,4.590)--(9.693,4.603)--(9.609,4.616)--(9.523,4.629)--(9.436,4.642)--(9.346,4.655)%
--(9.254,4.668)--(9.161,4.681)--(9.065,4.694)--(8.968,4.707)--(8.869,4.720)--(8.768,4.733)%
--(8.666,4.745)--(8.562,4.758)--(8.456,4.770)--(8.350,4.782)--(8.241,4.794)--(8.132,4.806)%
--(8.021,4.817)--(7.909,4.828)--(7.796,4.839)--(7.683,4.850)--(7.568,4.860)--(7.452,4.869)%
--(7.335,4.879)--(7.218,4.888)--(7.100,4.896)--(6.982,4.904)--(6.863,4.912)--(6.744,4.919)%
--(6.624,4.926)--(6.504,4.932)--(6.384,4.938)--(6.264,4.943)--(6.143,4.948)--(6.023,4.952)%
--(5.903,4.956)--(5.783,4.959)--(5.663,4.962)--(5.544,4.964)--(5.425,4.965)--(5.307,4.966)%
--(5.189,4.966)--(5.072,4.966)--(4.955,4.965)--(4.839,4.964)--(4.724,4.961)--(4.611,4.959)%
--(4.498,4.956)--(4.386,4.952)--(4.275,4.947)--(4.166,4.942)--(4.057,4.937)--(3.951,4.931)%
--(3.845,4.924)--(3.741,4.917)--(3.639,4.909)--(3.538,4.901)--(3.439,4.892)--(3.342,4.883)%
--(3.246,4.873)--(3.153,4.863)--(3.061,4.852)--(2.971,4.841)--(2.884,4.830)--(2.798,4.818)%
--(2.714,4.806)--(2.633,4.794)--(2.554,4.781)--(2.478,4.768)--(2.403,4.755)--(2.331,4.742)%
--(2.262,4.728)--(2.195,4.714)--(2.130,4.700)--(2.069,4.686)--(2.009,4.672)--(1.953,4.658)%
--(1.899,4.643)--(1.848,4.629)--(1.799,4.614)--(1.753,4.599)--(1.710,4.584)--(1.670,4.569)%
--(1.633,4.555)--(1.599,4.540)--(1.568,4.525)--(1.539,4.510)--(1.514,4.495)--(1.491,4.480)%
--(1.472,4.465)--(1.455,4.450)--(1.441,4.435)--(1.431,4.420)--(1.423,4.405)--(1.419,4.390)%
--(1.417,4.375)--(1.419,4.354)--(1.423,4.334)--(1.431,4.313)--(1.441,4.293)--(1.455,4.273)%
--(1.472,4.253)--(1.491,4.233)--(1.514,4.214)--(1.539,4.194)--(1.568,4.175)--(1.599,4.156)%
--(1.633,4.138)--(1.670,4.120)--(1.710,4.102)--(1.753,4.084)--(1.799,4.067)--(1.848,4.051)%
--(1.899,4.035)--(1.953,4.019)--(2.009,4.004)--(2.069,3.989)--(2.130,3.975)--(2.195,3.962)%
--(2.262,3.949)--(2.331,3.936)--(2.403,3.924)--(2.478,3.913)--(2.554,3.902)--(2.633,3.892)%
--(2.714,3.882)--(2.798,3.873)--(2.884,3.864)--(2.971,3.856)--(3.061,3.849)--(3.153,3.842)%
--(3.246,3.836)--(3.342,3.830)--(3.439,3.825)--(3.538,3.821)--(3.639,3.817)--(3.741,3.814)%
--(3.845,3.811)--(3.951,3.809)--(4.057,3.807)--(4.166,3.806)--(4.275,3.806)--(4.386,3.806)%
--(4.498,3.807)--(4.611,3.808)--(4.724,3.810)--(4.839,3.812)--(4.955,3.815)--(5.072,3.818)%
--(5.189,3.822)--(5.307,3.826)--(5.425,3.831)--(5.544,3.837)--(5.663,3.842)--(5.783,3.849)%
--(5.903,3.855)--(6.023,3.862)--(6.143,3.870)--(6.264,3.878)--(6.384,3.886)--(6.504,3.895)%
--(6.624,3.903)--(6.744,3.912)--(6.863,3.922)--(6.982,3.931)--(7.100,3.941)--(7.218,3.951)%
--(7.335,3.961)--(7.452,3.972)--(7.568,3.982)--(7.683,3.993)--(7.796,4.004)--(7.909,4.014)%
--(8.021,4.025)--(8.132,4.036)--(8.241,4.047)--(8.350,4.058)--(8.456,4.069)--(8.562,4.080)%
--(8.666,4.091)--(8.768,4.102)--(8.869,4.113)--(8.968,4.123)--(9.065,4.134)--(9.161,4.145)%
--(9.254,4.155)--(9.346,4.166)--(9.436,4.176)--(9.523,4.186)--(9.609,4.196)--(9.693,4.206)%
--(9.774,4.216)--(9.853,4.226)--(9.929,4.235)--(10.004,4.244)--(10.076,4.253)--(10.145,4.262)%
--(10.212,4.270)--(10.277,4.279)--(10.338,4.287)--(10.398,4.294)--(10.454,4.302)--(10.508,4.309)%
--(10.559,4.316)--(10.608,4.322)--(10.654,4.328)--(10.697,4.334)--(10.737,4.340)--(10.774,4.345)%
--(10.808,4.349)--(10.839,4.354)--(10.868,4.358)--(10.893,4.361)--(10.916,4.364)--(10.935,4.367)%
--(10.952,4.369)--(10.966,4.371)--(10.976,4.373)--(10.984,4.374)--(10.988,4.374)--cycle;
\draw[gp path] (10.990,4.375)--(10.988,4.375)--(10.984,4.375)--(10.976,4.375)--(10.966,4.375)%
--(10.952,4.375)--(10.935,4.376)--(10.916,4.376)--(10.893,4.376)--(10.868,4.377)--(10.839,4.377)%
--(10.808,4.378)--(10.774,4.379)--(10.737,4.379)--(10.697,4.380)--(10.654,4.381)--(10.608,4.382)%
--(10.559,4.383)--(10.508,4.384)--(10.454,4.385)--(10.398,4.386)--(10.338,4.387)--(10.277,4.388)%
--(10.212,4.390)--(10.145,4.391)--(10.076,4.393)--(10.004,4.394)--(9.929,4.396)--(9.853,4.398)%
--(9.774,4.400)--(9.693,4.402)--(9.609,4.405)--(9.523,4.407)--(9.436,4.411)--(9.346,4.414)%
--(9.254,4.419)--(9.161,4.424)--(9.065,4.429)--(8.968,4.435)--(8.869,4.443)--(8.768,4.451)%
--(8.666,4.460)--(8.562,4.471)--(8.456,4.483)--(8.350,4.496)--(8.241,4.511)--(8.132,4.528)%
--(8.021,4.547)--(7.909,4.567)--(7.796,4.590)--(7.683,4.614)--(7.568,4.641)--(7.452,4.670)%
--(7.335,4.701)--(7.218,4.734)--(7.100,4.769)--(6.982,4.807)--(6.863,4.847)--(6.744,4.888)%
--(6.624,4.932)--(6.504,4.977)--(6.384,5.023)--(6.264,5.071)--(6.143,5.120)--(6.023,5.170)%
--(5.903,5.221)--(5.783,5.272)--(5.663,5.323)--(5.544,5.374)--(5.425,5.424)--(5.307,5.473)%
--(5.189,5.522)--(5.072,5.569)--(4.955,5.614)--(4.839,5.658)--(4.724,5.699)--(4.611,5.738)%
--(4.498,5.774)--(4.386,5.808)--(4.275,5.838)--(4.166,5.865)--(4.057,5.889)--(3.951,5.909)%
--(3.845,5.926)--(3.741,5.939)--(3.639,5.948)--(3.538,5.953)--(3.439,5.955)--(3.342,5.953)%
--(3.246,5.948)--(3.153,5.938)--(3.061,5.925)--(2.971,5.909)--(2.884,5.889)--(2.798,5.866)%
--(2.714,5.840)--(2.633,5.811)--(2.554,5.779)--(2.478,5.745)--(2.403,5.708)--(2.331,5.668)%
--(2.262,5.627)--(2.195,5.583)--(2.130,5.538)--(2.069,5.492)--(2.009,5.443)--(1.953,5.394)%
--(1.899,5.344)--(1.848,5.292)--(1.799,5.240)--(1.753,5.187)--(1.710,5.134)--(1.670,5.080)%
--(1.633,5.026)--(1.599,4.972)--(1.568,4.918)--(1.539,4.863)--(1.514,4.809)--(1.491,4.754)%
--(1.472,4.700)--(1.455,4.646)--(1.441,4.591)--(1.431,4.537)--(1.423,4.483)--(1.419,4.429)%
--(1.417,4.375)--(1.419,4.320)--(1.423,4.267)--(1.431,4.213)--(1.441,4.160)--(1.455,4.108)%
--(1.472,4.058)--(1.491,4.008)--(1.514,3.960)--(1.539,3.913)--(1.568,3.868)--(1.599,3.825)%
--(1.633,3.784)--(1.670,3.745)--(1.710,3.709)--(1.753,3.674)--(1.799,3.642)--(1.848,3.613)%
--(1.899,3.585)--(1.953,3.560)--(2.009,3.538)--(2.069,3.518)--(2.130,3.500)--(2.195,3.484)%
--(2.262,3.471)--(2.331,3.459)--(2.403,3.450)--(2.478,3.442)--(2.554,3.437)--(2.633,3.433)%
--(2.714,3.430)--(2.798,3.429)--(2.884,3.430)--(2.971,3.431)--(3.061,3.434)--(3.153,3.438)%
--(3.246,3.442)--(3.342,3.448)--(3.439,3.454)--(3.538,3.461)--(3.639,3.468)--(3.741,3.475)%
--(3.845,3.483)--(3.951,3.492)--(4.057,3.500)--(4.166,3.509)--(4.275,3.518)--(4.386,3.527)%
--(4.498,3.536)--(4.611,3.545)--(4.724,3.554)--(4.839,3.564)--(4.955,3.573)--(5.072,3.582)%
--(5.189,3.592)--(5.307,3.601)--(5.425,3.610)--(5.544,3.620)--(5.663,3.629)--(5.783,3.639)%
--(5.903,3.649)--(6.023,3.658)--(6.143,3.668)--(6.264,3.678)--(6.384,3.689)--(6.504,3.699)%
--(6.624,3.709)--(6.744,3.720)--(6.863,3.731)--(6.982,3.742)--(7.100,3.753)--(7.218,3.764)%
--(7.335,3.776)--(7.452,3.788)--(7.568,3.800)--(7.683,3.812)--(7.796,3.825)--(7.909,3.837)%
--(8.021,3.850)--(8.132,3.863)--(8.241,3.877)--(8.350,3.890)--(8.456,3.904)--(8.562,3.918)%
--(8.666,3.932)--(8.768,3.947)--(8.869,3.961)--(8.968,3.976)--(9.065,3.991)--(9.161,4.006)%
--(9.254,4.021)--(9.346,4.036)--(9.436,4.051)--(9.523,4.066)--(9.609,4.082)--(9.693,4.097)%
--(9.774,4.112)--(9.853,4.127)--(9.929,4.142)--(10.004,4.156)--(10.076,4.171)--(10.145,4.185)%
--(10.212,4.199)--(10.277,4.212)--(10.338,4.226)--(10.398,4.238)--(10.454,4.251)--(10.508,4.263)%
--(10.559,4.274)--(10.608,4.285)--(10.654,4.296)--(10.697,4.305)--(10.737,4.315)--(10.774,4.323)%
--(10.808,4.331)--(10.839,4.339)--(10.868,4.345)--(10.893,4.351)--(10.916,4.357)--(10.935,4.362)%
--(10.952,4.365)--(10.966,4.369)--(10.976,4.371)--(10.984,4.373)--(10.988,4.374)--cycle;
%% coordinates of the plot area
%\gpdefrectangularnode{gp plot 1}{\pgfpoint{0.460cm}{1.503cm}}{\pgfpoint{11.947cm}{7.246cm}}
\end{tikzpicture}
%% gnuplot variables
\end{center}
\caption{Circular removal regions \footnotesize{(150 vertices, $D = 1$)}}
\centering\sffamily\footnotesize
The set of circular removal regions is the same for each pair of models, regardless of which is the old and the new one.
\end{figure}


\subsection{Practical application}

\todo[inline]{Describe how mesh remodelling is to be applied to the dynamics simulator, how the latter works (implicit vs explicit), in what ways remodelling can speed-up the process (element preservation), etc.. The time to perform simulation is 5 orders of magnitude above, so speed-up in remodelling is not that relevant.}