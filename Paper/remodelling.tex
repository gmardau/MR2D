\section{Mesh remodelling}

\paragraph{\textit{Note}}\textit{The mesh remodelling method described in this chapter can be applied to any kind of mesh. Even so, given that the whole work (and this document) is based on Delaunay triangulations, please consider the mesh to be of such type for a better understanding of the method and the way it is applied.}

\paragraph{Motivation} Mesh remodelling is a method which purpose is to replace one model with another in a given mesh. It was designed to be as \textit{robust}, \textit{simple} and \textit{general} as possible; in other words, to not be prone to errors arising from special cases, to modify the mesh using only basic, well established algorithms, and to not make any assumptions regarding the type of mesh or models. The main goal of mesh remodelling, derived from the practical application addressed in this work, is to maximise element preservation, i.e.\ the percentage of elements from the old mesh that were maintained in the new mesh.

\paragraph{Methodology} Since mesh remodelling relies solely on basic mesh modification procedures --- vertex insertion and removal --- the mesh is required either to be present at the locations of the new vertices or to allow their insertion outside its boundaries. The algorithm itself is relatively simple, almost trivial. It starts by inserting the initial discretisation vertices of the new model into the mesh. Then, a circular removal region is defined for each non-Steiner vertex of both models, inside of which all Steiner vertices are to be removed. The radius of such regions is determined by the distance between a vertex and the closest vertex of the other model and bounded from below by the maximum distance to its two current boundary neighbours. The computation of this distance can be easily accomplished using a quadtree. Additionally, the distance can be multiplied by a pre-defined value $D$, denominated \textit{distance factor}. Finally, all boundary vertices of the old model are removed from the mesh.

\paragraph{Mesh quality, refinement and gradation} Due to the algorithm's simplicity of operations, it is possible to maintain a Delaunay triangulation throughout the remodelling process simply by using the appropriate subroutines for vertex insertion and removal, which assures that the mesh retains its optimal properties and allows a subsequent Delaunay refinement process to be performed, thus guaranteeing that the quality of the new mesh is no worse than the last one. The mesh gradation however, cannot be the same as in a newly built mesh, in part because the length scale of the existing Steiner vertices was computed considering older models, at different distances. The value of $D$ can be increased to improve the gradation component. The computation of length scale for the new model's discretisation vertices is a combination of the non-Steiner boundary vertex and Steiner vertex variants; $LS_b$ and $LS_s$, respectively. The decision of which variant to employ is determined by the type of each neighbour.
\begin{equation*}
LS_b(v) = \min_{\text{neighbours }u_i} \left(
\begin{cases}
\dfrac{\|u_i-v\|}{R} & \text{, } u_i \text{ is non-Steiner}\\
LS(u_i) + \dfrac{\|u_i-v\|}{G} & \text{, otherwise}
\end{cases}
\right)
\end{equation*}

\paragraph{Advantages and disadvantages} The main advantage of mesh remodelling is the fact that it uses only subroutines that were needed to build the mesh in the first place, and therefore avoiding the inclusion of additional complexity to the implementation. When compared to mesh deformation methods, mesh remodelling has the convenience of actually removing elements --- as opposed to just translating and reshaping ---, thus producing a mesh with a number of elements more appropriate to the dimensions of new model. In the context of this work's practical application it also benefits from its ability to maintain a Delaunay triangulation, making it easier to guarantee mesh quality on every iteration. On the other hand, and despite being considerably faster, mesh remodelling cannot achieve the same value of mesh gradation as the mesh generation method, which results in a mesh with a different number of elements from the ideal --- usually more.